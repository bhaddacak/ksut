\section{Dutiyakaṇḍa}

\head{120}{120, 243. amhassa mamaṃ savibhattissa se.}
\headtrans{[There is] \pali{mamaṃ}-substitution for \pali{amha} together with vi\-bh\-atti because of \pali{sa}[-vibhatti].}
\sutdef{sabbasseva amhasaddassa savibhattissa mamaṃādeso hoti se vibhattimhi.}
\sutdeftrans{There is a substitution of \pali{mamaṃ} for the whole \pali{amha} together with vibhatti because of \pali{sa}-vibhatti.}
\example[0]{mamaṃ = amha + sa}
\example[0]{mamaṃ dīyate purisena\\{\upshape= [Something] is given to me by a man. (dat.)}}
\example{mamaṃ pariggaho\\{\upshape= My possession. (gen.)}}
\transnote{Do not be misled by \pali{savibhattissa}. It means \pali{saha vibhattissa}. To be precise, \pali{amhasadda} means the sound of `\pali{amha}.' It signifies the word itself. Here, \pali{sadda} is equivalent to \pali{liṅga} (nominal base, not gender) as we have seen earlier.}

\head{121}{121, 233. mayaṃ yomhi paṭhame.}
\headtrans{[There is] \pali{mayaṃ}-substitution because of nominative \pali{yo}-vibhatti.}
\sutdef{sabbasseva amhasaddassa savibhattissa mayaṃādeso hoti yomhi paṭhame.}
\sutdeftrans{There is a substitution of \pali{mayaṃ} for the whole \pali{amha} together with vibhatti because of nominative \pali{yo}-vibhatti.}
\example[0]{mayaṃ = amha + yo}
\example[0]{mayaṃ gacchāma {\upshape(We go.)}}
\example{mayaṃ dema {\upshape(We give.)}}

\head{122}{122, 99. ntussa nto.}
\headtrans{[There is] \pali{nto}-substitution for \pali{ntu}[-paccaya].}
\sutdef{sabbasseva ntupaccayassa savibhattissa ntoādeso hoti yomhi paṭhame.}
\sutdeftrans{There is a substitution of \pali{nto} for the whole \pali{ntu}-paccaya together with vibhatti because of nominative \pali{yo}-vibhatti.}
\example{guṇavanto = guṇavantu + yo}

\head{123}{123, 103. ntassa se vā.}
\headtrans{[There is] \pali{ntassa}-substitution because of \pali{sa}-vibhatti} sometimes.
\sutdef{sabbasseva ntupaccayassa savibhattissa ntassādeso hoti vā se vibhattimhi.}
\sutdeftrans{There is a substitution of \pali{ntassa} for the whole \pali{ntu}-paccaya together with vibhatti because of \pali{sa}-vibhatti sometimes.}
\example{sīlavantassa = sīlavantu + sa (= sīlavato)}

\head{124}{124, 98. ā simhi.}
\headtrans{[There is] \pali{ā}-substitution because of \pali{si}[-vibhatti].}
\sutdef{sabbasseva ntupaccayassa savibhattissa āādeso hoti simhi vibhatatimhi.}
\sutdeftrans{There is a substitution of \pali{ā} for the whole \pali{ntu}-paccaya together with vibhatti because of \pali{si}-vibhatti.}
\example[0]{guṇavā = guṇavantu + si}
\example[0]{paññavā = paññavantu + si}
\example[0]{sīlavā = sīlavantu + si}
\example[0]{balavā = balavantu + si}
\example[0]{dhanavā = dhanavantu + si}
\example[0]{matimā = matimantu + si}
\example[0]{satimā = satimantu + si}
\example{dhitimā = dhitimantu + si}

\head{125}{125, 198. aṃ napuṃsake.}
\headtrans{[There is] \pali{aṃ}-substitution because of a neuter [noun].}
\sutdef{sabbasseva ntupaccayassa savibhattissa aṃādeso hoti simhi vibhattimhi napuṃsake vattamānassa.}
\sutdeftrans{There is a substitution of \pali{aṃ} for the whole \pali{ntu}-paccaya together with vibhatti because of \pali{si}-vibhatti occurring in a neuter [noun].}
\example[0]{guṇavaṃ = guṇavantu + si {\upshape (nt.)}}
\example[0]{guṇavaṃ cittaṃ tiṭṭhati\\{\upshape= The virtuous mind stays.}}
\example[0]{rucimaṃ = rucimantu + si {\upshape (nt.)}}
\example{rucimaṃ pupphaṃ virocati\\{\upshape= A beautiful flower shines.}}
\transnote{A shown in the examples, the neuter nouns include adjectives as well.}

\head{126}{126, 101. avaṇṇā ca ge.}
\headtrans{[There is] \pali{a}-group-substitution also because of \pali{ga}-alias.}
\sutdef{sabbasseva ntupaccayassa savibhattissa aṃavaṇṇā ca honti ge pare.}
\sutdeftrans{There is also [a substitution of] \pali{aṃ} and \pali{a}-group (\pali{a, ā}) for the whole \pali{ntu}-paccaya together with vibhatti because of \pali{ga}-alias (vocative \pali{si}-vibhatti) behind.}
\example[0]{bho guṇavaṃ}
\example[0]{bho guṇava}
\example{bho guṇavā}
\transnote{See the definition of \pali{ga}-alias in \hyperref[sut:57]{Kacc 57}.}

\head{127}{127, 102. totitā sasmiṃnāsu.}
\headtrans{[There are substitutions of] \pali{to}, \pali{ti}, and \pali{tā} because of \pali{sa}, \pali{smiṃ}, and \pali{nā}[-vibhatti].}
\sutdef{sabbasseva ntupaccayassa savibhattissa totitāādesā honti vā sasmiṃnāiccetesu yathāsaṅkhyaṃ.}
\sutdeftrans{There are substitutions of \pali{to}, \pali{ti}, and \pali{tā} for the whole \pali{ntu}-paccaya together with vibhatti because of \pali{sa}, \pali{smiṃ}, and \pali{nā}[-vibhatti] respectively.}
\example[0]{guṇavato = guṇavantu + sa (= guṇavantassa)}
\example[0]{guṇavati = guṇavantu + smiṃ (= guṇavantasmiṃ)}
\example[0]{guṇavatā = guṇavantu + nā (= guṇavantena)}
\example[0]{satimato = satimantu + sa (= satimantassa)}
\example[0]{satimati = satimantu + smiṃ (= satimantasmiṃ)}
\example{satimatā = satimantu + nā (= satimantena)}

\head{128}{128, 104. naṃmhi taṃ vā.}
\headtrans{Because of \pali{naṃ}[-vibhatti], [there is] \pali{taṃ}-substitution sometimes.}
\sutdef{sabbasseva ntupaccayassa savibhattissa taṃādeso hoti vā naṃmhi vibhatatimhi.}
\sutdeftrans{There is a substitution of \pali{taṃ} for the whole \pali{ntu}-paccaya together with vibhatti sometimes because of \pali{naṃ}-vibhatti.}
\example[0]{guṇavataṃ = guṇavantu + naṃ (= guṇavantānaṃ)}
\example{satimataṃ = satimantu + naṃ (= satimantānaṃ)}

\head{129}{129, 222. imassidamaṃsisu napuṃsake.}
\headtrans{[There is] \pali{idaṃ}-substitution for \pali{ima} because of \pali{aṃ} and \pali{si}[-vibhatti] in a neuter [noun].}
\sutdef{sabbasseva imasaddassa savibhattissa idaṃādeso hoti vā aṃsisu napuñasake vattamānassa.}
\sutdeftrans{There is a substitution of \pali{idaṃ} for the whole \pali{ima} together with vibhatti sometimes because of \pali{aṃ} and \pali{si}[-vibhatti] occurring in a neuter [noun].}
\example[0]{idaṃ cittaṃ passasi = imaṃ cittaṃ passasi}
\example{idaṃ cittaṃ tiṭṭhati = imaṃ cittaṃ tiṭṭhati}
\transnote{We can break \pali{imassidamaṃsisu} in the formula to \pali{ima + sa + idaṃ + aṃ + si + su}.}

\head{130}{130, 225. amussāduṃ.}
\headtrans{For \pali{amu}, [there is] \pali{aduṃ}-substitution.}
\sutdef{sabbasseva amusaddassa savibhattissa aduṃādeso hoti aṃsisu napuṃsake vattamānassa.}
\sutdeftrans{There is a substitution of \pali{aduṃ} for the whole \pali{amu} together with vibhatti because of \pali{aṃ} and \pali{si}[-vibhatti] occurring in a neuter [noun].}
\example[0]{aduṃ pupphaṃ passasi}
\example{aduṃ pupphaṃ virocati}

\head{131}{131. itthipumanapuṃsakasaṅkhyaṃ.}
\headtrans{Numeral for a feminine, masculine, and neuter [noun].}
\sutdef{“itthipumanapuṃsakasaṅkhyaṃ” iccetaṃ adhikāratthaṃ veditabbaṃ.}
\sutdeftrans{This ``Numeral for a feminine, masculine, and neuter noun'' [sutta] should be known as having meaning of governing rule.}
\transnote{This should be understood as the heading of the following suttas.}

\head{132}{132, 228. yosu dvinnaṃ dve ca.}
\headtrans{Because of \pali{yo}[-vibhattis], [there is] \pali{dve}-substitution for \pali{dvi} also.}
\sutdef{dvinnaṃ saṅkhyānaṃ itthipumanapuṃsake vattamānānaṃ savibhattīnaṃ dve hoti yoiccetesu.}
\sutdeftrans{There is [a substitution of] \pali{dve} for the numeral \pali{dvi} (two) occurring in a feminine, masculine, and neuter [noun] together with vibhattis because of \pali{yo}[-vibhattis].}
\example[0]{dve itthiyo \upshape(nom./acc.)}
\example[0]{dve dhammā}
\example{dve rūpāni}
\transnote{In the examples, \pali{dve} = \pali{dvi} + \pali{yo}. This vibhatti has the same form in nominative and accusative (only plural). The outcome is identical in three genders. As marked by \pali{ca} (also), this sutta gives other terms for two, such as \pali{duve, dvaya, ubha, ubhaya,} and \pali{duvi}. These can also decline into cases as follows:}
\example[0]{duve = duve + yo}
\example[0]{dvayena = dvaya + nā}
\example[0]{dvayaṃ = dvaya + aṃ}
\example[0]{ubhinnaṃ = ubha + naṃ}
\example[0]{ubhayesaṃ = ubhaya + naṃ}
\example{duvinnaṃ = duvi + naṃ}

\head{133}{133, 230. ticatunnaṃ tisso catasso tayo cattāro tīṇi cattāri.}
\headtrans{For \pali{ti} and \pali{catu}, [there are] \pali{tisso}, \pali{catasso}, \pali{tayo}, \pali{cattāro}, \pali{tīṇi}, and \pali{cattāri}[-substitution].}
\sutdef{ticatunnaṃ saṅkhyānaṃ itthipumanapuṃsake vattamānānaṃ savibhattīnaṃ tisso catasso tayo cattāro tīṇi cattāri iccete ādesā honti yathāsaṅkhyaṃ yoiccetesu.}
\sutdeftrans{There are substitutions of \pali{tisso}, \pali{catasso}, \pali{tayo}, \pali{cattāro}, \pali{tīṇi}, and \pali{cattāri}, for the numeral \pali{ti} (three) and \pali{catu} (four) occurring in a feminine, masculine, and neuter [noun] together with vi\-bhattis respectively because of \pali{yo}[-vibhattis].}
\example[0]{tisso vedanā \upshape(nom./acc.)}
\example[0]{catasso disā}
\example[0]{tayo janā/jane}
\example[0]{cattāro purisā/purise}
\example[0]{tīṇi āyatanāni}
\example{cattāri ariyasaccāni}
\transnote{The first two examples are feminine, the next two masculine, and the last two neuter. All these use \pali{yo}-vibhattis (nom.\ and acc.).}

\head{134}{134, 251. pañcādīnamakāro.}
\headtrans{For \pali{pañca} and so on, [there is] \pali{a}-substitution.}
\sutdef{pañcādīnaṃ saṅkhyānaṃ itthipumanapuṃsake vattamānānaṃ savibhattissa antassa sarassa akāro hoti yoiccetesu.}
\sutdeftrans{There is a substitution of \pali{a} for the ending vowel of the numeral \pali{pañca} (five) and so on occurring in a feminine, masculine, and neuter [noun] together with vibhattis because of \pali{yo}[-vibhattis].}
\example[0]{pañca/pañca \upshape(nom./acc.)}
\example[0]{cha/cha}
\example[0]{satta/satta}
\example[0]{aṭṭha/aṭṭha}
\example[0]{nava/nava}
\example{dasa/dasa}
\transnote{The examples show that these numbers use the same form in nominative and accusative (only plural).}

\head{135}{135, 118. rājassa rañño rājino se.}
\headtrans{For \pali{rāja}, [there are] \pali{rañño}- and \pali{rājino}-substitution because of \pali{sa}[-vibhatti].}
\sutdef{sabbasseva rājasaddassa savibhattissa raññorājinoiccete ādesā honti se vibhattimhi.}
\sutdeftrans{There are substitutions of \pali{rañño} and \pali{rājino} for the whole \pali{rāja} together with vibhatti because of \pali{sa}-vibhatti.}
\example[0]{rañño = rāja + sa}
\example{rājino = rāja + sa}

\head{136}{136, 119. raññaṃ naṃmhi vā.}
\headtrans{[There is] \pali{raññaṃ}-substitution [for \pali{rāja}] because of \pali{naṃ}[-vibhatti] sometimes.}
\sutdef{sabbasseva rājasaddassa savibhattissa raññaṃādeso hoti vā naṃmhi vibhattimhi.}
\sutdeftrans{There is a substitution of \pali{raññaṃ} for the whole \pali{rāja} together with vibhatti sometimes because of \pali{naṃ}-vibhatti.}
\example{raññaṃ = rāja + naṃ (= rājūnaṃ)}
\transnote{For the rendition of \pali{rājūnaṃ}, see \hyperref[sut:169]{Kacc 169}.}

\head{137}{137, 116. nāmhi raññā vā.}
\headtrans{Because of \pali{nā}[-vibhatti], [there is] \pali{raññā}-substitution [for \pali{rāja}] sometimes.}
\sutdef{sabbasseva rājasaddassa savibhattissa raññāādeso hoti vā nāmhi vibhattimhi.}
\sutdeftrans{There is a substitution of \pali{raññā} for the whole \pali{rāja} together with vibhatti sometimes because of \pali{nā}-vibhatti.}
\example[0]{raññā = rāja + nā (= rājena)}

\head{138}{138, 121. smiṃmhi raññe rājini.}
\headtrans{Because of \pali{smiṃ}[-vibhatti], [there is] \pali{raññe}- and \pali{rājini}-substitution [for \pali{rāja}].}
\sutdef{sabbasseva rājasaddassa savibhattissa raññerājiniiccete ādesā honti smiṃmhi vibhattimhi.}
\sutdeftrans{There are substitutions of \pali{raññe} and \pali{rājini} for the whole \pali{rāja} together with vibhatti because of \pali{smiṃ}-vibhatti.}
\example[0]{raññe = rāja + smiṃ}
\example[0]{rājini = rāja + smiṃ}

\head{139}{139, 245. tumhamhākaṃ tayimayi.}
\headtrans{For \pali{tumha} [and \pali{amha}], [there are] \pali{tayi}- and \pali{mayi}-substitution [respectively because of \pali{smiṃ}-vibhatti].}
\sutdef{sabbesaṃ tumhaamhasaddānaṃ savibhattīnaṃ tayimayiiccete ādeso honti yathāsaṅkhyaṃ smiṃmhi vibhattimhi.}
\sutdeftrans{There are substitutions of \pali{tayi} and \pali{mayi} for the whole \pali{tumha} and \pali{amha} respectively together with vibhattis because of \pali{smiṃ}-vibhatti.}
\example[0]{tayi = tumha + smiṃ}
\example{mayi = amha + smiṃ}

\head{140}{140, 232. tvamahaṃ simhi ca.}
\headtrans{[There are] \pali{tvaṃ}- and \pali{ahaṃ}-substitution [for \pali{tumha} and \pali{amha} respectively] because of \pali{si}[-vibhatti] also.}
\sutdef{sabbesaṃ tumhaamhasaddānaṃ savibhattīnaṃ tvaṃahaṃiccete ādesā honti yathāsaṅkhyaṃ simhi vibhattimhi.}
\sutdeftrans{There are substitutions of \pali{tvaṃ} and \pali{ahaṃ} for the whole \pali{tumha} and \pali{amha} respectively together with vibhattis because of \pali{si}-vibhatti.}
\example[0]{tvaṃ = tumha + si}
\example{ahaṃ = amha + si}
\transnote{As marked by \pali{ca} (also), \pali{tuvaṃ} can also be found.}

\head{141}{141, 241. tavamama se.}
\headtrans{[There are] \pali{tava}- and \pali{mama}-substitution [for \pali{tumha} and \pali{amha} respectively] because of \pali{sa}[-vibhatti].}
\sutdef{sabbesaṃ tumhaamhasaddānaṃ savibhattīnaṃ tavamamaiccete ādesā honti yathāñasaṅkhyaṃ se vibhattimhi.}
\sutdeftrans{There are substitutions of \pali{tava} and \pali{mama} for the whole \pali{tumha} and \pali{amha} respectively together with vibhattis because of \pali{sa}-vibhatti.}
\example[0]{tava = tumha + sa}
\example{mama = amha + sa}

\head{142}{142, 242. tuyhaṃmayhañca.}
\headtrans{Also \pali{tuyhaṃ}- and \pali{mayhaṃ}-substitution [for \pali{tumha} and \pali{amha} respectively because of \pali{sa}-vibhatti].}
\sutdef{sabbesaṃ tumhaamhasaddānaṃ savibhattīnaṃ tuyhaṃmayhaṃiccete ādesā honti yathāsaṅkhyaṃ se vibhattimhi.}
\sutdeftrans{There are substitutions of \pali{tuyhaṃ} and \pali{mayhaṃ} for the whole \pali{tumha} and \pali{amha} respectively together with vibhattis because of \pali{sa}-vibhatti.}
\example[0]{tuyhaṃ = tumha + sa}
\example{mayhaṃ = amha + sa}

\head{143}{143, 235. taṃmamaṃmhi.}
\headtrans{[There are] \pali{taṃ}- and \pali{maṃ}-substitution [for \pali{tumha} and \pali{amha} respectively] because of \pali{aṃ}-vibhatti.}
\sutdef{sabbesaṃ tumhaamhasaddānaṃ savibhattīnaṃ taṃmaṃiccete ādesā honti yathāsaṅkhyaṃ aṃmhi vibhattimhi.}
\sutdeftrans{There are substitutions of \pali{taṃ} and \pali{maṃ} for the whole \pali{tumha} and \pali{amha} respectively together with vibhattis because of \pali{aṃ}-vibhatti.}
\example[0]{taṃ = tumha + aṃ}
\example{maṃ = amha + aṃ}
\transnote{To clarify, \pali{taṃmamaṃmhi} is \pali{taṃ + maṃ + aṃ + smiṃ}.}

\head{144}{144, 234. tavaṃmamañca navā.}
\headtrans{[There are] \pali{tavaṃ}- and \pali{mamaṃ}-substitution [for \pali{tumha} and \pali{amha} respectively because of \pali{aṃ}-vibhatti] sometimes.}
\sutdef{sabbesaṃ tumhaamhasaddānaṃ savibhattīnaṃ tavaṃmamaṃiccete ādesā honti navā yathāsaṅkhyaṃ aṃmhi vibhattimhi.}
\sutdeftrans{There are substitutions of \pali{tavaṃ} and \pali{mamaṃ} for the whole \pali{tumha} and \pali{amha} respectively together with vibhattis sometimes because of \pali{aṃ}-vibhatti.}
\example[0]{tavaṃ = tumha + aṃ}
\example{mamaṃ = amha + aṃ}

\head{145}{145, 238. nāmhi tayāmayā.}
\headtrans{Because of \pali{nā}[-vibhatti], [there are] \pali{tayā}- and \pali{mayā}-substitu\-tion [for \pali{tumha} and \pali{amha} respectively].}
\sutdef{sabbesaṃ tumhaamhasaddānaṃ savibhattīnaṃ tayāmayāiccete ādesā honti yathāsaṅkhyaṃ nāmhi vibhattimhi.}
\sutdeftrans{There are substitutions of \pali{tayā} and \pali{mayā} for the whole \pali{tumha} and \pali{amha} respectively together with vibhattis sometimes because of \pali{nā}-vibhatti.}
\example[0]{tayā = tumha + nā}
\example{mayā = amha + nā}

\head{146}{146, 236. tumhassa tuvaṃtvamaṃmhi.}
\headtrans{For \pali{tumha}, [there are] \pali{tuvaṃ}- and \pali{tvaṃ}-substitution because of \pali{aṃ}-vibhatti.}
\sutdef{sabbassa tumhasaddassa savibhattissa tuvaṃtvaṃiccete ādesā honti aṃmhi vibhattimhi.}
\sutdeftrans{There are substitutions of \pali{tuvaṃ} and \pali{tvaṃ} for the whole \pali{tumha} together with vibhatti because of \pali{aṃ}-vibhatti.}
\example[0]{tuvaṃ = tumha + aṃ}
\example{tvaṃ = tumha + aṃ}

\head{147}{147, 246. padato dutiyācatutthīchaṭṭhīsu vono.}
\headtrans{[When] behind [another] term, because of [plural] accusative, dative, and genitive case, [there are] \pali{vo}- and \pali{no}-substitution [for \pali{tumha} and \pali{amha} respectively].}
\sutdef{sabbesaṃ tumhaamhasaddānaṃ savibhattīnaṃ yadā padasmā paresaṃ vonoādesā honti navā yathāsaṅkhyaṃ dutiyācatutthīchaṭṭhīiccetesu bahuvacanesu.}
\sutdeftrans{There are substitutions of \pali{vo} and \pali{no} for the whole \pali{tumha} and \pali{amha} respectively together with vibhattis when following another term sometimes because of plural accusative, dative, and genitive case.}
\example[0]{pahāya vo bhikkhave gamissāmi\\{\upshape= Leaving you, monks, I will go. (acc.)}}
\example[0]{mā no ajja vikantiṃsu raññosūdā mahānase\\{\upshape= Today, [I wish] the royal cooks will not cut us into pieces in the kitchen. (acc.)}}
\example[0]{dhammaṃ vo bhikkhave desessāmi\\{\upshape= Monks, I will teach you the Dhamma. (dat.)}}
\example[0]{saṃvibhajetha no rajjena\\{\upshape= Distribute us with the kingdom. (dat.)}}
\example[0]{tuṭṭhosmi vo bhikkhave pakatiyā\\{\upshape= I am satisfied with your normal condition, monks. (gen.)}}
\example{satthā no bhagavā anuppatto\\{\upshape= Our master, the Blessed One, has arrived. (gen.)}}
\transnote{The use of aorist verb in the second example is noteworthy. It has future meaning stylistically. In the fifth example, \pali{tuṭṭhosmi} is \pali{tuṭṭho + asmi}.}

\head{148}{148, 247. temekavacanesu ca.}
\headtrans{Also \pali{te}- and \pali{me}-substitution because of singular [dative and genitive case].}
\sutdef{sabbesaṃ tumhaamhasaddānaṃ savibhattīnaṃ yadā padasmā paresaṃ temeādesā honti yathāsaṅkhyaṃ catutthīchaṭṭhīiccetesu ekavacanesu.}
\sutdeftrans{There are substitutions of \pali{te} and \pali{me} for the whole \pali{tumha} and \pali{amha} respectively together with vibhattis when following another term because of singular dative and genitive case.}
\example[0]{dadāmi te gāmavarāni pañca\\{\upshape= I give you five villages as a reward. (dat.)}}
\example[0]{dadāhi me gāmavaraṃ\\{\upshape= Give me a village as a reward. (dat.)}}
\example[0]{idaṃ te raṭṭhaṃ\\{\upshape= This [is] your kingdom. (gen.)}}
\example{ayaṃ me putto\\{\upshape= This [is] my son. (gen.)}}

\head{149}{149, 248. na aṃmhi.}
\headtrans{No [\pali{te}- and \pali{me}-substitution] because of \pali{aṃ}[-vibhatti].}
\sutdef{sabbesaṃ tumhaamhasaddānaṃ savibhattīnaṃ yadā padasmā paresaṃ temeādesā na honti aṃmhi vibhattimhi.}
\sutdeftrans{There are no substitutions of \pali{te} and \pali{me} for the whole \pali{tumha} and \pali{amha} together with vibhattis when following another term because of \pali{aṃ}-vibhatti (acc.).}
\example{passeyya taṃ vassasataṃ arogaṃ, so maṃ bravīti\\{\upshape= He said to me that [he] would like to see you being healthy for a hundren years.}}

\head{150}{150, 249. vā tatiye ca.}
\headtrans{Sometimes [\pali{te}- and \pali{me}-substitution] because of instrumental case also.}
\sutdef{sabbesaṃ tumhaamhasaddānaṃ savibhattīnaṃ yadā padasmā paresaṃ temeādesā honti vā yathāsaṅkhyaṃ tatiyekavacane pare.}
\sutdeftrans{There are substitutions of \pali{te} and \pali{me} for the whole \pali{tumha} and \pali{amha} respectively together with vibhattis when following another term sometimes because of singular instrumental [vibhatti (\pali{nā})] behind.}
\example[0]{kataṃ te pāpaṃ = kataṃ tayā pāpaṃ\\{\upshape= An evil action was done by you.}}
\example{kataṃ me pāpaṃ = kataṃ mayā pāpaṃ\\{\upshape= An evil action was done by me.}}
\transnote{For the rendition of \pali{tayā} and \pali{mayā}, see ``nāmhi tayāmayā'' (\hyperref[sut:145]{Kacc 145}).}

\head{151}{151, 250. bahuvacanesu vono.}
\headtrans{Because of plural [vibhattis], [there are] \pali{vo}- and \pali{no}-substitution.}
\sutdef{sabbesaṃ tumhaamhasaddānaṃ savibhattīnaṃ yadā padasmā paresaṃ vonoādesā honti yathāsaṅkhyaṃ tatiyābahuvacanesu paresu.}
\sutdeftrans{There are substitutions of \pali{vo} and \pali{no} for the whole \pali{tumha} and \pali{amha} respectively together with vibhattis when following another term because of plural instrumental [vibhatti (\pali{hi})] behind.}
\example[0]{kataṃ vo kammaṃ\\{\upshape= An evil action was done by you (all).}}
\example{kataṃ no kammaṃ\\{\upshape= An evil action was done by us.}}
\transnote{I am just wondering why it is not \pali{tatiyābahuvacane pare}, because there is only one vibhatti (\pali{hi}).}

\head{152}{152, 136. pumantassā simhi.}
\headtrans{For the ending [vowel] of \pali{puma}, [there is] \pali{ā}-substitution because of \pali{si}[-vibhatti].}
\sutdef{pumaiccevamantassa savibhattissa āādeso hoti simhi vi\-bh\-attimhi.}
\sutdeftrans{There is a substitution of \pali{ā} for the ending [vowel] of \pali{puma} together with vibhatti because of \pali{si}-vibhatti.}
\example{pumā tiṭṭhati {\upshape= A man stands.}}

\head{153}{153, 138. amālapanekavacane.}
\headtrans{[There is] \pali{aṃ}-substitution [for \pali{puma}] because of singular vocative [vibhatti].}
\sutdef{pumaiccevamantassa savibhattissa aṃādeso hoti ālapanekavacane pare.}
\sutdeftrans{There is a substitution of \pali{aṃ} for the ending [vowel] of \pali{puma} together with vibhatti because of singular vocative [vibhatti (\pali{si})] behind.}
\example{he pumaṃ {\upshape= Hey, man!}}

\head{154}{154. samāse ca vibhāsā.}
\headtrans{In a compound, [there is] also [\pali{aṃ}-substitution for \pali{puma}] optionally.}
\sutdef{pumaiccevamantassa samāse ca aṃādeso hoti vibhāsā samāse kate.}
\sutdeftrans{There is a substitution of \pali{aṃ} for the ending [vowel] of \pali{puma} in a compound also optionally when the compound was made.}
\example{itthī ca pumā ca napuṃsakaṃ ca itthipumannapuṃsakāni}
\transnote{In the example, look closely at \pali{puman} in the compound. It comes from \pali{pumaṃ} followed by \pali{na}. It also seems to me that \pali{vibhāsā} here is used as an indeclinable, otherwise it should be \pali{vibhāsāya}.}

\head{155}{155, 137. yosvāno.}
\headtrans{Because of \pali{yo}[-vibhattis], [there is] \pali{āno}-substitution [for \pali{puma}]}
\sutdef{pumaiccevamantassa savibhattissa ānoādeso hoti yosu vi\-bhattīsu.}
\sutdeftrans{There is a substitution of \pali{āno} for the ending [vowel] of \pali{puma} together with vibhatti because of \pali{yo}-vibhattis.}
\example[0]{pumāno {\upshape= [There are] men. (nom.)}}
\example{he pumāno {\upshape= Hey, men! (voc.)}}

\head{156}{156, 142. āne smiṃmhi vā.}
\headtrans{[There is] \pali{āne}-substitution [for \pali{puma}] because of \pali{smiṃ}[-vibhatti] sometimes.}
\sutdef{pumaiccevamantassa savibhattissa āneādeso hoti vā smiṃmhi vibhattimhi.}
\sutdeftrans{There is a substitution of \pali{āne} for the ending [vowel] of \pali{puma} together with vibhatti sometimes because of \pali{smiṃ}-vibhatti.}
\example{pumāne = puma + smiṃ (= pume)}

\head{157}{157, 140. hivibhattimhi ca.}
\headtrans{Because of \pali{hi}-vibhatti also.}
\sutdef{pumaiccevamantassa hivibhattimhi ca āneādeso hoti.}
\sutdeftrans{There is a substitution of \pali{āne} for the ending [vowel] of \pali{puma} also because of \pali{hi}-vibhatti.}
\example[0]{pumānehi = puma + hi}
\example{pumānebhi = puma + hi}
\transnote{As noted by \pali{ca} (also), there are some other words that decline likewise, to some extent at least. They are \pali{maghava}, \pali{yuva}, \pali{kamma}, and \pali{thāma}. I find these confusing, so it is better to consult another source as well.}

\head{158}{158, 143. susmimā vā.}
\headtrans{Because of \pali{su}[-vibhatti], [there is] \pali{ā}-substitution sometimes.}
\sutdef{pumaiccevamantassa suiccetasmiṃ vibhattimhi āādeso hoti vā.}
\sutdeftrans{There is a substitution of \pali{ā} for the ending [vowel] of \pali{puma} sometimes because of \pali{su}-vibhatti.}
\example{pumāsu = puma + su (= pumesu)}

\head{159}{159, 139. u nāmhi ca.}
\headtrans{[There is] \pali{u}-substitution (\pali{ā} also) because of \pali{nā}[-vibhatti] also.}
\sutdef{pumaiccevamantassa āuādesā honti vā nāmhi vibhattimhi.}
\sutdeftrans{There are substitutions of \pali{ā} and \pali{u} for the ending [vowel] of \pali{puma} sometimes because of \pali{nā}-vibhatti.}
\example[0]{pumānā = puma + nā (= pumena)}
\example{pumunā = puma + nā}

\head{160}{160, 197. a kammantassa ca.}
\headtrans{[There is] \pali{a}-substitution (\pali{u} also) for the ending of \pali{kamma} also.}
\sutdef{kammaiccevamantassa ca uaādesā honti vā nāmhi vibhattimhi.}
\sutdeftrans{There are substitutions of \pali{u} and \pali{a} for the ending [vowel] of \pali{kamma} sometimes because of \pali{nā}-vibhatti.}
\example[0]{kammunā = kamma + nā (= kammena)}
\example{kammanā = kamma + nā}

