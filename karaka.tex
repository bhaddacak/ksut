\chapter{Kāraka}

This part of grammar is difficult for new learners. You have to connect several puzzle pieces together to get a comprehensible picture. If you are very new to Pāli grammar, please make sure you understand the foundations well before going further. Otherwise, you can skip this part and go on to other chapters. When you finish all of them and come back here, your learning will be much easier.

To clarify, we normally do not translate \pali{kāraka} (literally, `doer') and use it as a technical term. There are also several terms that are used untranslated, mainly because their close English equivalents are hard to find. Therefore, translating the terms tends to make things more confusing.

Notable use cases described here will be summarized in Table \ref{tab:karaka} at the end of this chapter.\footnote{This does not include explanations packed in very long suttas, like \hyperref[sut:275]{Kacc 275} and \hyperref[sut:277]{277}.} To ease the startup, however, I list some names of kāraka you will meet soon (but not in this order) as follows:

\begin{compactitem}
\item \pali{Kattu}, mainly marked by nominative case.
\item \pali{Kamma}, mainly marked by accusative case.
\item \pali{Karaṇa}, mainly marked by instrumental case.
\item \pali{Sampadāna}, mainly marked by dative case.
\item \pali{Apādāna}, mainly marked by ablative case.
\item \pali{Okāsa}, mainly marked by locative case.
\end{compactitem}

\boxnote{There are important things to keep in mind:\\
(1) Kāraka and vibhatti are related but not the same thing.\\
(2) Kārakas are about functions, whereas vibhatti operations are about forms.\\
(3) One kāraka function can be achieved by different vibhatti operations, or other equivalent form like annexing with the suffix \pali{-to}, but the main relation can be the case, for example, the apādāna functions and ablative forms.
}

\head{271}{271, 88, 308. yasmādapeti bhayamādatte vā tadapādānaṃ.}
\headtrans{From where [one] leaves, fear arises, or [one] learns, that [is] \emph{apādāna}.}
\sutdef{yasmā vā apeti, yasmā vā bhayaṃ jāyate, yasmā vā ādatte, taṃ kārakaṃ apādānasaññaṃ hoti.}
\sutdeftrans{From where [one] leaves, or from where fear arises, or from where [one] learns, that kāraka is named \emph{apādāna}.}
\example[0]{gāmā apenti munayo\\{\upshape= Sages depart from the village.}}
\example[0]{nagarā niggato rājā\\{\upshape= The king departed from the city.}}
\example[0]{corā bhayaṃ jāyate\\{\upshape= Fear arises from thief.}}
\example{ācariyupajjhāyehi sikkhaṃ gaṇhāti sisso\\{\upshape= A student takes lesson from teacher and preceptor.}}
\transnote{In the formula, \pali{yasmādapeti} is \pali{yasmā + apeti}. The \pali{d} is an extra insertion. And \pali{tadapādānaṃ} is \pali{taṃ + apādānaṃ}. This sutta describes an application of ablative case. So, you are supposed to remember the forms of all cases explained formerly in the book of nāma.}

\head{272}{272, 309. dhātunāmānamupasaggayogādvīsvapi ca.}
\headtrans{In [an application] of roots and nouns, and in a binding of prefixes and so on, also [the kāraka is named \emph{apādāna}].}
\transnote{In the formula, we can break the chunk to \pali{dhātu + nāma + naṃ + upasagga + yoga + ādi + su + api}. Note that, if not specified explicitly, when we talk about nouns (\pali{nāma}) in Pāli, it includes pronouns and adjectives as well.}
\sutdef{dhātunāmānaṃ payoge ca upasaggayogādīsvapi ca taṃ kāra\-kaṃ apādānasaññaṃ hoti.}
\sutdeftrans{[Which term is] in an application of roots and nouns, and in a binding of prefixes and so on, that kāraka is named \emph{apādāna}.}
\bigbullet{(1) With \pali{parā + ji}}
\sutdef{dhātūnaṃ payoge tāva jiiccetassa dhātussa parāpubbassa payoge yo asaho, so apādānasañño hoti.}
\sutdeftrans{[For example], to the extent that an application of roots [is concerned], which [action means] not-defeating because of an application of the root \pali{ji} (defeat) and the prefix \pali{parā} (over, beyond), that [kāraka] is named \emph{apādāna}.}
\example{buddhasmā parājenti aññatitthiyā\\{\upshape= The heretics are defeated from (by) the Buddha.}}
\transnote{This is another application of ablative case when used with certain roots and prefixes. In the example, when \pali{parā + ji} means the opposite of \pali{ji}, it is used with an ablative term, hence the term is called \emph{apādāna}.}
\bigbullet{(2) With \pali{pa + bhū}}
\sutdef{bhūiccetassa dhātussa papubbassa payoge yato acchinnappabhavo, so apādānasañño hoti.}
\sutdeftrans{From which [action means] an unbroken state because of an application of the root \pali{bhū} and the prefix \pali{pa}, that [kāraka] is named \emph{apādāna}.}
\example[0]{himavatā pabhavanti pañca mahānadiyo\\{\upshape= The five great rivers originate from the Himalayas.}}
\example[0]{anavatattamhā pabhavanti mahāsarā\\{\upshape= The great lakes originate from the lake Anavatatta.}}
\example{aciravatiyā pabhavanti kunnadiyo\\{\upshape= Small rivers originate from the river Aciravatī.}}
\bigbullet{(3) In noun clauses}
\sutdef{nāmappayogepi taṃ kārakaṃ apādānasaññaṃ hoti.}
\sutdeftrans{Also in a noun clause, that kārana is named \emph{apādāna}.}
\example[0]{urasmā jāto putto\\{\upshape= The son born from the chest.}}
\example[0]{bhūmito niggato raso\\{\upshape= The taste gone out from the earth.}}
\example{ubhato sujāto putto mātito ca pitito ca\\{\upshape= The son well-born from both sides, from mother's and father's.}}
\transnote{In the context of this description, we can translate \pali{payoga} simply as \emph{clause} or \emph{sentence}, the product of the application/composition.}
\bigbullet{(4) With some prefixes}
\sutdef{upasaggayoge taṃ kārakaṃ apādānasaññaṃ hoti.}
\sutdeftrans{Because of an application of [certain] prefixes, that kāraka is named \emph{apādāna}.}
\example[0]{apa sālāya āyanti vāṇijā\\{\upshape= The merchants comes, leaving from a shed.}}
\example[0]{ā brahmalokā saddo abbhuggacchati\\{\upshape= The sound rises up to the Brahmā's world.}}
\example[0]{upari pabbatā devo vassati\\{\upshape= The rain falls over the mountain.}}
\example[0]{buddhasmā pati sāriputto dhammadesanāya bhikkhū ālapati temāsaṃ\\{\upshape= On behalf of the Buddha, Sāriputta addresses monks for teaching the Dhamma in three months.}}
\example[0]{ghatamassa telasmā pati dadāti\\{\upshape= [One] gives ghee instead of oil.}}
\example[0]{uppalamassa padumasmā pati dadāti\\{\upshape= [One] gives a water lily instead of a pink lotus.}}
\example{kanakamassa hiraññasmā pati dadāti\\{\upshape= [One] gives gold instead of silver.}}
\bigbullet{(5) In the middle}
\sutdef{ādiggahaṇena kārakamajjhepi pañcamīvibhatti hoti.}
\sutdeftrans{Concerning `and so on' [in the sutta], there is an ablative vibhatti in the middle of the kāraka.}
\example[0]{ito pakkhasmā vijjhati migaṃ luddako\\{\upshape= From/within this fortnight, the hunter is going to shoot a deer.}}
\example[0]{kosā vijjhati kuñjaraṃ\\{\upshape= [One] shoots an elephant from/within a Kosa [away].}}
\example{māsasmā bhuñjati bhojanaṃ\\{\upshape= [One] eats food from/since a month.}}
\bigbullet{(6) With some particles}
\sutdef{apiggahaṇena nipātapayogepi pañcamīvibhatti hoti dutiyā ca tatiyā ca.}
\sutdeftrans{Concerning `\pali{api}' [in the sutta], there is an ablative vibhatti, as well as accusative and instrumental, because of an application of particles.}
\example[0]{rahitā mātujā puññaṃ katvā dānaṃ deti\\= rahitā mātujaṃ puññaṃ katvā dānaṃ deti\\= rahitā mātujena puññaṃ katvā dānaṃ deti\\{\upshape= Having made merit without the son, he/she gives alms.}}
\example[0]{rite saddhammā kuto sukhaṃ labhati\\= rite saddhammaṃ kuto sukhaṃ labhati\\= rite saddhammena kuto sukhaṃ labhati\\{\upshape= Without the true doctrine, where does [one] get happiness?}}
\example[0]{te bhikkhū nānā kulā pabbajitā\\= te bhikkhū nānā kulaṃ pabbajitā\\= te bhikkhū nānā kulena pabbajitā\\{\upshape= Those monks went forth from various families.}}
\example{vinā saddhammā natthi añño koci nātho loke\footnote{In the text, this sentence is ``\pali{vinā saddhammā natthañño koci nātho loke vijjati},'' which strangely has two verbs.}\\= vinā saddhammaṃ natthi añño koci nātho loke\\= vinā saddhammena natthi añño koci nātho loke\\{\upshape= Without the true doctrine, there is no other any refuge in the world.}}
\transnote{By this explanation and examples provided, \pali{rahitā}, \pali{rite}, \pali{nānā}, and \pali{vinā} are particles. This also shows that other vibhattis, accusative and instrumental in this case, can have apādāna function.}
\bigbullet{(7) With other meanings}
\sutdef{caggahaṇena aññatthāpi pañcamīvibhatti hoti.}
\sutdeftrans{Concerning `\pali{ca}' [in the sutta], there is an ablative vibhatti also after some other meanings.}
\example{yatohaṃ bhagini ariyāya jātiyā jāto\\{\upshape= Sister, since I was born from the noble birth.}}
\transnote{The last point looks redundant to me. The use seems already covered by the previous explanations.}

\head{273}{273, 310. rakkhaṇatthānamicchitaṃ.}
\headtrans{[In] an expected [application of roots] meaning `to protect/prevent,' [that kāraka is named \emph{apādāna}].}
\sutdef{rakkhaṇatthānaṃ dhātūnaṃ payoge yaṃ icchitaṃ, taṃ kārakaṃ apādānasaññaṃ hoti.}
\sutdeftrans{Which [term was] expected in an application of roots meaning `to protect/prevent,' that kāraka is named \emph{apādāna}.}
\example[0]{kāke rakkhanti taṇḍulā\\{\upshape= [People] prevent crows from rice grain.}}
\example{yavā paṭisedhenti gāvo\\{\upshape= [People] prevent cows from barley.}}

\head{274}{274, 311. yena vādassanaṃ.}
\headtrans{Also by which disappearance, [that kāraka is named \emph{apādāna}].}
\sutdef{yena vā adassanamicchitaṃ, taṃ kārakaṃ apādānasaññaṃ hoti.}
\sutdeftrans{Also by which [term] disappearance [is] desired, that kāraka is named \emph{apādāna}.}
\example[0]{upajjhāyā antaradhāyati sisso\\{\upshape= The student disappears from the preceptor.}}
\example{mātarā ca pitarā ca antaradhāyati putto\\{\upshape= The son disappears from mother and father.}}

\head{275}{275,~312.~dūr\,antik\,addha\,kāla\,nimmāna\,tvālopa\,disāyoga\-vibhatt\,ārappayoga\,suddhap\,pamocana\,hetu\,vivittap\,pamāṇa\-pubbayoga\,bandhana\,guṇavacana\,pañha\,kathana\,thok\,ākattū\-su\ \ ca.}
\headtrans{In [the following applications] also: distantness, nearness, measurement of distance, measurement of time, elision of \pali{tvā}, use of direction, classification, abstinence, purity, release, cause, seclusion, measurement, use of \pali{pubba}, arrestment, expressing virtue, questioning, answering, small amount, and non-\pali{kattu}.}
\transnote{We can break the giant compound in this way: \pali{dūra + antika + addha + kāla + nimmāna + tvālopa + disāyoga + vibhatta + ārappayoga + suddha + pamocana + hetu + vivitta + pamāṇa + pubbayoga + bandhana + guṇavacana + pañha + kathana + thoka + akattu + su}.}
\sutdef{dūratthe, antikatthe, addhanimmāne, kālanimmāne, tvālope, disāyoge, vibhatte, ārappayoge, suddhe, pamocane, hetvatthe, vivittatthe, pamāṇe, pubbayoge, bandhanatthe, guṇavacane, pañhe, kathane, thoke, akattari ca iccetesvatthesu, payogesu ca, taṃ kārakaṃ apādānasaññaṃ hoti.}
\sutdeftrans{Also because of applications of these meanings [as follows]: distantness, nearness, measurement of distance, measurement of time, elision of \pali{tvā}, use of direction, classification, abstinence, purity, release, cause, seclusion, measurement, use of \pali{pubba} (before), arrestment, expressing virtue, questioning, answering, small amount, and non-\pali{kattu} (non-subject), that kāraka is named \emph{apādāna}.}
\bigbullet{(1) Distantness}
\example[0]{kīva dūro ito naḷakāragāmo\\{\upshape= How far from here [is] the village of basket-makers?}}
\example[0]{kīva dūrato naḷakāragāmo\\{\upshape= How far [is] the village of basket-makers?}}
\example{ārakā te moghapurisā imasmā dhammavinayā\\{\upshape= Those worthless men [are] far away from this teaching (doctrine and discipline).}\\= ārakā te moghapurisā imaṃ dhammavinayaṃ\\= ārakā te moghapurisā anena dhammavinayena}
\transnote{Here, \pali{ārakā} is an indeclinable particle.}
\bigbullet{(2) Nearness}
\example[0]{antikaṃ gāmā\\{\upshape= [A place] near (from) the village.}\\= āsannaṃ gāmā\\= samīpaṃ gāmā}
\example[0]{antikaṃ gāmaṃ\\{\upshape= [A place] near the village.}\\= āsannaṃ gāmaṃ\\= samīpaṃ gāmaṃ}
\example[0]{antikaṃ gāmena\\{\upshape= [A place] near (by) the village.}\\= āsannaṃ gāmena\\= samīpaṃ gāmena}
\example[0]{samīpaṃ saddhammā\\{\upshape= [A position] near (from) the true teaching.}}
\example[0]{samīpaṃ saddhammaṃ\\{\upshape= [A position] near the true teaching.}}
\example[0]{samīpaṃ saddhammena\\{\upshape= [A position] near (by) the true teaching.}}
\transnote{Adding \pali{ṭhānaṃ} to these examples can make them more understandable.}
\bigbullet{(3) Measurement of distance}
\example{ito mathurāya catūsu yojanesu saṅkassaṃ nāma nagaraṃ atthi, tattha bahū janā vasanti\\{\upshape= From this Mathurā (city) in four yojanas, there is a city named Saṅkassa. Many people live there.}}
\bigbullet{(4) Measurement of time}
\example[0]{ito bhikkhave ekanavutikappe vipassī nāma bhagavā loke udapādi\\{\upshape= From this time, monks, in 91 eons, the Blessed One named Vipassī arose in the world.}}
\example{ito tiṇṇaṃ māsānaṃ accayena parinibbāyissati\\{\upshape= From now, by the elapse of three months, [the Buddha] will attain the final release.}}
\bigbullet{(5) Elision of \pali{tvā}}
\example[0]{pāsādā saṅkameyya\\{\upshape= [He/she] may go across the mansion.}\\{\upshape This implies but elided} pāsādaṃ abhiruhitvā {\upshape (Having ascended the mension, \ldots)}}
\example{āsanā vuṭṭhaheyya\\{\upshape= [He/she] may get up from the seat.}\\{\upshape This implies but elided} āsane nisīditvā {\upshape (Having sat on the seat, \ldots)}}
\bigbullet{(6) Direction}
\example[0]{avīcito yāva uparibhavaggamantare bahū sattanikāyā vasanti\\{\upshape= In between from the hell Avīci up to the highest heaven, many groups of beings live.}}
\example[0]{yato khemaṃ tato bhayaṃ\\{\upshape= From where [it is] safe, there is danger.}}
\example[0]{puratthimato, dakkhiṇato, pacchimato, uttarato aggī pajjalanti\\{\upshape= From the East, from the South, from the West, from the North, the fires blaze.}}
\example[0]{yato assosuṃ bhagavantaṃ\\{\upshape= From where [they] heard the Blessed One.}}
\example{uddhaṃ pādatalā adho kesamatthakā\\{\upshape= Up from the sole of the foot, down from the top of the hair.}}
\bigbullet{(7) Classification/comparison}
\example[0]{yato paṇītataro vā visiṭṭhataro vā natthi\\{\upshape= [Which thing is] more excellent or more extraordinary, [that thing] does not exist.}}
\example{channavutīnaṃ pāsaṇḍānaṃ dhammānaṃ pavaraṃ yadidaṃ sugatavinayo\\{\upshape= Of the 96 heretical teachings, it is the Buddha's discipline [that is] excellent.}}
\transnote{The last example shows that genitive case can also be used in this sense.}
\bigbullet{(8) Abstinence}
\example[0]{gāmadhammā vasaladhammā asaddhammā ārati/virati/pa\-ṭivirati\\{\upshape= The abstinence from mundane teaching, low teaching, false teaching.}}
\example{pāṇātipātā veramaṇī\\{\upshape= The abstinence from killing living beings.}}
\bigbullet{(9) Purity}
\example[0]{lobhaniyehi dhammehi suddho asaṃsaṭṭho\\{\upshape= [One is] pure, dissociated from the condition causing greed.}}
\example{mātito ca pitito ca suddho asaṃsaṭṭho anupakuṭṭho\footnote{In the text, this is \pali{anupakuddho}.} agarahito\\{\upshape= [One is] pure from mother's and father's [side], unmixed [with other castes], uninsulted, unblamed.}}
\bigbullet{(10) Release}
\example[0]{parimutto dukkhasmāti vadāmi\\{\upshape= [I] say, ``[he/she] was released from suffering.''}}
\example[0]{muttosmi mārabandhanā\\{\upshape= [I] am free from shackles of death.}}
\example{na te muccanti maccunā\\{\upshape= They are not free from death.}}
\bigbullet{(11) Cause}
\example[0]{kasmā hetunā\\{\upshape= From what reason?}}
\example[0]{kena hetunā\\{\upshape= From (by) what reason?}}
\example[0]{kissa hetunā\\{\upshape= From (of) what reason?}}
\transnote{The examples above look ambiguous because \pali{hetunā} can be both instrumental and ablative case. The first example is a normal use of apādāna. The second and third can also be the case if we treat the term as ablative.}
\example[0]{kasmā nu tumhaṃ daharā na mīyare\\{\upshape= Why your youths do not die?}}
\example{kasmā idheva maraṇaṃ bhavissati\\{\upshape= Why the death will be just here?}}
\bigbullet{(12) Seclusion}
\example[0]{vivitto pāpakā dhammā\\{\upshape= [One] was secluded from an evil condition.}}
\example{vivicceva kāmehi vivicca akusalehi dhammehi\\{\upshape= Having been separated from pleasures, having been separated from unwholesome conditions.}}
\bigbullet{(13) Measurement}
\example{dīghaso navavidatthiyo sugatavidatthiyā pamāṇikā kāretabbā, majjhimassa purisassa aḍḍhateḷasahatthā\\{\upshape= By length, nine Buddha's spans [of the cloth] from measurement should be made, by a half of thirteen cubits of a medium-size man.}}
\bigbullet{(14) Use of \pali{pubba} (before)}
\example{pubbeva sambodhā\\{\upshape= Before the enlightenment.}}
\bigbullet{(15) Arrestment}
\example{satasmā bandho naro raññā iṇatthena\\{\upshape= A man was arrested by the king because of debt by a hundred.}\\= satena bandho naro raññā iṇatthena}
\bigbullet{(16) Expressing virtue}
\example[0]{puññāya sugatiṃ yanti\\{\upshape= [People] go to a good rebirth because of merit.}}
\example[0]{cāgāya vipulaṃ dhanaṃ\\{\upshape= The abundance of wealth [exists] because of giving up.}}
\example[0]{paññāya vimutti mano\\{\upshape= The mind emancipates because of wisdom.}}
\example{issariyāya janaṃ rakkhati rājā\\{\upshape= The king protects people because of rulership.}}
\bigbullet{(17) Questioning (with elided \pali{tvā})}
\example{abhidhammā pucchanti\\{\upshape= [They] ask [questions] from the Abhidhamma.}\\{\upshape This implies but elided} abhidhammaṃ sutvā {\upshape (Having heard the Abhidhamma, \ldots)}\\= abhidhammaṃ pucchanti\\= abhidhammena pucchanti}
\bigbullet{(18) Answering (with elided \pali{tvā})}
\example{abhidhammā kathayanti\\{\upshape= [They] answer from the Abhidhamma.}\\{\upshape This implies but elided} abhidhammaṃ sutvā {\upshape (Having heard the Abhidhamma, \ldots)}\\= abhidhammaṃ kathayanti\\= abhidhammena kathayanti}
\bigbullet{(19) Small amount}
\example[0]{thokā muccanti\\= thokena muccanti\\{\upshape= [They] are released because of little [effort].}}
\example[0]{appamattakā muccanti\\= appamattakena muccanti\\{\upshape= [They] are released because of little [effort].}}
\example{kicchā muccanti\\= kicchena muccanti\\{\upshape= [They] are released because of difficult [effort].}}
\transnote{The examples look ambiguous. For example, the first one can also mean ``They are released from little [suffering].''}
\bigbullet{(20) Non-\pali{kattu} (non-subject)}
\example{kammassa katattā upacitattā ussannattā vipulattā cakkhuviññāṇaṃ uppannaṃ hoti\\{\upshape= The eye-consciousness has arisen from the having done of an action, having collected, having accumulated, having increased.}}

\head{276}{276, 84, 302. yassa dātukāmo rocate dhārayate vā taṃ sampadānaṃ.}
\headtrans{To whom one having desire to give [gives], or [to whom a thing] gives delight, or [for whom one] carries [a thing], that [kāraka is named] \emph{sampadāna}.}
\sutdef{yassa vā dātukāmo, yassa vā rocate, yassa vā dhārayate, taṃ kārakaṃ sampadānasaññaṃ hoti.}
\sutdeftrans{To whom one having desire to give [gives], or to whom [a thing] gives delight, or for whom [one] carries [a thing], that kāraka is named \emph{sampadāna}.}
\example[0]{samaṇassa cīvaraṃ dadāti\\{\upshape= [One] gives a robe to a recluse.}}
\example[0]{samaṇassa rocate saccaṃ\\{\upshape= Truth gives delight to a recluse.}}
\example{devadattassa suvaṇṇacchattaṃ dhārayate yaññadatto\\{\upshape= Yaññadatta carries a golden parasol for Devadatta.}}
\transnote{The use of \pali{rocate} is unfamiliar here. We often find \pali{ruccati} used in this sense.}

\head{277}{277,~303.~silāgha\,hanu\,ṭhā\,sapa\,dhāra\,piha\,kudha\,duh\,iss\-osūya\,rādh\,ikkha\,paccāsuṇa\,anupatigiṇa\,pubbakattā\,rocan\-attha\,tadattha\,tumatth\,ālamattha\,maññān\,ādar\,appāṇini gatyatthakammani\ \ āsīsattha\,sammuti\,bhiyya\,sattamy- atthesu ca.}
\headtrans{[In an application of these roots, i.e.,] \pali{silāgha}, \pali{hanu}, \pali{ṭhā}, \pali{sapa}, \pali{dhāra}, \pali{piha}, \pali{kudha}, \pali{duha}, \pali{issa},, \pali{usūya}, \pali{rādha}, \pali{ikkha}; the previous subject of \pali{pati+su}, \pali{ā+su}, \pali{anu+gi}, \pali{pati+gi}; [in these senses, i.e.,] telling, of that [verb], \pali{tuṃ}[-paccaya], \pali{alaṃ}, to think, going, wishing, consent, excess, and locative meaning; [that kāraka is named \emph{sampadāna}].}
\sutdef{silāgha\,hanu\,ṭhā\,sapa\,dhāra\,piha\,kudha\,duha\,issaiccetesaṃ\\ dhātūnaṃ payoge, usūyatthānañca payoge, rādhikkhappayoge, paccāsuṇaanupatigiṇānaṃ pubbakattari, ārocanatthe, tadatthe, tumatthe, alamatthe, maññatippayoge anādare appāṇini, gatyatthānaṃ dhātūnaṃ kammani, āsīsatthe ca sammuti\,bhiyya\-sattamyatthesu ca, taṃ kārakaṃ sampadānasaññaṃ hoti.}
\sutdeftrans{In an application of these roots, i.e, \pali{silāgha}, \pali{hanu}, \pali{ṭhā}, \pali{sapa}, \pali{dhāra}, \pali{piha}, \pali{kudha}, \pali{duha}, and \pali{issa}; in an application of the meaning of \pali{usūya}; in an application of [the root] \pali{rādha} and \pali{ikkha}; in the previous subject of [the root] \pali{su} with \pali{pati} and \pali{ā}, and [the root] \pali{gi} (or \pali{ge}) with \pali{anu} and \pali{pati}; in the sense of telling; in the sense of that [verb]; in the sense of \pali{tuṃ}[-paccaya]; in the sense of \pali{alaṃ} (enough, suitable); in an application of \pali{maññati} (to think) about disrespect and lifeless thing; in an action of roots having senses of going; in the sense of wishing, consent, excess, and locative meaning; that kāraka is named \emph{sampadāna}.}
\bigbullet{(1) With \pali{silāgha}}
\example[0]{buddhassa silāghate\\{\upshape= [One] praises the Buddha.}}
\example[0]{sakaṃ upajjhāyassa silāghate\\{\upshape= [One] praises one's own preceptor.}}
\example{tava silāghate mama silāghate\\{\upshape= [One] praises you, praises me.}}
\bigbullet{(2) With \pali{hanu}}
\example{hanute tuyhameva, hanute mayhameva\\{\upshape= [One] just deceives you, deceives me.}}
\transnote{In Rūpa 303, \pali{apalapatīti attho} is added to explain this example. Possibly, \pali{apalapati} (\pali{apa + lapati}) means ``to talk away.'' However, in Sadd-Dhā 1047, we find \pali{hanu apanayane}. Literally, \pali{apanayati} (\pali{apa + nayati}) means ``to lead away.''}

\boxnote{How come the unused? If you have time to seach for the uses of the two roots above in the Pāli canon, you will realize the peculiarity.\\
\hspace{5mm}\bullet\ The root \pali{silāgha} is never used in the canon, even in the main commentaries.\\
\hspace{5mm}\bullet\ I found fives instances of \pali{silāgha} used in Vinayā\-laṅkāla (a commentary on Vinayasaṅgaha), particularly in Vnl\,226. This includes its definition, \pali{silāgha katthane}.\\
\hspace{5mm}\bullet\ The root \pali{hanu} in this sense is never found in the canon and commentaries. This is why its exact definition is difficult to find. We mostly see \pali{hanu} or \pali{hanukā} used in the meaning of `the jaw.'\\
\hspace{5mm}\bullet\ How do such roots appear in the formula is mysterious. Moreover, the arrangement of this sutta is strange indeed. Perhaps, it underwent several revisions, and the last retouching seems recent.
}

\bigbullet{(3) With \pali{ṭhā}}
\example[0]{upatiṭṭheyya sakyaputtānaṃ vaḍḍhakī\\{\upshape= The carpenter should stand nearby (attend) the Sakyan sons.}}
\example{bhikkhussa bhuñjantassa pānīyena vā vidhūpanena vā upatiṭṭheyya\\{\upshape= [One] should look after a monk who is eating with water and fanning.}}
\bigbullet{(4) With \pali{sapa}}
\example{tuyhaṃ sapate, mayhaṃ sapate\\{\upshape= [One] curses you, curses me.}}
\bigbullet{(5) With \pali{dhāra}}
\example{suvaṇṇaṃ te dhārayate\\{\upshape= The gold carries (debt) for you.}}
\transnote{This example is explained by \pali{iṇaṃ dhārayatīti attho} in Rūpa 303.}
\bigbullet{(6) With \pali{piha}}
\example[0]{buddhassa aññatitthiyā pihayanti\\{\upshape= The heretics envy the Buddha.}}
\example[0]{devā dassanakāmā te\\{\upshape= The gods [are ones] having desire to see you.}}
\example[0]{yato icchāmi bhaddantassa\\{\upshape= Because [I] prefer the venerable one.}}
\example{samiddhānaṃ pihayanti daliddā\\{\upshape= The poor envy the successful ones.}}
\bigbullet{(7) With \pali{kudha}, \pali{duha}, \pali{issa}, and \pali{usūya}}
\example[0]{kodhayati devadattassa\\{\upshape= [One] feels anger at Devadatta.}}
\example[0]{tassa kujjha mahāvīra, mā raṭṭhaṃ vinassa idaṃ\\{\upshape= Having felt anger at that [king], O the great hero (hermit), [please] do not destroy this kingdom.}}
\example[0]{duhayati disānaṃ megho\\{\upshape= The cloud fills/destroys the directions (= obscures the sky).}}
\example[0]{titthiyā samaṇānaṃ issayanti guṇagiddhena lābhagiddhena\\{\upshape= The heretics envy the recluses by craving for virtue, by craving for gift.}}
\example[0]{dujjanā guṇavantānaṃ usūyanti guṇagiddhena\\{\upshape= Bad people envy the virtuous ones by craving for virtue.}}
\example{kā usūyā vijānataṃ\\{\upshape= What [benefit is] the envy at knowledgeable ones?}}
\bigbullet{(8) With \pali{rādha} and \pali{ikkha}}
\sutdef{rādhaikkhaiccetesaṃ dhātūnaṃ payoge yassa akathitassa pucchanaṃ kammavikkhyāpanatthañca, taṃ kārakaṃ sampadānasaññaṃ hoti, dutiyā ca.}
\sutdeftrans{Of which unspoken asking and to make an action manifest in an application of the root \pali{rādha} and \pali{ikkha}, that kāraka is named \emph{sampadāna}, the accusative case [can be used] also.}
\transnote{It not exactly that the roots mentioned are used. This can be a reverse of them like \pali{ārādha}, or other terms in a similar sense, such as \pali{aparajjhati}, \pali{dassana}, and \pali{pekkha} as we see in the examples provided.}
\example[0]{ārādhohaṃ rañño\\= ārādhohaṃ rājānaṃ\\{\upshape= I [am one who is] satisfied with the king.}}
\transnote{The above example is dubious. In Rūpa 303, it is \pali{ārādho me rañño} (My satisfaction in the king [exists]), which makes better sense.}
\example[0]{kyāhaṃ ayyānaṃ aparajjhāmi\\= kyāhaṃ ayye aparajjhāmi\\{\upshape= Do I offend against the venerables?}}
\example[0]{cakkhuṃ janassa dassanāya taṃ viya maññe\\{\upshape= I think that [person/thing is] like an eye for seeing a person.}}
\example{āyasmato upālittherassa upasampadāpekkho upatisso\\= āyasmantaṃ upālittheraṃ upasampadāpekkho upatisso\\{\upshape= Upatissa [is one who is] looking forward to an ordination under the Venerable Upāli.}}
\bigbullet{(9) With \pali{su} and \pali{gi}}
\sutdef{paccāsuṇaanupatigiṇānaṃ pubbakattari suṇotissa paccāyoge yassa kammuno pubbassa yo kattā, so sampadānasañño hoti.}
\sutdeftrans{[According to] ``in the previous subject \ldots,'' which agent [is] of the previous action in an application of \pali{pati} and \pali{ā} upon the root \pali{su}, that [agent applied in a subsequent sentence] is named \emph{sampadāna}.}
\transnote[0]{This point is likely to confuse the reader. Let us unpack it slowly. The \pali{paccāsuṇaanupatigiṇānaṃ pubbakattari} is part of the formula. It is put here as a back-referencing. The first compound can be broken to \pali{pati + ā + su + ṇa + anu + pati + gi + ṇa + naṃ}. Here, \pali{pati} and \pali{ā} and \pali{anu} are prefixes, \pali{su} and \pali{gi} are roots, \pali{ṇa} is the root's group paccaya, and \pali{naṃ} is plural genitive vibhatti.}
\transnote{The most problematic word here is \pali{pubbakattā} (its locative form is \pali{pubbakattari}). As clearly shown in the examples below, this means `the previous actor.' In grammatical terms, we call this `agent' or `subject' of the previous sentence. With \pali{pubba} (previous), it means that there must be an action before the subsequent use of the agent to be called `sampadāna.' Hence, be aware that \pali{kammuno pubbassa} does not mean `of the previous object' or `of the first object' but `of the previous action.'}
\example[0]{bhagavā bhikkhū etadavoca\\{\upshape= The Blessed One said this to monks.}}
\sutdef{bhikkhūti akathitakammaṃ, etanti kathitakammaṃ. yassa kammuno pubbassa yo kattā, so ‘bhagavā’ti “yo karoti sa kattā”ti suttavacanena kattusañño. evaṃ yassa kammuno pubbassa yo kattā, so sampadānasañño hoti.}
\sutdeftrans{[In the above sentence,] `\pali{bhikkhū}' is unspoken object (the secondary object or the detination of the verb `to say'), `\pali{etaṃ}' is spoken object (the object of the verb `to say'). Which [is] the subject of the previous action, it is `\pali{bhagavā},' regarded as `subject' according to the sutta ``What does [the action], it [is] subject.'' Thus, which [is] the subject of the previous action, it is named \emph{sampadāna} [subsequently].}
\transnote{The sentence above is regarded as `the previous action.' So, the agent or subject of the action is \pali{bhagavā}. This is not yet called `sampadāna.'}
\example[0]{te bhikkhū bhagavato paccassosuṃ\\{\upshape= Those monks replied the Buddha.}}
\example{āsuṇanti buddhassa bhikkhū\\{\upshape= The monks listen attentively to the Buddha.}}
\transnote{The two examples above make use of the previous agent (\pali{bhagavā} in this case) with verbs formed by the criterion, i.e., \pali{pati + su} and \pali{ā + su}. The dative forms of the previous agent, i.e., \pali{bhagavato} and \pali{buddhassa} are now called `sampadāna.' The case of \pali{gi} below can also be understood in the same way.}
\sutdef{giṇassa anupatiyoge yassa kammuno pubbassa yo kattā, so sampadānasañño hoti.}
\sutdeftrans{Which subject [is] of the previous action in an application of \pali{anu} and \pali{pati} upon \pali{gi}, that is named \emph{sampadāna}.}
\example{bhikkhu janaṃ dhammaṃ sāveti, tassa bhikkhuno jano anu\-giṇāti, tassa bhikkhuno jano patigiṇāti\\{\upshape= A monk makes a person hear the teaching. The person repeats after that monk. The person responds to that monk.}}
\sutdef[0]{yo vadeti sa ‘kattā’ti, vuttaṃ ‘kamman’ti vuccati;}
\sutdef{yo paṭiggāhako tassa, ‘sampadānaṃ’ vijāniyā.}
\sutdeftrans[0]{Which [person] speaks, that [person is] `subject.'}
\sutdeftrans[0]{[Which is] spoken, [that] is called 'object.'}
\sutdeftrans[0]{Which [person is] the recipient of that [speaker],}
\sutdeftrans{[That person] should be know as `sampadāna.'}
\bigbullet{(10) In the sense of `telling'}
\example[0]{ārocayāmi vo bhikkhave\\{\upshape= [I] tell you (all), monks.}}
\example[0]{āmantayāmi vo bhikkhave\\{\upshape= [I] advise you (all), monks.}}
\example[0]{paṭivedayāmi vo bhikkhave\\{\upshape= [I] inform you (all), monks.}}
\example[0]{ārocayāmi te mahārāja\\{\upshape= [I] tell you, Your Majesty.}}
\example[0]{āmantayāmi te mahārāja\\{\upshape= [I] advise you, Your Majesty.}}
\example{paṭivedayāmi te mahārāja\\{\upshape= [I] inform you, Your Majesty.}}
\bigbullet{(11) In the sense of that [verb]}
\example[0]{ūnassa pāripūriyā taṃ cīvaraṃ nikkhipitabbaṃ\\{\upshape= That robe should be kept for the fulfilment of a shortage.}}
\example{buddhassa atthāya, dhammassa atthāya, saṅghassa atthāya, jīvitaṃ pariccajāmi\\{\upshape= For the benefit of the Buddha, for the benefit of the Dhamma, for the benefit of the Saṅgha, [I] give up the life.}}
\bigbullet{(12) In the sense of \pali{tuṃ}-paccaya}
\example[0]{lokānukampāya atthāya hitāya sukhāya devamanussānaṃ buddho loke uppajjati\\{\upshape= The Buddha arises in the world for the compassion of the world, for the benefit, the wellbeing, [and] the happiness of gods and human beings.}}
\example{bhikkhūnaṃ phāsuvihārāya vinayo paññatto\\{\upshape= The discipline was laid down for the easy-living of monks.}}
\transnote{This definition is misleading. It is not about the infinitive or \pali{tuṃ}-paccaya, but rather its sense of purpose.}
\bigbullet{(13) In the sense of \pali{alaṃ}}
\sutdef{alamiti arahati paṭikkhittesu}
\sutdeftrans{Thus \pali{alaṃ} [means] worthiness and prohibition.}
\example[0]{alaṃ me buddho\\{\upshape= The Buddha [is] worthy for me.}}
\example[0]{alaṃ me rajjaṃ\\{\upshape= The kingdom [is] worthy for me.}}
\example[0]{alaṃ bhikkhu pattassa\\{\upshape= The monk [is] suitable for the bowl.}}
\example[0]{alaṃ mallo mallassa\\{\upshape= [This] wrestler [is] suitable for [that] wrestler.}\\= arahati mallo mallassa}
\example[0]{alaṃ te rūpaṃ karaṇīyaṃ\\{\upshape= [That] made-up body [is] enough for you.}}
\example{alaṃ me hiraññasuvaṇṇena\\{\upshape= [It is] enough for me with silver and gold.}}
\bigbullet{(14) With \pali{maññati} (to think)}
\example[0]{kaṭṭhassa tuvaṃ maññe\\{\upshape= I think you [are] a piece of wood.}}
\example{kaliṅgarassa tuvaṃ maññe\\{\upshape= I think you [are] a log.}}
\transnote{In other senses rather than to express disrespect and to make a simile of a lifeless thing, accusative case is used instead, as shown in the examples below.}
\example[0]{suvaṇṇaṃ viya taṃ maññe\\{\upshape= I think you [are] like gold.}}
\example{gadrabhaṃ tuvaṃ maññe\\{\upshape= I think you [are] a donkey.}}
\bigbullet{(15) With `going'}
\example[0]{gāmassa pādena gato\\= gāmaṃ pādena gato\\{\upshape= [One] went to the village by foot.}}
\example[0]{nagarassa pādena gato\\= nagaraṃ pādena gato\\{\upshape= [One] went to town by foot.}}
\example[0]{appo saggāya gacchati\\= appo saggaṃ gacchati\\{\upshape= A little [being] goes to heaven.}}
\example[0]{saggassa gamanena\\= saggaṃ gamanena\\{\upshape= By going to heaven.}}
\example{mūlāya paṭikasseyya saṅgho\\= mūlaṃ paṭikasseyya saṅgho\\{\upshape= The Saṅgha should draw [the monk] back to the original state.}}
\bigbullet{(16) With wishing}
\example[0]{āyasmato dīghāyuko hotu\\{\upshape= May [you] the Venerable be long-lived.}}
\example[0]{bhaddaṃ bhavato hotu\\{\upshape= May [you] Sir be lucky.}}
\example{anāmayaṃ bhavato hotu\\{\upshape= May [you] Sir be healthy.}}
\bigbullet{(17) With consent}
\example[0]{sādhu sammuti me tassa bhagavato dassanāya\\{\upshape= The consent for the seeing the Blessed One to me [is] good.}}
\example{aññatra saṅghasammutiyā bhikkhussa vippavatthuṃ na vaṭ\-ṭati\\{\upshape= Except by the consent from the Saṅgha, to live without [the triple robes] for a monk is not suitable.}}

\boxnote{A noteworthy example.\\
\hspace{5mm}\bullet\ The idiom ``\pali{vippavatthuṃ na vaṭṭati}'' is never used in the canon. We find the use of \pali{vaṭṭati} mostly in commentaries and later texts.\\
\hspace{5mm}\bullet\ The instance of this example was possibly adapted from Vinayavichaya (Vnv\,602--3), but see also in Khuddasikkhā (Khds\,48).\\
\hspace{5mm}\bullet\ In Bhikkhuvibhaṅga, we find this instead: ``\pali{Idaṃ me, bhante, cīvaraṃ atirekachārattaṃ vippavuṭṭhaṃ, aññatra bhikkhusammutiyā, nissaggiyaṃ}'' (Buv1\,654).\\
\hspace{5mm}\bullet\ So, it is reasonable to date this example around the time of Buddhaghosa, if not later.
}

\newpage
\sutdeftrans{(18) With excessive degree}
\example{bhiyyoso mattāyaṃ\\{\upshape= Beyond measurement.}}
\sutdeftrans{(19) With locative senses}
\example[0]{tuyhañcassa āvi karomi\\{\upshape= [I] also make manifest that [condition] in you.}}
\example{tassa me sakko pāturahosi\\{\upshape= The king of devas appeared in that me.}}
\transnote{These two examples are difficult to understand. If we hold that dative forms can be used in locative senses, my translations are close. But what they really mean is unclear.}

\head{278}{278, 93, 320. yodhāro tamokāsaṃ.}
\headtrans{Which [is] a support, that is \pali{okāsa}.}
\sutdef{yo ādhāro, taṃ okāsasaññaṃ hoti. svādhāro catubbidho byāpiko, opasilesiko, vesayiko, sāmīpiko cāti.}
\sutdeftrans{Which thing [is] a support [of other things], that is named \pali{okāsa}[-kāraka]. That support is fourfold, namely \pali{byāpika} (permeative), \pali{opasilesika} (touching), \pali{vesayika} (regional), and \pali{sāmī\-pika} (proximal).}
\bigbullet{(1) Permeative aspect}
\example[0]{jalesu khīraṃ tiṭṭhati\\{\upshape= Milk stands (= mixes) in the water.}}
\example[0]{tilesu telaṃ\\{\upshape= Oil in sesame seeds.}}
\example{ucchūsu raso\\{\upshape= Taste in sugarcanes.}}
\bigbullet{(2) Touching aspect}
\example[0]{pariyaṅke rājā seti\\{\upshape= The king sleep on the couch.}}
\example{āsane upaviṭṭho saṅgho\\{\upshape= The Saṅgha sat down on the seat.}}
\bigbullet{(3) Regional aspect}
\example[0]{bhūmīsu manussā caranti\\{\upshape= Human beings travel on the earth.}}
\example[0]{antalikkhe vāyū vāyanti\\{\upshape= Winds blow in the sky.}}
\example{ākāse sakuṇā pakkhandanti\\{\upshape= Birds fly in the air.}}
\bigbullet{(4) Proximal aspect}
\example[0]{vane hatthino caranti\\{\upshape= Elephants travel in the forest.}}
\example[0]{gaṅgāyaṃ ghoso tiṭṭhati\\{\upshape= Mr.\,Ghosa stands nearby the Ganges.}\footnote{In Rūpa\,320, this example is ``\pali{gaṅgāyaṃ ghoso vasati}'' instead, with this explanation, ``\pali{gaṅgāya samīpe vajo vasatīti attho}.'' Hence, \pali{ghaso} here means ``cattle-pen.''}}
\example[0]{vaje gāvo duhanti\\{\upshape= Cows are milked in the pen.}}
\example{sāvatthiyaṃ viharati jetavane\\{\upshape= [One] lives in the Jeta grove nearby Sāvatthī.}}
\transnote{The regional aspect (\pali{vesayika = visaya + ṇika}) needs some explanation. Here, \pali{visaya} means ``realm, domain, or region.'' It is normal that human beings move around the earth, so do winds in the sky and birds in the air, because these are their territory. Hence, \pali{jalesu macchā caranti} can also be added. I suspect that \pali{vane hatthino caranti} should be also put into this group because it can be seen likewise (the elephant example is absent in Rūpa\,320).}

\head{279}{279, 82, 292. yena vā kayirate taṃ karaṇaṃ.}
\headtrans{By which [way an action] is done, that is \pali{karaṇa}.}
\sutdef{yena vā kayirate, yena vā passati, yena vā suṇāti, taṃ kārakaṃ karaṇasaññaṃ hoti.}
\sutdeftrans{By which [tool an action] is done, or by which [organ one] sees, or by which [organ one] hear, that kāraka in named \pali{karaṇa}.}
\example[0]{dāttena vīhiṃ lunāti\\{\upshape= By a sickle, [one] cuts a rice paddy.}}
\example[0]{vāsiyā kaṭṭhaṃ tacchati\\{\upshape= By a hatchet, [one] chops a piece of wood.}}
\example[0]{pharasunā rukkhaṃ chindati\\{\upshape= By an axe, [one] cuts down a tree.}}
\example[0]{kudālena pathaviṃ khaṇati\\{\upshape= By a spade, [one] digs the ground.}}
\example[0]{satthena kammaṃ karoti\\{\upshape= By a knife, [one] does work.}}
\example{cakkhunā rūpaṃ passati\\{\upshape= By the eye, [one] sees an image.}}
\transnote{To be complete, \pali{sotena saddaṃ suṇāti} (By the ear, [one] hears sound) can be added.}

\head{280}{280, 75, 285. yaṃ karoti taṃ kammaṃ.}
\headtrans{What [one] makes, that is \pali{kamma}.}
\sutdef{yaṃ vā karoti, yaṃ vā passati, yaṃ vā suṇāti, taṃ kārakaṃ kammasaññaṃ hoti.}
\sutdeftrans{What [one] makes (or do), what [one] sees, or what [one] hears, that kāraka in named \pali{kamma}.}
\example[0]{chattaṃ karoti\\{\upshape= [One] makes an umbrella.}}
\example[0]{rathaṃ karoti\\{\upshape= [One] makes a chariot.}}
\example[0]{rūpaṃ passati\\{\upshape= [One] sees an image.}}
\example[0]{saddaṃ suṇāti\\{\upshape= [One] hears sound.}}
\example[0]{kaṇṭakaṃ maddati\\{\upshape= [One] tramples on a thorn.}}
\example{visaṃ gilati\\{\upshape= [One] swallows a poison.}}

\head{281}{281, 77, 294. yo karoti sa kattā.}
\headtrans{Who does, that is \pali{kattā}.}
\sutdef{yo karoti, so kattusañño hoti.}
\sutdeftrans{Which [person] does, that [person] is named \pali{kattu}[-kāraka].}
\example[0]{ahinā daṭṭho naro\\{\upshape= A man was bitten by a snake.}}
\example[0]{garuḷena hato nāgo\\{\upshape= A serpent was killed by a garuda.}}
\example[0]{buddhena jito māro\\{\upshape= The Lord of Death was defeated by the Buddha.}}
\example[0]{upaguttena māro bandho\\{\upshape= The Lord of Death was captured by Ven.\,Upagutta.}}
\transnote[0]{The examples provided make things a lttle confusing. One simple example can be ``\pali{naro gacchati}'' (a man goes). Then we can call this \pali{naro} in nominative case as kattu-kāraka or simply \emph{kattā}.}
\transnote{In the book's examples, there are two kattās in each case. Technically, they are called \pali{abhihitakattā} and \pali{anabhihitakattā}. The former is the subject of the sentences, marked by nominative case. The later the instrumental actors that in fact do the actions in passive structure (see \hyperref[sut:288]{Kacc 288} below). Hence, \pali{nara}, \pali{nāga}, and \pali{māra} are abhihitakattā, whereas \pali{ahi}, \pali{garuḷa}, \pali{buddha}, and \pali{upagutta} are anabhihitakattā. All of these are kattā is this sutta, but they are of two different functions.}

\head{282}{282, 295. yo kāreti sa hetu.}
\headtrans{Who makes [someone do], that [person] is \pali{hetu}.}
\sutdef{yo kattāraṃ kāreti, so hetusañño hoti, kattā ca.}
\sutdeftrans{Which [person] makes the doer [do things], that [person] is named \pali{hetu}, also [called] \pali{kattā}.}
\example{so puriso taṃ purisaṃ kammaṃ kāreti\\= so puriso tena purisena kammaṃ kāreti\\= so puriso tassa purisassa kammaṃ kāreti\\{\upshape= That man makes that [another] man do an action.}}
\transnote{Hence, \pali{so puriso} in the examples is called \pali{hetukattā} (causative subject). Note that the actual doer of the action can be of accusative, instrumental, or gentinive case.}

\head{283}{283, 91, 316. yassa vā pariggaho taṃ sāmī.}
\headtrans{Who [has] possession, that is \pali{sāmī}.}
\sutdef{yassa vā pariggaho, taṃ sāmīsaññaṃ hoti.}
\sutdeftrans{Which [person has] possession, that [person] is named \pali{sāmī}.}
\example[0]{tassa bhikkhuno paṭivīso\\{\upshape= A share of that monk.}}
\example[0]{tassa bhikkhuno patto\\{\upshape= A bowl of that monk.}}
\example[0]{tassa bhikkhuno cīvaraṃ\\{\upshape= A robe of that monk.}}
\example{attano mukhaṃ\\{\upshape= The face of one's own.}}

\head{284}{284, 65, 283. liṅgatthe paṭhamā.}
\headtrans{In [a mere designation of] a noun's meaning, [there is a] nominative [vibhatti].}
\sutdef{liṅgatthābhidhānamatte paṭhamāvibhatti hoti.}
\sutdeftrans{In a mere designation of a noun's meaning, there is a nominative vibhatti.}
\example[0]{puriso {\upshape ([There is] a man.)}}
\example[0]{purisā {\upshape ([There are] men.)}}
\example[0]{eko {\upshape ([There is] one [thing].)}}
\example[0]{dve {\upshape ([There are] two [things].)}}
\example[0]{ca, vā {\upshape (and/or)}}
\example{he, ahe, re, are {\upshape (O/Hey!)}}
\transnote[0]{In the definititon, \pali{liṅgatthābhidhānamatte} can be broken to \pali{liṅga + attha + abhidhāna + matta + smiṃ}. Here, the use of `mere' (\pali{matta}) stresses the fact that just the meaning of the noun is expressed, not other functions such as kattā, kamma, or karaṇa.}
\transnote[0]{If you look at the examples closely, you will realize that \pali{liṅga} in this sense is not genders. It is a nominal base before the term gets any meaning. Once a nominative vibhatti is applied, the term acquires certain meaning. For example, \pali{puriso} designates a singular masculine person, \pali{purisā} plural masculine persons.}
\transnote{As explained in \hyperref[sut:221]{Kacc 221}, the last two examples show that particles also have vibhattis but they are elided.}

\boxnote{On \pali{liṅga}.\\
\hspace{5mm}\bullet\ In Rūpa 283, this technical term is explained as \pali{ettha ca līnaṃ aṅganti liṅgaṃ, apākaṭo avayavo} (Here, \pali{liṅga} [is] a concealed part, an unseen limb).\\
\hspace{5mm}\bullet\ The meaning and function of a \pali{liṅga} is unseen before a vibhatti is applied, because we cannot say yet it will be masuline, feminine, or neuter; singular or plural; kattā, kamma, karaṇa, etc.\\
\hspace{5mm}\bullet\ The simplest kind of sentences in Pāli is called \pali{liṅgattha (liṅgassa attho liṅgattho)} by a utilization of nominative vibhatti. It just asserts that a certain entity exists, but its other grammatical function is not specified.
}

\head{285}{285, 70. ālapane ca.}
\headtrans{In addressing also, [there is a nominative vibhatti].}
\sutdef{ālapanatthādhike liṅgatthābhidhānamatte ca paṭhamāvibhatti hoti.}
\sutdeftrans{In a mere designation of a noun's meaning in addresing, also there is a nominative vibhatti.}
\example[0]{bho purisa, bho rāja, he sakhe {\upshape (singular)}}
\example{bhavanto purisā, bhavanto rājāno, he sakhino {\upshape (plural)}}
\transnote{In the definition, \pali{ālapanatthādhike} is \pali{ālapana + attha + adhika + smiṃ}. We have \pali{adhika} (exceeding) here to emphasize that the meaning is just used for addressing, beyond other functions such as kattā or kamma.}

\head{286}{286, 83, 291. karaṇe tatiyā.}
\headtrans{In karaṇa[-karaka], [there is an] instrumental [vibhatti].}
\sutdef{karaṇakārake tatiyāvibhatti hoti.}
\sutdeftrans{In karaṇa-karaka, there is an instrumental vibhatti.}
\example[0]{agginā kuṭiṃ jhāpeti\\{\upshape= [One] causes a hut to burn with fire.}}
\example[0]{manasā ce paduṭṭhena, manasā ce pasannena\\{\upshape= If [one does] with wicked mind \ldots; if [one does] with faithful mind \ldots}}
\example{kāyena kammaṃ karoti\\{\upshape= [One] does an action with the body.}}
\transnote{For the definition of karaṇa, see \hyperref[sut:279]{Kacc 279}.}

\head{287}{287, 299. sahādiyoge ca.}
\headtrans{In an application of \pali{saha} and so on, also [there is an instrumental vibhatti].}
\sutdef{sahādiyogatthe ca tatiyāvibhatti hoti.}
\sutdeftrans{In an application of \pali{saha} and so on, also there is an instrumental vibhatti.}
\example[0]{sahāpi gaggena saṅgho uposathaṃ kareyya, vināpi gaggena\\{\upshape= The Saṅgha should do the Uposatha (Vinaya recitation), even with or without Ven.\,Gagga.}}
\example[0]{mahatā bhikkhusaṅghena saddhiṃ\\{\upshape= Together with a large group of monks.}}
\example{sahassena samaṃ mitā\\{\upshape= [Things] were measured equally to one thousand.}}

\head{288}{288, 78, 293. kattari ca.}
\headtrans{In [the function of] kattā, also [there is an instrumental vibhatti].}
\sutdef{kattari ca tatiyāvibhatti hoti.}
\sutdeftrans{In [the function of] kattā, also there is an instrumental vibhatti.}
\example[0]{raññā hato poso\\{\upshape= A man was killed by the king.}}
\example[0]{yakkhena dinno varo\\{\upshape= A wish was given by a demon.}}
\example{ahinā daṭṭho naro\\{\upshape= A man was bitten by a snake.}}
\transnote{To be precise, these instrumental actors are called \pali{anabhihitakattā}. The definition of kattā can be found in \hyperref[sut:281]{Kacc 281}.}

\head{289}{289, 297. hetvatthe ca.}
\headtrans{In the meaning of cause, also [there is an instrumental vibhatti].}
\sutdef{hetvatthe ca tatiyāvibhatti hoti.}
\sutdeftrans{In the meaning of cause, also there is an instrumental vibhatti.}
\example[0]{annena vasati\\{\upshape= [One] lives on account of food.}}
\example[0]{dhammena vasati\\{\upshape= [One] lives on account of dhamma.}}
\example[0]{vijjāya vasati\\{\upshape= [One] lives on account of knowledge.}}
\example{sakkārena vasati\\{\upshape= [One] lives on account of honor.}}

\head{290}{290, 298. sattamyatthe ca.}
\headtrans{In the locative meaning, also [there is an instrumental vibhatti].}
\sutdef{sattamyatthe ca tatiyāvibhatti hoti.}
\sutdeftrans{In the locative meaning, also there is an instrumental vibhatti.}
\example[0]{tena kālena {\upshape (in that time)}}
\example{tena samayena {\upshape (in that occasion)}}

\head{291}{291, 299. yenaṅgavikāro.}
\headtrans{A deformation [is maked] by which part, [in that part there is an instrumental vibhatti].}
\sutdef{yena byādhimatā aṅgena aṅgino vikāro lakkhīyate. tattha tatiyāvibhatti hoti.}
\sutdeftrans{A deformation of the body is marked by which part that the sickness is known, in that [part] there is an instrumental vibhatti.}
\example[0]{akkhinā kāṇo\\{\upshape= [One is] blind by the eye.}}
\example[0]{hatthena kuṇī\\{\upshape= [One is] deformed by the hand.}}
\example[0]{pādena khañjo\\{\upshape= [One is] lame by the foot.}}
\example{piṭṭhiyā khujjo\\{\upshape= [One is] hunched by the back.}}

\head{292}{292, 300. visesane ca.}
\headtrans{In [the meaning of] attributive modifiers also.}
\sutdef{visesanatthe ca tatiyāvibhatti hoti.}
\sutdeftrans{In the meaning of attributive modifiers, also there is an instrumental vibhatti.}
\example[0]{gottena gotamo nātho\\{\upshape= The Lord (Buddha) is Gotama by clan.}}
\example[0]{suvaṇṇena abhirūpo\\{\upshape= [One is] handsome by golden (complexion).}}
\example{tapasā uttamo\\{\upshape= [One is] the greatest by austerity.}}

\head{293}{293, 85, 301. sampadāne catutthī.}
\headtrans{In sampadāna[-kāraka], [there is a] dative [vibhatti].}
\sutdef{sampadānakārake catutthīvibhatti hoti.}
\sutdeftrans{In [the sense of] sampadāna-kāraka, there is a dative vibhatti.}
\example[0]{buddhassa vā dhammassa vā saṅghassa vā dānaṃ deti\\{\upshape= [One] gives an offering to the Buddha, the Dhamma, and the Saṅgha.}}
\example{dātā hoti samaṇassa vā brāhmaṇassa vā\\{\upshape= [One] is a donor to an ascetic and a brahmin.}}
\transnote{For the detailed definition of sampadāna, see \hyperref[sut:276]{Kacc 276} and \hyperref[sut:277]{277}.}

\head{294}{294, 305. namoyogādīsvapi ca.}
\headtrans{In an application of \pali{namo} and so on, also [there is a dative vibhatti].}
\sutdef{namoyogādīsvapi ca catutthīvibhatti hoti.}
\sutdeftrans{In an application of \pali{namo} and so on, also there is a dative vibhatti.}
\example[0]{namo te buddha vīratthu\\{\upshape= O the great Buddha, may the homage be to you.}}
\example[0]{sotthi pajānaṃ\\{\upshape= [May] blessing be to people.}}
\example[0]{namo karohi nāgassa\\{\upshape= Pay homage to the serpent/great man.}}
\example{svāgataṃ te mahārāja\\{\upshape= Welcome to you, Your Majesty.}}
\transnote{In the first example, \pali{vīratthu} is \pali{vīra} (vocative) and \pali{atthu} (imperative verb `to be').}

\head{295}{295, 89, 307. apādāne pañcamī.}
\headtrans{In apādāna[-kāraka], [there is an] ablative [vibhatti].}
\sutdef{apādānakārake pañcamīvibhatti hoti.}
\sutdeftrans{In [the sense of] apādāna-kāraka, there is an ablative vibhatti.}
\example[0]{pāpā cittaṃ nivāraye\\{\upshape= [One] should restrain the mind from wickedness.}}
\example[0]{abbhā muttova candimā\\{\upshape= The moon was released from the cloud.}}
\example{bhayā muccati so naro\\{\upshape= That man is free from danger.}}
\transnote{For the detailed definition of apādāna, see \hyperref[sut:271]{Kacc 271}--\hyperref[sut:275]{275}.}

\head{296}{296, 314. kāraṇatthe ca.}
\headtrans{In the meaning of cause, also [there is an ablative vibhatti].}
\sutdef{kāraṇatthe ca pañcamīvibhatti hoti.}
\sutdeftrans{In the meaning of cause, also there is an ablative vibhatti.}
\example{ananubodhā appaṭivedhā catunnaṃ ariyasaccānaṃ yathābhūtaṃ adassanā\\{\upshape= Because of not-understanding, not-penetrating the four noble truths, not-seeing [a thing] as it is.}}

\head{297}{297, 76, 284. kammatthe dutiyā.}
\headtrans{In the meaning of kamma, [there is an] accusative [vibhatti].}
\sutdef{kammatthe dutiyāvibhatti hoti.}
\sutdeftrans{In the meaning of kamma, there is an accusative vibhatti.}
\example[0]{gāvaṃ hanati\\{\upshape= [One] kills a cow.}}
\example[0]{vīhayo lunāti\\{\upshape= [One] reaps rice paddies.}}
\example[0]{satthaṃ karoti\\{\upshape= [One] makes a sword.}}
\example[0]{ghaṭaṃ karoti\\{\upshape= [One] makes ghee.}}
\example[0]{rathaṃ karoti\\{\upshape= [One] makes a chariot.}}
\example[0]{dhammaṃ suṇāti\\{\upshape= [One] listens to the Dhamma.}}
\example[0]{buddhaṃ pūjeti\\{\upshape= [One] reveres the Buddha.}}
\example[0]{vācaṃ bhāsatī\\{\upshape= [One] talks a speech.}}
\example[0]{taṇḍulaṃ pacati\\{\upshape= [One] cooks rice.}}
\example{coraṃ ghāteti\\{\upshape= [One] kills a thief.}}
\transnote{For the definition of kamma, see \hyperref[sut:280]{Kacc 280}.}

\head{298}{298, 287. kāladdhānamaccantasaṃyoge.}
\headtrans{In an application of continuity of time and distance, [there is an accusative vibhatti].}
\sutdef{kāladdhānaṃ accantasaṃyoge dutiyāvibhatti hoti.}
\sutdeftrans{In an application of continuity of time and distance, there is an accusative vibhatti.}
\example[0]{māsaṃ maṃsodanaṃ bhuñjati\\{\upshape= [One] eats meat and rice throughout a month.}}
\example[0]{saradaṃ ramaṇīyā nadī\\{\upshape= The river is delightful throughout the autumn.}}
\example[0]{māsaṃ sajjhāyati\\{\upshape= [One] studies throughout a month.}}
\example[0]{yojanaṃ vanarāji\\{\upshape= The forest area [is] one yojana [long].}}
\example[0]{yojanaṃ dīgho pabbato\\{\upshape= The mountain [is] one yojana long.}}
\example{kosaṃ sajjhāyati\\{\upshape= [One] chants [a sutta] throughout a kosa long.}}
\transnote[0]{When a term related to time or distance is marked by an accusative vibhatti, it works like an adverb denoting continuity. The last example may mean the sound is heard throughout the length, or the chanter does it continuously while walking in that length.}
\transnote{When the actions do not happen continuously, locative case is used instead. Here are examples from Rūpa 287, \pali{māse māse bhuñjati} ([One] eats every month), \pali{yojane yojane vihāraṃ patiṭ\-ṭhāpesi} ([One] had a residence built every a yojana long).}

\head{299}{299, 288. kammappavacanīyayutte.}
\headtrans{In an application of \pali{kammappavacanīya}, [there is an accusative vibhatti].}
\sutdef{kammappavacanīyayutte dutiyāvibhatti hoti.}
\sutdeftrans{In an application of \pali{kammappavacanīya} (certain particles and prefixes), there is an accusative vibhatti.}
\transnote{The technical term `\pali{kammappavacanīya}' is abstruse. It makes this sutta in Kaccāyana lame because we barely understand it. Here is the definition found in Rūpa 288.}
\sutdef{kammappavacanīyehi nipātopasaggehi yutte yoge sati liṅgamhā dutiyāvibhatti hoti.}
\sutdeftrans{When there is an application/composition by particles and prefixes called \pali{kammappavacanīya}, there is an accusative vibhatti behind the noun.}
\transnote{A little better, now we know that the term is used to name some particles and prefixes. There are several of them, e.g., \pali{pati}, \pali{pari}, \pali{abhi}, and \pali{anu}, which be able to use independently. Here is the definition of the term from the same place.}
\sutdef{kammaṃ pavacanīyaṃ yesaṃ te kammappavacanīyā}
\sutdeftrans{An action should be specified by which [particles or prefixes], those [are called] \pali{kammappavacanīya}.}
\transnote{To stress, \pali{kamma} here means action, not object (kamma-kāraka).}
\example{taṃ kho pana bhavantaṃ gotamaṃ evaṃ kalyāṇo kittisaddo abbhuggato\\{\upshape= The good reputation of that Honorable Gotama has spread over thus.}}
\transnote{In this example, \pali{abbhuggato} is \pali{abhi} and \pali{uggato}. This \pali{abhi} is called \pali{kammappavacanīya}. Hence, we see accusative forms, i.e., \pali{taṃ}, \pali{bhavantaṃ}, and \pali{gotamaṃ}. I suspect that genitive forms of these can also do the job in a more comprehensible way.}
\example{pabbajitamanupabbajiṃsu\\{\upshape= [People] went forth after an ordained one.}}
\transnote{This example is \pali{pabbajitaṃ + anu + pabbajiṃsu}. Here, \pali{anu} is called \pali{kammappavacanīya}. So, we see \pali{pabbajitaṃ} in accusative case. If we treat \pali{anupabbajati} as a whole word that needs an object, the sentence makes clear sense without any extra (but confusing) explanation.}

\boxnote{Kammappavacanīya (karmapravacanīya) is in fact Pāṇini's idea.\\
\hspace{5mm}\bullet\ We can see its definitions in Pāṇ 1.4.83--98.\\
\hspace{5mm}\bullet\ Here is another elucidation: karmapravacanīya literally means ``that which is to be announced by the action.'' It practically amounts to a preposition.\footnote{\citealp[p.~134]{roodbergen:gramdict}}\\
\hspace{5mm}\bullet\ As a matter of fact, this kind of use is barely found in the Pāli canon. The sutta seems to serve only its own presence by borrowing some terrific ideas from Sanskrit. Its relevance to the real Pāli text is light.
}

\head{300}{300,~286.~gati\,buddhi\,bhuja\,paṭha\,hara\,kara\,sayādīnaṃ\ \ kārite\ \ vā.}
\headtrans{For \pali{gati}, \pali{buddhi}, \pali{bhuja}, \pali{paṭha}, \pali{hara}, \pali{kara}, \pali{saya}, and so on, [there is an accusative vibhatti] sometimes because of causative [structure].}
\sutdef{gatibuddhibhujapaṭhaharakarasayādīnaṃ payoge kārite dutiyāvibhatti hoti vā.}
\sutdeftrans{In an application of [roots such as] \pali{gati}, \pali{buddhi}, \pali{bhuja}, \pali{paṭha}, \pali{hara}, \pali{kara}, \pali{saya}, and so on, there is an accusative vi\-bhatti sometimes because of causative [structure].}
\example{puriso purisaṃ gāmaṃ gāmayati\\{\upshape= A man makes [another] man go to the village.}\\= puriso purisena gāmaṃ gāmayati\\= puriso purisassa gāmaṃ gāmayati}

\head{301}{301, 92, 315. sāmismiṃ chaṭṭhī.}
\headtrans{In [the sense of] \pali{sāmī}, [there is a] genitive [vibhatti].}
\sutdef{sāmismiṃ chaṭṭhīvibhatti hoti.}
\sutdeftrans{In [the sense of] \pali{sāmī} (owner), there is a genitive vibhatti.}
\example[0]{tassa bhikkhuno paṭivīso\\{\upshape= [There is] a share of that monk.}}
\example[0]{tassa bhikkhuno patto\\{\upshape= [There is] a bowl of that monk.}}
\example{tassa bhikkhuno cīvaraṃ\\{\upshape= [There is] a robe of that monk.}}
\transnote{For the definition of \pali{sāmī}, see \hyperref[sut:283]{Kacc 283}. Traditionally, sāmī is not kāraka because it relates noun to noun, not noun to verb.}

\head{302}{302, 94, 319. okāse sattamī.}
\headtrans{In okāsa[-kāraka], [there is a] locative [vibhatti].}
\sutdef{okāsakārake sattamīvibhatti hoti}.
\sutdeftrans{In okāsa-kāraka, there is a locative vibhatti.}
\example[0]{gambhīre odakantike (nidhiṃ nidheti)\\{\upshape= [One] buries treasure in a deep [place] near water.}}
\example[0]{pāpasmiṃ ramati mano\\{\upshape= The mind enjoys in wrongdoing.}}
\example{bhagavati brahmacariyaṃ vussati kulaputto\\{\upshape= A son of family lives the religious life in [the Dhamma of] the Blessed One.}}
\transnote{For the definition of \pali{okāsa}, see \hyperref[sut:278]{Kacc 278}.}

\head{303}{303,~321.~sām\,issar\,ādhipati\,dāyāda\,sakkhī\,patibhū\,pasuta\\kusalehi\ \ ca.}
\headtrans{In [an application of] \pali{sāmī}, \pali{issara}, \pali{adhipati}, \pali{dāyāda}, \pali{sakkhī}, \pali{patibhū}, \pali{pasuta}, and \pali{kusala}, also [there is a genitive vibhatti, locative as well].}
\sutdef{sāmī\,issara\,adhipati\,dāyāda\,sakkhī\,patibhū\,pasuta\,kusalaic\-cetehi payoge chaṭṭhīvibhatti hoti, sattamī ca.}
\sutdeftrans{In an application of \pali{sāmī}, \pali{issara}, \pali{adhipati}, \pali{dāyāda}, \pali{sakkhī}, \pali{patibhū}, \pali{pasuta}, and \pali{kusala}, there is a genitive vibhatti, locative as well.}
\example[0]{goṇānaṃ sāmī, goṇesu sāmī\\{\upshape= the owner of cattle}}
\example[0]{goṇānaṃ issaro, goṇesu issaro\\{\upshape= the master of cattle}}
\example[0]{goṇānaṃ adhipati, goṇesu adhipati\\{\upshape= the ruler of cattle}}
\example[0]{goṇānaṃ dāyādo, goṇesu dāyādo\\{\upshape= the heir of cattle}}
\example[0]{goṇānaṃ sakkhī, goṇesu sakkhī\\{\upshape= the witness of cattle}}
\example[0]{goṇānaṃ patibhū, goṇesu patibhū\footnote{Not found in Pāli dictionaries, this is \pali{pratibhū} in Sanskrit. The term means `surety' in Macdonell's dictionary, also `security' and `bail' in Monior-Williams.}\\{\upshape= the guarantor of cattle}}
\example[0]{goṇānaṃ pasūto, goṇesu pasūto\footnote{In the text, the term is \pali{pasuto}.}\\{\upshape= [a calf] born in the cattle}}
\example{goṇānaṃ kusalo, goṇesu kusalo\\{\upshape= an expert in cattle}}

\head{304}{304, 322. niddhāraṇe ca.}
\headtrans{In \pali{niddhāraṇa}, also [there is a genitive vibhatti, locative as well].}
\sutdef{niddhāraṇatthe ca chaṭṭhīvibhatti hoti, sattamī ca.}
\sutdeftrans{In the meaning of \pali{niddhāraṇa}, also there is a genitive vibhatti, locative as well.}
\example[0]{kaṇhā gāvīnaṃ sampannakhīratamā\\{\upshape= Of the cows, the black [one has] the fullest milk.}}
\example[0]{kaṇhā gāvīsu sampannakhīratamā\\{\upshape= Among the cows, \ldots}}
\example[0]{sāmā nārīnaṃ dassanīyatamā\\{\upshape= Of the women, the brown-skinned one [is] the most beautiful.}}
\example[0]{sāmā nārīsu dassanīyatamā\\{\upshape= Among the women, \ldots}}
\example[0]{manussānaṃ khattiyo sūratamo\\{\upshape= Of human beings, the warrior [caste is] the most courageous.}}
\example[0]{manussesu khattiyo sūratamo\\{\upshape= Of human beings, \ldots}}
\example[0]{pathikānaṃ dhāvanto sīghatamo\\{\upshape= Of the pedestrians, the running [one] is the fastest.}}
\example{pathikesu dhāvanto sīghatamo\\{\upshape= Among the pedestrians, \ldots}}
\transnote{This sutta introduces another technical term you have to know. In Rūpa 322, it is explained as \pali{nīharitvā dhāraṇaṃ niddhāraṇaṃ} (The holding [after] having singled out [is] \pali{niddhāraṇa}). It is a selection from a group, then the selected item is made outstanding.}

\head{305}{305, 323. anādare ca.}
\headtrans{In \pali{anādara}, also [there is a genitive vibhatti, locative as well].}
\sutdef{anādare chaṭṭhīvibhatti hoti, sattamī ca.}
\sutdeftrans{In [the structure of] \pali{anādara}, there is a genitive vibhatti, locative as well.}
\example{rudato dārakassa pabbaji\\= rudantasmiṃ dārake pabbaji\\{\upshape= While the boy [was] crying, [the father] went forth.}}
\transnote{Literally, \pali{anādara} means `disrespect.' This term in technical sense means a kind of absolute constructions, a way to make complex sentences.}

\head{306}{306, 289. kvaci dutiyā chaṭṭhīnamatthe.}
\headtrans{In some places, an accusative [vibhatti is used in] genitive meaning.}
\sutdef{chaṭṭhīnamatthe kvaci dutiyāvibhatti hoti.}
\sutdeftrans{In genitive meaning, in some places there is an accusative vibhatti.}
\example{apissu maṃ aggivessana tisso upamā paṭibhaṃsu\\{\upshape= And then, Aggivessana, my three similes came to mind.}}
\transnote{In the example, \pali{maṃ} is in accusative but has genitive meaning (= \pali{mama}).}

\head{307}{307, 290. tatiyāsattamīnañca.}
\headtrans{In the meaning of instrumental and locative case, also [in some places there is an accusative vibhatti].}
\sutdef{tatiyāsattamīnaṃ atthe ca kvaci dutiyāvibhatti hoti.}
\sutdeftrans{In the meaning of instrumental and locative case, also in some places there is an accusative vibhatti.}
\example[0]{sace maṃ samaṇo gotamo ālapissati\\{\upshape= If the ascetic Gotama will talk with me, \ldots}}
\example[0]{tvañca maṃ nābhibhāsasi\\{\upshape= You also did not talk with me.}}
\example[0]{pubbaṇhasamayaṃ nivāsetvā\\{\upshape= Having dressed in the morning, \ldots}}
\example{ekaṃ samayaṃ bhagavā\\{\upshape= In one occasion, the Blessed One, \ldots}}
\transnote{In the first two examples, \pali{maṃ} is used in instrumental meaning. In the last two, \pali{samayaṃ} is used in locative meaning.}

\head{308}{308, 317. chaṭṭhī ca.}
\headtrans{[In the meaning of instrumental and locative case], also [in some places there is an] genitive [vibhatti].}
\sutdef{tatiyāsattamīnaṃ atthe ca kvaci chaṭṭhīvibhatti hoti.}
\sutdeftrans{In the meaning of instrumental and locative case, also in some places there is an genitive vibhatti.}
\example[0]{kato me kalyāṇo, kataṃ me pāpaṃ\\{\upshape= A good action was done by me, a bad action was done by me.}}
\example[0]{kusalā naccagītassa sikkhitā cāturitthiyo\\{\upshape= The four girls was trained skillfully in dancing and singing.}}
\example{kusalo tvaṃ rathassa aṅgapaccaṅgānaṃ\\{\upshape= You are skillful in various components of chariot.}}
\transnote{In the first example, \pali{me} has instrumental meaning (this is ambiguous because it can be either case). In the second, \pali{naccagītassa} is genitive form used in locative sense. In the last one, \pali{aṅgapaccaṅgānaṃ} is also in locative sense.}

\head{309}{309, 318. dutiyāpañcamīnañca.}
\headtrans{In [the meaning of] accusative and ablative case, also [in some places there is an genitive vibhatti].}
\sutdef{dutiyāpañcamīnañca atthe kvaci chaṭṭhīvibhatti hoti.}
\sutdeftrans{In the meaning of accusative and ablative case, in some places there is an genitive vibhatti.}
\example[0]{tassa bhavanti vattāro\footnote{This sentence is unusual. If \pali{bhavanti} is changed to \pali{bhāsanti}, the sentence will make better sense.}\\{\upshape= There are speakers [talking to] that [person].}}
\example[0]{sahasā kammassa kattāro\\{\upshape= [There are] doers [doing] an action suddenly.}}
\example[0]{assavanatā dhammassa parihāyanti\\{\upshape= [People] fall away because of not hearing the Dhamma.}}
\example[0]{kinnu kho ahaṃ tassa sukhassa bhāyāmi\\{\upshape= Do I fear that happiness?}}
\example[0]{sabbe tasanti daṇḍassa, sabbe bhāyanti maccuno\\{\upshape= All are afraid of punishment, all fear death.}}
\example[0]{bhīto catunnaṃ āsīvisānaṃ ghoravisānaṃ\\{\upshape= [One] feared four venomous, dreadful snake.}}
\example{bhāyāmi ghoravisassa nāgassa\\{\upshape= [I] fear a dreadfully venomous serpent.}}
\transnote{The first two examples show genitive forms used in accusative sense. The rest are in ablative sense, particularly when used with the verb `to fear.'}

\head{310}{310, 324. kammakaraṇanimittatthesu sattamī.}
\headtrans{In the meaning of \pali{kamma}, \pali{karaṇa}, and \pali{nimitta}, [there is a] locative [vibhatti].}
\sutdef{kammakaraṇanimittatthesu sattamīvibhatti hoti.}
\sutdeftrans{In the meaning of \pali{kamma}, \pali{karaṇa}, and \pali{nimitta} (cause), there is a locative vibhatti.}
\example[0]{sundarāvuso ime ājīvakā bhikkhūsu abhivādenti\\{\upshape= O friend, these nice naked ascetics bow down to monks.}}
\example[0]{hatthesu piṇḍāya caranti, pattesu piṇḍāya caranti, pathesu gacchanti\\{\upshape= [They] go for alms with hands, go for alms with bowls, go by the paths.}}
\example{dīpi cammesu haññate, kuñjaro dantesu haññate\\{\upshape= A leopard is killed because of pelts, an elephant is killed because of tusks.}}
\transnote{In the first example, \pali{bhikkhūsu} is used as kamma. In the second, \pali{hatthesu}, \pali{pattesu}, and \pali{pathesu} are used as karaṇa. In the last, \pali{cammesu} and \pali{dantesu} are used as cause.}

\head{311}{311, 325. sampadāne ca.}
\headtrans{In \pali{sampadāna}, also [there is a locative vibhatti].}
\sutdef{sampadāne ca sattamīvibhatti hoti.}
\sutdeftrans{In \pali{sampadāna}, also there is a locative vibhatti.}
\example[0]{saṅghe dinnaṃ mahapphalaṃ\\{\upshape= [The gift] given to the Saṅgha [brings] great result.}}
\example[0]{saṅghe gotamī dehi\\{\upshape= O Gotamī, give to the Saṅgha.}}
\example{saṅghe te dinne ahañceva pūjito bhavissāmi\\{\upshape= When [alms] was given by you to the Saṅgha, I will be a revered [one by you].}}

\head{312}{312, 326. pañcamyatthe ca.}
\headtrans{In ablative meaning, also [there is a locative vibhatti].}
\sutdef{pañcamyatthe ca sattamīvibhatti hoti.}
\sutdeftrans{In ablative meaning, also there is a locative vibhatti.}
\example{kadalīsu gaje rakkhanti\\{\upshape= [People] prevent elephants from banana trees.}}

\head{313}{313, 327. kālabhāvesu ca.}
\headtrans{In time and state, there is a locative vibhatti.}
\sutdef{kālabhāvesu ca kattari payujjamāne sattamīvibhatti hoti.}
\sutdeftrans{When the subject is being applied with time and state, there is a locative vibhatti.}
\example[0]{pubbaṇhasamaye gato, sāyanhasamaye āgato\\{\upshape= [One] went in the morning, [then] came in the evening.}}
\example[0]{bhikkhūsu bhojīyamānesu gato, bhuttesu āgato\\{\upshape= [One] went while monks are eating, came when [they] have eaten.}}
\example{gosu duyhamānesu gato, duddhāsu āgato\\{\upshape= [One] went while the cows are being milked, came when [they] have been milked.}}
\transnote{As shown in the last two examples, \pali{bhāva} or state here means a concurrent action. So, this is a way to make complex sentences, much like \pali{anādara}-sentence mentioned in \hyperref[sut:305]{Kacc 305}. Technically, this is called \pali{lakkhaṇa}-sentence.}

\head{314}{314, 328. upadhādhikissaravacane.}
\headtrans{In [an application of] \pali{upa} and \pali{adhi} in an expression of excess and rulership, [there is a locative vibhatti].}
\sutdef{upaadhiiccetesaṃ payoge adhikaissaravacane sattamīvibhatti hoti.}
\sutdeftrans{In an application of \pali{upa} and \pali{adhi} in an expression of excess and rulership, there is a locative vibhatti.}
\example[0]{upa khāriyaṃ doṇo, upa nikkhe kahāpaṇaṃ\\{\upshape= A doṇa is greater than a khāri, a kahāpaṇa is greater than a nikkha.}}
\example[0]{adhi brahmadatte pañcālā\\{\upshape= The people of Pañcāla are ruled over by king Brahmadatta.}}
\example[0]{adhi naccesu gotamī\\{\upshape= Gotamī rules over dancers.}}
\example{adhi devesu buddho\\{\upshape= The Buddha rules over gods.}}
\transnote{In the second example, the logic seems reverse. But Rūpa 328 asserts the meaning by ``\pali{brahmadattissarā pañcālāti attho}.'' In a straight form, it should be ``\pali{adhi pañcalesu brahmadatto},'' though. In fact, this example is really old. It can be found in Patañjali's Mahābhāṣya, in the commentary to Pāṇ 1.4.97.\footnote{See \citealp[p.~303]{sharma:asta2}.}}

\head{315}{315, 329. maṇḍitussukkesu tatiyā.}
\headtrans{In [the meaning of] \pali{maṇḍita} and \pali{ussukka}, [there is an] instrumental [vibhatti, locative as well].}
\sutdef{maṇḍitaussukkaiccetesvatthesu tatiyāvibhatti hoti, sattamī ca.}
\sutdeftrans{In the meaning of \pali{maṇḍita} and \pali{ussukka}, there is an instrumental vibhatti, locative as well.}
\example[0]{ñāṇena pasīdito\\{\upshape= [One] was bright/faithful by insight.}}
\example[0]{ñāṇasmiṃ pasīdito\\{\upshape= [One] was bright/faithful in wisdom.}}
\example[0]{ñāṇena ussukko tathāgato\\{\upshape= The Buddha [is] zealous by insight}}
\example{ñāṇasmiṃ ussukko tathāgatagotto\\{\upshape= Disciples of the Buddha [are] zealous in wisdom.}}
\transnote{In Rūpa 329, \pali{maṇḍita} is defined as ``\pali{maṇḍitasaddo panettha pasannatthavācako}.'' That is why we see \pali{pasīdito} in the examples. And \pali{ussukka} is defined as ``\pali{ussukkasaddo saīhattho},'' hence \pali{saha + īha}.}

\newpage
{\footnotesize
\begin{longtable}{%
		@{}>{\raggedright\arraybackslash}p{0.12\linewidth}%
		*{8}{>{\centering\arraybackslash}p{0.07\linewidth}}@{}}
\caption{Grammatical use cases summarized}\label{tab:karaka}\\
\toprule
\bfseries Func. & \bfseries Def. & \bfseries Nom. & \bfseries Acc. & \bfseries Ins. & \bfseries Dat. & \bfseries Abl. & \bfseries Gen. & \bfseries Loc. \\ \midrule
\endfirsthead
\multicolumn{9}{c}{\tablename\ \thetable: Grammatical use cases summarized (contd\ldots)}\\
\toprule
\bfseries Func. & \bfseries Def. & \bfseries Nom. & \bfseries Acc. & \bfseries Ins. & \bfseries Dat. & \bfseries Abl. & \bfseries Gen. & \bfseries Loc. \\ \midrule
\endhead
\bottomrule
\ltblcontinuedbreak{9}
\endfoot
\bottomrule
\endlastfoot
%
kattā & \hyperref[sut:281]{281} \hyperref[sut:282]{282} & & & \hyperref[sut:288]{288} & & & & \\
kamma & \hyperref[sut:280]{280} & & \hyperref[sut:297]{297} & & & & \hyperref[sut:309]{309} & \hyperref[sut:310]{310} \\
karaṇa & \hyperref[sut:279]{279} & & \hyperref[sut:307]{307} & \hyperref[sut:286]{286} & & & \hyperref[sut:308]{308} & \hyperref[sut:310]{310} \\
\mbox{sampadāna} & \hyperref[sut:276]{276} \hyperref[sut:277]{277} & & & & \hyperref[sut:293]{293} & & & \hyperref[sut:311]{311} \\
apādāna & \hyperref[sut:271]{271} \hyperref[sut:272]{272} \hyperref[sut:273]{273} \hyperref[sut:274]{274} \hyperref[sut:275]{275} & & & & & \hyperref[sut:295]{295} & \hyperref[sut:309]{309} & \hyperref[sut:312]{312} \\
sāmī & \hyperref[sut:283]{283} & & \hyperref[sut:306]{306} & & & & \hyperref[sut:301]{301} & \\
okāsa & \hyperref[sut:278]{278} & & \hyperref[sut:307]{307} & \hyperref[sut:290]{290} & & & \hyperref[sut:308]{308} & \hyperref[sut:302]{302} \\
hetu & \hyperref[sut:282]{282} & & & \hyperref[sut:289]{289} & & \hyperref[sut:296]{296} & & \hyperref[sut:310]{310} \\
liṅgatta & & \hyperref[sut:284]{284} & & & & & & \\
ālapana & & \hyperref[sut:285]{285} & & & & & & \\
accanta & & & \hyperref[sut:298]{298} & & & & & \\
\mbox{kammappavacanīya} & & & \hyperref[sut:299]{299} & & & & & \\
gati\,buddhi\,etc. & & & \hyperref[sut:300]{300} & & & & & \\
rahitā\,rite\,vinā\,nānā & & & \hyperref[sut:272]{272} & \hyperref[sut:272]{272} & & \hyperref[sut:272]{272} & & \\
saha\,saddhiṃ & & & & \hyperref[sut:287]{287} & & & & \\
\mbox{aṅgavikāra} & & & & \hyperref[sut:291]{291} & & & & \\
visesana & & & & \hyperref[sut:292]{292} & & & & \\
namo & & & & & \hyperref[sut:294]{294} & & & \\
parā\,+\,ji & & & & & & \hyperref[sut:272]{272} & & \\
pa\,+\,bhū & & & & & & \hyperref[sut:272]{272} & & \\
apa\,ā\,upari\,pati & & & & & & \hyperref[sut:272]{272} & & \\
sāmī\,issara\,etc. & & & & & & & \hyperref[sut:303]{303} & \hyperref[sut:303]{303} \\
\mbox{niddhāraṇa} & & & & & & & \hyperref[sut:304]{304} & \hyperref[sut:304]{304} \\
anādara & & & & & & & \hyperref[sut:305]{305} & \hyperref[sut:305]{305} \\
lakkhaṇa & & & & & & & & \hyperref[sut:313]{313} \\
upa\,adhi & & & & & & & & \hyperref[sut:314]{314} \\
maṇḍita\,ussukka & & & & \hyperref[sut:315]{315} & & & & \hyperref[sut:315]{315} \\
\end{longtable}
}

