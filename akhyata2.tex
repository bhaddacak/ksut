\section{Dutiyakaṇḍa}

\begin{longtable}{%
		>{\itshape\raggedright\arraybackslash}p{0.15\linewidth}%
		>{\raggedright\arraybackslash}p{0.25\linewidth}%
		>{\raggedright\arraybackslash}p{0.38\linewidth}}
\caption{Paccayas used in verbal formations}\label{tab:akhypacc}\\
\toprule
\upshape\bfseries \mbox{Paccaya} & \bfseries Product & \bfseries Suttas \\ \midrule
\endfirsthead
\multicolumn{3}{c}{\footnotesize\tablename\ \thetable: Paccayas used in verbal formations (contd\ldots)}\\
\toprule
\upshape\bfseries \mbox{Paccaya} & \bfseries Product & \bfseries Suttas \\ \midrule
\endhead
\bottomrule
\ltblcontinuedbreak{3}
\endfoot
\bottomrule
\endlastfoot
%
āya & denominative & \hyperref[sut:435]{435} \\
āra & denominative & \hyperref[sut:439]{439} \\
āla & denominative & \hyperref[sut:439]{439} \\
īya & denominative & \hyperref[sut:436]{436} \\
īya & desiderative & \hyperref[sut:437]{437} \\
kha & desiderative & \hyperref[sut:433]{433}, \hyperref[sut:434]{434}, \hyperref[sut:473]{473} \\
cha & desiderative & \hyperref[sut:433]{433}, \hyperref[sut:434]{434}, \hyperref[sut:472]{472} \\
ṇaya & causative & \hyperref[sut:438]{438}, \hyperref[sut:483]{483}, \hyperref[sut:484]{484}, \hyperref[sut:515]{515}, \hyperref[sut:523]{523} \\
ṇaya & denominative & \hyperref[sut:439]{439} \\
ṇāpaya & causative & \hyperref[sut:438]{438}, \hyperref[sut:483]{483}, \hyperref[sut:484]{484}, \hyperref[sut:515]{515}, \hyperref[sut:523]{523} \\
ṇāpe & causative & \hyperref[sut:438]{438}, \hyperref[sut:483]{483}, \hyperref[sut:484]{484}, \hyperref[sut:515]{515}, \hyperref[sut:523]{523} \\
ṇe & causative & \hyperref[sut:438]{438}, \hyperref[sut:483]{483}, \hyperref[sut:484]{484}, \hyperref[sut:515]{515}, \hyperref[sut:523]{523} \\
ya & \mbox{bhāva \& kamma} & \hyperref[sut:440]{440}, \hyperref[sut:441]{441}, \hyperref[sut:442]{442}, \hyperref[sut:443]{443}, \hyperref[sut:487]{487}, \hyperref[sut:488]{488}, \hyperref[sut:502]{502}, \hyperref[sut:503]{503} \\
sa & desiderative & \hyperref[sut:433]{433}, \hyperref[sut:434]{434}, \hyperref[sut:467]{467}, \hyperref[sut:474]{474} \\
\end{longtable}


\begin{longtable}{%
		>{\raggedright\arraybackslash}p{0.05\linewidth}%
		>{\raggedright\arraybackslash}p{0.16\linewidth}%
		>{\itshape\raggedright\arraybackslash}p{0.24\linewidth}%
		>{\raggedright\arraybackslash}p{0.25\linewidth}}
\caption{Root-groups and \pali{vikaraṇa}-paccayas}\label{tab:vikapacc}\\
\toprule
\multicolumn{2}{c}{\upshape\bfseries Group} & \upshape\bfseries Paccayas & \bfseries Suttas \\ \midrule
\endfirsthead
\multicolumn{4}{c}{\footnotesize\tablename\ \thetable: Root-groups and \pali{vikaraṇa}-paccayas (contd\ldots)}\\
\toprule
\multicolumn{2}{c}{\upshape\bfseries Group} & \upshape\bfseries Paccayas & \bfseries Suttas \\ \midrule
\endhead
\bottomrule
\ltblcontinuedbreak{4}
\endfoot
\bottomrule
\endlastfoot
%
I & \paliroot{bhū} & a & \hyperref[sut:445]{445}, \hyperref[sut:510]{510} \\
II & \paliroot{rudha} & a (+ ṃ) & \hyperref[sut:446]{446} \\
III & \paliroot{divu} & ya & \hyperref[sut:447]{447}, (\hyperref[sut:444]{444})\\
IV & \paliroot{su} & ṇu, ṇā, uṇā & \hyperref[sut:448]{448} \\
V & \paliroot{kī} & nā & \hyperref[sut:449]{449} \\
VI & \paliroot{gaha} & ppa, ṇhā & \hyperref[sut:450]{450} \\
VII & \paliroot{tanu} & o, yira & \hyperref[sut:451]{451}, \hyperref[sut:511]{511} \\
VIII & \paliroot{cura} & ṇe, ṇaya & \hyperref[sut:452]{452} \\
\end{longtable}

\newpage
\head{432}{432, 362. dhātuliṅgehi parā paccayā.}
\headtrans{[There are] paccayas after roots and nominal bases.}
\sutdef{dhātuliṅgaiccetehi parā paccayā honti.}
\sutdeftrans{There are paccayas after roots and nominal bases.}
\example[0]{karoti (\paliroot{kara} + o + ti) \\{\upshape= [One] does.}}
\example{gacchati (\paliroot{gamu} + a + ti) \\{\upshape= [One] goes.}}
\transnote{To form a verb, there are a number of verbal paccayas, called \pali{vikaraṇa}, classified by root groups. In the above example, \pali{o} and \pali{a} are these paccayas from different groups.}
\example{yo koci karoti, taṃ añño “karohi karohi” iccevaṃ bravīti, atha vā karontaṃ payojayati kāreti \\{\upshape= Whoever does [an action], another one tells that [person] thus ``Do! Do!'' Or [he/she] makes [that person] the one doing [the action], hence \pali{kāreti}.}}
\transnote{Here is another formation of verb called \emph{causative}: someone makes another do something. For example, \pali{kāreti} comes from \pali{\paliroot{kara} + ṇe + ti}. Here, \pali{ṇe} is a paccaya used to form a causative verb, also \pali{ṇaya}, \pali{ṇāpe}, and \pali{ṇāpaya} (see \hyperref[sut:438]{Kacc 438} below). Likewise \pali{payojayati (pa + \paliroot{yuj} + ṇaya + ti)} is a causative verb as well.}
\example[0]{saṅgho pabbatamiva attānamācarati pabbatāyati \\(pabbata + āya + ti) \\{\upshape= The Saṅgha behaves itself like a mountain, hence \pali{pabbatāyati}.}}
\example[0]{taḷākaṃ samuddamiva attānamācarati samuddāyati \\(samudda + āya + ti) \\{\upshape= The lake behaves itself like the sea, hence \pali{samuddāyati}.}}
\example{saddo cicciṭamiva attānamācarati cicciṭāyati \\(cicciṭa + āya + ti) \\{\upshape= The sound behaves itself like `chit chit,' hence \pali{cicciṭāyati}}}
\transnote{The three examples above illustrate another form of verb called \emph{denominative}, a verb derived from noun. Here, \pali{āya} is a paccaya that converts a noun into a verb (see \hyperref[sut:435]{Kacc 435}). The last instance also shows how to create \emph{onomatopoeia} in Pāli.}
\example{vasiṭṭhassa apaccaṃ vāsiṭṭho \\(vasiṭṭha + ṇa + si) \\{\upshape= [One] is Vasiṭṭha's offspring, hence \pali{vāsiṭṭho}.}}
\transnote{This example shows a paccaya (\pali{ṇa}) after a nominal base. The outcome is called secondary derivative (taddhita).}

\head{433}{433, 528. tija\,gupa\,kita\,mānehi\ \ kha\,cha\,sā\ \ vā.}
\headtrans{After \pali{tija}, \pali{gupa}, \pali{kita}, and \pali{māna}, [there are] \pali{kha}-, \pali{cha}-, and \pali{sa}-paccaya sometimes.}
\sutdef{tijagupakitamānuiccetehi dhātūhi khachasaiccete paccayā honti vā.}
\sutdeftrans{There are \pali{kha}-, \pali{cha}-, and \pali{sa}-paccaya sometimes after these roots: \pali{tija}, \pali{gupa}, \pali{kita}, and \pali{māna}.}
\example[0]{titikkhati (\paliroot{tija} + kha + ti) \\{\upshape= [One] endures.}}
\example[0]{jigucchati (\paliroot{gupa} + cha + ti) \\{\upshape= [One] detests.}}
\example[0]{tikicchati (\paliroot{kita} + cha + ti) \\{\upshape= [One] cures.}}
\example{vīmaṃsati (vi + \paliroot{māna} + sa + ti) \\{\upshape= [One] investigates.}}
\transnote{When the three paccayas operate, sometimes the roots become reduplicated, more on this in \hyperref[sut:458]{Kacc 458} onward.}

\head{434}{434, 534. bhuja\,ghasa\,hara\,su\,pādīhi\ \ tumicchatthesu.}
\headtrans{After \pali{bhuja}, \pali{ghasa}, \pali{hara}, \pali{su}, \pali{pā} and so on, [there are \pali{kha}, \pali{cha}, and \pali{sa}] in the senses of \pali{tuṃ} and desire.}
\sutdef{bhujaghasaharasupāiccevamādīhi dhātūhi tumicchatthesu khachasaiccete paccayā honti vā.}
\sutdeftrans{There are \pali{kha}-, \pali{cha}-, and \pali{sa}-paccaya sometimes in the senses of \pali{tuṃ}-paccaya (inf.) and desire after these roots: \pali{bhuja}, \pali{ghasa}, \pali{hara}, \pali{su}, \pali{pā} and so on.}
\example[0]{bhottumicchati bubhukkhati (\paliroot{bhuja} + kha + ti) \\{\upshape= [One] desires to eat, hence \pali{bubhukkhati}.}}
\example[0]{ghasitumicchati jighacchati (\paliroot{ghasa} + cha + ti) \\{\upshape= [One] desires to eat, hence \pali{jighacchati}.}}
\example[0]{haritumicchati jigīsati (\paliroot{hara} + sa + ti) \\{\upshape= [One] desires to take away, hence \pali{jigīsati}.}}
\example[0]{sotumicchati sussūsati (\paliroot{su} + sa + ti) \\{\upshape= [One] desires to hear, hence \pali{sussūsati}.}}
\example{pātumicchati pivāsati (\paliroot{pā} + sa + ti) \\{\upshape= [One] desires to drink, hence \pali{pivāsati}.}}
\transnote{This form of verb is called \emph{desiderative}. We often see reduplication here.}

\head{435}{435, 536. āya nāmato kattūpamānādācāre.}
\headtrans{[There is] \pali{āya}-paccaya in the sense of behavior after a noun as metaphorical actor.}
\sutdef{nāmato kattūpamānā ācāratthe āyapaccayo hoti.}
\sutdeftrans{There is \pali{āya}-paccaya in the sense of behavior after a noun as metaphorical actor.}
\example[0]{saṅgho pabbatamiva attānamācarati pabbatāyati \\(pabbata + āya + ti) \\{\upshape= The Saṅgha behaves itself like a mountain, hence \pali{pabbatāyati}.}}
\example[0]{taḷākaṃ samuddamiva attānamācarati samuddāyati \\(samudda + āya + ti) \\{\upshape= The lake behaves itself like the sea, hence \pali{samuddāyati}.}}
\example{saddo cicciṭamiva attānamācarati cicciṭāyati \\(cicciṭa + āya + ti) \\{\upshape= The sound behaves itself like `chit chit,' hence \pali{cicciṭāyati}}}
\transnote{This is a formation of denominative verbs.}

\head{436}{436, 537. īyūpamānā ca.}
\headtrans{[There is] \pali{īya}-paccaya after a metaphor also.}
\sutdef{nāmato upamānā ācāratthe ca īyapaccayo hoti.}
\sutdeftrans{There is also \pali{īya}-paccaya in the sense of behavior after a noun as metaphor.}
\example[0]{achattaṃ chattamiva ācarati chattīyati \\(chatta + īya + ti) \\{\upshape= Having no parasol, [one] behaves like having a parasol, hence \pali{chattīyati}.}}
\example{aputtaṃ puttamiva ācarati puttīyati \\(putta + īya + ti) \\{\upshape= Having no child, [one] behaves like having a child, hence \pali{puttīyati}.}}
\transnote{This is another formation of denominative verbs.}

\head{437}{437, 538. nāmamhātticchatthe.}
\headtrans{[There is] \pali{īya}-paccaya after a noun in the sense of desire for oneself.}
\sutdef{nāmamhā attano icchatthe īyapaccayo hoti.}
\sutdeftrans{There is \pali{īya}-paccaya after a noun in the sense of desire for oneself.}
\example[0]{attano pattamicchati pattīyati (patta + īya + ti) \\{\upshape= [One] desires a bowl for oneself, hence \pali{pattīyati}.}}
\example[0]{vatthīyati (vatta + īya + ti) \\{\upshape= [One] desires a cloth for oneself.}}
\example[0]{parikkhārīyati (parikkhāra + īya + ti) \\{\upshape= [One] desires a requisite for oneself.}}
\example[0]{cīvarīyati (cīvara + īya + ti) \\{\upshape= [One] desires a robe for oneself.}}
\example[0]{dhanīyati (dhana + īya + ti) \\{\upshape= [One] desires wealth for oneself.}}
\example{ghaṭīyati (ghaṭa + īya + ti) \\{\upshape= [One] desires a pot for oneself.}}

\head{438}{438, 540. dhātūhi ṇeṇayaṇāpeṇāpayā kāritāni hetvatthe.}
\headtrans{After roots, [there are] \pali{ṇe}-, \pali{ṇaya}-, \pali{ṇāpe}-, and \pali{ṇāpaya}-paccaya, [called] \pali{kārita}, in the sense of cause.}
\sutdef{sabbehi dhātūhi ṇeṇayaṇāpeṇāpayaiccete paccayā honti kāritasaññā ca hetvatthe.}
\sutdeftrans{There are \pali{ṇe}-, \pali{ṇaya}-, \pali{ṇāpe}-, and \pali{ṇāpaya}-paccaya after all roots in the sense of cause, also called \pali{kārita}-paccaya.}
\example{yo koci karoti, taṃ añño “karohi karohi” iccevaṃ bravīti, atha vā karontaṃ payojayati kāreti/kārayati/kārāpeti/kārāpayati \\(\paliroot{kara} + ṇe/ṇaya/ṇāpe/ṇāpaya + ti) \\{\upshape= Whoever does [an action], another one tells that [person] thus ``Do! Do!'' Or [he/she] makes [that person] the one doing [the action], hence \pali{kāreti/kārayati/kārāpeti/kārāpayati}.}}
\example{ye keci karonti, te aññe “karotha karotha” iccevaṃ bruvanti kārenti/kārayanti/kārāpenti/kārāpayanti \\(\paliroot{kara} + ṇe/ṇaya/ṇāpe/ṇāpaya + nti) \\{\upshape= Whichever people do [an action], other ones tell those [people] thus ``Do! Do!'', hence \pali{kārenti/kārayanti/kārāpenti/kārāpayanti}.}}
\example{yo koci pacati, taṃ añño “pacāhi pacāhi” iccevaṃ bravīti, atha vā pacantaṃ payojayati pāceti/pācayati/pācāpeti/pācāpayati \\(\paliroot{paca} + ṇe/ṇaya/ṇāpe/ṇāpaya + ti) \\{\upshape= Whoever cooks, another one tell that [person] thus ``Cook! Cook!'' Or [he/she] makes [that person] the one cooking, hence \pali{pāceti/pācayati/pācāpeti/pācāpayati}.}}
\example{ye keci pacanti, te aññe “pacatha pacatha” iccevaṃ bruvanti pācenti/pācayanti/pācāpenti/pācāpayanti \\(\paliroot{paca} + ṇe/ṇaya/ṇāpe/ṇāpaya + nti) \\{\upshape= Whichever people cook, other ones tell those [people] thus ``Cook! Cook!'', hence \pali{pācenti/pācayanti/pācāpenti/pācāpayanti}.}}
\example{bhaṇeti/bhaṇayati/bhaṇāpeti/bhaṇāpayati \\(\paliroot{bhaṇa} + ṇe/ṇaya/ṇāpe/ṇāpaya + ti) \\{\upshape= [One] makes [another one] speak.}}
\example{bhaṇenti/bhaṇayanti/bhaṇāpenti/bhaṇāpayanti \\(\paliroot{bhaṇa} + ṇe/ṇaya/ṇāpe/ṇāpaya + nti) \\{\upshape= [They] make [other ones] speak.}}
\transnote{These are the paccayas used in the formation of causative verbs. Normally, these cause vuddhi-strength in the products, due to \pali{ṇa}-anubandha. See \hyperref[sut:483]{Kacc 483}, also \hyperref[sut:400]{Kacc 400} and \hyperref[sut:405]{405}.}

\head{439}{439, 539. dhāturūpe nāmasmā ṇayo ca.}
\headtrans{In a form of root, after a noun [there is] \pali{ṇaya}-paccaya also.}
\sutdef{tasmā nāmasmā ṇayapaccayo hoti kāritasañño ca dhāturūpe sati.}
\sutdeftrans{When a form of root exists, there is \pali{ṇaya}-paccaya after that noun. [This is] also called \pali{kārita}.}
\example[0]{hatthinā atikkamati maggaṃ atihatthayati \\(ati + hatthī + ṇaya + ti) \\{\upshape= [One] crosses over the path by an elephant, hence \pali{atihatthayati}.}}
\example[0]{vīṇāya upagāyati gītaṃ upavīṇayati (upa + vīṇā + ṇaya + ti) \\{\upshape= [One] sings a song along with a lute, hence \pali{upavīṇayati}.}}
\example[0]{daḷhaṃ karoti vīriyaṃ daḷhayati (daḷha + ṇaya + ti) \\{\upshape= [One] makes an effort strengthened, hence \pali{daḷhayati}.}}
\example{visuddhā hoti ratti visuddhayati (visuddha + ṇaya + ti) \\{\upshape= The night is purified, hence \pali{visuddhayati}.}}
\transnote{As marked by \pali{ca}, other two paccayas can also be used: \pali{āra} and \pali{āla}. Note that most examples in this sutta are not found in the main Pāli texts.}
\example[0]{santaṃ karoti santārati (santa + ṇaya + ti) \\{\upshape= [One] makes peace, hence \pali{santārati}.}}
\example{upakkamaṃ karoti upakkamālati (upakkama + ṇaya + ti) \\{\upshape= [One] makes an endeavor, hence \pali{upakkamālati}.}}
\transnote{This is another way to form a denominative verb. As we have learnt from the examples, \pali{dhāturūpa} means simply taking the noun as a form of root, then \pali{ṇaya} can be applied. The final products are intended to be used as normal verbs, not exactly like causatives. But a certain sense of causative is there. For example, \pali{atihatthayati} has the elephant as the target of command.}

\head{440}{440, 445. bhāvakammesu yo.}
\headtrans{In [the structure of] \pali{bhāva} and \pali{kamma}, [there is] \pali{ya}-paccaya.}
\sutdef{sabbehi dhātūhi bhāvakammesu yapaccayo hoti.}
\sutdeftrans{There is \pali{ya}-paccaya after all roots in \pali{bhāva} (impersonal passive) and \pali{kamma} (passive) [structure].}
\example[0]{ṭhīyate (\paliroot{ṭhā + ya + te}) \\{\upshape= Standing is done.}}
\example[0]{bujjhate (\paliroot{budha + ya + te}) \\{\upshape= Knowing is done/[Something] is known.}}
\example[0]{paccate (\paliroot{paca + ya + te}) \\{\upshape= Cooking is done/[Something] is cooked.}}
\example[0]{labbhate (\paliroot{labha + ya + te}) \\{\upshape= Obtaining is done/[Something] is obtained.}}
\example[0]{karīyate (\paliroot{kara + ya + te}) \\{\upshape= Making is done/[Something] is made.}}
\example[0]{yujjate (\paliroot{yuja + ya + te}) \\{\upshape= Applying is done/[Something] is applied.}}
\example{uccate (\paliroot{vaca + ya + te}) \\{\upshape= Saying is done/[Something] is said.}}
\transnote{This is the way to form a passive verb. We normally use attanopada vibhattis in these structures (\hyperref[sut:453]{Kacc 453}), but parassapada forms are more often found in the texts. The hard part of applying \pali{ya}-paccaya is that the results can be a little unpredictable. In some cases, \pali{ya} is retained, sometimes with \pali{i}- or \pali{ī}-insertion (see also \hyperref[sut:502]{Kacc 502}). In some, it makes a double consonant. Yet in some, it deforms the roots. You may have to make yourselves familiar with various forms of passive verbs. See also \hyperref[sut:441]{Kacc 441}--\hyperref[sut:443]{443} below.}

\head{441}{441,~447.~tassa~cavaggayakāravakārattaṃ~sadhātvantassa.}
\headtrans{For that [\pali{ya}-paccaya, there is the state of being] \pali{ca}-group, \pali{ya}, and \pali{va} together with the ending of roots.}
\sutdef{tassa yapaccayassa cavaggayakāravakārattaṃ hoti dhātūnaṃ antena saha yathāsambhavaṃ.}
\sutdeftrans{There is the state of being \pali{ca}-group, \pali{ya}, and \pali{va} for that \pali{ya}-paccaya appropriately together with the ending of roots.}
\example[0]{vuccate, vuccante, uccate, uccante (\paliroot{vaca} + ya + [n]te)}
\example[0]{paccate, paccante (\paliroot{paca} + ya + [n]te)}
\example[0]{majjate, majjante (\paliroot{mada} + ya + [n]te)}
\example[0]{yujjate, yujjante (\paliroot{yuja} + ya + [n]te)}
\example[0]{bujjhate, bujjhante (\paliroot{budha} + ya + [n]te)}
\example[0]{kujjhate, kujjhante (\paliroot{kudha} + ya + [n]te)}
\example[0]{ujjhate, ujjhante (\paliroot{udha}\footnote{As a root, \pali{udha} is not found, but \pali{ujjha ussagge} (forsaking).} + ya + [n]te)}
\example[0]{haññate, haññante (\paliroot{hana} + ya + [n]te)}
\example[0]{kayyate, kayyante (\paliroot{kī} + ya + [n]te)}
\example{dibbate, dibbante (\paliroot{divu} + ya + [n]te)}
\transnote{Why palatal \pali{ca}-group? That is because \pali{ya} itself is palatal semi-vowel. There is some observable pattern here, i.e., when roots ending with \pali{-dha} meet \pali{ya}, they become \pali{-jjha}. Other transformations can be harder to predict. In \pali{va} case, it is often changed to \pali{ba}, as we see in \pali{dibbate}.\footnote{If you are curious about interchangeability of \pali{v} and \pali{b}, see Whit 2\S50a.}}

\enlargethispage*{5mm}
\boxnote{There are no standard names for Pāli roots.\\
\hspace{5mm}\bullet\ As a matter of fact, verbal roots are a construction of grammarians. What we really have are final forms of verbs in use. Grammarians reconstruct roots from verbs to figure out or regulate the underlying rules.\footnote{``The verbal roots are not words in their own right but convenient grammatical fictions.'' (\citealp[p.~21]{coulson:sanskrit})}\\
\hspace{5mm}\bullet\ Unlike Sanskit, which has well-defined root names since Pāṇini, Pāli has no standard for calling roots. Different grammarians define different names, often with slightly different ending vowels. So, do not too much fuss over the names. Even in this Kaccāyana, some roots are called inconsistently.\\
\hspace{5mm}\bullet\ That is to say, \paliroot{gama} and \paliroot{gamu} are the same, even so is \paliroot{gam} (Pāli grammarians do not end a term with consonant, though). Likewise, \paliroot{diva}, \paliroot{divu}, and \paliroot{div} are the same.\\
\hspace{5mm}\bullet\ In \hyperref[sut:512]{Kacc 512}, it is said that multi-vowel roots will have their ending elided first. For example, \pali{gamu} becomes \pali{gam} before undergoing other operations.\\
\hspace{5mm}\bullet\ However, be careful with roots and their formation or transformation. They can be confusing sometimes. For example, \pali{yujjati} (from \paliroot{yuja}) means `to be applied,' but \pali{yujjhati} (from \paliroot{yudha}) means `to fight.'
}

\head{442}{442, 448. ivaṇṇāgamo vā.}
\headtrans{[There is] an insertion of \pali{i}-group sometimes.}
\sutdef{sabbehi dhātūhi yamhi paccaye, pare ivaṇṇāgamo hoti vā.}
\sutdeftrans{Because of \pali{ya}-paccaya behind all roots, there is sometimes an insertion of \pali{i}-group before.}
\example[0]{karīyate, karīyati (\paliroot{kara} + ya + te/ti)}
\example{gacchīyate, gacchīyati (\paliroot{gama} + ya + te/ti)}
\transnote{From a Thai source, the examples are \pali{kariyate, kariyati, gacchiyate,} and \pali{gacchiyati} instead.}

\head{443}{443, 449. pubbarūpañca.}
\headtrans{[There is] an assimilation [of \pali{ya}] to the preceding form also.}
\sutdef{sabbehi dhātūhi yapaccayo pubbarūpamāpajjate vā.}
\sutdeftrans{The \pali{ya}-paccaya behind all roots assimilates to the preceding form sometimes.}
\example[0]{vuḍḍhate (\paliroot{vaḍḍha} + ya + te)}
\example[0]{phallate (\paliroot{phala} + ya + te)}
\example[0]{dammate (\paliroot{dama} + ya + te)}
\example[0]{sakkate (\paliroot{saka} + ya + te)}
\example[0]{labbhate (\paliroot{labha} + ya + te)}
\example{dissate (\paliroot{disa} + ya + te)}

\head{444}{444, 501. tathā kattari ca.}
\headtrans{By that manner [the \pali{ya}-paccaya operates in impersonal and passice structure, it should be done] in active structure also.}
\sutdef{yathā heṭṭhā bhāvakammesu yapaccayassa ādeso hoti tathā kattaripi yapaccayassa ādeso kātabbo.}
\sutdeftrans{By which manner the operation of \pali{ya}-paccaya exists in impersonal and passive structure mentioned above, by that manner the operation of \pali{ya}-paccaya should be done likewise in active structure.}
\example[0]{bujjhati (\paliroot{budha} + ya + ti)}
\example[0]{vijjhati (\paliroot{vidha} + ya + ti)}
\example[0]{maññati (\paliroot{mana} + ya + ti)}
\example{sibbati (\paliroot{sivu} + ya + ti)}
\transnote{All the verbs above are in active form. Their root are of group III (\paliroot{divu}), which has \pali{ya} as the group paccaya (\pali{vikaraṇa}). So, the operation is similar to the cases described earlier. See \hyperref[sut:447]{Kacc 447} below.}

\head{445}{445, 433. bhūvādito a.}
\headtrans{After [the root-group of] \paliroot{bhū} and so on, [there is] \pali{a}-paccaya.}
\sutdef{bhūiccevamādito dhātugaṇato apaccayo hoti kattari.}
\sutdeftrans{There is \pali{a}-paccaya after the root-group of \paliroot{bhū} and so on in active structure.}
\example[0]{bhavati (\paliroot{bhū} + a + ti)}
\example[0]{paṭhati (\paliroot{paṭha} + a + ti)}
\example[0]{pacati (\paliroot{paca} + a + ti)}
\example{jayati (\paliroot{ji} + a + ti)}
\transnote{This is the root-group I having \pali{a} as \pali{vikaraṇa}-paccaya.}

\head{446}{446, 509. rudhādito niggahitapubbañca.}
\headtrans{After [the root-group of] \paliroot{rudha} and so on, [there is \pali{a}-paccaya] with a niggahita at the first syllable also.}
\sutdef{rudhaiccevamādito dhātugaṇato apaccayo hoti kattari, pubbe niggahitāgamo hoti.}
\sutdeftrans{There is \pali{a}-paccaya after the root-group of \paliroot{rudha} and so on in active structure. There is an insertion of niggahita at the first syllable [also].}
\example[0]{rundhati (\paliroot{rudha} + a + ti)}
\example[0]{chindati (\paliroot{chida} + a + ti)}
\example{bhindati (\paliroot{bhida} + a + ti)}
\transnote{This is the root-group II having \pali{a} as \pali{vikaraṇa}-paccaya. When the niggahita is inserted, it then becomes the last character of the consonant group (\hyperref[sut:31]{Kacc 31}). Hence, \pali{ruṃdha} becomes \pali{rundha}, \pali{chiṃda} becomes \pali{chinda}, and \pali{bhiṃda} becomes \pali{bhinda}.}
\transnote{As marked by \pali{ca}, the author added \pali{i}, \pali{ī}, \pali{e}, and \pali{o} as the group paccayas producing \pali{rundhiti, rundhīti, rundheti, run\-dh\-oti, sumbhoti,} and \pali{parisumbhoti}. This can make things confusing.}

\head{447}{447, 510. divādito yo.}
\headtrans{After [the root-group of] \paliroot{divu} and so on, [there is] \pali{ya}-paccaya.}
\sutdef{divuiccevamādito dhātugaṇato yapaccayo hoti kattari.}
\sutdeftrans{There is \pali{ya}-paccaya after the root-group of \paliroot{divu} and so on in active structure.}
\example[0]{dibbati (\paliroot{divu} + ya + ti)}
\example[0]{sibbati (\paliroot{sivu} + ya + ti)}
\example[0]{yujjhati (\paliroot{yudha} + ya + ti)}
\example[0]{vijjhati (\paliroot{vidha} + ya + ti)}
\example{bujjhati (\paliroot{budha} + ya + ti)}
\transnote{This is the root-group III having \pali{ya} as \pali{vikaraṇa}-paccaya. The operation of \pali{ya} is the same as described in impersonal and passive structure.}

\head{448}{448, 512. svādito ṇu ṇā uṇā ca.}
\headtrans{After [the root-group of] \paliroot{su} and so on, [there are] \pali{ṇu}-, \pali{ṇā}-, and \pali{uṇā}-paccaya.}
\sutdef{suiccevamādito dhātugaṇato ṇu ṇā uṇāiccete paccayā honti kattari.}
\sutdeftrans{There are \pali{ṇu}-, \pali{ṇā}-, and \pali{uṇā}-paccaya after the root-group of \paliroot{su} and so on in active structure.}
\example[0]{abhisuṇoti (abhi + \paliroot{su} + ṇu + ti)}
\example[0]{abhisuṇāti (abhi + \paliroot{su} + ṇā + ti)}
\example[0]{saṃvuṇoti (saṃ + \paliroot{vu} + ṇu + ti)}
\example[0]{saṃvuṇāti (saṃ + \paliroot{vu} + ṇā + ti)}
\example[0]{āvuṇoti (ā + \paliroot{vu} + ṇu + ti)}
\example[0]{āvuṇāti (ā + \paliroot{vu} + ṇā + ti)}
\example[0]{pāpuṇoti (pa + \paliroot{apa} + ṇu + ti)}
\example{pāpuṇāti (pa + \paliroot{apa} + uṇā + ti)}
\transnote{This is the root-group IV having \pali{ṇu, ṇā,} and \pali{uṇā} as \pali{vikaraṇa}-paccaya. Note that the \pali{ṇa} in the paccayas is part of them, not \pali{anubandha} (see page \pageref{box:anubandha}). So, it causes no vuddhi-strength here.}

\head{449}{449, 513. kiyādito nā.}
\headtrans{After [the root-group of] \paliroot{kī} and so on, [there is] \pali{nā}-paccaya.}
\sutdef{kīiccevamādito dhātugaṇato nāpaccayo hoti kattari.}
\sutdeftrans{There is \pali{nā}-paccaya after the root-group of \paliroot{kī} and so on in active structure.}
\example[0]{kiṇāti (\paliroot{kī} + nā + ti)}
\example[0]{jināti (\paliroot{ji} + nā + ti)}
\example[0]{dhunāti (\paliroot{dhu} + nā + ti)}
\example[0]{munāti (\paliroot{muna} + nā + ti)}
\example[0]{lunāti (\paliroot{lu} + nā + ti)}
\example{punāti (\paliroot{pu} + nā + ti)}
\transnote{This is the root-group V having \pali{nā} as \pali{vikaraṇa}-paccaya. Sometimes \pali{nā} becomes \pali{ṇā}, so both \pali{kināti} and \pali{kiṇāti} are found.}

\head{450}{450, 517. gahādito ppaṇhā.}
\headtrans{After [the root-group of] \paliroot{gaha} and so on, [there are] \pali{ppa}- and \pali{ṇhā}-paccaya.}
\sutdef{gahaiccevamādito dhātugaṇato ppaṇhāiccete paccayā honti kattari.}
\sutdeftrans{There are \pali{ppa}- and \pali{ṇhā}-paccaya after the root-group of \paliroot{gaha} and so on in active structure.}
\example[0]{gheppati (\paliroot{gaha} + ppa + ti)}
\example{gaṇhāti (\paliroot{gaha} + ṇhā + ti)}
\transnote{This is the root-group VI having \pali{ppa} and \pali{ṇhā} as \pali{vikaraṇa}-paccaya. This \pali{ṇa} is not \pali{anubandha} as well. The description of the transformation from \pali{gaha} to \pali{ghe} is found in \hyperref[sut:489]{Kacc 489} (\pali{gahassa ghe ppe}).}

\head{451}{451, 520. tanādito oyirā.}
\headtrans{After [the root-group of] \paliroot{tanu} and so on, [there are] \pali{o}- and \pali{yira}-paccaya.}
\sutdef{tanuiccevamādito dhātugaṇato oyiraiccete paccayā honti kattari.}
\sutdeftrans{There are \pali{o}- and \pali{yira}-paccaya after the root-group of \paliroot{tanu} and so on in active structure.}
\example[0]{tanoti, tanohi (\paliroot{tanu} + o + ti/hi)}
\example[0]{karoti, karohi (\paliroot{kara} + o + ti/hi)}
\example{kayirati, kayirāhi (\paliroot{kara} + yira + ti/hi)}
\transnote{This is the root-group VII having \pali{o} and \pali{yira} as \pali{vikaraṇa}-paccaya.}

\head{452}{452, 525. curādito ṇeṇayā.}
\headtrans{After [the root-group of] \paliroot{cura} and so on, [there are] \pali{ṇe}- and \pali{ṇaya}-paccaya.}
\sutdef{curaiccevamādito dhātugaṇato ṇeṇayaiccete paccayā honti kattari, kāritasaññā ca.} 
\sutdeftrans{There are \pali{ṇe}- and \pali{ṇaya}-paccaya after the root-group of \paliroot{cura} and so on in active structure. [They are] called \pali{kārita}-paccaya also.}
\example[0]{coreti (\paliroot{cura} + ṇe + ti)}
\example[0]{corayati (\paliroot{cura} + ṇaya + ti)}
\example[0]{cinteti (\paliroot{cinta} + ṇe + ti)}
\example[0]{cintayati (\paliroot{cinta} + ṇaya + ti)}
\example[0]{manteti (\paliroot{manta} + ṇe + ti)}
\example{mantayati (\paliroot{manta} + ṇaya + ti)}
\transnote{This is the root-group VIII having \pali{ṇe} and \pali{ṇaya} as \pali{vikaraṇa}-paccaya. Note that the \pali{ṇa} in these paccayas is real \pali{anubandha}, so it causes vuddhi-strength in the first syllable. Because the paccayas are also used in causative formation, when verbs in this group are used in causative structure, they look the same as their active form.}

\head{453}{453, 444. attanopadāni bhāve ca kammani.}
\headtrans{There are terms of \pali{attanopada} in impersonal passive and passive structure.}
\sutdef{bhāve ca kammani ca attanopadāni honti.}
\sutdeftrans{In impersonal passive and passive structure, there are terms of \pali{attanopada}.}
\example[0]{uccate, uccante (\paliroot{vaca} + ya + [n]te) \\{\upshape= Saying is done.}}
\example[0]{majjate, majjante (\paliroot{vaca} + ya + [n]te) \\{\upshape= Being intoxicated is done.}}
\example[0]{yujjate, yujjante (\paliroot{yuja} + ya + [n]te) \\{\upshape= Appling is done/[Something] is applied.}}
\example[0]{kujjhate, kujjhante (\paliroot{kuda} + ya + [n]te) \\{\upshape= Getting angry is done.}}
\example[0]{labbhate, labbhante (\paliroot{labha} + ya + [n]te) \\{\upshape= Obtaining is done/[Something] is obtained.}}
\example[0]{kayyate, kayyante (\paliroot{kī} + ya + [n]te) \\{\upshape= Buying is done/[Something] is bought.}}

\head{454}{454, 440. kattari ca.}
\headtrans{In active structure also, [there are terms of \pali{attanopada}].}
\sutdef{kattari ca attanopadāni honti.}
\sutdeftrans{In active structure, there are also terms of \pali{attanopada}.}
\example[0]{maññate {\upshape ([One] thinks.)}}
\example[0]{rocate {\upshape ([One] shines.)}}
\example[0]{socate {\upshape ([One] mourns.)}}
\example[0]{bujjhate {\upshape ([One] knows.)}}
\example{jāyate {\upshape ([One] is born/arises.)}}

\head{455}{455, 530. dhātuppaccayehi vibhattiyo.}
\headtrans{After paccayas [designated after] roots, [there are] vibhattis.}
\sutdef{dhātuniddiṭṭhehi paccayehi khādikāritantehi vibhattiyo honti.}
\sutdeftrans{There are vibhattis after paccayas designated after roots, i.e., \pali{kha} and so on to the kārita (causative makers).}
\example[0]{titikkhati (\paliroot{tija} + kha + ti)}
\example[0]{jigucchati (\paliroot{gupa} + cha + ti)}
\example[0]{vīmaṃsati (vi + \paliroot{māna} + sa + ti)}
\example[0]{samuddāyati (samudda + āya + ti)}
\example[0]{puttīyati (putta + īya + ti)}
\example[0]{kāreti (\paliroot{kara} + ṇe + ti)}
\example{pāceti (\paliroot{paca} + ṇe + ti)}
\transnote{Up to this point, the reader should be able to tell paccayas and vibhattis apart. The latter group of suffixes function as tense and mood identifiers. They are the last part of a finally formed verb. We have already learnt about verbal vibhattis in the first section. In the context of verbal formation, we have two main groups of paccayas: (1) root-group paccayas (\pali{vikaraṇa}) as described in \hyperref[sut:445]{Kacc 445}--\hyperref[sut:452]{452}, and (2) paccayas after roots (\pali{dhātuniddiṭṭha}) as described in \hyperref[sut:435]{Kacc 435}--\hyperref[sut:439]{439}. This sutta just affirms that verbal vibhattis are also annexed to paccayas of the latter group as well.}

\head{456}{456, 430. kattari parassapadaṃ.}
\headtrans{In active structure, [there is] a \pali{parassapada}-vibhatti.}
\sutdef{kattari parassapadaṃ hoti.}
\sutdeftrans{There is a \pali{parassapada}-vibhatti in active structure.}
\example[0]{karoti (\paliroot{kara} + o + ti)}
\example[0]{pacati (\paliroot{paca} + a + ti)}
\example[0]{paṭhati (\paliroot{paṭha} + a + ti)}
\example{gacchati (\paliroot{gamu} + a + ti)}

\head{457}{457, 424. bhūvādayo dhātavo.}
\headtrans{\pali{Bhū} and so on [are called] \pali{dhātu} (root).}
\sutdef{bhūiccevamādayo ye saddagaṇā, te dhātusaññā honti.}
\sutdeftrans{Which groups of word such as \pali{bhū} and so on [exist], they are called \pali{dhātu} (root).}
\example[0]{bhavati, bhavanti (\paliroot{bhū} + a + [n]ti)}
\example[0]{carati, caranti (\paliroot{cara} + a + [n]ti)}
\example[0]{pacati, pacanti (\paliroot{paca} + a + [n]ti)}
\example[0]{cintayati, cintayanti (\paliroot{cinta} + ṇaya + [n]ti)}
\example[0]{hoti, honti (\paliroot{hū} + a + [n]ti)}
\example{gacchati, gacchanti (\paliroot{gamu} + a + [n]ti)}

