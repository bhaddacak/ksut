\markboth{}{Afterword}
\cleardoublepage
\phantomsection
\addcontentsline{toc}{chapter}{Afterword}
\chapter*{Afterword}\label{chap:afterword}

If the learners have read through the book to this point, they should come up with these observations:

\begin{itemize}
\item Grammatical system laid down by Kaccāyana is not complete. It supposes the existing knowledge of Sanskrit. Several things are left out because they are expected to be known in advance, such as prefixes and particles (\pali{upasagganipātā}).\footnote{This lack is fulfilled by Rūpasiddhi in \pali{Opasaggikapada} (before Rūpa 282). I also have been thinking about translating the whole book of Rūpasiddhi because it is really worth reading in full. I cannot promise anything yet at this time.} Moreover, the learners are supposed to know how to put fragmented grammatical units together to make them meaningful in certain way.\footnote{Learning Pāli by Pāli itself is an infeasible task. No one can do that. There must be a starter, either a primer textbook in vernacular languages or a learning class that the teacher can communicate effectively with students.}
\item Many rules are an adaptation, if not a direct taking, from Sanskrit grammars, mainly from Aṣṭādhyāyī and Kātantra. However, sometimes the author goes against Sanskrit grammars by proposing his own logic of analysis.
\item The author did not take the real use of terms in Pāli scriptures as primary consideration. So, we see a lot of unused terms, and we do not see terms worthy to analyze, like \pali{nibbāna}, \pali{arhanta}, etc.
\item Peculiarity in phrasing and inconsistency in analyses can be found occasionally. This may show that there were more than one author in charge, particularly in example part.
\item The arrangement of suttas could be better. This shows that the text was gradually accumulated until we see it in final form.
\item Recensions of the text from different maintainers can be different, slightly in the formula and description part, and sometimes substantially in example part.
\end{itemize}

If you read the text closely and try to translate it as I have done, you can find more points to add up. And those who have certain beliefs related to the text should read through it at least once before they assert their position.

Regardless of what conclusion you may reach, Kaccāyana's grammar is important to our understanding in Pāli. It should be studied after you have tackled the basic grammar well enough. Then this will reinforce the understanding considerably.

In conclusion, let me wrap up and reflect upon my writing. The first full release of this book has been finished in four months. The writing process is fast but longer than I expected. By the initial intention, I just want to make sense of the Kaccāyana's suttas, but obscure explanations direct me to Sanskrit grammars. That cost me extra time in dealing with Sanskrit also.

At first, I wanted to make a full translation of the text after finishing this essential translation. But I think it is not necessary now because the leftover can be easily read by the learners themselves after they are familiar with the overall style. And what I left out, if not too trivial to be mentioned, is mostly too quirky to be bothered with.

As you have seen, I always make myself visible as the translator/expositor endowed with prejudices and limited knowledge, unlike many translated works I have read which have godlike translators. For why the visibility of translator matters, see my \emph{Pāli Text Reading: A Handbook}.\footnote{\url{bhaddacak.github.io/ptr}}

One factor that helps me finish my work quickly in offline environment is my handy program, \textsc{Pāli\,Platform}, especially with the help from DPD. The reader should know and learn to use this also. But be very careful with DPD because it can easily mislead us, particularly when searching terms ending with `\pali{ti}.' We cannot be sure whether the terms found are verbs or just nouns with \pali{iti} elided. And when we find a term ending with \pali{ā}, it is not always a feminine noun or a kind of irregular formations or indeclinables. That may be the price we pay for a tight net, so we get a lot of tiny fish only to obscure a real big one.

After this I will spend some time enhance the program with some useful features, and, of course, with newly edited Kaccāyana, plus some other texts. When the program has its new release, I will pick Saddanīti, which closer to Kaccāyana in articulation than Moggallāna, into the process. That will be an exciting work to come.
