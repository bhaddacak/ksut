\section{Pañcamakaṇḍa}

\head{247}{247, 261. tvādayo vibhattisaññāyo.}
\headtrans{The \pali{to}[-paccaya] and so on [are] regarded as vibhatti.}
\sutdef{toādi yesaṃ paccayānaṃ, te honti tvādayo. te paccayā tvādayo vibhattisaññāva daṭṭhabbā.}
\sutdeftrans{[Those] such as \pali{to} and so on of which paccayas, these [paccayas] are \pali{to} and so on. These paccayas, such as \pali{to} and so on, should be seen to be regarded as vibhatti.}
\example[0]{sabbato = sabba + to}
\example[0]{yato = ya + to}
\example[0]{tato = ta + to}
\example[0]{kuto = kiṃ + to}
\example[0]{ato = ima + to}
\example[0]{ito = ima + to}
\example[0]{sabbadā = sabba + dā}
\example[0]{yadā = ya + dā}
\example[0]{tadā = ta + dā}
\example[0]{kadā = kiṃ + dā}
\example[0]{idha = ima + dha}
\example{idāni = ima + dāni}
\transnote{This can mean that when these paccayas are used, they also carry out the vibhatti function. I call terms with these paccayas `suffixed indeclinables.' Traditionally put, there are fifteen paccayas that behave in this way:}
\example{to, tra, tha, dhi, va, hiṃ, haṃ, ha, dha, hiñcanaṃ, dā, dācanaṃ, dāni, rahi, dhunā}

\head{248}{248, 260. kvaci to pañcamyatthe.}
\headtrans{In some places, \pali{to}[-paccaya] [is] in ablative sense.}
\sutdef{kvaci topaccayo hoti pañcamyatthe.}
\sutdeftrans{In some places, \pali{to}-paccaya is in ablative sense.}
\example[0]{sabbato {\upshape (from all)}}
\example[0]{yato {\upshape (from which)}}
\example[0]{tato {\upshape (from that)}}
\example[0]{kuto {\upshape (from where)}}
\example[0]{ato {\upshape (from here)}}
\example{ito {\upshape (from here)}}

\head{249}{249, 266. tratha sattamiyā sabbanāmehi.}
\headtrans{After pronouns, \pali{tra} and \pali{tha}[-paccaya] [are] in locative [sense].}
\sutdef{trathaiccete paccayā honti sattamyatthe sabbanāmehi.}
\sutdeftrans{After pronouns, \pali{tra}- and \pali{tha}-paccaya are in locative sense.}
\example[0]{sabbatra = sabbattha {\upshape (in all)}}
\example[0]{yatra = yattha {\upshape (in which)}}
\example{tatra = tattha {\upshape (in that)}}

\head{250}{250, 268. sabbato dhi.}
\headtrans{After \pali{sabba}, [there is] \pali{dhi}[-paccaya] [in locative sense].}
\sutdef{sabbaiccetasmā dhipaccayo hoti kvaci sattamyatthe.}
\sutdeftrans{In some places, there is \pali{dhi}-paccaya after \pali{sabba} in locative sense.}
\example{sabbadhi = sabbasmiṃ {\upshape (in all)}}

\head{251}{251, 269. kiṃsmā vo.}
\headtrans{After \pali{kiṃ}, [there is] \pali{va}[-paccaya] [in locative sense].}
\sutdef{kimiccetasmā vapaccayo hoti sattamyatthe.}
\sutdeftrans{After \pali{kiṃ}, there is \pali{va}-paccaya in locative sense.}
\example{kva = kiṃ + va {\upshape (where)}}
\transnote{For why \pali{kiṃ} becomes \pali{ka}, see ``kissa ka ve ca'' (\hyperref[sut:227]{Kacc 227}).}

\head{252}{252, 271. hiṃ haṃ hiñcanaṃ.}
\headtrans{[There are] \pali{hiṃ}, \pali{haṃ}, and \pali{hiñcanaṃ}[-paccaya] [after \pali{kiṃ} in locative sense].}
\sutdef{kimiccetasmā hiṃhaṃhiñcanaṃiccete paccayā honti sattamyatthe.}
\sutdeftrans{There are \pali{hiṃ}-, \pali{haṃ}-, and \pali{hiñcanaṃ}-paccaya after \pali{kiṃ} in locative sense.}
\example[0]{kuhiṃ = kiṃ + hiṃ {\upshape (where)}}
\example[0]{kuhaṃ = kiṃ + haṃ}
\example{kuhiñcanaṃ = kiṃ + hiñcanaṃ}
\transnote{For why \pali{kiṃ} becomes \pali{ku}, see ``ku hiṃhaṃsu ca'' \hyperref[sut:228]{Kacc 228}.}

\head{253}{253, 273. tamhā ca.}
\headtrans{Also [\pali{hiṃ} and \pali{haṃ}] after \pali{ta}.}
\sutdef{tamhā ca hiṃhaṃiccete paccayā honti sattamyatthe.}
\sutdeftrans{There are also \pali{hiṃ}- and \pali{haṃ}-paccaya after \pali{ta} in locative sense.}
\example[0]{tahiṃ = ta + hiṃ {\upshape (there, in that place)}}
\example{tahaṃ = ta + haṃ}

\head{254}{254, 274. imasmā hadhā ca.}
\headtrans{After \pali{ima}, [there are] \pali{ha} and \pali{dha}[-paccaya] also.}
\sutdef{imasmā hadhaiccete paccayā honti sattamyatthe.}
\sutdeftrans{There are \pali{ha}- and \pali{dha}-paccaya after \pali{ima} in locative sense.}
\example[0]{iha = ima + ha {\upshape (here)}}
\example{idha = ima + dha}

\head{255}{255, 275. yato hiṃ.}
\headtrans{After \pali{ya}, [there is] \pali{hiṃ}[-paccaya].}
\sutdef{tasmā yato hiṃpaccayo hoti sattamyatthe.}
\sutdeftrans{There is \pali{hiṃ}-paccaya after that \pali{ya} in locative sense.}
\example{yahiṃ = ya + hiṃ {\upshape (in which place)}}

\head{256}{256. kāle.}
\headtrans{In time.}
\sutdef{“kāle” iccetaṃ adhikāratthaṃ veditabbaṃ.}
\sutdeftrans{This ``In time'' [sutta] should be known as having meaning of governing rule.}
\transnote{This means the following suttas (257--260) are mainly concerning with time rather than location.}

\head{257}{257, 279. kiṃsabbaññekayakuhi dādācanaṃ.}
\headtrans{After \pali{kiṃ}, \pali{sabba}, \pali{añña}, \pali{eka}, \pali{ya}, and \pali{ku}, [there are] \pali{dā} and \pali{dācanaṃ}[-paccaya].}
\sutdef{kiṃsabbaaññaekayakuiccetehi dādācanaṃiccete paccayā honti kāle sattamyatthe.}
\sutdeftrans{There are \pali{dā}- and \pali{dācanaṃ}-paccaya after \pali{kiṃ}, \pali{sabba}, \pali{añña}, \pali{eka}, \pali{ya}, and \pali{ku} in temporal locative sense.}
\example[0]{kadā = kiṃ + dā {\upshape (when)}}
\example[0]{sabbadā = sabba + dā {\upshape (always)}}
\example[0]{aññadā = añña + dā {\upshape (at other time)}}
\example[0]{ekadā = eka + dā {\upshape (at one time, once)}}
\example[0]{yadā = ya + dā {\upshape (at which time)}}
\example{kudācanaṃ = ku + dācanaṃ {\upshape (at some time, ever)}}

\head{258}{258, 278. tamhā dāni ca.}
\headtrans{After \pali{ta}, [there are] \pali{dāni} [and \pali{dā}-paccaya] also.}
\sutdef{taiccetasmā dānidāiccete paccayā honti kāle sattamyatthe.}
\sutdeftrans{There are \pali{dāni}- and \pali{dā}-paccaya after \pali{ta} in temporal locative sense.}
\example[0]{tadāni = ta + dāni {\upshape (at that time)}}
\example{tadā = ta + dā}

\head{259}{259, 279. imasmā rahidhunādāni ca.}
\headtrans{After \pali{ima}, [there are] \pali{rahi}, \pali{dhunā}, and \pali{dāni}[-paccaya] also.}
\sutdef{imasmā rahidhunādāniiccete paccayā honti kāle sattamyatthe.}
\sutdeftrans{There are \pali{rahi}-, \pali{dhunā}-, and \pali{dāni}-paccaya after \pali{ima} in temporal locative sense.}
\example[0]{etarahi = ima + rahi {\upshape (at this time, now)}}
\example[0]{adhunā = ima + dhunā}
\example{idāni = ima + dāni}
\transnote{For why \pali{ima} becomes \pali{eta}, see \hyperref[sut:236]{Kacc 236}. For why \pali{ima} becomes \pali{a}, see \hyperref[sut:235]{Kacc 235}. And for why \pali{ima} becomes \pali{i}, see \hyperref[sut:234]{Kacc 234}.}

\head{260}{260, 277. sabbassa so dāmhi vā.}
\headtrans{For \pali{sabba}, [there is] \pali{sa}-substitution because of \pali{dā}[-paccaya] sometimes.}
\sutdef{sabbaiccetassa sakārādeso hoti vā dāmhi paccaye pare.}
\sutdeftrans{There is a substitution of \pali{sa} for \pali{sabba} sometimes because of \pali{dā}-paccaya behind.}
\example{sadā = sabba + dā (= sabbadā)}

\head{261}{261, 369. avaṇṇo ye lopañca.}
\headtrans{The \pali{a}-group [becomes] elided also because of \pali{ya}[-paccaya].}
\sutdef{avaṇṇo ye paccaye pare lopamāpajjate.}
\sutdeftrans{The \pali{a}-group [vowels] become elided because of \pali{ya}-paccaya behind.}
\example[0]{bāhussaccaṃ = bahussuta + ṇya (+ si)}
\example[0]{paṇḍiccaṃ = paṇḍita + ṇya (+ si)}
\example[0]{vepullaṃ = vipula + ṇya (+ si)}
\example[0]{kāruññaṃ = karuṇā + ṇya (+ si)}
\example[0]{kosallaṃ = kusala + ṇya (+ si)}
\example[0]{sāmaññaṃ = samaṇa + ṇya (+ si)}
\example{sohajjaṃ = suhada + ṇya (+ si)}
\transnote{This sutta is somewhat out of place. It mentions \pali{ya} in \pali{ṇya}-paccaya of secondary derivation. To understand the examples, you have to know how \pali{ṇ} and \pali{ya} work. Briefly put, \pali{ṇ} causes the vuddhi strength (\pali{ba} becomes \pali{bā} in the first example), and it disappears. Then \pali{ya} removes \pali{a} from the ending by this sutta, yielding \pali{tya} (in the first example), and it makes \pali{tya} become \pali{cca}. See also \hyperref[sut:269]{Kacc 269} below, For \pali{ṇya}-paccaya, see \hyperref[sut:360]{Kacc 360}. And for more on this, see Appendix I in PNL.}

\head{262}{262, 391. vuḍḍhassa jo iyiṭṭhesu.}
\headtrans{For \pali{vuḍḍha}, [there is] \pali{ja}-substitution because of \pali{iya} and \pali{iṭṭha}[-paccaya].}
\sutdef{sabbasseva vuḍḍhasaddassa joādeso hoti iyaiṭṭhaiccetesu paccayesu.}
\sutdeftrans{There is a substitution of \pali{ja} for the whole \pali{vuḍḍha} because of \pali{iya}- and \pali{iṭṭha}-paccaya.}
\example[0]{jeyyo = vuḍḍha + iya + si {\upshape (older)}}
\example{jeṭṭho = vuḍḍha + iṭṭha + si {\upshape (oldest)}}

\head{263}{263, 392. pasatthassa so ca.}
\headtrans{For \pali{pasattha}, [there are] also \pali{sa} [and \pali{ja}-substitution because of \pali{iya} and \pali{iṭṭha}].}
\sutdef{sabbasseva pasatthasaddassa soādeso hoti, jādeso ca iyaiṭṭhaiccetesu paccayesu.}
\sutdeftrans{There is a substitution of \pali{sa}, as well as \pali{ja}, for the whole \pali{pasattha} because of \pali{iya}- and \pali{iṭṭha}-paccaya.}
\example[0]{seyyo = pasattha + iya + si {\upshape (better)}}
\example[0]{seṭṭho = pasattha + iṭṭha + si {\upshape (best)}}
\example[0]{jeyyo = pasattha + iya + si {\upshape (better)}}
\example{jeṭṭho = pasattha + iṭṭha + si {\upshape (best)}}
\transnote{From this sutta and the previous one, we know that \pali{jeyya} and \pali{jeṭṭha} have two denotations.}

\head{264}{264, 393. antikassa nedo.}
\headtrans{For \pali{antika}, [there is] \pali{neda}-substitution.}
\sutdef{sabbassa antikasaddassa nedādeso hoti iyaiṭṭhaiccetesu paccayesu.}
\sutdeftrans{There is a substitution of \pali{neda} for the whole \pali{antika} because of \pali{iya}- and \pali{iṭṭha}-paccaya.}
\example[0]{nediyo = antika + iya + si {\upshape (nearer)}}
\example{nediṭṭho = antika + iṭṭha + si {\upshape (nearest)}}

\head{265}{265, 394. bāḷhassa sādho.}
\headtrans{For \pali{bāḷha}, [there is] \pali{sādha}-substitution.}
\sutdef{sabbassa bāḷhasaddassa sādhādeso hoti iyaiṭṭhaiccetesu paccayesu.}
\sutdeftrans{There is a substitution of \pali{sādha} for the whole \pali{bāḷha} because of \pali{iya}- and \pali{iṭṭha}-paccaya.}
\example[0]{sādhiyo = bāḷha + iya + si {\upshape (stronger)}}
\example{sādhiṭṭho = bāḷha + iṭṭha + si {\upshape (strongest)}}

\head{266}{266, 395. appassa kaṇa.}
\headtrans{For \pali{appa}, [there is] \pali{kaṇa}-substitution.}
\sutdef{sabbassa appasaddassa kaṇaādeso hoti iyaiṭṭhaiccetesu paccayesu.}
\sutdeftrans{There is a substitution of \pali{kaṇa} for the whole \pali{appa} because of \pali{iya}- and \pali{iṭṭha}-paccaya.}
\example[0]{kaṇiyo = appa + iya + si {\upshape (less)}}
\example{kaṇiṭṭho = appa + iṭṭha + si {\upshape (least)}}

\head{267}{267, 396. yuvānañca.}
\headtrans{For \pali{yuva} also.}
\sutdef{sabbassa yuvasaddassa kaṇaādeso hoti iyaiṭṭhaiccetesu paccayesu.}
\sutdeftrans{There is a substitution of \pali{kaṇa} for the whole \pali{yuva} because of \pali{iya}- and \pali{iṭṭha}-paccaya.}
\example[0]{kaṇiyo = yuva + iya + si {\upshape (younger)}}
\example{kaṇiṭṭho = yuva + iṭṭha + si {\upshape (youngest)}}
\transnote{This and the previous sutta show that the terms also have two denotations.}

\head{268}{268, 397. vantumantuvīnañca lopo.}
\headtrans{[There is] elision of \pali{vantu}, \pali{mantu}, and \pali{vī}[-paccaya] [because of \pali{iya}- and \pali{iṭṭha}-paccaya].}
\sutdef{vantumantuvīiccetesaṃ paccayānaṃ lopo hoti iyaiṭṭhaiccetesu paccayesu.}
\sutdeftrans{There is elision of \pali{vantu}-, \pali{mantu}-, and \pali{vī}-paccaya because of \pali{iya}- and \pali{iṭṭha}-paccaya.}
\example[0]{guṇiyo = guṇavantu + iya + si {\upshape (more virtuous)}}
\example[0]{guṇiṭṭho = guṇavantu + iṭṭha + si {\upshape (most virtuous)}}
\example[0]{satiyo = satimantu + iya + si {\upshape (more mindful)}}
\example[0]{satiṭṭho = satimantu + iṭṭha + si {\upshape (most mindful)}}
\example[0]{medhiyo = medhāvī+ iya + si {\upshape (wiser)}}
\example{medhiṭṭho = medhāvī+ iṭṭha + si {\upshape (wisest)}}

\head{269}{269, 401. yavataṃ talaṇadakārānaṃ byañjanāni calañajakārattaṃ.}
\headtrans{Having \pali{ya} behind, \pali{ta}-, \pali{la}-, \pali{ṇa}-, and \pali{da}-consonant [become] \pali{ca}, \pali{la}, \pali{ña}, and \pali{ja} [respectively].}
\sutdef{yakāravantānaṃ talaṇadakārānaṃ byañjanāni calañajakārattamāpajjante yathāsaṅkhyaṃ.}
\sutdeftrans{Having \pali{ya} behind, \pali{ta}-, \pali{la}-, \pali{ṇa}-, and \pali{da}-consonant become \pali{ca}, \pali{la}, \pali{ña}, and \pali{ja} respectively.}
\example[0]{bāhussaccaṃ = bahussuta + ṇya (+ si)}
\example[0]{paṇḍiccaṃ = paṇḍita + ṇya (+ si)}
\example[0]{vepullaṃ = vipula + ṇya (+ si)}
\example[0]{kāruññaṃ = karuṇā + ṇya (+ si)}
\example[0]{kosallaṃ = kusala + ṇya (+ si)}
\example[0]{nepuññaṃ = nipuṇa + ṇya (+ si)}
\example[0]{sāmaññaṃ = samaṇa + ṇya (+ si)}
\example{sohajjaṃ = suhada + ṇya (+ si)}
\transnote{This sutta helps us understand \hyperref[sut:261]{Kacc 261} better.}

\head{270}{270,~120.~amhatumhanturājabrahmattasakhasatthupitā\-dīhi smā nāva.}
\headtrans{The \pali{smā}[-vibhatti] after \pali{amha}, \pali{tumha}, \pali{ntu}, \pali{rāja}, \pali{brahma}, \pali{atta}, \pali{sakha}, \pali{satthu}, \pali{pitu}, and so on [should be seen like] \pali{nā}[-vibhatti].}
\sutdef{amhatumhanturājabrahmaattasakhasatthupituiccevamādīhi smāvacanaṃ nāva daṭṭhabbaṃ.}
\sutdeftrans{The \pali{smā}-vibhatti after \pali{amha}, \pali{tumha}, \pali{ntu}, \pali{rāja}, \pali{brahma}, \pali{atta}, \pali{sakha}, \pali{satthu}, \pali{pitu}, and so on should be seen like \pali{nā}-vibhatti.}
\example[0]{mayā = amha + smā {\upshape (from me)}}
\example[0]{tayā = tumha + smā {\upshape (from you)}}
\example[0]{guṇavatā = guṇavantu + smā {\upshape (from virtuous one)}}
\example[0]{raññā = rāja + smā {\upshape (from the king)}}
\example[0]{brahmunā = brahma + smā {\upshape (from the Brahman)}}
\example[0]{attanā = atta + smā {\upshape (from self)}}
\example[0]{sakhinā = sakha + smā {\upshape (from friend)}}
\example[0]{satthārā = satthu + smā {\upshape (from master)}}
\example[0]{pitarā = pitu + smā {\upshape (from father)}}
\example[0]{mātarā = mātu + smā {\upshape (from mother)}}
\example[0]{bhātarā = bhātu + smā {\upshape (from brother)}}
\example[0]{dhītarā = dhītu + smā {\upshape (from daughter)}}
\example[0]{kattārā = kattu + smā {\upshape (from the doer)}}
\example{vattārā = vattu + smā {\upshape (from the speaker)}}

