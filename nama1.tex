\chapter{Nāma}

The second book of Kaccāyana is on nouns (\pali{nāma}). Precisely, this includes pronouns and adjectives as well. The main part of the content is about how to decline and recognize words in various cases. This book is not well organized. You may need to jump around to make the pictures cohesive.

The first section, Sutta 52--119, starts with the general ideas about nominal base and vibhattis. Then some aliases are defined to be used subsequently. Toward the end, numerous cases of declensions are explained, mostly on nouns.

The second section, Sutta 120--160, is mostly on declensions of pronouns, such as \pali{tumha}, \pali{amha}, and some numbers. There are also treatments for some irregular nouns, such as those formed by \pali{ntu}-paccaya, \pali{rāja}, \pali{puma}, etc.

The third section, Sutta 161--210, continues on declensions of pronouns and some irregular nouns, such as the \pali{mano}-group, \pali{brahma}, \pali{atta}, \pali{sakha}, \pali{rāja}, \pali{satthu}, \pali{pitu}, etc. 

The forth section, Sutta 211--246, continue on declensions of nouns and pronouns, with a few treatments of some derivations.

The fifth section, Sutta 247--270, continue on declensions. Several paccayas are also brought into play.

\section{Paṭhamakaṇḍa}

\head{52}{52, 60. jinavacanayuttaṃ hi.}
\headtrans{Applicable to the teaching of the Buddha.}
\sutdef{“jinavacanayuttaṃ hi” iccetaṃ adhikāratthaṃ veditabbaṃ.}
\sutdeftrans{This ``Applicable to the teaching of the Buddha'' [sutta] should be known as having meaning of governing rule.}
\transnote{The recursive definition makes this sutta sounds cryptic. An explanation of this is the suttas before this as well as the suttas to come are applicable to the teaching.}

\head{53}{53, 61. liṅgañca nippajjate.}
\headtrans{The nominal base is laid out [suitably for the teaching].}
\sutdef{yathā yathā jinavacanayuttaṃ hi liṅgaṃ, tathā tathā idha liṅgañca nippajjate.}
\sutdeftrans{In whichever way the nominal base is applicable to the teaching of the Buddha, in such a way the nominal base is laid out here.}
\example{eso no satthā, brahmā, attā, sakhā, rājā}
\transnote{Thitzana translates \pali{liṅga} here as genders.\footnote{\citealp[pp.~183--4]{thitzana:kacc2}} It should be read as nominal base,\footnote{See \pali{liṅga}~4 in DPD.} the sound (\pali{sadda}) before it has a meaning by appending a vibhatti.\footnote{See Chapter 17 of PNL.} From the examples given, we can say that the way these irregular forms are given in the suttas to come corresponds to their uses in the texts. This can also show that these forms are archaic and really in use in the past.}

\head{54}{54, 62. tato ca vibhattiyo.}
\headtrans{[There are] vibhattis after those [nominal bases].}
\sutdef{tato jinavacanayuttehi liṅgehi vibhattiyo parā honti.}
\sutdeftrans{There are vibhattis after those nominal bases applicable to the teaching of the Buddha.}
\transnote{It is better not to translate \pali{vibhatti}. So, we use it as a technical term. A vibhatti is more than a suffix. It is an operator that can change nominal bases in various ways as we shall see in due course.}

\head{55}{55, 63. si yo, aṃ yo, nā hi, sa naṃ, smā hi, sa naṃ, smiṃ su.}
\headtrans{[The vibhattis are] \pali{si yo, aṃ yo, nā hi, sa naṃ, smā hi, sa naṃ, smiṃ su.}}
\sutdef{kā ca pana tāyo vibhattiyo? si yo iti paṭhamā, aṃ yo iti dutiyā, nā hi iti tatiyā, sa naṃ iti catutthī, smā hi iti pañcamī, sa naṃ iti chaṭṭhī, smiṃ su iti sattamī.}
\sutdeftrans{What are these vibhattis? \pali{Si yo} [are] nominative, \pali{aṃ yo} [are] accusative, \pali{nā hi} [are] instrumental, \pali{sa naṃ} [are] dative, \pali{smā hi} [are] ablative, \pali{sa naṃ} [are] genitive, \pali{smiṃ su} [are] locative.}
\transnote{Vibhattis come in pairs (for singular and plural nouns) for seven cases plus vocative (\pali{ālapana}). All these are summarized in Table \ref{tab:vibhatti}.}

\begin{table}[!hbt]
\centering
\caption{Vibhattis for nouns}
\label{tab:vibhatti}
\bigskip
\begin{tabular}{ll*{2}{>{\itshape}l}} \toprule
\multicolumn{2}{c}{\textbf{Cases}} & \bfseries\upshape Singular & \bfseries\upshape Plural \\ \midrule
paṭhamā & nominative (nom.) & si & yo \\
dutiyā & accusative (acc.) & aṃ & yo \\
tatiyā & instrumental (ins.) & nā & hi \\
catutthī & dative (dat.) & sa & naṃ \\
pañcamī & ablative (abl.) & smā & hi \\
chaṭṭhī & genitive (gen.) & sa & naṃ \\
sattamī & locative (loc.) & smiṃ & su \\
ālapana & vocative (voc.) & si & yo \\
\bottomrule
\end{tabular}
\end{table}

\head{56}{56, 64. tadanuparodhena.}
\headtrans{By the way not going against that [teaching of the Buddha].}
\sutdef{yathā yathā tesaṃ jinavacanānaṃ anuparodho. tathā tathā idha liṅgañca nippajjate.}
\sutdeftrans{In whichever way not going against those teachings of the Buddha, in such a way the nominal base is laid out here.}
\transnote{We can break down the formula to \pali{taṃ + ana + uparodhena}.}

\head{57}{57, 71. ālapane si ga sañño.}
\headtrans{In vocative case \pali{si} [is] named \pali{ga}.}
\sutdef{ālapanatthe si gasañño hoti.}
\sutdeftrans{In the meaning of vocative case, \pali{si}-vibhatti is named \pali{ga}.}
\example[0]{ayye = ayyā + si}
\example[0]{kaññe = kaññā + si}
\example{kharādiye = kharādiyā + si}
\transnote{In the three examples, they are used to address a woman, like Madam, for example. This sutta says that if \pali{si} is used in locative case, it will be called \pali{ga}. But we will never see both \pali{si} and \pali{ga} because they get deleted by ``\pali{sesato lopaṃ gasipi}'' (\hyperref[sut:220]{Kacc 220}). We can see this as a kind of making an alias for subsequent uses. Table \ref{tab:alias} below shows aliases used here. Applications of \pali{ga}-alias can be found in \hyperref[sut:113]{Kacc 113}, \hyperref[sut:126]{126}, \hyperref[sut:243]{243}, for example.}

\begin{table}[!hbt]
\centering
\caption{Aliases for nouns}
\label{tab:alias}
\bigskip
\begin{tabular}{>{\itshape}llc} \toprule
\bfseries\upshape Alias & \bfseries\upshape Description & \bfseries\upshape Sutta \\ \midrule
ga & vocative \pali{si}-vibhatti & \hyperref[sut:57]{57} \\
jha & masculine bases ending with \pali{i} or \pali{ī} & \hyperref[sut:58]{58} \\
la & masculine bases ending with \pali{u} or \pali{ū} & \hyperref[sut:58]{58} \\
pa & feminine bases ending with \pali{i}, \pali{ī}, \pali{u}, or \pali{ū} & \hyperref[sut:59]{59} \\
gha & feminine bases ending with \pali{ā} & \hyperref[sut:60]{60} \\
\bottomrule
\end{tabular}
\end{table}

\head{58}{58, 29. ivaṇṇuvaṇṇā jhalā.}
\headtrans{[Vowels] in \pali{i}-group and \pali{u}-group [are] \pali{jha} and \pali{la}.}
\sutdef{ivaṇṇuvaṇṇāiccete jhalasaññā honti yathāsaṅkhyaṃ.}
\sutdeftrans{[Vowels] in [masculine] \pali{i}-group and \pali{u}-group are named \pali{jha} and \pali{la} respectively.}
\example[0]{isino = isi + sa}
\example[0]{aggino = aggi + sa}
\example[0]{gahapatino = gahapatī + sa}
\example[0]{daṇḍino = daṇḍī + sa}
\example[0]{setuno = setu + sa}
\example[0]{ketuno = ketu + sa}
\example[0]{bhikkhuno = bhikkhu + sa}
\example[0]{sayambhuno = sayambhū + sa}
\example{abhibhuno = abhibhū + sa}
\transnote{The examples show singular dative and genitive case of masculine terms ending with \pali{i}-group and \pali{u}-group. Applications of \pali{jha}- and \pali{la}-alias can be seen in \hyperref[sut:70]{Kacc 70}, \hyperref[sut:71]{71}, \hyperref[sut:82]{82}, \hyperref[sut:96]{96}, \hyperref[sut:97]{97}, \hyperref[sut:116]{116}, \hyperref[sut:117]{117}, \hyperref[sut:119]{119}, \hyperref[sut:215]{215}, \hyperref[sut:224]{224}, \hyperref[sut:245]{245}, for example.}

\head{59}{59, 182. te itthikhyā po.}
\headtrans{Those [\pali{i}-group and \pali{u}-group are named] \pali{pa}, [when] signifying feminine nouns.}
\sutdef{te ivaṇṇuvaṇṇā yadā itthikhyā, tadā pasaññā honti.}
\sutdeftrans{When those [vowels] in \pali{i}-group and \pali{u}-group [are] signifiers of feminine nouns, then they are named \pali{pa}.}
\example[0]{rattiyā = ratti + nā}
\example[0]{itthiyā = itthī + nā}
\example[0]{dhenuyā = dhenu + nā}
\example{vadhuyā = vadhū + nā}
\transnote{Applications of this \pali{pa}-alias can be seen in \hyperref[sut:68]{Kacc 68}, \hyperref[sut:72]{72}, \hyperref[sut:82]{82}, \hyperref[sut:112]{112}, \hyperref[sut:179]{179}, \hyperref[sut:216]{216}, \hyperref[sut:223]{223}, \hyperref[sut:245]{245}, for example.}

\head{60}{60, 177. ā gho.}
\headtrans{[The signifier] \pali{ā} [is named] \pali{gha}.}
\sutdef{ākāro yadā itthikhyo, tadā ghasañño hoti.}
\sutdeftrans{When \pali{ā} [is] a signifier of feminine nouns, then they are named \pali{gha}.}
\example[0]{saddhāya = saddhā + nā}
\example[0]{kaññāya = kaññā + nā}
\example[0]{vīṇāya = vīṇā + nā}
\example[0]{gaṅgāya = gaṅgā + nā}
\example[0]{disāya = disā + nā}
\example[0]{sālāya = sālā + nā}
\example[0]{mālāya = mālā + nā}
\example[0]{tulāya = tulā + nā}
\example[0]{dolāya = dolā + nā}
\example[0]{pabhāya = pabhā + nā}
\example[0]{sobhāya = sobhā + nā}
\example[0]{paññāya = paññā + nā}
\example[0]{karuṇāya = karuṇā + nā}
\example[0]{nāvāya = nāvā + nā}
\example{kapālikāya = kapālikā + nā}
\transnote{Applications of \pali{gha}-alias can be seen in \hyperref[sut:66]{Kacc 66}, \hyperref[sut:84]{84}, \hyperref[sut:111]{111}, \hyperref[sut:114]{114}, \hyperref[sut:179]{179}, \hyperref[sut:216]{216}, for example.}

\head{61}{61, 86. sāgamo se.}
\headtrans{Insertion of \pali{sa} in \pali{sa}[-vibhatti].}
\sutdef{sakārāgamo hoti se vibhattimhi.}
\sutdeftrans{There is \pali{sa}-insertion in \pali{sa}-vibhatti.}
\example[0]{purisassa = purisa + sa}
\example[0]{aggissa = aggi + sa}
\example[0]{isissa = isi + sa}
\example[0]{daṇḍissa = daṇḍī + sa}
\example[0]{bhikkhussa = bhikkhu + sa}
\example[0]{sayambhussa = sayambhū + sa}
\example{abhibhussa = abhibhū + sa}

\head{62}{62, 206. saṃsāsvekavacanesu ca.}
\headtrans{Also [\pali{sa}-insertion] because of singular \pali{saṃ} and \pali{sā}.}
\sutdef{saṃsāsu ekavacanesu vibhattādesesu sakārāgamo hoti.}
\sutdeftrans{There is \pali{sa}-insertion because of singular \pali{saṃ} and \pali{sā} produced by vibhatti-operation.}
\example[0]{etissaṃ = etā + smiṃ}
\example[0]{etissā = etā + sa}
\example[0]{imissaṃ = imā + smiṃ}
\example[0]{imissā = imā + sa}
\example[0]{tissaṃ = tā + smiṃ}
\example[0]{tissā = tā + sa}
\example[0]{tassaṃ = tā + smiṃ}
\example[0]{tassā = tā + sa}
\example[0]{yassaṃ = yā + smiṃ}
\example[0]{yassā = yā + sa}
\example[0]{amussaṃ = amu + smiṃ}
\example{amussā = amu + sa}
\transnote{For why \pali{smiṃ} becomes \pali{saṃ} and \pali{sa} becomes \pali{sā}, see ``ghapato smiṃsānaṃ saṃsā'' (\hyperref[sut:179]{Kacc 179}).}

\head{63}{63, 217. etimāsami.}
\headtrans{[The ending of] \pali{etā} and \pali{imā} [becomes] \pali{i} [because of] \pali{saṃ}.}
\sutdef{etāimāiccetesamanto saro ikāro hoti saṃsāsu ekavacanesu vibhattādesesu.}
\sutdeftrans{The ending vowel of \pali{etā} and \pali{imā} becomes \pali{i} because of singular \pali{saṃ} and \pali{sā} produced by vibhatti-operation.}
\example[0]{etissaṃ = etā + smiṃ}
\example[0]{etissā = etā + sa}
\example[0]{imissaṃ = imā + smiṃ}
\example{imissā = imā + sa}

\head{64}{64, 216. tassā vā.}
\headtrans{Sometimes [it happens to] \pali{tā}.}
\sutdef{tassā itthiyaṃ vattamānassa antassa ākārassa ikāro hoti vā saṃsāsu ekavacanesu vibhattādesesu.}
\sutdeftrans{When the \pali{ā}-ending of \pali{tā} exists in feminine [nouns], [it] becomes \pali{i} sometimes because of singular \pali{saṃ} and \pali{sā} produced by vibhatti-operation.}
\example[0]{tissaṃ = tā + smiṃ}
\example{tissā = tā + sa}

\head{65}{65, 215. tato sassa ssāya.}
\headtrans{After \pali{tā}, [there is] \pali{ssāya} for \pali{sa}[-vibhatti].}
\sutdef{tato tā etā imāto sassa vibhattissa ssāyādeso hoti vā.}
\sutdeftrans{There is a substitution of \pali{ssāya} for \pali{sa}-vibhatti after \pali{tā, etā, } and \pali{imā} sometimes.}
\example[0]{tissāya = tā + sa}
\example[0]{etissāya = etā + sa}
\example{imissāya = imā + sa}

\head{66}{66, 205. gho rassaṃ.}
\headtrans{The \pali{gha}-alias [becomes] shortened.}
\sutdef{gho rassamāpajjate saṃsāsu ekavacanesu vibhattādesesu.}
\sutdeftrans{The \pali{gha}-alias (feminine \pali{ā}) becomes shortened because of singular \pali{saṃ} and \pali{sā} produced by vibhatti-operation.}
\example[0]{tassaṃ = tā + smiṃ}
\example[0]{tassā = tā + sa}
\example[0]{yassaṃ = yā + smiṃ}
\example[0]{yassā = yā + sa}
\example[0]{sabbassaṃ = sabbā + smiṃ}
\example{sabbassā = sabbā + sa}
\transnote{For the definition of \pali{gha}-alias, see \hyperref[sut:60]{Kacc 60}.}

\head{67}{67, 229. no ca dvādito naṃmhi.}
\headtrans{[There is] \pali{na}-insertion after \pali{dvi} and so on because of \pali{naṃ}.}
\sutdef{dviiccevamādito saṅkhyāto nakārāgamo hoti naṃmhi vibhattimhi.}
\sutdeftrans{There is an insertion of \pali{na} after numbers, such as \pali{dvi} and so on, because of \pali{naṃ}-vibhatti.}
\example[0]{dvinnaṃ = dvi + naṃ}
\example[0]{tinnaṃ = ti + naṃ}
\example[0]{catunnaṃ = catu + naṃ}
\example[0]{pañcannaṃ = pañca + naṃ}
\example[0]{channaṃ = cha + naṃ}
\example[0]{sattannaṃ = satta + naṃ}
\example[0]{aṭṭhannaṃ = aṭṭha + naṃ}
\example[0]{navannaṃ = nava + naṃ}
\example{dasannaṃ = dasa + naṃ}

\head{68}{68, 184. amā pato smiṃsmānaṃ vā.}
\headtrans{[There are] \pali{aṃ}- and \pali{ā}-substitution after \pali{pa}-alias for \pali{smiṃ} and \pali{smā} sometimes.}
\sutdef{paiccetasmā smiṃsmāiccetesaṃ aṃāādesā honti vā yathāsaṅkhyaṃ.}
\sutdeftrans{There are substitutions of \pali{aṃ} and \pali{ā} for \pali{smiṃ}- and \pali{smā}-vibhatti respectively after \pali{pa}-alias (feminine \pali{i-} and \pali{u-}group) sometimes.}
\example[0]{matyaṃ = mati + smiṃ}
\example[0]{matiyaṃ = mati + smiṃ}
\example[0]{matyā = mati + smā}
\example[0]{matiyā = mati + smā}
\example[0]{nikatyaṃ = nikati + smiṃ}
\example[0]{nikatiyaṃ = nikati + smiṃ}
\example[0]{nikatyā = nikati + smā}
\example[0]{nikatiyā = nikati + smā}
\example[0]{vikatyaṃ = vikati + smiṃ}
\example[0]{vikatiyaṃ = vikati + smiṃ}
\example[0]{vikatyā = vikati + smā}
\example[0]{vikatiyā = vikati + smā}
\example[0]{viratyaṃ = virati + smiṃ}
\example[0]{viratiyaṃ = virati + smiṃ}
\example[0]{viratyā = virati + smā}
\example[0]{viratiyā = virati + smā}
\example[0]{ratyaṃ = rati + smiṃ}
\example[0]{ratiyaṃ = rati + smiṃ}
\example[0]{ratyā = rati + smā}
\example[0]{ratiyā = rati + smā}
\example[0]{puthabyaṃ = puthavi + smiṃ}
\example[0]{puthaviyaṃ = puthavi + smiṃ}
\example[0]{puthabyā = puthavi + smā}
\example[0]{puthaviyā = puthavi + smā}
\example[0]{pavatyaṃ = pavatti + smiṃ}
\example[0]{pavatyā = pavatti + smā}
\example[0]{pavattiyaṃ = pavatti + smiṃ}
\example{pavattiyā = pavatti + smā}
\transnote{For the definition of \pali{pa}-alias, see \hyperref[sut:59]{Kacc 59}. For \pali{i} becomes \pali{ya}, see ``pasaññassa ca'' (\hyperref[sut:72]{Kacc 72}).}

\head{69}{69, 186. ādito o ca.}
\headtrans{After \pali{ādi}, [there is] \pali{o}-substitution also.}
\sutdef{ādiiccetasmā smiṃvacanassa aṃoādesā honti vā.}
\sutdeftrans{There are substitutions of \pali{aṃ} and \pali{o} for \pali{smiṃ}-vibhatti after \pali{ādi}.}
\example[0]{ādiṃ = ādi + smiṃ}
\example{ādo = ādi + smiṃ}
\transnote{Here, \pali{smiṃvacanassa} is equivalent to \pali{smiṃvibhattissa}. We will see similar uses quite frequently in the course.}

\head{70}{70, 30. jhalānamiyuvā sare vā.}
\headtrans{[There are] \pali{iya}- and \pali{uva}-substitution for \pali{jha}- and \pali{la}-alias because of vowel sometimes.}
\sutdef{jhalaiccetesaṃ iya uvaiccete ādesā honti vā sare pare yathāsaṅkhyaṃ.}
\sutdeftrans{There are substitutions of \pali{iya} and \pali{uva} for \pali{jha}-alias (masculine \pali{i}-group) and \pali{la}-alias (masculine \pali{u}-group) respectively sometimes because of the other vowel.}
\example[0]{tiyantaṃ = ti + antaṃ}
\example[0]{pacchiyāgāre = pacchi + āgāre}
\example[0]{aggiyāgāre = aggi + āgāre}
\example[0]{bhikkhuvāsane = bhikkhu + āsane}
\example{puthuvāsane = puthu + āsane}
\transnote{The definition of \pali{jha}- and \pali{la}-alias is in \hyperref[sut:58]{Kacc 58}.}

\head{71}{71, 505. yavakārā ca.}
\headtrans{Also \pali{ya}- and \pali{va}-substitution [for \pali{jha}- and \pali{la}-alias].}
\sutdef{jhalānaṃ yakāra vakārādesā honti sare pare yathāsaṅkhyaṃ.}
\sutdeftrans{There are substitutions of \pali{ya} and \pali{va} for \pali{jha}- and \pali{la}-alias respectively because of the other vowel.}
\example[0]{agyāgāraṃ = aggi + āgāraṃ}
\example[0]{cakkhvāyatanaṃ = cakkhu + āyatanaṃ}
\example{svāgataṃ = su + āgataṃ}
\transnote{Why does double \pali{gg} become single \pali{g}? We can see similar cases quite often. Some suggest ``byañjano ca visaññogo'' (\hyperref[sut:41]{Kacc 41}), but its definition seems unfit here.}

\head{72}{72, 185. pasaññassa ca.}
\headtrans{Also [\pali{ya}-substitution] for \pali{pa}-alias (only \pali{i}).}
\sutdef{pasaññassa ca ivaṇṇassa vibhattādese sare pare yakārādeso hoti.}
\sutdeftrans{There is a substitution of \pali{ya} for \pali{i}-group of \pali{pa}-alias (feminine \pali{i} and \pali{u}) because of the other vowel produced by vibhatti-operation.}
\example[0]{puthabyā = puthavī + smā}
\example[0]{ratyā = ratti + smā}
\example{matyā = mati + smā}
\transnote{For the definition of \pali{pa}-alias, see \hyperref[sut:59]{Kacc 59}.}

\head{73}{73, 174. gāva se.}
\headtrans{Substitution of \pali{āva} for \pali{go} because of \pali{sa}[-vibhatti].}
\sutdef{goiccetassa okārassa āvādeso hoti se vibhattimhi.}
\sutdeftrans{There is a substitution of \pali{āva} for \pali{o} of \pali{go} because of \pali{sa}-vibhatti.}
\example{gāvassa = go + sa}

\head{74}{74, 169. yosu ca.}
\headtrans{Also [\pali{āva}-substitution] for \pali{yo}[-vibhattis].}
\sutdef{goiccetassa okārassa āvādeso hoti yoiccetesu paresu.}
\sutdeftrans{There is a substitution of \pali{āva} for \pali{o} of \pali{go} because of \pali{yo}-vibhattis.}
\example[0]{gāvo = go + yo}
\example{gāvī = go + yo}

\head{75}{75, 170. avamhi ca.}
\headtrans{Also \pali{ava}-substitution because of \pali{aṃ}[-vibhatti].}
\sutdef{goiccetassa okārassa āvaavaiccete ādesā honti aṃmhi vibhattimhi.}
\sutdeftrans{There are substitutions of \pali{āva} and \pali{ava} for \pali{o} of \pali{go} because of \pali{aṃ}-vibhatti.}
\example[0]{gāvaṃ = go + aṃ}
\example{gavaṃ = go + aṃ}

\head{76}{76, 171. āvassu vā.}
\headtrans{Sometimes \pali{u}-substitution for \pali{āva}.}
\sutdef{āvaiccetassa gāvādesassa antasarassa ukārādeso hoti vā aṃmhi vibhattimhi.}
\sutdeftrans{There is a substitution of \pali{u} for the ending vowel of \pali{gāva}, i.e. \pali{āva}, sometimes because of \pali{aṃ}-vibhatti.}
\example{gāvuṃ = go + aṃ}

\head{77}{77, 175. tato namaṃ patimhālutte ca samāse.}
\headtrans{After that [\pali{go}], [there is] \pali{aṃ}-substitution for \pali{naṃ}[-vibhatti] because of the following \pali{pati} of unelided compound also.}
\sutdef{tato gosaddato naṃvacanassa aṃādeso hoti, goiccetassa okārassa avādeso hoti patimhi pare alutte ca samāse.}
\sutdeftrans{There is a substitution of \pali{aṃ} for \pali{naṃ}-vibhatti after that word \pali{go}. There is also a substitution of \pali{ava} for \pali{o} of \pali{go} because of the following \pali{pati} of unelided compound.}
\example{gavaṃpati = go + naṃ + pati}
\transnote{This is another sutta trying to treat a single instance.}

\head{78}{78, 3. o sare ca.}
\headtrans{Also [\pali{ava}-substitution] for \pali{o} because of vowel.}
\sutdef{goiccetassa okārassa avādeso hoti samāse ca sare pare.}
\sutdeftrans{There is a substitution of \pali{ava} for \pali{o} of \pali{go} because of compound and the other vowel.}
\example[0]{gavassakaṃ = go + assakaṃ}
\example[0]{gaveḷakaṃ = go + eḷakaṃ}
\example{gavājinaṃ = go + ajinaṃ}

\head{79}{79, 46. tabbiparītūpapade byañjane ca.}
\headtrans{[There is] a reversal of that [\pali{ava}] when [nearby] a certain term because of consonant.}
\sutdef{tassa avasaddassa yadā upapade tiṭṭhamānassa tassa okārassa viparīto hoti byañjane pare.}
\sutdeftrans{There is a reversal of \pali{o} [to \pali{u}] when \pali{ava} [which is changed from that \pali{o}] stands nearby a certain term because of the other consonant.}
\example[0]{uggate = ava + gate}
\example[0]{uggacchati = ava + gacchati}
\example{uggahetvā = ava + gahetvā}
\transnote{We can break \pali{tabbiparītūpapade} in the formula to \pali{ta + viparīto + upapade}. The definition is hard to translate, but the examples make it clearer. Thitzana reads \pali{upapada} as \pali{upasagga}. I follow Thai translations here, which make better sense for me.}

\head{80}{80, 173. goṇa naṃmhi vā.}
\headtrans{[There is] \pali{goṇa}-substitution because of \pali{naṃ}[-vibhatti] sometimes.}
\sutdef{sabbasseva gosaddassa goṇādeso hoti vā naṃmhi vibhattimhi.}
\sutdeftrans{There is a substitution of \pali{goṇa} for the whole \pali{go} sometimes because of \pali{naṃ}-vibhatti.}
\example{goṇānaṃ = go + naṃ}

\head{81}{81, 172. suhināsu ca.}
\headtrans{Also because of \pali{su}-, \pali{hi}-, and \pali{nā}[-vibhatti].}
\sutdef{suhināiccetesu sabbassa gosaddassa goṇādeso hoti vā.}
\sutdeftrans{There is a substitution of \pali{goṇa} for the whole \pali{go} sometimes because of \pali{su}-, \pali{hi}-, and \pali{nā}-vibhatti.}
\example[0]{goṇesu = go + su}
\example[0]{goṇehi = go + hi}
\example[0]{goṇebhi = go + hi}
\example{goṇena = go + nā}

\head{82}{82, 149. aṃmo niggahitaṃ jhalapehi.}
\headtrans{After \pali{jha}-, \pali{la}-, and \pali{pa}-alias \pali{aṃ}[-vibhatti] and \pali{ma} [become] niggahita.}
\sutdef{aṃvacanassa makārassa ca jhalapaiccetehi niggahitaṃ hoti.}
\sutdeftrans{There is [a substitution] of niggahita for \pali{aṃ}-vibhatti and \pali{ma} after \pali{jha}-, \pali{la}-, and \pali{pa}-alias.}
\example[0]{aggiṃ = aggi + aṃ}
\example[0]{isiṃ = isi + aṃ}
\example[0]{gahapatiṃ = gahapati + aṃ}
\example[0]{daṇḍiṃ = daṇḍī + aṃ}
\example[0]{mahesiṃ = mahesī + aṃ}
\example[0]{bhikkhuṃ = bhikkhu + aṃ}
\example[0]{paṭuṃ = paṭu + aṃ}
\example[0]{sayambhuṃ = sayambhū + aṃ}
\example[0]{abhibhuṃ = abhibhū + aṃ}
\example[0]{rattiṃ = ratti + aṃ}
\example[0]{itthiṃ = itthī + aṃ}
\example[0]{vadhuṃ = vadhū + aṃ}
\example[0]{pulliṅgaṃ = puma + liṅgaṃ}
\example[0]{pumbhāvo = puma + bhāvo}
\example{puṅkokilo = puma + kokilo}
\transnote{For the definition of \pali{jha}-, \pali{la}-, \pali{pa}-alias, see \hyperref[sut:58]{Kacc 58} and \hyperref[sut:59]{59}. For the last three examples, see also \hyperref[sut:222]{Kacc 222}.}

\head{83}{}83, 67. saralopomādesapaccayādimhi saralope tu pakati.
\headtrans{[There is] an elision of vowel because of \pali{aṃ}-vibhatti, substitution, and suffix, etc. When the vowel is elided, [there is] no [further] change.}
\sutdef{saralopo hoti amādesapaccayādimhi. saralope tu pakati hoti.}
\sutdeftrans{There is an elision of vowel because of \pali{aṃ}-vibhatti, substitution, and suffix, etc. When the vowel is elided, there is no [further] change.}
\example[0]{purisaṃ = purisa + aṃ}
\example[0]{pāpaṃ = pāpa + aṃ}
\example[0]{purise = purisa + yo}
\example[0]{pāpe = pāpa + yo}
\example[0]{pāpiyo = pāpa + iya (+ si)}
\example{pāpiṭṭho = pāpa + iṭṭha (+ si)}
\transnote{We can break \pali{saralopomādesapaccayādimhi} down to \pali{saralopo + aṃ + ādesa + paccaya + ādi + smiṃ}. Seeing the examples makes the sutta more understandable. The elision here is applied to the preceding vowel (\pali{a} in these examples). The first two examples are the case of \pali{aṃ}-vibhatti. The next two examples are the case of substitution. And the last two are the case of suffix (\pali{paccaya}).}

\head{84}{84, 144. agho rassamekavacanayosvapi ca.}
\headtrans{A non-\pali{gha} vowel gets shortened because of singular [vibhattis] and \pali{yo}-vibhatti also.}
\sutdef{agho saro rassamāpajjate ekavacanayoiccetesu.}
\sutdeftrans{A non-\pali{gha} vowel (not feminine \pali{ā}) gets shortened because of singular [vibhattis] and \pali{yo}-vibhatti.}
\example[0]{itthiṃ = itthī + aṃ}
\example[0]{itthiyo = itthī + yo}
\example[0]{itthiyā = itthī + nā}
\example[0]{vadhuṃ = vadhū + aṃ}
\example[0]{vadhuyo = vadhū + yo}
\example[0]{vadhuyā = vadhū + nā}
\example[0]{daṇḍiṃ = daṇḍī + aṃ}
\example[0]{daṇḍino = daṇḍī + yo}
\example[0]{daṇḍinā = daṇḍī + nā}
\example[0]{sayambhuṃ = sayambhū + aṃ}
\example[0]{sayambhuvo = sayambhū + yo}
\example{sayambhunā = sayambhū + nā}
\transnote{See the definition of \pali{gha}-alias in \hyperref[sut:60]{Kacc 60}.}

\head{85}{85, 150. na sismimanapuṃsakāni.}
\headtrans{Non-neuter genders [are not shortened] because of \pali{si}[-vibhatti].}
\sutdef{sismiṃ anapuṃsakāni liṅgāni na rassamāpajjante.}
\sutdeftrans{Non-neuter genders are not shortened because of \pali{si}[-vibhatti].}
\example[0]{itthī = itthī + si}
\example[0]{bhikkhunī = bhikkhunī + si}
\example[0]{vadhū = vadhū + si}
\example[0]{daṇḍī = daṇḍī + si}
\example{sayambhū = sayambhū + si}
\transnote{We do not see \pali{si}-vibhatti because it get deleted by ``\pali{sesato lopaṃ gasipi}'' (\hyperref[sut:220]{Kacc 220}).}

\head{86}{86, 227. ubhādito naminnaṃ.}
\headtrans{After \pali{ubha} and so on, \pali{naṃ} [becomes] \pali{innaṃ}.}
\sutdef{ubhaiccevamādito saṅkhyāto naṃvacanassa innaṃ hoti.}
\sutdeftrans{There is [a substitution of] \pali{innaṃ} for \pali{naṃ}-vibhatti after numbers such as \pali{ubha} and so on.}
\example[0]{ubhinnaṃ = ubha + naṃ}
\example{duvinnaṃ = duve + naṃ}

\head{87}{87, 231. iṇṇamiṇṇannaṃ tīhi saṅkhyāhi.}
\headtrans{[There are] \pali{iṇṇaṃ}- and \pali{iṇṇannaṃ}-substitution after the number \pali{ti}.}
\sutdef{naṃvacanassa iṇṇaṃ iṇṇannaṃ iccete ādesā honti tīhi saṅkhyāhi.}
\sutdeftrans{There are substitutions of \pali{iṇṇaṃ} and \pali{iṇṇannaṃ} for \pali{naṃ}-vibhatti after the number \pali{ti}.}
\example[0]{tiṇṇaṃ = ti + ṇaṃ}
\example{tiṇṇannaṃ = ti + naṃ}

\head{88}{88, 147. yosu katanikāralopesu dīghaṃ.}
\headtrans{[All vowels get] elongated because of \pali{yo}-vibhatti that was changed to \pali{ni} or elided.}
\sutdef{sabbe sarā yosu katanikāralopesu dīghamāpajjante.}
\sutdeftrans{All vowels get elongated because of \pali{yo}-vibhatti that was changed to \pali{ni} or elided.}
\example[0]{aggī = aggi + yo}
\example[0]{bhikkhū = bhikkhu + yo}
\example[0]{rattī = ratti + yo}
\example[0]{yāgū = yāgu + yo}
\example[0]{aṭṭhī = aṭṭhi + yo}
\example[0]{aṭṭhīni = aṭṭhi + yo}
\example[0]{āyū = āyu + yo}
\example[0]{āyūni = āyu + yo}
\example[0]{sabbāni = sabba + yo}
\example[0]{yāni = ya + yo}
\example[0]{tāni = ta + yo}
\example[0]{kāni = ka + yo}
\example[0]{katamāni = katama + yo}
\example[0]{etāni = eta + yo}
\example[0]{amūni = amu + yo}
\example{imāni = ima + yo}
\transnote{It is obvious that the \pali{ni} case is for neuter words.}

\head{89}{89, 87. sunaṃhisu ca.}
\headtrans{Also [because of] \pali{su}-, \pali{naṃ}-, and \pali{hi}[-vibhatti].}
\sutdef{sunaṃhiiccetesu sabbe sarā dīghamāpajjante.}
\sutdeftrans{All vowels get elongated because of \pali{su}-, \pali{naṃ}-, and \pali{hi}[-vibhatti].}
\example[0]{aggīsu = aggi + su}
\example[0]{aggīnaṃ = aggi + naṃ}
\example[0]{aggīhi = aggi + hi}
\example[0]{rattīsu = ratti + su}
\example[0]{rattīnaṃ = ratti + naṃ}
\example[0]{rattīhi = ratti + hi}
\example[0]{bhikkhūsu = bhikkhu + su}
\example[0]{bhikkhūnaṃ = bhikkhu + naṃ}
\example[0]{bhikkhūhi = bhikkhu + hi}
\example{purisānaṃ = purisa + naṃ}

\head{90}{90, 252. pañcādīnamattaṃ.}
\headtrans{[The ending of] \pali{pañca} and so on [becomes] \pali{a}.}
\sutdef{pañcādīnaṃ saṅkhyānaṃ anto attamāpajjate sunaṃhiiccetesu.}
\sutdeftrans{The ending of numbers such as \pali{pañca} and so on [to aṭṭhārasa] becomes \pali{a} because of \pali{su}-, \pali{naṃ}-, and \pali{hi}[-vibhatti].}
\example[0]{pañcasu = pañca + su}
\example[0]{pañcannaṃ = pañca + naṃ}
\example[0]{pañcahi = pañca + hi}
\example[0]{chasu = cha + su}
\example[0]{channaṃ = cha + naṃ}
\example[0]{chahi = cha + hi}
\example[0]{sattasu = satta + su}
\example[0]{sattannaṃ = satta + naṃ}
\example[0]{sattahi = satta + hi}
\example[0]{aṭṭhasu = aṭṭha + su}
\example[0]{aṭṭhannaṃ = aṭṭha + naṃ}
\example[0]{aṭṭhahi = aṭṭha + hi}
\example[0]{navasu = nava + su}
\example[0]{navannaṃ = nava + naṃ}
\example[0]{navahi = nava + hi}
\example[0]{dasasu = dasa + su}
\example[0]{dasannaṃ = dasa + naṃ}
\example{dasahi = dasa + hi}
\transnote{To make \pali{a} accusative, it becomes \pali{attaṃ} (and similarly \pali{ettaṃ} for \pali{e}, see below).}

\head{91}{91, 194. patissinīmhi.}
\headtrans{For \pali{pati}, [the ending becomes \pali{a}] because of \pali{inī}.}
\sutdef{patissanto attamāpajjate inīmhi paccaye pare.}
\sutdeftrans{The ending [vowel] of \pali{pati} becomes \pali{a} because of \pali{inī}-paccaya behind.}
\example{gahapatānī = gahapati + inī}
\transnote{The \pali{inī}-paccaya is mentioned in ``patibhikkhurājīkārantehi inī'' (\hyperref[sut:240]{Kacc 240}).}

\head{92}{92, 100. ntussanto yosu ca.}
\headtrans{Also for the ending of \pali{ntu} because of \pali{yo}[-vibhatti] and so on.}
\sutdef{ntupaccayassa anto attamāpajjate sunaṃhiyoiccetesu paresu.}
\sutdeftrans{The ending [vowel] of \pali{ntu}-paccaya becomes \pali{a} because of \pali{su}-, \pali{naṃ}-, \pali{hi}-, and \pali{yo}-vibhatti behind.}
\example[0]{guṇavantesu = guṇavantu + su}
\example[0]{guṇavantānaṃ = guṇavantu + naṃ}
\example[0]{guṇavantehi = guṇavantu + hi}
\example[0]{guṇavantā = guṇavantu + yo}
\example{guṇavante = guṇavantu + yo}

\head{93}{93, 106. sabbassa vā aṃsesu.}
\headtrans{The whole [\pali{ntu} becomes \pali{a}] sometimes because of \pali{aṃ}[-vibhatti] [and \pali{sa}].}
\sutdef{sabbasseva ntupaccayassa attaṃ hoti vā aṃsaiccetesu.}
\sutdeftrans{The whole \pali{ntu}-paccaya becomes \pali{a} sometimes because of \pali{aṃ}- and \pali{sa}-vibhatti.}
\example[0]{satimaṃ = satimantu + aṃ}
\example[0]{satimantaṃ = satimantu + aṃ}
\example[0]{bandhumaṃ = bandhumantu + aṃ}
\example[0]{bandhumantaṃ = bandhumantu + aṃ}
\example[0]{satimassa = satimantu + sa}
\example[0]{satimato = satimantu + sa}
\example[0]{bandhumassa = bandhumantu + sa}
\example{bandhumato = bandhumantu + sa}

\head{94}{94, 105. simhi vā.}
\headtrans{Because of \pali{si}-vibhatti sometimes.}
\sutdef{ntupaccayassa antassa attaṃ hoti vā simhi vibhattimhi.}
\sutdeftrans{The ending [vowel] of \pali{ntu}-paccaya becomes \pali{a} sometimes because of \pali{si}-vibhatti.}
\example{himavanto = himavantu + si}
\transnote{For \pali{si}-vibhatti becomes \pali{o}, see \hyperref[sut:104]{Kacc 104}.}

\head{95}{95, 145. aggissini.}
\headtrans{[The ending of] \pali{aggi} [becomes] \pali{ini}.}
\sutdef{aggissantassa ini hoti vā simhi vibhattimhi.}
\sutdeftrans{The ending [vowel] of \pali{aggi} becomes \pali{ini} sometimes because of \pali{si}-vibhatti.}
\example{aggini = aggi + si}

\head{96}{96, 148. yosvakatarasso jho.}
\headtrans{Because of \pali{yo}[-vibhatti], the un-shortened \pali{jha}-alias [becomes \pali{a}].}
\sutdef{yosu akatarasso jho attamāpajjate.}
\sutdeftrans{Because of \pali{yo}[-vibhatti], the un-shortened [ending of] \pali{jha}-alias (masculine \pali{i}) becomes \pali{a}.}
\example[0]{aggayo = aggi + yo}
\example[0]{munayo = muni + yo}
\example[0]{isayo = isi + yo}
\example{gahapatayo = gahapati + yo}
\transnote{See the definition of \pali{jha}-alias in \hyperref[sut:58]{Kacc 58}.}

\head{97}{97, 156. vevosu lo ca.}
\headtrans{Because of \pali{ve} and \pali{vo}, [the unshortened] \pali{la}-alias [becomes \pali{a}] also.}
\sutdef{vevoiccetesu akatarasso lo attamāpajjate.}
\sutdeftrans{Because of \pali{ve} and \pali{vo} (substitution of \pali{yo}-vibhatti), the un-shortened [ending of] \pali{la}-alias (masculine \pali{u}) becomes \pali{a}.}
\example[0]{bhikkhave = bhikkhu + yo}
\example[0]{bhikkhavo = bhikkhu + yo}
\example[0]{hetave = hetu + yo}
\example{hetavo = hetu + yo}
\transnote{See the definition of \pali{la}-alias in \hyperref[sut:58]{Kacc 58}. For \pali{yo} becomes \pali{ve} and \pali{vo}, see ``akatarassā lato yvālapanassa vevo'' (\hyperref[sut:116]{Kacc 116}).}

\head{98}{98, 186. mātulādīnamānattamīkāre.}
\headtrans{[The ending of] \pali{mātula} and so on [becomes] \pali{āna} because of \pali{ī}[-paccaya].}
\sutdef{mātulaiccevamādīnaṃ anto ānattamāpajjate īkāre paccaye pare.}
\sutdeftrans{The ending of \pali{mātula} and so on becomes \pali{āna} because of \pali{ī}-paccaya behind.}
\example[0]{mātulānī = mātula + ī}
\example[0]{ayyakānī = ayyaka + ī}
\example{varuṇānī = varuṇa + ī}

\head{99}{99, 81. smāhismiṃnaṃ mhābhimhi vā.}
\headtrans{[Substitutions of] \pali{mhā}, \pali{bhi}, and \pali{mhi} for \pali{smā}, \pali{hi}, and \pali{smiṃ} sometimes.}
\sutdef{sabbato liṅgato smāhismiṃ iccetesaṃ mhābhimhiiccete ādesā honti vā yathāsaṅkhyaṃ.}
\sutdeftrans{There are substitutions of \pali{mhā}, \pali{bhi}, and \pali{mhi} for \pali{smā}, \pali{hi}, and \pali{smiṃ} respectively after all nominal bases sometimes.}
\example[0]{purisamhā = purisa + smā (= purisasmā)}
\example[0]{purisebhi = purisa + hi (= purisehi)}
\example{purisamhi = purisa + smiṃ (= purisasmiṃ)}

\head{100}{100, 214. na timehi katākārehi.}
\headtrans{No substitutions after \pali{ta} and \pali{ima} [that have been] made to \pali{a}.}
\sutdef{taimaiccetehi katākārehi smāsmiṃ (naṃ)mhāmhiiccete ādesā neva honti.}
\sutdeftrans{There are no substitutions of \pali{mhā} and \pali{mhi} because of \pali{smā}- and \pali{smiṃ}-vibhatti after \pali{ta} and \pali{ima} [that have been] made to \pali{a}.}
\example[0]{asmā = ta + smā}
\example[0]{asmiṃ = ta + smiṃ}
\example[0]{asmā = ima + smā}
\example{asmiṃ = ima + smiṃ}
\transnote{This sutta prevents ambiguity but it also makes \pali{ta} and \pali{ima} undifferentiated in these cases. In the definition, \pali{naṃ} seems out of place.}

\head{101}{101, 80. suhisvakāro e.}
\headtrans{Because of \pali{su} and \pali{hi}[-vibhatti], \pali{a} [becomes] \pali{e}.}
\sutdef{suhiiccetesu akāro ettamāpajjate.}
\sutdeftrans{The letter \pali{a} becomes \pali{e} because of \pali{su} and \pali{hi}[-vibhatti].}
\example[0]{sabbesu = sabba + su}
\example[0]{yesu = ya + su}
\example[0]{tesu = ta + su}
\example[0]{kesu = ka + su}
\example[0]{purisesu = purisa + su}
\example[0]{imesu = ima + su}
\example[0]{kusalesu = kusala + su}
\example[0]{tumhesu = tumha + su}
\example[0]{amhesu = amha + su}
\example[0]{sabbehi = sabba + hi}
\example[0]{yehi = ya + hi}
\example[0]{tehi = ta + hi}
\example[0]{kehi = ka + hi}
\example[0]{purisehi = purisa + hi}
\example[0]{imehi = ima + hi}
\example[0]{kusalehi = kusala + hi}
\example[0]{tumhehi = tumha + hi}
\example{amhehi = amha + hi}
\transnote{In the formula, \pali{suhisvakāro} can be broken to \pali{suhisu + akāro}. The former part is in locative case.}

\head{102}{102, 202. sabbanāmānaṃ naṃmhi ca.}
\headtrans{Also [\pali{a}] of pronouns [becomes \pali{e}] because of \pali{naṃ}[-vibhatti].}
\sutdef{sabbesaṃ sabbanāmānaṃ anto akāro ettamāpajjate naṃmhi vibhattimhi.}
\sutdeftrans{The ending \pali{a} of all pronouns becomes \pali{e} because of \pali{naṃ}-vibhatti.}
\example[0]{sabbesaṃ = sabba + naṃ}
\example[0]{sabbesānaṃ = sabba + naṃ}
\example[0]{yesaṃ = ya + naṃ}
\example[0]{yesānaṃ = ya + naṃ}
\example[0]{tesaṃ = ta + naṃ}
\example[0]{tesānaṃ = ta + naṃ}
\example[0]{imesaṃ = ima + naṃ}
\example[0]{imesānaṃ = ima + naṃ}
\example[0]{kesaṃ = ka + naṃ}
\example[0]{kesānaṃ = ka + naṃ}
\example[0]{itaresaṃ = itara + naṃ}
\example[0]{itaresānaṃ = itara + naṃ}
\example[0]{katamesaṃ = katama + naṃ}
\example{katamesānaṃ = katama + naṃ}

\head{103}{103, 79. ato nena.}
\headtrans{After \pali{a}, \pali{nā}[-vibhatti] [becomes] \pali{ena}.}
\sutdef{tasmā akārato nāvacanassa enādeso hoti.}
\sutdeftrans{There is a substitution of \pali{ena} for \pali{nā}-vibhatti after that \pali{a}.}
\example[0]{sabbena = sabba + nā}
\example[0]{yena = ya + nā}
\example[0]{tena = ta + nā}
\example[0]{kena = ka + nā}
\example[0]{anena = ta + nā}
\example[0]{purisena = purisa + nā}
\example{rūpena = rūpa + nā}

\head{104}{104, 66. so.}
\headtrans{[There is a substitution of] \pali{o} for \pali{si}[-vibhatti].}
\sutdef{tasmā akārato sivacanassa okārādeso hoti.}
\sutdeftrans{There is a substitution of \pali{o} for \pali{si}-vibhatti after that \pali{a}.}
\example[0]{sabbo = sabba + si}
\example[0]{yo = ya + si}
\example[0]{so = ta + si}
\example[0]{ko = ka + si}
\example[0]{amuko = amuka + si}
\example{puriso = purisa + si}
\transnote{In the formula, \pali{so} means \pali{si + o}.}

\head{105}{105. so vā.}
\headtrans{Sometimes \pali{so} [for \pali{nā}].}
\sutdef{tasmā akārato nāvacanassa soādeso hoti vā.}
\sutdeftrans{There is a substitution of \pali{so} for \pali{nā}-vibhatti after that \pali{a} sometimes.}
\example[0]{atthaso = attha + nā}
\example[0]{byañjanaso = byañjana + nā}
\example[0]{akkharaso = akkhara + nā}
\example[0]{suttaso = sutta + nā}
\example[0]{padaso = pada + nā}
\example[0]{yasaso = yasa + nā}
\example[0]{upāyaso = upāya + nā}
\example[0]{sabbaso = sabba + nā}
\example[0]{thāmaso = thāma + nā}
\example{ṭhānaso = ṭhāna + nā}

\head{106}{106, 313. dīghorehi.}
\headtrans{[Sometimes \pali{so}] after \pali{dīgha} and \pali{ora}.}
\sutdef{dīghaoraiccetehi smāvacanassa soādeso hoti vā.}
\sutdeftrans{There is a substitution of \pali{so} for \pali{smā}-vibhatti after \pali{dīgha} and \pali{ora} sometimes.}
\example[0]{dīghaso = dīgha + smā}
\example{oraso = ora + smā}

\head{107}{107, 69. sabbayonīnamāe.}
\headtrans{For the whole \pali{yo}[-vibhatti] and \pali{ni}, [there are] \pali{ā}- and \pali{e}-substitution.}
\sutdef{tasmā akārato sabbesaṃ yonīnaṃ āeādesā honti vā yathāsaṅkhyaṃ.}
\sutdeftrans{There are substitutions of \pali{ā} and \pali{e} for the whole \pali{yo}[-vibhatti] and \pali{ni} after that \pali{a} sometimes.}
\example[0]{purisā = purisa + yo}
\example[0]{purise = purisa + yo}
\example[0]{rūpā = rūpa + yo}
\example{rūpe = rūpa + yo}
\transnote{As for \pali{rūpa} is neuter, \pali{yo}-vibhatti becomes \pali{ni}, as shown by ``ato niccaṃ'' (\hyperref[sut:218]{Kacc 218}). Then \pali{ni} becomes \pali{ā} and \pali{e} by this sutta.}

\head{108}{108, 90. smāsmiṃnaṃ vā.}
\headtrans{[For] \pali{smā} and \pali{smiṃ}[-vibhatti] sometimes, [there are \pali{ā}- and \pali{e}-substitution].}
\sutdef{tasmā akārato sabbesaṃ smāsmiṃiccetesaṃ āeādesā honti vā yathāsaṅkhyaṃ.}
\sutdeftrans{There are substitutions of \pali{ā} and \pali{e} for the whole \pali{smā} and \pali{smiṃ}[-vibhatti] respectively after that \pali{a} sometimes.}
\example[0]{purisā = purisa + smā}
\example{purise = purisa + smiṃ}

\head{109}{109, 304. āya catutthekavacanassa tu.}
\headtrans{[There is] \pali{āya}-substitution for the singular dative vibhatti.}
\sutdef{tasmā akārato catutthekavacanassa āyādeso hoti vā.}
\sutdeftrans{There is a substitution of \pali{āya} for the singular dative vibhatti (\pali{sa}) after that \pali{a} sometimes.}
\example[0]{atthāya = attha + sa}
\example[0]{hitāya = hita + sa}
\example{sukhāya = sukha + sa}
\transnote{This can be a clue that differentiates dative words from genitive ones.}

\head{110}{110, 201. tayo neva ca sabbanāmehi.}
\headtrans{[For] the three [vibhattis, there are] no [substitutions] after pronouns.}
\sutdef{tehi sabbanāmehi akārantehi smāsmiṃsaiccetesaṃ tayo āeāyādesā neva honti.}
\sutdeftrans{There are no substitutions of the three \pali{ā}, \pali{e}, and \pali{āya} for \pali{smā}, \pali{smiṃ}, and \pali{sa}[-vibhatti] after that \pali{a} of pronouns.}
\example[0]{sabbasmā = sabba + smā}
\example[0]{sabbasmiṃ = sabba + smiṃ}
\example[0]{sabbassa = sabba + sa}
\example[0]{yasmā = ya + smā}
\example[0]{yasmiṃ = ya + smiṃ}
\example[0]{yassa = ya + sa}
\example[0]{tasmā = ta + smā}
\example[0]{tasmiṃ = ta + smiṃ}
\example[0]{tassa = ta + sa}
\example[0]{kasmā = ka + smā}
\example[0]{kasmiṃ = ka + smiṃ}
\example[0]{kassa = ka + sa}
\example[0]{imasmā = ima + smā}
\example[0]{imasmiṃ = ima + smiṃ}
\example{imassa = ima + sa}
\transnote{This means, for example, there are no ablative \pali{sabbā}, locative \pali{sabbe}, and dative \pali{sabbāya}.}

\head{111}{111, 179. ghato nādīnaṃ.}
\headtrans{After \pali{gha}-alias, [there is \pali{āya}-substitution] for \pali{nā}[-vibhatti] and so on.}
\sutdef{tasmā ghato nādīnamekavacanānaṃ vibhattigaṇānaṃ āyādeso hoti.}
\sutdeftrans{There is a substitution of \pali{āya} for the singular vibhattis such as \pali{nā} and so on after that \pali{gha}-alias (feminine \pali{ā}).}
\example[0]{kaññāya kataṃ kammaṃ\\{\upshape= An action was done by the girl. (ins.)}}
\example[0]{kaññāya dīyate dhanaṃ\\{\upshape= Money is given to the girl. (dat.)}}
\example[0]{kaññāya nissaṭaṃ vatthaṃ\\{\upshape= A cloth was stripped from the girl. (abl.)}}
\example[0]{kaññāya pariggaho\\{\upshape= A possession of the girl. (gen.)}}
\example{kaññāya patiṭṭhitaṃ sīlaṃ\\{\upshape= Morality was established in the girl. (loc.)}}
\transnote{See the definition of \pali{gha}-alias in \hyperref[sut:60]{Kacc 60}.}

\head{112}{112, 183. pato yā.}
\headtrans{After \pali{pa}-alias, [there is] \pali{yā}-substitution [for \pali{nā}-vibhatti and so on].}
\sutdef{tasmā pato nādīnamekavacanānaṃ vibhattigaṇānaṃ yāādeso hoti.}
\sutdeftrans{There is a substitution of \pali{yā} for the singular vibhattis such as \pali{nā} and so on after that \pali{pa}-alias (feminine \pali{i}- and \pali{u}-group).}
\example[0]{rattiyā = ratti + nā/sa/smā/smiṃ}
\example[0]{itthiyā = itthī + nā/sa/smā/smiṃ}
\example[0]{deviyā = devī + nā/sa/smā/smiṃ}
\example[0]{dhenuyā = dhenu + nā/sa/smā/smiṃ}
\example[0]{yāguyā = yāgu + nā/sa/smā/smiṃ}
\example{vadhuyā = vadhū + nā/sa/smā/smiṃ}
\transnote{See the definition of \pali{pa}-alias in \hyperref[sut:59]{Kacc 59}.}

\head{113}{113, 132. sakhato gasse vā.}
\headtrans{After \pali{sakha}, [there are substitutions of] \pali{e} [and so on] for \pali{ga}-alias.}
\sutdef{tasmā sakhato gassa akāra ākāra ikāra īkāra ekārādesā honti vā.}
\sutdeftrans{There are substitutions of \pali{a}, \pali{ā}, \pali{i}, \pali{ī}, and \pali{e} for \pali{ga}-alias (vocative \pali{si}-vibhatti) after that \pali{sakha} sometimes.}
\example[0]{bho sakha}
\example[0]{bho sakhā}
\example[0]{bho sakhi}
\example[0]{bho sakhī}
\example{bho sakhe}
\transnote{See the definition of \pali{ga}-alias in \hyperref[sut:57]{Kacc 57}.}

\head{114}{114, 178. ghate ca.}
\headtrans{Also \pali{e}-substitution after \pali{gha}-alias.}
\sutdef{tasmā ghato gassa ekārādeso hoti.}
\sutdeftrans{There is a substitution of \pali{e} for \pali{ga}-alias after that \pali{gha}-alias (feminine \pali{ā}).}
\example[0]{bhoti ayye}
\example[0]{bhoti kaññe}
\example{bhoti kharādiye}
\transnote{See the definition of \pali{gha}-alias in \hyperref[sut:60]{Kacc 60}.}

\head{115}{115, 181. na ammādito.}
\headtrans{No [\pali{e}-substitution] after \pali{amma} and so on.}
\sutdef{tato ammādito gassa ekārattaṃ na hoti.}
\sutdeftrans{There is no \pali{e}-substitution for \pali{ga}-alias after that \pali{amma} and so on.}
\example[0]{bhoti ammā}
\example[0]{bhoti annā}
\example[0]{bhoti ambā}
\example{bhoti tātā}
\transnote{There are no \pali{amme}, \pali{anne}, \pali{ambe}, \pali{tāte} in vocative sense, so to speak.}

\head{116}{116, 157. akatarassā lato yvālapanassa vevo.}
\headtrans{After unshortened \pali{la}-alias, [there are] \pali{ve}- and \pali{vo}-substitution for the vocative \pali{yo}[-vibhatti].}
\sutdef{tasmā akatarassā lato yvālapanassa vevoādesā honti.}
\sutdeftrans{There are substitutions of \pali{ve} and \pali{vo} for the vocative \pali{yo}[-vibhatti] after that unshortened \pali{la}-alias (masculine \pali{u}-group).}
\example[0]{bhikkhave = bhikkhu + yo}
\example[0]{bhikkhavo = bhikkhu + yo}
\example[0]{hetave = hetu + yo}
\example{hetavo = hetu + yo}
\transnote{See the definition of \pali{la}-alias in \hyperref[sut:58]{Kacc 58}.}

\head{117}{117, 124. jhalato sassa no vā.}
\headtrans{After \pali{jha}-alias and \pali{la}-alias, [there is] \pali{no}-substitution for \pali{sa}[-vibhatti].}
\sutdef{tasmā jhalato sassa vibhattissa noādeso hoti vā.}
\sutdeftrans{There is a substitution of \pali{no} for \pali{sa}-vibhatti after that \pali{jha}- and \pali{la}-alias}
\example[0]{aggino = aggi + sa}
\example[0]{sakhino = sakhī + sa}
\example[0]{daṇḍino = daṇḍī + sa}
\example[0]{bhikkhuno = bhikkhu + sa}
\example{sayambhuno = sayambhū + sa}
\transnote{See the definition of \pali{jha}-alias and \pali{la}-alias in \hyperref[sut:58]{Kacc 58}.}

\head{118}{118, 146. ghapato ca yonaṃ lopo.}
\headtrans{After \pali{gha}-, \pali{pa}-alias [and so on], [there is] an elision of \pali{yo}[-vibhatti].}
\sutdef{tehi ghapajhalaiccetehi yonaṃ lopo hoti vā.}
\sutdeftrans{There is an elision of \pali{yo}[-vibhatti] after those \pali{gha}-, \pali{pa}-, \pali{jha}, and \pali{la}-alias sometimes.}
\example[0]{kaññā = kaññā + yo}
\example[0]{rattī = ratti + yo}
\example[0]{itthī = itthī + yo}
\example[0]{yāgū = yāgu + yo}
\example[0]{vadhū = vadhū + yo}
\example[0]{aggī = aggi + yo}
\example[0]{bhikkhū = bhikkhu + yo}
\example[0]{sayambhū = sayambhū + yo}
\example[0]{aṭṭhī = aṭṭhi + yo}
\example{āyū = āyu + yo}
\transnote{For the definition of these aliases, see \hyperref[sut:58]{Kacc 58}, \hyperref[sut:59]{59}, and \hyperref[sut:60]{60}.}

\head{119}{119, 155. lato vokāro ca.}
\headtrans{After \pali{la}-alias, [there is a substitution of] \pali{vo} also.}
\sutdef{tasmā lato yonaṃ vokāro hoti vā.}
\sutdeftrans{There is [a substitution of] \pali{vo} for \pali{yo}[-vibhatti] after that \pali{la}-alias sometimes.}
\example[0]{bhikkhavo = bhikkhū}
\example{sayambhuvo = sayambhū}
\transnote{See the definition of \pali{la}-alias in \hyperref[sut:58]{Kacc 58}.}

