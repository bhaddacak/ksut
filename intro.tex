\markboth{}{Introduction}
\cleardoublepage
\phantomsection
\addcontentsline{toc}{chapter}{A minimal introduction}
\chapter*{A minimal introduction}

Kaccāyanabyākaraṇa (Grammar of Kaccāyana) is the oldest Pāli grammar textbook, possibly composed in Sri Lanka. There are many controversies around this book, including who is the author, when it was composed, etc. I do not want to distract the reader with disputable issues.\footnote{I have a brief introduction to the three grammar books in Appendix A of \emph{Pāli for New Learners}. That should be enough for most readers. See \url{bhaddacak.github.io/pnl}.} I also have no proselytizing purpose to put forward.

Let us look at undeniable facts when comparing the book to Sanskrit grammar. As listed below, it is clear that some suttas in Kaccāyana follow Pāṇini's Aṣṭādhyāyī and Śarvavarman's Kātantra. The latter is said to have more influence upon the work.

\bigskip
{\small\noindent
Pāṇ 2.3.5 kālādhvanoratyantasaṃyoge\\
Kacc 298 kāladdhānamaccantasaṃyoge\\[1mm]
Pāṇ 2.3.8 karmapravacanīyayukte dvitīyā\\
Kacc 299 kammappavacanīyayutte\\[1mm]
Kāt 2.214 yato'paiti bhayamādatte vā tadapādānam\\
Kacc 271 yasmādapeti bhayamādatte vā tadapādānaṃ\\[1mm]
Kāt 2.216 yasmai ditsā rocate dhārayate vā tat saṃpradānam\\
Kacc 276 yassa dātukāmo rocate dhārayate vā taṃ\\[1mm]
Kāt 2.259 nāmnāṃ samāso yuktārthaḥ\\
Kacc 316 nāmānaṃ samāso yuttattho\\
}

This can say that, at least for some suttas, the book can not be dated earlier than Śarvavarman's time, around the 2nd or 1st century BCE. And this also shows that the book is not original in its entirety.

For other issues, the book will attest itself when you finish all suttas. You can see how disorganized and inconsistent it is. This tells us that the book was not a product of one-time composition. As clearly shown by examples given in suttas, some of them have a scent of recency.

Still, no one can deny the importance of Kaccāyana grammar. In the present book, I will focus only on its text and put other issues aside. Sometimes, however, I also add noteworthy remarks.

The most widespread Pāli source of Kaccāyana is the corpus of Chaṭṭha Saṅgāyana Tipiṭaka, in which the book contains 673 suttas. But in an older source, two other suttas are also present.\footnote{See \citealp[p.~221]{thitzana:kacc1}.} I therefore include these to make the book complete (see \hyperref[sut:243a]{Kacc 243a} and \hyperref[sut:243b]{243b}). So we have totally 675 suttas here. I also use Thai sources in parallel. When discrepancies occur, they will be informed clearly. These are supposedly very few.

As far as I know, there are only a few English translations of Kaccāyana. The oldest one is by James d'Alwis.\footnote{\citealp{dalwis:kach}} Despite the lengthy introduction, d'Alwis translated only the book on verb (\pali{ākhyātakappa}), 4 chapters, 118 suttas totally.\footnote{In the modern numbering, they are suttas number 406--523.} For this author did not retain the original Pāli suttas (the formulas were translated), to locate a sutta needs some effort. The overall translation is readable but not complete.

A translation of all suttas was done later by Satis Chandra Acharyya.\footnote{\citealp{satis:kacc}} This book is hard to follow by those who are not familiar with Devanagari script. Also the translation is not complete, but it can be close to what you will see in the present book, in a more accessible way.

A full-blown translation has been undertaken recently by A.\,Thitzana.\footnote{\citealp{thitzana:kacc2}} This one is the most complete to date and very friendly to modern readers.

Why is another translation needed after all? The main reason is that Thitzana's translations are imbued with explanations. This does not help new students understand the suttas as they are. Also, this author tends to make things complicated than necessary. And above all, this translation is proprietary, does not belong to the public.

When I read word-for-word Thai translations of Kaccāyana, I really feel at home and see the picture more clearly.\footnote{Thai books on Kaccāyana that guide me in the process are translations by Phra Ratchapariyattimoli (Somsak Upasamo), and in a lesser extent by Phra Maha Thitipong Uttamapañño and by Phra Maha Sompong Mudito. When doubts occur I also consult translations of Padarūpasiddhi by Phra Gandhasārābhivaṃsa and the authors mentioned above.} To make the reader feels in the same way, then I render the text in a simpler way facilitating new learners who want to exercise their effort in their own way.

On style of translation, it will be similar to what I wrote in my previous Pāli books, particularly in \emph{Pāli Text Reading: A Handbook}.\footnote{\url{bhaddacak.github.io/ptr}} I will make it clean as much as possible. It may be a little verbose, not beautiful, but understandable. When the translation sounds cryptic, I will add a short explanation. However, I do not utilize the whole process as guided by the book. Only clean translation is applied, no deep analytic parts like discourse analysis. But some grammatical suggestions are also given.

In the main content of the book, each sutta will be put into the same format as follows:

\begin{itemize}
\item The sutta head (formula) in Pāli is retained.
\item The translation of head is marked by \faHeart[regular].
\item The definition (\pali{vutti}) in Pāli is retained, marked by \faAngleRight.
\item The translation of definition in full is marked by \faAngleDoubleRight.
\item Examples (\pali{ud\=ahara\d na}) will be given only when neccesary, marked by $\triangleright$.
\item Other information (\pali{payoga}), such as an additional explanation, analysis, or questions \& answers, is excluded, except some rare cases that I find the information is useful (but there is no direct translation in this part).
\item Text outside the sutta body is excluded.
\item The author's translation notes or additional explanations will be added when needed, marked by \faLightbulb[regular].
\end{itemize}

Normally, each sutta of Kaccāyana has two numbers. The first one is the running number in Kaccāyana itself. The rest is the number in Padarūpasiddhi related to the sutta. The latter can be multiple in case the sutta may have several relations, or absent completely if no such relations.

One problem concerning the formulas is the use of long compounds. For reasonably comprehensible compounds, I will keep the words connected as a whole unit, and if necessary I will provide a decomposition of them. In a very long compound, as found in \hyperref[sut:275]{Kacc 275} and \hyperref[sut:277]{277} for example, I will break the word with small spaces to ease the reading. In this case, keep in mind that the parts should be seen as connected. In some rare cases, the formula is treated as a list of words, not grammatical units, see \hyperref[sut:55]{Kacc 55} for example.

\section*{Notes on Sanskrit grammar}

After I went through a half of the book, I found that some grammatical points are really mysterious to us. Explanations from commentaries and related works sometimes help, but not always so. This inevitably makes me bring Sanskrit grammar into play, even though my knowledge of Sanskrit is still limited.

This can also put hardship upon readers, but I see no other way better than this. It is always better to know the origins (as far as we can trace). And as a matter of fact, the author of Kaccāyana supposed that the learners are familiar with Sanskrit grammar to some extent, as shown in \hyperref[sut:9]{Kacc 9} (\pali{parasamaññā payoge}), for example.

Does this mean we have to master Sanskrit before fully making sense of Pāli grammar? If you major in philology, this can be the case. However, I will not go that far, because I myself started learning Pāli without knowing Sanskrit. In my treatments of certain issues, I will take Sanskrit grammar into account just enough to elucidate the points at hand. This also enable readers of scholarly type go deeper into the topic on their own.

To make this less intimidating, we should be familiar with related Sanskrit literature first. Unquestionably, the greatest Sanskrit grammarian is Pāṇini (around the 4th century BCE). His Aṣṭādhyāyī (the eight chapters) marked the period called classical Sanskrit from Vedic. This work had great impacts on subsequent Indic languages, including Pāli.

Aṣṭādhyāyī contains only sūtras or formulas (3,995 totally) that have to be explained. The earliest commentary on this is Vārttikas by Kātyāyana (around the 3rd century BCE). This work seems to be lost, but Mahābhāṣya of Patañjali (around the 2nd century BCE) preserved 1,245 sūtras of Vārttikas for us.\footnote{\citealp[p.~426]{keith:history}} A full commentary on Aṣṭādhyāyī was written later by Jayāditya and Vāmana in the 7th century AD. It was named Kāśikā Vṛtti.

However, not only the Pāṇinian grammar did influence Kaccāyana, but also, even in a greater extent, the oldest post-Pāṇinian grammatical work called Kātantra (little treatise) by Śarvavarman (around the 2nd or 1st century BCE). This work has a commentary written by Durgasiṅha (around the 9th century AD).

The works and names mentioned above will be brought into account occasionally when needed. For a vast digital collection of Sanskrit texts, see \href{https://gretil.sub.uni-goettingen.de/}{GRETIL}.\footnote{\url{gretil.sub.uni-goettingen.de}} You can find Aṣṭādhyāyī, Mahābhāṣya, and Kāśikā here.\footnote{See also links to resources in \hyperref[chap:abbrev]{Abbreviations}.}

Learning Sanskrit is definitely difficult. But for Pāli knowers, you have already gone more than a half way. It is far easier to learn the language with the knowledge you have because many things are still relevant. Moreover, resources for Sanskrit learning are much richer and have better quality than the Pāli counterparts.
