\markboth{}{Introduction}
\cleardoublepage
\phantomsection
\addcontentsline{toc}{chapter}{A minimal introduction}
\chapter*{A minimal introduction}

Kaccāyanabyākaraṇa (Grammar of Kaccāyana) is the oldest Pāli grammar textbook, possibly composed in Sri Lanka. There are many controversies around this book, including who is the author, when it was composed, whether it followed a Sanskrit model, etc. I do not want to distract the reader with disputable issues.\footnote{I have a brief introduction to the three grammar books in Appendix A of \emph{Pāli for New Learners}. That should be enough for most readers. See \url{bhaddacak.github.io/pnl}.} I also have no proselytizing purpose to put forward.

Let us use our common sense and think about it logically. As a matter of fact, any grammar books (and dictionaries) come after the language. They are records of the actual language uses at a time. When the usages undergo changes, grammatical rules and definitions of terms are updated accordingly. The rules therefore by no means dictate the uses of the language.

By this fact, the author of Kaccāyana cannot be the one who lived in the far past, supposedly in the Buddha's time when the language we call Pāli did not yet take its form as we see today. An established set of grammatical rules has to be formed not earlier than when the language takes its final rigid form. That is to say, positing that a Pāli grammar book was written before the Pāli canon was crystallized is questionable.\footnote{The issue of when exactly the Pāli canon was closed for further editing is debatable. By the malleable nature of languages, including Pāli, thinking that the language was frozen from the early time is illogical. Once a language is in use, it continuously changes. This can affect the Pāli canon on the condition that the meaning of the teachings were transmitted, not just sound patterns.}

Another point, it is impossible that one can understand the formulas without their definition.\footnote{See \hyperref[sut:9]{Kacc 9}, for example. Also consider \hyperref[sut:15]{Kacc 15} versus \hyperref[sut:25]{Kacc 25}, and very terse formulas like \hyperref[sut:104]{Kacc 104} and \hyperref[sut:105]{Kacc 105}.} That is to say, composing all formulas at one time and then adding definitions by another person at another time is unthinkable. However, it seems likely that examples and additional explanations were added subsequently, at least some parts of them. Moreover, the whole book seems not to be a product of one-time composition because inconsistencies can be found occasionally. The final version, as we have it, is likely to have been gradually developed upon an old framework.

Still, no one can deny the importance of Kaccāyana grammar. In the present book, I will only focus on its text and put other issues aside. Sometimes, however, I also add noteworthy remarks. (I also have many things that I want to believe they are true, but my rigorous reasoning does not allow that. This is exactly what I have learned from the Buddha's methodology.) 

The most widespread Pāli source of Kaccāyana is the corpus of Chaṭṭha Saṅgāyana Tipiṭaka, in which the book contains 673 suttas. But in an older source, two other suttas are also present.\footnote{See \citealp[p.~221]{thitzana:kacc1}.} I therefore include these to make the book complete (see \hyperref[sut:243a]{Kacc 243a} and \hyperref[sut:243b]{243b}). So we have totally 675 suttas here. I also use Thai sources in parallel. When discrepancies occur, they will be informed clearly. These are supposedly very few.

As far as I know, there are only a few English translations of Kaccāyana. The oldest one is by James d'Alwis.\footnote{\citealp{dalwis:kach}} Despite the lengthy introduction, d'Alwis translated only the book on verb (\pali{ākhyātakappa}), 4 chapters, 118 suttas totally.\footnote{In the modern numbering, they are suttas number 406--523.} For this author did not retain the original Pāli suttas (the formulas were translated), to locate a sutta needs some effort. The overall translation is readable but not complete.

A translation of all suttas was done later by Satis Chandra Acharyya.\footnote{\citealp{satis:kacc}} This book is hard to follow by those who are not familiar with Devanagari script. Also the translation is not complete, but it can be close to what you will see in the present book, in a more accessible way.

A full-blown translation has been undertaken recently by A.\,Thitzana.\footnote{\citealp{thitzana:kacc2}} This one is the most complete to date and very friendly to modern readers.

Why is another translation needed after all? The main reason is that Thitzana's translations are imbued with explanations. This does not help new students understand the suttas as they are. Also, this author tends to make things complicated than necessary. And above all, this translation is proprietary, does not belong to the public.

When I read word-for-word Thai translations of Kaccāyana, I really feel at home and see the picture more clearly.\footnote{Thai books on Kaccāyana that guide me in the process are translations by Phra Ratchapariyattimoli (Somsak Upasamo), and in a lesser extent by Phra Maha Thitipong Uttamapañño and by Phra Maha Sompong Mudito. When doubts occur I also consult translations of Padarūpasiddhi by Phra Gandhasārābhivaṃsa and the authors mentioned above.} To make the reader feels in the same way, then I render the text in a simpler way facilitating new learners who want to exercise their effort in their own way.

On style of translation, it will be similar to what I wrote in my previous Pāli books, particularly in \emph{Pāli Text Reading: A Handbook}.\footnote{\url{bhaddacak.github.io/ptr}} I will make it clean as much as possible. It may be a little verbose, not beautiful, but understandable. When the translation sounds cryptic, I will add a short explanation. However, I do not utilize the whole process as guided by the book. Only clean translation is applied, no deep analytic parts like discourse analysis. But some grammatical suggestions are also given.

Each sutta will be put into the same format as follows:

\begin{itemize}
\item The sutta head (formula) in Pāli is retained.
\item The translation of head is marked by \faHeart[regular].
\item The definition (\pali{vutti}) in Pāli is retained, marked by \faAngleRight.
\item The translation of definition in full is marked by \faAngleDoubleRight.
\item Examples (\pali{ud\=ahara\d na}) will be given only when neccesary, marked by $\triangleright$.
\item Other information (\pali{payoga}), such as an additional explanation, analysis, or questions \& answers, is excluded, except some rare cases that I find the information is useful (but there is no direct translation in this part).
\item Text outside the sutta body is excluded.
\item The author's translation notes or additional explanations will be added when needed, marked by \faLightbulb[regular].
\end{itemize}

Normally, each sutta of Kaccāyana has two numbers. The first one is the running number in Kaccāyana itself. The rest is the number in Padarūpasiddhi related to the sutta. The latter can be multiple in case the sutta may have several relations, or absent completely if no such relations.

One problem concerning the formulas is the use of long compounds. For reasonably comprehensible compounds, I will keep the words connected as a whole unit, and if necessary I will provide a decomposition of them. In a very long compound, as found in \hyperref[sut:275]{Kacc 275} and \hyperref[sut:277]{277} for example, I will break the word with small spaces to ease the reading. In this case, keep in mind that the parts should be seen as connected. In some rare cases, the formula is treated as a list of words, not grammatical units, see \hyperref[sut:55]{Kacc 55} for example.
