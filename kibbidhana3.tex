\section{Tatiyakaṇḍa}

\head{571}{571, 624. paccayādaniṭṭhā nipātanā sijjhanti.}
\headtrans{The paccayas not come to a conclusion are accomplished from the give.}
\sutdef{saṅkhyānāmasamāsataddhitākhyātakitakappamhi sappaccayā ye saddā aniṭṭhaṅgatā, te sādhanena nirakkhitvā sakehi sakehi nāmehi nipātanā sijjhanti.}
\sutdeftrans{Which terms together with paccayas in the sections of numeral, noun, compound, secondary derivation, verb, and primary derivation did not come to a conclusion, those [terms] are accomplished from the given by their own name each, having laid down by the procedure.}
\transnote{Concerning \pali{nipātanā}, please see \hyperref[sut:391]{Kacc 391} first. The examples in this sutta are redundant and a little messy, so I skip them altogether.}

\head{572}{572, 625. sāsadisato tassa riṭṭho ca.}
\headtrans{After \pali{sāsa} and \pali{disa}, [there is] \pali{riṭṭha}-substitution for \pali{ta}-paccaya also.}
\sutdef{sāsadisaiccevamādīhi dhātūhi tapaccayassa riṭṭhādeso hoti ṭhāne.}
\sutdeftrans{There is a substitution of \pali{riṭṭha} for \pali{ta}-paccaya after roots such as \pali{sāsa}, \pali{disa} and so on in a suitable place.}
\example[0]{anusiṭṭho (anu + \paliroot{sāsa} + ta) so mayā  \\{\upshape= The person was taught by me.}}
\example{diṭṭhaṃ (\paliroot{disa} + ta) me rūpaṃ \\{\upshape= The image was seen by me.}}
\transnote{As marked by \pali{ca}, for passive paccayas and \pali{tuṃ} they can also have \pali{raṭṭha}- and \pali{raṭṭhuṃ}-substitution.}
\example[0]{dassanīyaṃ daṭṭhabbaṃ (\paliroot{disa} + tabba) \\{\upshape= [It is] worthy to be seen, hence \pali{daṭṭhabba}.}}
\example{daṭṭhuṃ (\paliroot{disa} + tuṃ) vihāraṃ gacchanti samaṇānaṃ \\{\upshape= [They] go to the temple to see the ascetics.}}

\head{573}{573, 626. sādisantapucchabhanjahansādīhi ṭṭho.}
\headtrans{After [roots] ending with \pali{sa} and so on such as \pali{puccha}, \pali{bhanja}, \pali{hansa} and so on, [there is] \pali{ṭṭha}-substitution [for \pali{ta}-paccaya] together with the preceding [consonant].}
\sutdef{sakārantapucchabhanjahansaiccevamādīhi dhātūhi tapaccayassa sahādibyañjanena ṭṭhādeso hoti ṭhāne.}
\sutdeftrans{There is a substitution of \pali{ṭṭha} together with the preceding consonant for \pali{ta}-paccaya after roots ending with \pali{sa} and so on such as \pali{puccha}, \pali{bhanja}, \pali{hansa} and so on in a suitable place.}
\example[0]{tuṭṭho (\paliroot{tusa} + ta) \\{\upshape= [One] was satisfied.}}
\example[0]{ahinā daṭṭho (\paliroot{daṃsa} + ta) naro \\{\upshape= A man was bitten by a snake.}}
\example[0]{mayā puṭṭho (\paliroot{puccha} + ta) \\{\upshape= Asking was done by me.}}
\example[0]{bhaṭṭho (\paliroot{bhañja} + ta) \\{\upshape= [It] was broken.}}
\example[0]{pabhaṭṭho (pa + \paliroot{bhañja} + ta) \\{\upshape= [It] was destroyed.}}
\example[0]{haṭṭho (\paliroot{hasa} + ta) \\{\upshape= [One] was joyful.}}
\example[0]{pahaṭṭho (pa + \paliroot{hasa} + ta) \\{\upshape= [One] was delighted.}}
\example{yiṭṭho (\paliroot{yaja} + ta) \\{\upshape= [It] was sacrificed.}}
\transnote{Note that the correct root form of \pali{bhanja} is \paliroot{bhañja}, and for \pali{hansa} it is \paliroot{hasa}. For the last instance, see also \hyperref[sut:610]{Kacc 610}.}

\head{574}{574, 613. vasato uṭṭha.}
\headtrans{After \pali{vasa}, [there is] \pali{uṭṭha}-substitution.}
\sutdef{vasaiccetamhā dhātumhā takārapaccayassa sahādibyañjanena uṭṭhādeso hoti ṭhāne.}
\sutdeftrans{There is a substitution of \pali{uṭṭha} together with the preceding consonant for \pali{ta}-paccaya after the root \pali{vasa} in a suitable place.}
\example{vassaṃ vuṭṭho (\paliroot{vasa} + ta) \\{\upshape= [One] stayed [in] the rain retreat.}}

\head{575}{575, 614. vassa vā vu.}
\headtrans{For \pali{vasa} sometimes \pali{va} [becomes] \pali{u} [because of \pali{ta}-paccaya].}
\sutdef{vasasseva dhātussa tapaccaye pare vakārassa ukārādeso hoti vā.}
\sutdeftrans{There is a substitution of \pali{u} for \pali{va} of the root \pali{vasa} sometimes bacause of \pali{ta}-paccaya behind.}
\example{vusitaṃ/uṭṭhaṃ/vuṭṭhaṃ (\paliroot{vasa} + ta) brahmacariyaṃ \\{\upshape= The religious life was completed/lived.}}

\head{576}{576, 607. dhaḍhabhahehi dhaḍhā ca.}
\headtrans{After [roots ending with] \pali{dha}, \pali{ḍha}, \pali{bha}, and \pali{ha}, [there are] \pali{dha}- and \pali{ḍha}-substitution also.}
\sutdef{dhaḍhabhahaiccevamantehi dhātūhi takārapaccayassa yath\-ākkamaṃ dhaḍhādesā honti.}
\sutdeftrans{There are substitutions of \pali{dha} and \pali{ḍha} for \pali{ta}-paccaya after roots ending with \pali{dha}, \pali{ḍha}, \pali{bha}, and \pali{ha} respectively.}
\example[0]{buddho (\paliroot{budha} + ta) bhagavā \\{\upshape= Buddha, the Blessed One}}
\example[0]{vaḍḍho (\paliroot{vaḍḍha} + ta) bhikkhu \\{\upshape= The monk prospered.}}
\example[0]{laddhaṃ (\paliroot{labha} + ta) me patthacīvaraṃ \\{\upshape= Bowl and robe was obtained by me.}}
\example{agginā daḍḍhaṃ (\paliroot{daha} + ta) vanaṃ \\{\upshape= The forest was burnt by fire.}}

\head{577}{577, 628. bhanjato ggo ca.}
\headtrans{After \pali{bhanja}, [there is] \pali{gga}-substitution also.}
\sutdef{bhanjato dhātumhā takārapaccayassa ggoādeso hoti sahādi\-byañjanena.}
\sutdeftrans{There is a substitution of \pali{gga} for \pali{ta}-paccaya after the root \pali{bhanja} together with the preceding consonant.}
\example{bhaggo (\paliroot{bhañja} + ta) \\{\upshape= broken}}
\transnote{The root should be seen as \paliroot{bhañja}.}

\head{578}{578, 560. bhujādīnamanto no dvi ca.}
\headtrans{[There is] no ending of \pali{bhuja} and so on, also [there is] a duplication [of \pali{ta}-paccaya].}
\sutdef{bhujaiccevamādīnaṃ dhātūnaṃ anto no hoti, tapaccayassa ca dvibhāvo hoti.}
\sutdeftrans{There is no ending of roots such as \pali{bhuja} and so on, also there is a duplication of \pali{ta}-paccaya.}
\example[0]{bhutto, bhuttavā, bhuttāvī (\paliroot{bhuja} + ta/tavantu/tāvī) \\{\upshape= having eaten}}
\example[0]{catto (\paliroot{caja} + ta) \\{\upshape= given up}}
\example[0]{satto (\paliroot{saja} + ta) \\{\upshape= attached}}
\example[0]{ratto (\paliroot{rañja} + ta) \\{\upshape= delighted in}}
\example[0]{yutto (\paliroot{yuja} + ta) \\{\upshape= applied}}
\example{vivitto (vi + \paliroot{vica} + ta) \\{\upshape= detached}}

\head{579}{579, 629. vaca vāvu.}
\headtrans{[For] \pali{vaca}, sometimes \pali{va} [becomes] \pali{u}.}
\sutdef{vacaiccetassa dhātvassa vakārassa ukārādeso hoti anto cakāro no hoti, tapaccayassa ca dvebhāvo hoti vā.}
\sutdeftrans{There is a substitution of \pali{u} for \pali{va} of the root \pali{vaca}, [also] there is no the ending \pali{ca}. Also there is a duplication of \pali{ta}-paccaya sometimes.}
\example{vuttaṃ/uttaṃ (\paliroot{vaca} + ta) bhagavatā \\{\upshape= said by the Blessed One}}

\head{580}{580, 630. gupādīnañca.}
\headtrans{For \pali{gupa} and so on also, [there is no ending].}
\sutdef{gupaiccevamādīnaṃ dhātūnaṃ anto ca byañjano no hoti, tapaccayassa ca dvebhāvo hoti.}
\sutdeftrans{There is also no ending consonant of roots such as \pali{gupa} and so on, also there is a duplication of \pali{ta}-paccaya.}
\example[0]{sugutto (su + \paliroot{gupa} + ta) \\{\upshape= well-guarded}}
\example[0]{catto (\paliroot{caja} + ta) \\{\upshape= given up}}
\example[0]{litto (\paliroot{lipa} + ta) \\{\upshape= smeared}}
\example[0]{santatto (saṃ + \paliroot{tapa} + ta) \\{\upshape= scorched}}
\example[0]{utto (\paliroot{vaca} + ta) \\{\upshape= said}}
\example[0]{vivitto (vi + \paliroot{vica} + ta) \\{\upshape= detached}}
\example{sitto (\paliroot{sica} + ta) \\{\upshape= sprinkled}}

\head{581}{581, 616. tarādīhi iṇṇo.}
\headtrans{After \pali{tara} and so on, [there is] \pali{iṇṇa}-substitution.}
\sutdef{taraiccevamādīhi dhātūhi tapaccayassa iṇṇādeso hoti, anto ca byañjano no hoti.}
\sutdeftrans{There is a substitution of \pali{iṇṇa} for \pali{ta}-paccaya after roots such as \pali{tara} and so on, also there is no ending consonant.}
\example[0]{tiṇṇo (\paliroot{tara} + ta) \\{\upshape= crossed over}}
\example[0]{uttiṇṇo (u + \paliroot{tara} + ta) \\{\upshape= crossed over}}
\example[0]{sampuṇṇo (saṃ + \paliroot{pūra} + ta) \\{\upshape= well-filled}}
\example[0]{tuṇṇo (\paliroot{tura} + ta) \\{\upshape= hasted}}
\example[0]{parijiṇṇo (pari + \paliroot{jara} + ta) \\{\upshape= degenerated}}
\example{ākiṇṇo (ā + \paliroot{kira} + ta) \\{\upshape= scattered around}}
\transnote{In the text, all these examples are described by present-tense verbs. They look out of place, but this may remind us that sometimes present meaning is intended.}

\head{582}{582, 631. bhidādito innaannaīṇā vā.}
\headtrans{After \pali{bhida} and so on, [there are] \pali{inna}-, \pali{anna}-, and \pali{īṇa}-substitution sometimes.}
\sutdef{bhidiiccevamādīhi dhātūhi tapaccayassa innaannaīṇādesā honti vā, anto ca byañjano no hoti.}
\sutdeftrans{There are substitutions of \pali{inna}, \pali{anna}, and \pali{īṇa} for \pali{ta}-paccaya after roots such as \pali{bhidi} and so on sometimes, also there is no ending consonant.}
\example[0]{bhinno (\paliroot{bhida} + ta) \\{\upshape= broken}}
\example[0]{chinno (\paliroot{chida} + ta) \\{\upshape= cut}}
\example[0]{ucchinno (u + \paliroot{chida} + ta) \\{\upshape= cut off}}
\example[0]{dinno (\paliroot{dā} + ta) \\{\upshape= given}}
\example[0]{nisinno (ni + \paliroot{sada} + ta) \\{\upshape= sat}}
\example[0]{suchanno (su + \paliroot{chada} + ta) \\{\upshape= well-covered}}
\example[0]{khinno (\paliroot{khī} + ta) \\{\upshape= exhausted}}
\example[0]{runno (\paliroot{ruda} + ta) \\{\upshape= lamented}}
\example{khīṇā (\paliroot{khī} + ta) \\{\upshape= worn away}}
\transnote{Unlike the previous sutta, the examples are described by a variety of verb forms in the text. This can show us that \pali{ta}-paccaya is versatile in use. From the examples, \pali{khinna} and \pali{khīṇa} mean the same thing, but they may have different idiomatic uses. From a Thai source, \pali{runna} is described by \paliroot{rudha} instead, hence `obstructed/prevented.'}

\head{583}{583, 617. susapacasakato kkhakkā ca.}
\headtrans{After \pali{susa}, \pali{paca}, \pali{saka} and so on, [there are] \pali{kkha}- and \pali{kka}-substitution also.}
\sutdef{susapacasakaiccevamādīhi dhātūhi tapaccayassa kkhakkā\-desā honti, anto ca byañjano no hoti.}
\sutdeftrans{There are substitutions of \pali{kkha} and \pali{kka} for \pali{ta}-paccaya after roots such as \pali{susa}, \pali{paca}, \pali{saka} and so on, also there is no ending consonant.}
\example[0]{sukkhaṃ (\paliroot{susa} + ta) \\{\upshape= dried}}
\example[0]{pakkaṃ (\paliroot{paca} + ta) \\{\upshape= cooked/ripe}}
\example{sakko (\paliroot{saka} + ta) \\{\upshape= being capable}}

\head{584}{584, 618. pakkamādīhi nto ca.}
\headtrans{After \pali{pakkama} and so on, [there is] \pali{nta}-substitution also.}
\sutdef{pakkamaiccevamādīhi dhātūhi tapaccayassa ntoādeso hoti, anto ca no hoti.}
\sutdeftrans{There is a substitution of \pali{nta} for \pali{ta}-paccaya after roots such as \pali{pakkama} and so on, also there is no ending [consonant].}
\example[0]{pakkanto (pa + \paliroot{kamu} + ta) \\{\upshape= gone away}}
\example[0]{vibbhanto (vi + \paliroot{bhamu} + ta) \\{\upshape= wrongly-turned/strayed}}
\example[0]{saṅkanto (saṃ + \paliroot{kamu} + ta) \\{\upshape= moved over}}
\example[0]{khanto (\paliroot{khamu} + ta) \\{\upshape= endured}}
\example[0]{santo (\paliroot{sama} + ta) \\{\upshape= calmed}}
\example[0]{danto (\paliroot{dama} + ta) \\{\upshape= trained}}
\example{vanto (\paliroot{vama} + ta) \\{\upshape= ejected}}

\head{585}{585, 619. janādīnamā timhi ca.}
\headtrans{[The ending of] \pali{jana} and so on [becomes] \pali{ā} because of \pali{ti}-paccaya [and \pali{ta}] also.}
\sutdef{janaiccevamādīnaṃ dhātūnaṃ antassa byañjanassa āttaṃ hoti tapaccaye pare, timhi ca.}
\sutdeftrans{The ending consonant of roots such as \pali{jana} and so on becomes \pali{ā} because of \pali{ta}-paccaya behind, because of \pali{ti}[-paccaya] also.}
\example[0]{jāto (\paliroot{jana} + ta) \\{\upshape= born}}
\example{jāti (\paliroot{jana} + ti) \\{\upshape= birth}}

\head{586}{586, 600. gamakhanahanaramādīnamanto.}
\headtrans{The ending of \pali{gama}, \pali{khana}, \pali{hana}, \pali{ramu} and so on [gets elided].}
\sutdef{gamakhanahanaramuiccevamādīnaṃ dhātūnaṃ anto byañjano no hoti vā tapaccaye pare timhi ca.}
\sutdeftrans{There is no ending consonant of roots such as \pali{gama}, \pali{khana}, \pali{hana}, \pali{ramu} and so on sometimes because of \pali{ta}-paccaya behind, \pali{ti} also.}
\example[0]{sugato (su + \paliroot{gamu} + ta) \\{\upshape= well-gone}}
\example[0]{sugati (su + \paliroot{gamu} + ti) \\{\upshape= a good destination}}
\example[0]{khataṃ (\paliroot{khana} + ta) \\{\upshape= dug}}
\example[0]{khati (\paliroot{khana} + ti) \\{\upshape= digging}}
\example[0]{upahataṃ (upa + \paliroot{hana} + ta) \\{\upshape= injured}}
\example[0]{upahati (upa + \paliroot{hana} + ti) \\{\upshape= hurting}}
\example[0]{rato (\paliroot{ramu} + ta) \\{\upshape= delighted}}
\example[0]{rati (\paliroot{ramu} + ti) \\{\upshape= enjoyment}}
\example[0]{mato (\paliroot{mana} + ta) \\{\upshape= understood}}
\example{mati (\paliroot{mana} + ti) \\{\upshape= wisdom}}

\head{587}{587, 632. rakāro ca.}
\headtrans{The \pali{ra} [ending] also [gets elided].}
\sutdef{rakāro ca dhātūnamantabhūto no hoti tapaccaye pare timhi ca.}
\sutdeftrans{There is no \pali{ra} ending of roots because of \pali{ta}-paccaya behind, \pali{ti} also.}
\example[0]{pakato (pa + \paliroot{kara} + ta) \\{\upshape= produced}}
\example[0]{pakati (pa + \paliroot{kara} + ti) \\{\upshape= natural state}}
\example[0]{visato (vi + \paliroot{sara} + ta) \\{\upshape= forgotten}}
\example{visati (vi + \paliroot{sara} + ti) \\{\upshape= forgetfulness}}
\transnote{For \pali{visata} and \pali{visati}, see \pali{vismṛta} and \pali{vismṛti} in MWD.}

\head{588}{588, 620. ṭhāpānamiī ca.}
\headtrans{[The \pali{ā} ending of] \pali{ṭhā} and \pali{pā} [becomes] \pali{i} and \pali{ī} also.}
\sutdef{ṭhāpāiccetesaṃ dhātūnaṃ antassa ākārassa ikāraīkārādesā honti yathāsaṅkhyaṃ tapaccaye pare, timhi ca.}
\sutdeftrans{There are substitutions of \pali{i} and \pali{ī} for the \pali{ā} ending of the roots \pali{ṭhā} and \pali{pā} respectively because of \pali{ta}-paccaya behind, \pali{ti} also.}
\example[0]{ṭhito (\paliroot{ṭhā} + ta) \\{\upshape= stood}}
\example[0]{ṭhiti (\paliroot{ṭhā} + ti) \\{\upshape= stability}}
\example[0]{pīto (\paliroot{pā} + ta) \\{\upshape= drunk (water)}}
\example{pīti (\paliroot{pā} + ti) \\{\upshape= drinking/delight}}

\head{589}{589, 621. hantehi ho hassa ḷo vā adahanahānaṃ.}
\headtrans{After \pali{ha} ending [\pali{ta}-paccaya becomes] \pali{ha}, [then] \pali{ḷa} sometimes for non-\pali{daha} and non-\pali{naha} [roots].}
\sutdef{hakārantehi dhātūhi tapaccayassa hakārādeso hoti, hakārassa dhātvantassa ḷo hoti vā adahanahānaṃ.}
\sutdeftrans{There is a substitution of \pali{ha} for \pali{ta}-paccaya after roots ending with \pali{ha}. The \pali{ha} ending of a root becomes \pali{ḷa} sometimes for non-\pali{daha} and non-\pali{naha} [roots].}
\example[0]{āruḷho (ā + \paliroot{ruha} + ta) \\{\upshape= ascended}}
\example[0]{gāḷho (\paliroot{gāha} + ta) \\{\upshape= immersed}}
\example[0]{bāḷho (\paliroot{baha} + ta) \\{\upshape= intense}}
\example{mūḷho (\paliroot{muha} + ta) \\{\upshape= confused}}

