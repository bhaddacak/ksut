\section{Pañcamakaṇḍa}

\head{607}{607, 578. niggahita saṃyogādi no.}
\headtrans{The [letter] \pali{na} in the front of conjunct consonants [becomes] nigghita.}
\sutdef{saṃyogādibhūto nakāro niggahitamāpajjate.}
\sutdeftrans{The letter \pali{na} in the front of conjunct consonants becomes nigghita.}
\example[0]{raṅgo (\paliroot{ranja} + ṇa)}
\example[0]{bhaṅgo (\paliroot{bhanja} + ṇa)}
\example{saṅgo (\paliroot{sanja} + ṇa)}
\transnote{The letter \pali{na} becomes niggahita by this sutta. Then it changes to the nasal letter of the groups by \hyperref[sut:31]{Kacc 31}. The ending \pali{ja} becomes \pali{ga} by \hyperref[sut:623]{Kacc 623}.}

\head{608}{608, 623. sabbattha ge gī.}
\headtrans{In all cases, \pali{ge} [becomes] \pali{gī}.}
\sutdef{geiccetassa dhātussa gīādeso hoti sabbattha ṭhāne.}
\sutdeftrans{There is a substitution of \pali{gī} for the root \pali{ge} in all cases.}
\example[0]{gītaṃ (\paliroot{ge} + ta) \\{\upshape= recited}}
\example{gāyati (\paliroot{ge} + a + ti) \\{\upshape= [One] sings/recites.}}

\head{609}{609, 484. sadassa sīdattaṃ.}
\headtrans{For \pali{sada}, [there is] \pali{sīda}-substitution.}
\sutdef{sadaiccetassa dhātussa sīdādeso hoti sabbattha ṭhāne.}
\sutdeftrans{There is a substitution of \pali{sīda} for the root \pali{sada} in all cases.}
\example[0]{nisinno (ni + \paliroot{sada} + ta) \\{\upshape= sat}}
\example{nisīdati (ni + \paliroot{sada} + a + ti) \\{\upshape= [One] sits.}}

\head{610}{610, 627. yajassa sarassi ṭṭhe.}
\headtrans{For the vowel of \pali{yaja}, [there is] \pali{i}-substitution because of \pali{ṭṭha}.}
\sutdef{yajaiccetassa dhātussa sarassa ikārādeso hoti ṭṭhe pare.}
\sutdeftrans{There is a substitution of \pali{i} for the vowel of the root \pali{yaja} because of \pali{ṭṭha} behind.}
\example{yiṭṭho (\paliroot{yaja} + ta) \\{\upshape= sacrificed}}
\transnote{With \pali{ta}-paccaya, the root ending becomes \pali{ṭṭha} by \hyperref[sut:573]{Kacc 573}. Then \pali{ya} becomes \pali{yi} by this sutta.}

\head{611}{611, 608. hacatutthānamantānaṃ do dhe.}
\headtrans{For the ending \pali{ha} and the fourth [of the letter groups] of roots, [there is] \pali{da}-substitution because of \pali{dha}.}
\sutdef{hacatutthānaṃ dhātvantānaṃ do ādeso hoti dhe pare.}
\sutdeftrans{There is a substitution of \pali{da} for the ending \pali{ha} and the fourth [of the letter groups] of roots because of \pali{dha} behind.}
\example[0]{sannaddho (saṃ + \paliroot{naha} + ta)}
\example[0]{kuddho (\paliroot{gudha} + ta)}
\example[0]{yuddho (\paliroot{yudha} + ta)}
\example[0]{siddho (\paliroot{sidha} + ta)}
\example[0]{laddho (\paliroot{labha} + ta)}
\example{āraddho (ā + \paliroot{rabha} + ta)}
\transnote{The \pali{dha} mentioned here comes from a substitution of \pali{ta}-paccaya (see \hyperref[sut:576]{Kacc 576}).}

\head{612}{612, 615. ḍo ḍhakāre.}
\headtrans{[For the ending \pali{ha}, there is] \pali{ḍa}-substitution because of \pali{ḍha}.}
\sutdef{hacatutthānaṃ dhātvantānaṃ ḍo ādeso hoti ḍhakāre pare.}
\sutdeftrans{There is a substitution of \pali{ḍa} for the ending \pali{ha} and the fourth [of the letter groups] of roots because of \pali{ḍha} behind.}
\example[0]{daḍḍho (\paliroot{daha} + ta)}
\example{vuḍḍho (\paliroot{vaḍḍha} + ta)}
\transnote{See also \hyperref[sut:576]{Kacc 576}.}

\head{613}{613, 583. gahassa ghara ṇe vā.}
\headtrans{For \pali{gaha}, [there is] \pali{ghara}-substitution because of \pali{ṇa} sometimes.}
\sutdef{gahaiccetassa dhātussa sabbassa gharādeso hoti vā ṇapaccaye pare.}
\sutdeftrans{There is a substitution of \pali{ghara} for the whole root of \pali{gaha} sometimes because of \pali{ṇa}-paccaya behind.}
\example{gharaṃ (\paliroot{gaha} + ṇa) \\{\upshape= a house}}
\transnote{An analytic sentence of this is ``\pali{dabbasambhāraṃ gaṇhātīti gharaṃ}'' ([It] holds building materials, hence `house').}

\head{614}{614, 581. dahassa do ḷaṃ.}
\headtrans{For \pali{daha}, \pali{da} [becomes] \pali{ḷa}.}
\sutdef{dahaiccetassa dhātussa dakāro ḷattamāpajjate vā ṇapaccaye pare.}
\sutdeftrans{The letter \pali{da} of the root \pali{daha} becomes \pali{ḷa} sometimes because of \pali{ṇa}-paccaya behind.}
\example{pariḷāho (pari + \paliroot{daha} + ṇa)}

\head{615}{615, 586. dhātvantassa lopo kvimhi.}
\headtrans{For the root's ending, [there is] an elision because of \pali{kvi}-paccaya.}
\sutdef{dhātvantassa byañjanassa lopo hoti kvimhi paccaye pare.}
\sutdeftrans{There is an elision of the ending consonant of a root because of \pali{kvi}-paccaya behind.}
\example[0]{bhujena gacchatīti bhujago (bhuja + \paliroot{gamu} + kvi) \\{\upshape= [It] goes by coil/bending, hence \pali{bhujaga} (snake).}}
\example[0]{urena gacchatīti urago (ura + \paliroot{gamu} + kvi) \\{\upshape= [It] goes by the chest, hence \pali{uraga} (snake).}}
\example[0]{turago (tura + \paliroot{gamu} + kvi) \\{\upshape= [It goes quickly, hence] \pali{turaga} (horse).}}
\example{saṅkho (saṃ + \paliroot{khana} + kvi) \\{\upshape= [It digs well, hence] \pali{saṅkha} (conch).}}

\head{616}{616, 587. vidante ū.}
\headtrans{At the end of \pali{vida}, [there is] \pali{ū}-insertion.}
\sutdef{vidaiccetassa dhātussa ante ūkārāgamo hoti kvimhi paccaye pare.}
\sutdeftrans{There is an insertion of \pali{ū} at the end of the root \pali{vida} because of \pali{kvi}-paccaya behind.}
\example{lokaṃ vidati jānātīti lokavidū (loka + \paliroot{vida} + kvi) \\{\upshape= [One] understands [and] knows the world, hence \pali{lokavidū}.}}

\head{617}{617, 633. namakarānamantānaṃ niyuttatamhi.}
\headtrans{For the [root's] ending of \pali{na}, \pali{ma}, \pali{ka}, or \pali{ra}, [there is] no [elision] because of \pali{ta} applied with \pali{i}.}
\sutdef{nakāramakārakakārarakārānaṃ dhātvantānaṃ lopo na hoti ikārayutte tapaccaye pare.}
\sutdeftrans{There is no elision for the root's ending of \pali{na}, \pali{ma}, \pali{ka}, or \pali{ra} because of \pali{ta}-paccaya behind applied with \pali{i}[-insertion].}
\example[0]{hanituṃ (\paliroot{hana} + tuṃ)}
\example[0]{gamito (\paliroot{gamu} + ta)}
\example[0]{ramito (\paliroot{rama} + ta)}
\example[0]{sakito (\paliroot{saka} + ta)}
\example[0]{sarito (\paliroot{sara} + ta)}
\example{karitvā (\paliroot{kara} + tvā)}

\head{618}{618, 571. na kagattaṃ cajā ṇvumhi.}
\headtrans{The letters \pali{ca} and \pali{ja} do not become \pali{ka} and \pali{ga} because of \pali{ṇvu}.}
\sutdef{cakārajakārā kakāragakārattaṃ nāpajjante ṇvumhi paccaye pare.}
\sutdeftrans{The letters \pali{ca} and \pali{ja} do not become \pali{ka} and \pali{ga} because of \pali{ṇvu}-paccaya behind.}
\example[0]{pācako (\paliroot{paca} + ṇvu) \\{\upshape= a cook}}
\example{yājako (\paliroot{yaja} + ṇvu) \\{\upshape= a sacrificer}}
\transnote{See also \hyperref[sut:623]{Kacc 623} below.}

\head{619}{619, 573. karassa ca tattaṃ tusmiṃ.}
\headtrans{[The ending of] \pali{kara} also [becomes] \pali{ta} because of \pali{tu}.}
\sutdef{karaiccetassa dhātussa antassa rakārassa takārattaṃ hoti tupaccaye pare.}
\sutdeftrans{There is [a substitution of] \pali{ta} for the ending \pali{ra} of the root \pali{kara} because of \pali{tu}-paccaya behind.}
\example{kattā (\paliroot{kara} + tu)}

\head{620}{620, 549. tuṃtunatabbesu vā.}
\headtrans{Because of \pali{tuṃ}-, \pali{tuna}-, and \pali{tabba}-paccaya sometimes, [also \pali{ra} of \pali{kara} becomes \pali{ta}].}
\sutdef{karaiccetassa dhātussa antassa rakārassa takārattaṃ hoti vā tuṃtunatabbaiccetesu paccayesu.}
\sutdeftrans{There is [a substitution of] \pali{ta} for the ending \pali{ra} of the root \pali{kara} sometimes because of \pali{tuṃ}-, \pali{tuna}-, and \pali{tabba}-paccaya behind.}
\example[0]{kattuṃ = kātuṃ (\paliroot{kara} + tuṃ)}
\example[0]{kattuna = kātuna (\paliroot{kara} + tuna)}
\example{kattabbaṃ = kātabbaṃ (\paliroot{kara} + tabba)}

\head{621}{621, 553. kāritaṃ viya ṇānubandho.}
\headtrans{The \pali{ṇa}-anubandha [is] like the kārita-paccaya.}
\sutdef{ṇakārānubandho paccayo kāritaṃ viya daṭṭhabbo vā.}
\sutdeftrans{The paccaya having \pali{ṇa}-anubandha should be seen like the kārita-paccaya (causative maker) sometimes.}
\example[0]{dāho (\paliroot{daha} + ṇa)}
\example[0]{deho (\paliroot{diha} + ṇa)}
\example[0]{vāho (\paliroot{vaha} + ṇa)}
\example[0]{bāho (\paliroot{baha} + ṇa)}
\example[0]{cāgo (\paliroot{caja} + ṇa)}
\example[0]{vāro (\paliroot{vara} + ṇa)}
\example[0]{cāro (\paliroot{cara} + ṇa)}
\example[0]{parikkhāro (pari + \paliroot{kara} + ṇa)}
\example[0]{dāyako (\paliroot{dā} + ṇvu)}
\example[0]{nāyako (\paliroot{nī} + ṇvu)}
\example[0]{lāvako (\paliroot{lu} + ṇvu)}
\example[0]{bhāvako (\paliroot{bhū} + ṇvu)}
\example[0]{kārī (\paliroot{kara} + ṇī)}
\example[0]{ghātī (\paliroot{hana} + ṇī)}
\example{dāyī (\paliroot{dā} + ṇī)}
\transnote{The similarity here seems to mean the vuddhi-strength transformation.}

\head{622}{622, 570. anakā yuṇvūnaṃ.}
\headtrans{[There are] \pali{ana}- and \pali{aka}-substitution for \pali{yu}- and \pali{ṇvu}-paccaya.}
\sutdef{yuṇvuiccetesaṃ paccayānaṃ anaakaiccete ādesā honti.}
\sutdeftrans{There are substitutions of \pali{ana} and \pali{aka} for \pali{yu}- and \pali{ṇvu}-paccaya [respectively].}
\example[0]{nandanaṃ (\paliroot{nanda} + yu)}
\example{kārako (\paliroot{kara} + ṇvu)}

\head{623}{623, 554. kagā cajānaṃ.}
\headtrans{[There are] \pali{ka}- and \pali{ga}-substitution for \pali{ca}- and \pali{ja}-ending.}
\sutdef{cajaiccetesaṃ dhātvantānaṃ kakāragakārādesā honti ṇānu\-bandhe paccaye pare.}
\sutdeftrans{There are substitutions of \pali{ka} and \pali{ga} for the root's endings of \pali{ca} and \pali{ja} [respectively] because of paccayas behind with \pali{ṇa}-anubandha.}
\example[0]{pāko (\paliroot{paca} + ṇa)}
\example{yogo (\paliroot{yuja} + ṇa)}

\setcounter{table}{2}
\begin{longtable}{%
		>{\raggedright\arraybackslash}p{0.25\linewidth}%
		>{\raggedright\arraybackslash}p{0.42\linewidth}}
\caption{Roots having particular treatments}\label{tab:roots-kita-irr}\\
\toprule
\upshape\bfseries Root & \bfseries Suttas \\ \midrule
\endfirsthead
\multicolumn{2}{c}{\footnotesize\tablename\ \thetable: Roots having particular treatments (contd\ldots)}\\
\toprule
\upshape\bfseries Root & \bfseries Suttas \\ \midrule
\endhead
\bottomrule
\ltblcontinuedbreak{2}
\endfoot
\bottomrule
\endlastfoot
%
\paliroot{kamu}, \ldots & \hyperref[sut:584]{584} \\
\paliroot{kara} & \hyperref[sut:542]{542}, \hyperref[sut:594]{594}, \hyperref[sut:595]{595}, \hyperref[sut:619]{619}, \hyperref[sut:620]{620} \\
\paliroot{khana}, \ldots & \hyperref[sut:586]{586}, \hyperref[sut:596]{596} \\
\paliroot{gamu} & \hyperref[sut:534]{534}, \hyperref[sut:570]{570}, \hyperref[sut:586]{586}, \hyperref[sut:596]{596} \\
\paliroot{gamu} (\pali{ā}-) & \hyperref[sut:569]{569} \\
\paliroot{gupa}, \ldots & \hyperref[sut:580]{580} \\
\paliroot{ge} & \hyperref[sut:608]{608} \\
\paliroot{jana}, \ldots & \hyperref[sut:585]{585} \\
\paliroot{ji} & \hyperref[sut:558]{558} \\
\paliroot{ṭhā} & \hyperref[sut:588]{588} \\
\paliroot{tara}, \ldots & \hyperref[sut:581]{581} \\
\paliroot{dā} (\pali{dadā}) & \hyperref[sut:526]{526} \\
\paliroot{disa} & \hyperref[sut:572]{572}, \hyperref[sut:599]{599} \\
\paliroot{dhara}, \ldots & \hyperref[sut:531]{531} \\
\paliroot{pā} & \hyperref[sut:567]{567}, \hyperref[sut:588]{588} \\
\paliroot{puccha}, \ldots & \hyperref[sut:573]{573} \\
\paliroot{bhañja} & \hyperref[sut:577]{577} \\
\paliroot{bhikkha}, \ldots & \hyperref[sut:535]{535} \\
\paliroot{bhida}, \ldots & \hyperref[sut:582]{582} \\
\paliroot{bhuja}, \ldots & \hyperref[sut:578]{578} \\
\paliroot{bhū} & \hyperref[sut:543]{543} \\
\paliroot{māna}, \ldots & \hyperref[sut:568]{568} \\
\paliroot{yaja} & \hyperref[sut:610]{610} \\
\paliroot{ranja} & \hyperref[sut:590]{590} \\
\paliroot{vaca} & \hyperref[sut:579]{579} \\
\paliroot{vasa} & \hyperref[sut:574]{574}, \hyperref[sut:575]{575} \\
\paliroot{vida} & \hyperref[sut:616]{616} \\
\paliroot{sada} & \hyperref[sut:609]{609} \\
\paliroot{sāsa}, \ldots & \hyperref[sut:566]{566}, \hyperref[sut:572]{572} \\
\paliroot{supa} & \hyperref[sut:559]{559} \\
\paliroot{susa}, \ldots & \hyperref[sut:583]{583} \\
\paliroot{hana} & \hyperref[sut:536]{536}, \hyperref[sut:591]{591}, \hyperref[sut:592]{592}, \hyperref[sut:596]{596} \\
\paliroot{hana} (\pali{saṃ}-) & \hyperref[sut:538]{538} \\
\end{longtable}

