\markboth{}{Preface}
\cleardoublepage
\phantomsection
\addcontentsline{toc}{chapter}{Preface}
\chapter*{Preface}

When my critical works have been well settled, I have started thinking about new projects. It seems that after I finished the Pāli learning books and program, my motivation to do difficult work subsides considerably. I have spent more time meditating recently, as well as reading books I wished I had time to finish them.

Still, I feel that something is unfinished.

I have no interest in translating a mass of texts in the Pāli canon because it is far better to encourage people to read the Pāli canon by themselves using good guidelines and tools that I had spent a lot of energy and time to make. With the knowledge I put into my books and capacities of the program, it is possible that Pāli learners can read any part of texts in the canon, with an effort, of course (`possible' does not mean `effortless').

However, when we consider the traditional grammar books, the situation is different. Without a guide, it is very difficult that new learners will get a good understanding about the textbooks. As a result, they will get hardship to gain deep understanding in Pāli language. This means it would be better if we have an easily accessible translation of grammar textbooks.

Unfortunately, translating traditional grammar books is admittedly a painstaking task that scares away knowledgeable authors, particularly those who get no financial gain or a kind of promotions from the work. When I have idle time and contemplate over the issue, it comes up that the task is really worth taking seriously. It is not so critical as my previous works, but important nonetheless to the field of Pāli studies as a whole.

Here comes my new project on textbook translation. The initial feasible plan is that the grammatical suttas of the main three schools of Pāli grammar will be translated into English using my style of rendering (to keep the structure as much as possible). This will start with Kaccāyana first, then Saddanīti and Moggallāna respectively. At the end of the day, I hope I can link all of them together to gain a comprehensive picture of Pāli grammar.

By `suttas' I mean that only the formulas and descriptions (\pali{vutti}) will be taken into consideration. Examples will be picked up just enough to illustrate the points. For Saddanīti, this means only the sutta part (Chapter 1--7 of Suttamālā) is taken.

Like other books of mine, I will treat the products of the project as public properties, hence open-source products. I have a hope that there will be a community that can continue and enhance my works. But when the community will catch up with my works (or whether there will be such a community at all) is not my concern.

Another reason that I initiate this project is my eyes now get deteriorated quickly when used for a long time. This is a sign telling us that what should be done should be taken right away before it is too late.

For the readers who are new to Pāli grammar, you are supposed to be familiar with the basic ideas first, otherwise you will be disoriented, barely able to see the big picture and grasp the essence. A recommended primer of the subject is \emph{Pāli for New Learners}.\footnote{\url{bhaddacak.github.io/pnl}}

In the process of writing books related to this project, I would like to thank Antonio Costanzo for providing me a digitized text of Kātantra in Roman script.
