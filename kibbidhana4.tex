\section{Catutthakaṇḍa}

\head{590}{590, 579. ṇamhi ranjassa jo bhāvakaraṇesu.}
\headtrans{Because of \pali{ṇa}-paccaya, [there is] \pali{ja}-substitution for \pali{ranja} in the senses of state and cause.}
\sutdef{ṇamhi paccaye pare ranjaiccetassa dhātussa antabhūtassa njakārassa jo ādeso hoti bhāvakaraṇesu.}
\sutdeftrans{There is a substitution of \pali{ja} for \pali{nja} ending of the root \pali{ranja} because of \pali{ṇa}-paccaya behind in the senses of state and cause.}
\example[0]{rañjanaṃ rāgo (\paliroot{ranja} + ṇa) \\{\upshape= Finding pleasure, hence \pali{rāga} (lust).}}
\example{rañjanti etenāti rāgo (\paliroot{ranja} + ṇa) \\{\upshape= [They] find pleasure with that, hence \pali{rāga} (lust).}}
\transnote{We do not see \pali{ja} here because it turns into \pali{ga} by \hyperref[sut:623]{Kacc 623} (\pali{kagā cajānaṃ}).}

\head{591}{591, 544. hanassa ghāto.}
\headtrans{For \pali{hana}, [there is] \pali{ghāta}-substitution [because of \pali{ṇa}-paccaya].}
\sutdef{hanaiccetassa dhātussa sabbassa ghātādeso hoti ṇamhi paccaye pare.}
\sutdeftrans{There is a substitution of \pali{ghāta} for the whole root of \pali{hana} because of \pali{ṇa}-paccaya behind.}
\example[0]{upahanatīti upaghāto (upa + \paliroot{hana} + ṇa) \\{\upshape= [One] destroys, hence \pali{upaghāta} (destroyer).}}
\example{gāvo hanatīti goghātako (go + \paliroot{hana} + ṇa) \\{\upshape= [One] kills cows, hence \pali{goghātaka} (butcher).}}

\head{592}{592, 503. vadho vā sabbattha.}
\headtrans{[There is] \pali{vadha}-substitution [for \pali{hana}] sometimes in all cases.}
\sutdef{hanaiccetassa dhātussa vadhādeso hoti vā sabbattha ṭhānesu.}
\sutdeftrans{There is a substitution of \pali{vadha} for the root \pali{hana} sometimes in all cases.}
\example[0]{hanatīti vadho/vadhako (\paliroot{hana} + ṇa/ṇvu) \\{\upshape= [One] kills, hence \pali{vadha/vadhaka} (killer).}}
\example{avadhi = ahani (\paliroot{hana} + ī) \\{\upshape= [One] killed.}}
\transnote{The last example shows that not only in nouns does the substitution occur, but also in verbs.}

\head{593}{593, 564. ākārantānamāyo.}
\headtrans{For \pali{ā}-ending [roots, there is] \pali{āya}-substitution.}
\sutdef{ākārantānaṃ dhātūnaṃ antassa ākārassa āyādeso hoti ṇamhi paccaye pare.}
\sutdeftrans{There is a substitution of \pali{āya} for \pali{ā} ending of roots ending with \pali{ā} because of \pali{ṇa}-paccaya behind.}
\example[0]{dadātīti dāyako (\paliroot{dā} + ṇvu) \\{\upshape= [One] gives, hence \pali{dāyaka} (giver).}}
\example[0]{dānaṃ dātuṃ sīlaṃ yassāti dānadāyī (dāna + \paliroot{dā} + ṇī) \\{\upshape= Whoever [has] a habit to give alms, hence \pali{dānadāyī} (charitable giver).}}
\example[0]{majjaṃ dātuṃ sīlaṃ yassāti majjadāyī (majja + \paliroot{dā} + ṇī) \\{\upshape= Whoever [has] a habit to give an intoxicant, hence \pali{dānadāyī} (intoxicant giver).}}
\example{nagaraṃ yātuṃ sīlaṃ yassāti nagarayāyī (nagara + \paliroot{yā} + ṇī) \\{\upshape= Whoever [has] a habit to go to town, hence \pali{nagarayāyī} (town-goer).}}
\transnote{The \pali{ṇa}-paccaya mentioned here covers all paccayas having \pali{ṇa}-anubandha.}

\head{594}{594, 582. purasamupaparīhi karotissa khakharā vā tapaccayesu ca.}
\headtrans{After \pali{pura}, \pali{saṃ}, \pali{upa}, and \pali{pari}, [there are] \pali{kha}- and \pali{khara}-substitution for \pali{kara} sometimes because of \pali{ta}-paccaya also.}
\sutdef{purasaṃupapariiccetehi karotissa dhātussa khakharādesā honti vā tapaccaye pare, ṇamhi ca.}
\sutdeftrans{There are substitutions of \pali{kha} and \pali{khara} for the root \pali{kara} after \pali{pura}, \pali{saṃ}, \pali{upa}, and \pali{pari} sometimes because of \pali{ta}-paccaya behind, \pali{ṇa}[-paccaya] also.}
\example[0]{pure karīyatīti purakkhato (pura + \paliroot{kara} + ta) \\{\upshape= [One] is made to the front, hence \pali{purakkhata} (honored).}}
\example[0]{sammā karīyatīti saṅkhato (saṃ + \paliroot{kara} + ta) \\{\upshape= [It] is made proper, hence \pali{saṅkhata} (constructed).}}
\example[0]{upagantvā karīyatīti upakkhato (upa + \paliroot{kara} + ta) \\{\upshape= Having been approached, [it] is made, hence \pali{upakkhata} (undertaken).}}
\example[0]{parisamantato karotīti parikkhāro (pari + \paliroot{kara} + ṇa) \\{\upshape= [One] does from all-around, hence \pali{parikkhāra} (requisite).}}
\example{saṃkarīyatīti saṅkhāro (saṃ + \paliroot{kara} + ṇa) \\{\upshape= [It] is made together, hence \pali{saṅkhāra} (construction).}}

\head{595}{595, 637. tavetunādīsu kā.}
\headtrans{Because of \pali{tave}-, \pali{tuna}-paccaya and so on, [there is] \pali{kā}-substitution [for \pali{kara}].}
\sutdef{tavetunaiccevamādīsu paccayesu karotissa dhātussa kāādeso hoti vā.}
\sutdeftrans{There is a substitution of \pali{kā} for the root \pali{kara} because of \pali{tave}-, \pali{tuna}-paccaya and so on.}
\example[0]{kātave (\paliroot{kara} + tave)}
\example[0]{kātuṃ = kattuṃ (\paliroot{kara} + tuṃ)}
\example{kātuna = kattuna (\paliroot{kara} + tuna)}

\head{596}{596, 551. gamakhanahanādīnaṃ tuṃtabbādīsu na.}
\headtrans{For \pali{gama}, \pali{khana}, \pali{hana} and so on, because of \pali{tuṃ}-, \pali{tabba}-paccaya and so on, [there is] \pali{na}-substitution.}
\sutdef{gamakhanahanaiccevamādīnaṃ dhātūnaṃ antassa nakāro hoti vā tuṃtabbādīsu paccayesu.}
\sutdeftrans{There is [a substitution of] \pali{na} for the ending of roots such as \pali{gama}, \pali{khana}, \pali{hana} and so on sometimes because of \pali{tuṃ}-, \pali{tabba}-paccaya and so on.}
\example[0]{gantuṃ = gamituṃ (\paliroot{gamu} + tuṃ)}
\example[0]{gantabbaṃ = gamitabbaṃ (\paliroot{gamu} + tabba)}
\example[0]{khantuṃ = khanituṃ (\paliroot{khana} + tuṃ)}
\example[0]{khantabbaṃ = khanitabbaṃ (\paliroot{khana} + tabba)}
\example[0]{hantuṃ = hanituṃ (\paliroot{hana} + tuṃ)}
\example[0]{hantabbaṃ = hanitabbaṃ (\paliroot{hana} + tabba)}
\example[0]{mantuṃ = manituṃ (\paliroot{mana} + tuṃ)}
\example{mantabbaṃ = manitabbaṃ (\paliroot{mana} + tabba)}

\head{597}{597, 641. sabbehi tunādīnaṃ yo.}
\headtrans{After all [roots], for \pali{tuna}-paccaya and so on, [there is] \pali{ya}-substitution.}
\sutdef{sabbehi dhātūhi tunādīnaṃ paccayānaṃ yakārādeso hoti vā.}
\sutdeftrans{There is a substitution of \pali{ya} for \pali{tuna}-paccaya and so on after all roots sometimes.}
\example[0]{abhivandiya = abhivanditvā (abhi + \paliroot{vanda} + tvā)}
\example[0]{ohāya = ohitvā (o + \paliroot{hā} + tvā)}
\example[0]{upanīya = upanetvā (upa + \paliroot{nī} + tvā)}
\example[0]{passiya = passitvā (\paliroot{disa} + tvā)}
\example[0]{uddissa = uddisitvā (u + \paliroot{disa} + tvā)}
\example{ādāya = ādiyitvā (ā + \paliroot{dā} + tvā)}

\head{598}{598, 643. canantehi raccaṃ.}
\headtrans{After [roots ending with] \pali{ca} and \pali{na}, [there is] \pali{racca}-substitu\-tion.}
\sutdef{cakāranakārantehi dhātūhi tunādīnaṃ paccayānaṃ raccā\-deso hoti vā.}
\sutdeftrans{There is a substitution of \pali{racca} for \pali{tuna}-paccaya and so on sometimes after roots ending with \pali{ca} and \pali{na}.}
\example[0]{vivicca (vi + \paliroot{vica} + tuna/tvā/tvāna)}
\example[0]{āhacca (ā + \paliroot{hana} + tuna/tvā/tvāna)}
\example{upahacca (upa + \paliroot{hana} + tuna/tvā/tvāna)}
\transnote{The \pali{ra} in \pali{racca} works like \pali{ra}-anubandha, hence it causes the elision of the last syllable.}

\head{599}{599, 644. disā svānasvāntalopo ca.}
\headtrans{After \pali{disa}, [there are] \pali{svāna}- and \pali{svā}-substitution, also an elision of the ending.}
\sutdef{disaiccetāya dhātuyā tunādīnaṃ paccayānaṃ svānasvādesā honti, antalopo ca.}
\sutdeftrans{There are substitutions of \pali{svāna} and \pali{svā} for \pali{tuna}-paccaya and so on after the root \pali{disa}. [There is] an elision of the ending also.}
\example{disvāna, disvā (\paliroot{disa} + tuna/tvā/tvāna)}

\head{600}{600,~645.~ma\,ha\,da\,bhehi\ \ mma\,yha\,jja\,bbha\,ddhā\ \ ca.}
\headtrans{After [roots ending with] \pali{ma}, \pali{ha}, \pali{da}, and \pali{bha}, [there are] \pali{mma}-, \pali{yha}-, \pali{jja}-, \pali{bbha}-, and \pali{ddha}-substitution [for \pali{tuna}-paccaya and so on].}
\sutdef{mahadabhaiccevamantehi dhātūhi tunādīnaṃ paccayānaṃ mmayhajjabbhaddhāādesā honti vā antalopo ca.}
\sutdeftrans{There are substitutions of \pali{mma}, \pali{yha}, \pali{jja}, \pali{bbha}, and \pali{ddha} for \pali{tuna}-paccaya and so on after roots ending with \pali{ma}, \pali{ha}, \pali{da}, and \pali{bha} sometimes. [There is] an elision of the ending also.}
\example[0]{āgamma = āgamitvā (ā + \paliroot{gamu} + tuna/tvā/tvāna)}
\example[0]{okkamma = okkamitvā (o + \paliroot{kamu} + tuna/tvā/tvāna)}
\example[0]{paggayha = paggaṇhitvā (pa + \paliroot{gaha} + tuna/tvā/tvāna)}
\example[0]{uppajja = uppajjitvā (upa + \paliroot{pada} + tuna/tvā/tvāna)}
\example[0]{ārabbha = ārabhitvā (ā + \paliroot{rabha} + tuna/tvā/tvāna)}
\example{āraddha = ārabhitvā (ā + \paliroot{rabha} + tuna/tvā/tvāna)}

\head{601}{601, 334. taddhitasamāsakitakā nāmaṃ vā tavetunādīsu ca.}
\headtrans{The terms [described in the sections of] secondary derivation, compound, and primary derivation should be seen as nominal sometimes but [except those ending with] \pali{tave}, \pali{tuna}, and so on.}
\sutdef{taddhitasamāsakitakaiccevamantā saddā nāmaṃva daṭṭha\-bbā tavetunatvānatvādipaccayante vajjetvā.}
\sutdeftrans{The terms [described in the sections of] secondary derivation, compound, and primary derivation should be seen as nominal except those ending with \pali{tave}, (\pali{tuṃ},) \pali{tuna}, \pali{tvāna}, \pali{tvā} and so on.}
\transnote{They are nominal in the sense that they can be declined with vibhattis, but the products of the mentioned paccayas are indeclinable.}

\head{602}{602, 6. dumhi garu.}
\headtrans{Because of double letter, [it is] heavy syllable.}
\sutdef{dumhi akkhare yo pubbo akkharo, so garukova daṭṭhabbo.}
\sutdeftrans{Which the preceding letter [exists], that [letter] should be seen as heavy syllable because of double letter.}
\example{bhitvā, chitvā, datvā, hutvā}
\transnote{As shown by the examples, \pali{tvā} is heavy syllable.}

\head{603}{603, 7. dīgho ca.}
\headtrans{A long vowel also [should be seen as heavy syllable].}
\sutdef{dīgho ca saro garukova daṭṭhabbo.}
\sutdeftrans{A long vowel also should be seen as heavy syllable.}
\example{āhāro, nadī, vadhū, te dhammā, opanayiko}
\transnote{Thus, the vowels \pali{ā}, \pali{ī}, \pali{ū}, \pali{e}, and \pali{o} are all heavy.}

\head{604}{604, 684. akkharehi kāra.}
\headtrans{After letters, [there is] \pali{kāra}-paccaya.}
\sutdef{akkharatthehi akkharābhidheyyehi kārapaccayo hoti payoge sati.}
\sutdeftrans{There is \pali{kāra}-paccaya after letters, designating the letter [itself], when [a letter is] used.}
\example{akāro, ākāro, yakāro}

\head{605}{605, 647. yathāgamamikāro.}
\headtrans{Correspondingly to the tradition, [there is] \pali{i}-insertion.}
\sutdef{yathāgamaṃ sabbadhātūhi sabbapaccayesu ikārāgamo hoti.}
\sutdeftrans{There is an insertion of \pali{i} after all roots because of all paccayas correspondingly to the tradition.}
\example{kāriyaṃ, bhavitabbaṃ, janitabbaṃ, viditabbaṃ, karitvā, icchitaṃ}
\transnote{For \pali{yathāgama}, see MWD.}

\head{606}{606, 642. dadhantato yo kvaci.}
\headtrans{After [a root] ending with \pali{da} or \pali{dha}, [there is] \pali{ya}-insertion in some places [because of \pali{tuna}-paccaya and so on].}
\sutdef{dakāradhakārantāya dhātuyā yathāgamaṃ yakārāgamo hoti kvaci tunādīsu paccayesu.}
\sutdeftrans{There is an insertion of \pali{ya} after a root ending with \pali{da} or \pali{dha} in some places because of \pali{tuna}-paccaya and so on, correspondingly to the tradition.}
\example[0]{buddho loke uppajjitvā (u + \paliroot{pada} + ya + tvā) \\{\upshape= The Buddha, having arisen in the world.}}
\example{dhammaṃ bujjhitvā (\paliroot{budha} + ya + tvā) \\{\upshape= The Dhamma, having been known.}}
\transnote{The sentences shown in the examples are not complete, but in some contexts the sentences are indeed finished. Hence, we can read them as ``The Buddha has arisen'' and ``The Dhamma has been known''.}

