\chapter{Kibbidhāna}

This book of Kaccāyana is all about \pali{kita}-paccayas, a kind of an arrangement of them (\pali{kita + vidhāna}). On the word, \pali{kita} (Skt.\ \pali{kṛt}) is indeed old and before Pāṇini.\footnote{\citealp[p.~126]{abhyankar:gramdict}} It denotes suffixes (paccayas) added to verbal bases (roots) to make new words, resulting in both verbals and nominals. Scholars call this process \emph{primary derivation}, in contrast with secondary derivation (taddhita), which nominal bases are used instead.

The main task of the learning is to recognize and be familiar with a number of paccayas and their behavior. Some paccayas may be identical with those of taddhita, such as \pali{ṇa}, but they belong to a different set. So, we do not mix up the two processes of derivation, but a similarity of formations can be seen.

Paccayas used in primary derivation are numerous indeed (including those in \pali{Uṇādikappa}). They can be divided function-wise into two groups: verbal (\pali{tabba, anīya,} etc.) and nominal (\pali{ṇvu, kvi,} etc.). Yet they can also be divided meaning-wise into: \hyperref[sut:545]{\pali{kicca}} (passive), i.e., \pali{tabba, anīya, ṇya, teyya (tayya),} and \pali{ricca} (\hyperref[sut:540]{Kacc 540}--\hyperref[sut:544]{544}); and \hyperref[sut:546]{\pali{kita}} (active), the rest of those (but some of them can still be used in a passive way).

To ease the finding, I show all paccayas used in kita formations with their relevant suttas in Table \ref{tab:kitapacc}. This can help the reader considerably.

The book is divided into five sections, loosely organized, described as follows:

The first section, Sutta 524--549, starts with \pali{ṇa} and ends with \pali{yu}. Two of the most important verbal paccayas (\pali{tabba} and \pali{anīya}) are introduced here. That is to say, the section covers all passive (kicca) paccayas.

The second section, Sutta 550--570, introduces the rest of verbal paccayas, plus some for nominals. Because verbal derivatives are really important, I summarize all of them in Table \ref{tab:kitaverb} at the beginning of the section.

The third section, Sutta 571--589, mostly has a variety of treatments for \pali{ta}-paccaya, the most important and most used of all.

The fourth section, Sutta 590--606, continues on the treatments for difficult paccayas. Several suttas here are about the \pali{tunādi} (\pali{tuna}, \pali{tvā}, and \pali{tvāna}).

The fifth section, Sutta 607--623, continues on peculiar operations of paccayas, mostly about \pali{ta} and \pali{ṇa}.

By the fact that several roots behave distinctively due to certain paccayas, I therefore summarize some notable roots in Table \ref{tab:roots-kita} at the end of the chapter. 

\boxnote{Explanation on the paccaya table.\\
\hspace{5mm}\bullet\ Paccayas are ordered by Pāli letters.\\
\hspace{5mm}\bullet\ The last column shows the suttas the paccaya appears.\\
\hspace{5mm}\bullet\ The \emph{Verbal} column marks whether the product is also used as a non-finite verb. See Table \ref{tab:kitaverb} for a more elaborate summary of verbal derivatives.\\
\hspace{5mm}\bullet\ The products of non-verbal paccayas are only used as nominals. But a verbal product can be a nominal sometimes, depending on its context.\\
\hspace{5mm}\bullet\ The \emph{Passive} column marks whether the product has impersonal and passive meaning, partly or exclusively.\\
\hspace{5mm}\bullet\ The items marked with an asterisk (*) are not found in the CST (Myanmar) edition, but found in Thai tradition.\\
\hspace{5mm}\bullet\ The items marked with a question mark (?) seem not to have a proper elucidation.\\
\hspace{5mm}\bullet\ Note that Sutta No.\ 624 and greater are of the Uṇādi part.
}

\setcounter{table}{0}
\begin{longtable}{%
		>{\itshape\raggedright\arraybackslash}p{0.15\linewidth}%
		>{\centering\arraybackslash}p{0.12\linewidth}%
		>{\centering\arraybackslash}p{0.12\linewidth}%
		>{\raggedright\arraybackslash}p{0.40\linewidth}}
\caption{Paccayas for primary derivation}\label{tab:kitapacc}\\
\toprule
\upshape\bfseries \mbox{Paccaya} & \bfseries \mbox{Verbal} & \bfseries \mbox{Passive} & \bfseries Suttas \\ \midrule
\endfirsthead
\multicolumn{4}{c}{\footnotesize\tablename\ \thetable: Paccayas for primary derivation (contd\ldots)}\\
\toprule
\upshape\bfseries \mbox{Paccaya} & \bfseries \mbox{Verbal} & \bfseries \mbox{Passive} & \bfseries Suttas \\ \midrule
\endhead
\bottomrule
\ltblcontinuedbreak{4}
\endfoot
\bottomrule
\endlastfoot
%
a & & & \hyperref[sut:525]{525}, \hyperref[sut:526]{526}, \hyperref[sut:527]{527}, \hyperref[sut:553]{553} \\
anīya & \checkmark & \checkmark & \hyperref[sut:540]{540} \\
anta & \checkmark & & \hyperref[sut:565]{565} \\
ama & & & \hyperref[sut:666]{666} \\
ala & & & \hyperref[sut:665]{665} \\
āna & & & \hyperref[sut:655]{655} \\
āni & & & \hyperref[sut:645]{645} \\
āvī & & & \hyperref[sut:527]{527}, \hyperref[sut:532]{532} \\
i & & & \hyperref[sut:551]{551}, \hyperref[sut:669]{669} \\
ika & & & \hyperref[sut:570]{570} \\
idda & & & \hyperref[sut:661]{661} \\
ina & & & \hyperref[sut:558]{558}, \hyperref[sut:559]{559} \\
ira & & & \hyperref[sut:661]{661} \\
isa & & & \hyperref[sut:673]{673} \\
īvara & & & \hyperref[sut:668]{668} \\
ussa & & & \hyperref[sut:673]{673} \\
ūra & & & \hyperref[sut:670]{670} \\
ka & & & \hyperref[sut:661]{661}, \hyperref[sut:663]{663}, \hyperref[sut:664]{664} \\
kāra & & & \hyperref[sut:604]{604} \\
kta & & & \hyperref[sut:626]{626}, \hyperref[sut:643]{643} \\
kvi & & & \hyperref[sut:530]{530}, \hyperref[sut:615]{615}, \hyperref[sut:616]{616}, \hyperref[sut:639]{639} \\
kha & & & \hyperref[sut:560]{560}, \hyperref[sut:625]{625} \\
ghiṇ? & & & \hyperref[sut:651]{651} \\
ṭha & & & \hyperref[sut:659]{659}, \hyperref[sut:672]{672} \\
ḍha & & & \hyperref[sut:659]{659} \\
ṇa & & & \hyperref[sut:524]{524}, \hyperref[sut:528]{528}, \hyperref[sut:529]{529}, \hyperref[sut:590]{590}, \hyperref[sut:591]{591}, \hyperref[sut:593]{593}, \hyperref[sut:594]{594}, \hyperref[sut:613]{613}, \hyperref[sut:614]{614}, \hyperref[sut:640]{640}, \hyperref[sut:654]{654} \\
ṇitta & & & \hyperref[sut:657]{657} \\
ṇimā & & & \hyperref[sut:644]{644} \\
ṇī & & & \hyperref[sut:532]{532}, \hyperref[sut:636]{636}, \hyperref[sut:651]{651} \\
ṇu & & & \hyperref[sut:650]{650}, \hyperref[sut:671]{671} \\
ṇuka & & & \hyperref[sut:536]{536} \\
ṇya & & \checkmark & \hyperref[sut:541]{541}, \hyperref[sut:543]{543}, \hyperref[sut:544]{544} \\
ṇvu & & & \hyperref[sut:527]{527}, \hyperref[sut:618]{618}, \hyperref[sut:622]{622}, \hyperref[sut:641]{641}, \hyperref[sut:652]{652} \\
ta & \checkmark & \checkmark & \hyperref[sut:555]{555}, \hyperref[sut:556]{556}, \hyperref[sut:557]{557}, \hyperref[sut:572]{572}, \hyperref[sut:573]{573}, \hyperref[sut:574]{574}, \hyperref[sut:575]{575}, \hyperref[sut:576]{576}, \hyperref[sut:577]{577}, \hyperref[sut:578]{578}, \hyperref[sut:579]{579}, \hyperref[sut:580]{580}, \hyperref[sut:581]{581}, \hyperref[sut:582]{582}, \hyperref[sut:583]{583}, \hyperref[sut:584]{584}, \hyperref[sut:585]{585}, \hyperref[sut:586]{586}, \hyperref[sut:587]{587}, \hyperref[sut:588]{588}, \hyperref[sut:594]{594}, \hyperref[sut:611]{611}, \hyperref[sut:612]{612}, \hyperref[sut:617]{617} \\
ta & & & \hyperref[sut:650]{650}, \hyperref[sut:656]{656} \\
tabba & \checkmark & \checkmark & \hyperref[sut:540]{540}, \hyperref[sut:596]{596}, \hyperref[sut:620]{620} \\
tayya* & & \checkmark & \hyperref[sut:541]{541} \\
tavantu & \checkmark & & \hyperref[sut:555]{555} \\
tave & \checkmark & & \hyperref[sut:561]{561}, \hyperref[sut:595]{595}, \hyperref[sut:601]{601} \\
tāvī & \checkmark & & \hyperref[sut:555]{555} \\
ti & & & \hyperref[sut:552]{552}, \hyperref[sut:553]{553}, \hyperref[sut:585]{585}, \hyperref[sut:586]{586}, \hyperref[sut:587]{587}, \hyperref[sut:588]{588}, \hyperref[sut:658]{658} \\
tu & & & \hyperref[sut:527]{527}, \hyperref[sut:532]{532}, \hyperref[sut:619]{619}, \hyperref[sut:652]{652}, \hyperref[sut:667]{667}, \hyperref[sut:671]{671} \\
tuka & & & \hyperref[sut:569]{569} \\
tuna & \checkmark & & \hyperref[sut:564]{564}, \hyperref[sut:595]{595}, \hyperref[sut:597]{597}, \hyperref[sut:598]{598}, \hyperref[sut:599]{599}, \hyperref[sut:600]{600}, \hyperref[sut:601]{601}, \hyperref[sut:606]{606}, \hyperref[sut:620]{620} \\
tuṃ & \checkmark & & \hyperref[sut:561]{561}, \hyperref[sut:562]{562}, \hyperref[sut:563]{563}, \hyperref[sut:595]{595}, \hyperref[sut:596]{596}, \hyperref[sut:601]{601}, \hyperref[sut:620]{620}, \hyperref[sut:637]{637} \\
teyya & & \checkmark & \hyperref[sut:541]{541} \\
tti & & & \hyperref[sut:658]{658} \\
ttima & & & \hyperref[sut:644]{644} \\
traṇa & & & \hyperref[sut:656]{656} \\
tvā & \checkmark & & \hyperref[sut:564]{564}, \hyperref[sut:597]{597}, \hyperref[sut:598]{598}, \hyperref[sut:599]{599}, \hyperref[sut:600]{600}, \hyperref[sut:601]{601}, \hyperref[sut:606]{606} \\
tvāna & \checkmark & & \hyperref[sut:564]{564}, \hyperref[sut:597]{597}, \hyperref[sut:598]{598}, \hyperref[sut:599]{599}, \hyperref[sut:600]{600}, \hyperref[sut:601]{601}, \hyperref[sut:606]{606} \\
tha & & & \hyperref[sut:628]{628}, \hyperref[sut:660]{660} \\
thu & & & \hyperref[sut:644]{644} \\
da & & & \hyperref[sut:661]{661} \\
du & & & \hyperref[sut:667]{667} \\
dha & & & \hyperref[sut:661]{661} \\
nu & & & \hyperref[sut:671]{671} \\
nusa & & & \hyperref[sut:673]{673} \\
ntu & & & \hyperref[sut:655]{655} \\
ma & & & \hyperref[sut:628]{628} \\
man & & & \hyperref[sut:627]{627} \\
māna & \checkmark & \checkmark & \hyperref[sut:565]{565} \\
māna & & & \hyperref[sut:655]{655} \\
ya & & & \hyperref[sut:632]{632} \\
yāṇa & & & \hyperref[sut:633]{633} \\
yu & & \checkmark & \hyperref[sut:533]{533}, \hyperref[sut:547]{547}, \hyperref[sut:548]{548}, \hyperref[sut:549]{549}, \hyperref[sut:553]{553}, \hyperref[sut:622]{622}, \hyperref[sut:641]{641}, \hyperref[sut:650]{650} \\
ra & & & \hyperref[sut:538]{538} \\
ratthu & & & \hyperref[sut:566]{566} \\
ramma & & & \hyperref[sut:531]{531} \\
rātu & & & \hyperref[sut:568]{568} \\
ricca & & \checkmark & \hyperref[sut:542]{542} \\
ritu & & & \hyperref[sut:567]{567}, \hyperref[sut:568]{568} \\
ririya & & & \hyperref[sut:554]{554} \\
rū & & & \hyperref[sut:534]{534}, \hyperref[sut:535]{535} \\
la & & & \hyperref[sut:632]{632} \\
lāṇa & & & \hyperref[sut:633]{633} \\
ssaṃ & & & \hyperref[sut:655]{655} \\
\end{longtable}

\section{Paṭhamakaṇḍa}
\raggedbottom

\head{524}{524, 561. dhātuyā kammādimhi ṇo.}
\headtrans{After a root having \pali{kamma} in the front, [there is] \pali{ṇa}-paccaya.}
\sutdef{dhātuyā kammādimhi ṇapaccayo hoti.}
\sutdeftrans{There is \pali{ṇa}-paccaya after a root having \pali{kamma} (object/action) in the front.}
\example[0]{kammaṃ karotīti kammakāro (kamma + \paliroot{kara} + ṇa) \\{\upshape= [One] does an action, hence \pali{kammakāra} (workman).}}
\example[0]{kumbhakāro (kumbha + \paliroot{kara} + ṇa) \\{\upshape= a potter}}
\example[0]{mālākāro (mālā + \paliroot{kara} + ṇa) \\{\upshape= a wreath maker}}
\example[0]{kaṭṭhakāro (kaṭṭha + \paliroot{kara} + ṇa) \\{\upshape= a fire-wood maker}}
\example[0]{rathakāro (ratha + \paliroot{kara} + ṇa) \\{\upshape= a chariot maker}}
\example[0]{rajatakāro (rajata + \paliroot{kara} + ṇa) \\{\upshape= a silversmith}}
\example[0]{suvaṇṇakāro (suvaṇṇa + \paliroot{kara} + ṇa) \\{\upshape= a goldsmith}}
\example[0]{pattaggāho (patta + \paliroot{gaha} + ṇa) \\{\upshape= a bowl-holder/beggar}}
\example[0]{tantavāyo (tanta + \paliroot{ve} + ṇa) \\{\upshape= a weaver}}
\example[0]{dhaññamāyo (dhañña + \paliroot{mā} + ṇa) \\{\upshape= a grain-measurer}}
\example[0]{dhammakāmo (dhamma + \paliroot{kamu} + ṇa) \\{\upshape= a Dhamma-lover}}
\example{dhammacāro (dhamma + \paliroot{cara} + ṇa) \\{\upshape= a Dhamma-practitioner}}
\transnote{Like samāsa and taddhita, a product of kita can be explained by an analytic sentence. It can be as simple as shown in the first example, or a more elaborate one.}
\transnote{This paccaya has \pali{ṇa}-anubandha, so we see vuddhi-strength upon the roots. For \pali{ṇa}-paccaya and vuddhi-strength, see \hyperref[sut:400]{Kacc 400} and \hyperref[sut:405]{405}.}
\transnote{The products of this and most of the rest in this section (except \pali{tabba} and \pali{anīya}) are nominals. Please keep in mind that when an example is analyzed, its nominal vibhatti (mostly nominative) is left out to make the display less distracting.}

\boxnote{An analytic sentence (\pali{viggahavākya}) of a derivative tells its application of meaning.\\
\hspace{5mm}\bullet\ By `application of meaning' I means \pali{sādhana} used by Pāli teachers. There are seven of them, i.e., \pali{kattu}-, \pali{kamma}-, \pali{bhāva}-, \pali{karaṇa}-, \pali{sampadāna}-, \pali{apadāna}-, and \pali{adhikaraṇa}-sādhana. Putting \pali{bhāva}-sādhana aside, we will see an agreement between these and kārakas (see Chapter \ref{chap:karaka}).\\
\hspace{5mm}\bullet\ I will not much used \pali{sādhana} here because it tends to make things complicated than necessary.\\
\hspace{5mm}\bullet\ The application of meaning of a word can be seen in its analytic sentence, as illustrated by the followings:\\
\hspace{8mm} $\triangleright$\ \pali{dadāti so iti dāyako (\paliroot{dā} + ṇvu)}\\
\hspace{5mm} = giver (note the nominative)\\
\hspace{8mm} $\triangleright$\ \pali{karīyate taṃ iti kammaṃ (\paliroot{kara} + ramma)}\\
\hspace{5mm} = action (note the accusative)\\
\hspace{8mm} $\triangleright$\ \pali{gacchīyate iti gamanaṃ (\paliroot{gamu} + yu)}\\
\hspace{5mm} = going (note the impersonal passive verb)\\
\hspace{8mm} $\triangleright$\ \pali{karoti tena iti karaṇaṃ (\paliroot{kara} + yu)}\\
\hspace{5mm} = cause (note the intrumental)\\
\hspace{8mm} $\triangleright$\ \pali{tiṭṭhanti tasmiṃ iti ṭhānaṃ (\paliroot{ṭhā} + yu)}\\
\hspace{5mm} = base (note the locative)
}

\head{525}{525, 565. saññāyama nu.}
\headtrans{When naming, [there is] \pali{a}-paccaya, [also] \pali{nu}-insertion.}
\sutdef{saññāyamabhidheyyāyaṃ dhātuyā kammādimhi akārapaccayo hoti, nāmamhi ca nukārāgamo hoti.}
\sutdeftrans{When naming [a person], there is \pali{a}-paccaya after a root having \pali{kamma} in the front. There is also an insertion of \pali{nu} after the name.}
\example[0]{ariṃ dametīti arindamo (ari + \paliroot{dama} + a) \\{\upshape= [One] tames the enemy, hence \pali{arindama} (King Arindama, the enemy-tamer).}}
\example[0]{vessaṃ taratīti vessantaro (vessa + \paliroot{tara} + a) \\{\upshape= [One] crosses the merchant street, hence \pali{vessantara} (King Vessantara).}}
\example[0]{taṇhaṃ karotīti taṇhaṅkaro (taṇhā + \paliroot{kara} + a) \\{\upshape= [One] does away with craving, hence \pali{taṇhaṅkara} (the Buddha Taṇhaṅkara).}}
\example[0]{medhaṃ karotīti medhaṅkaro (medhā + \paliroot{kara} + a) \\{\upshape= [One] creates wisdom, hence \pali{medhaṅkara} (the Buddha Me\-dh\-aṅkara).}}
\example[0]{saraṇaṃ karotīti saraṇaṅkaro (saraṇa + \paliroot{kara} + a) \\{\upshape= [One] creates refuge, hence \pali{saraṇaṅkara} (the Buddha Saraṇ\-aṅkara).}}
\example{dīpaṃ karotīti dīpaṅkaro (dīpa + \paliroot{kara} + a) \\{\upshape= [One] creates light/support, hence \pali{dīpaṅkara} (the Buddha Dīpaṅkara).}}
\transnote{We do not see \pali{nu} because it becomes niggahita by \hyperref[sut:537]{Kacc 537} (\pali{nu niggahitaṃ padante}). Then it assimilates to a nasal letter by \hyperref[sut:31]{Kacc 31} (\pali{vaggantaṃ vā vagge}).}

\head{526}{526, 567. pure dadā ca iṃ.}
\headtrans{When \pali{pura} [is in the front, there is \pali{a}-paccaya] after \pali{dadā}. [The ending] \pali{a} becomes \pali{iṃ} also.}
\sutdef{purasadde ādimhi dadaiccetāya dhātuyā akārapaccayo hoti, purasaddassa akārassa ca iṃ hoti.}
\sutdeftrans{When \pali{pura} [is] in the front, there is \pali{a}-paccaya after the root \pali{dadā}. The [ending] \pali{a} of \pali{pura} becomes \pali{iṃ} also.}
\example{pure dānaṃ adāsīti purindado (pura + dadā + a) \\{\upshape= [One] gave alms in the prior time, hence \pali{purindada} (King Purindada).}}
\transnote{The product of this is also a name. It seems that this sutta treats only a peculiar case of \pali{pure} and \pali{dadā} (reduplicated form of \paliroot{dā}), not \paliroot{dā} in general.}

\head{527}{527, 568. sabbato ṇvutvāvī vā.}
\headtrans{After all [roots, there are \pali{a}-,] \pali{ṇvu}-, \pali{tu}-, and \pali{āvī}-paccaya.}
\sutdef{sabbato dhātuto kammādimhi vā akammādimhi vā akāraṇvutuāvīiccete paccayā honti.}
\sutdeftrans{There are \pali{a}-, \pali{ṇvu}-, \pali{tu}-, and \pali{āvī}-paccaya after all roots, with or without \pali{kamma} in the front.}

\bigbullet{(1) With \pali{a}-paccaya}
\example[0]{taṃ karotīti takkaro (ta + \paliroot{kara} + a) \\{\upshape= [One] does that, hence \pali{takkara} (that-doer).}}
\example[0]{hitaṃ karotīti hitakaro (hita + \paliroot{kara} + a) \\{\upshape= [One] does good, hence \pali{hitakara} (good-doer).}}
\example[0]{vineti ettha etenāti vā vinayo (vi + \paliroot{nī} + a) \\{\upshape= [It] leads to here or by this [way], hence \pali{vinaya} (discipline).}}
\example{nissāya naṃ vasatīti nissayo (ni + \paliroot{sī} + a) \\{\upshape= [One] lives depending on that [thing/person], hence \pali{nissaya} (support/teacher).}}

\bigbullet{(2) With \pali{ṇvu}-paccaya}
\example[0]{rathaṃ karotīti rathakārako (ratha + \paliroot{kara} + ṇvu) \\{\upshape= [One] makes a chariot, hence \pali{rathakāraka} (chariot-maker).}}
\example[0]{annaṃ dadātīti annadāyako (anna + \paliroot{dā} + ṇvu) \\{\upshape= [One] gives food, hence \pali{annadāyaka} (food-giver).}}
\example[0]{vineti satteti vināyako (vi + \paliroot{nī} + ṇvu) \\{\upshape= [One] leads beings, hence \pali{vināyaka} (leader).}}
\example[0]{karotīti kārako (\paliroot{kara} + ṇvu) \\{\upshape= [One] does, hence \pali{kāraka} (doer).}}
\example[0]{dadātīti dāyako (\paliroot{dā} + ṇvu) \\{\upshape= [One] gives, hence \pali{dāyaka} (giver).}}
\example{netīti nāyako (\paliroot{nī} + ṇvu) \\{\upshape= [One] leads, hence \pali{nāyaka} (leader).}}
\transnote{Here, \pali{ṇa} is anubandha, so vuddhi-strength is produced. For the substitution of \pali{aka}, see \hyperref[sut:622]{Kacc 622}.}

\bigbullet{(3) With \pali{tu}-paccaya}
\example[0]{taṃ karotīti takkattā (ta + \paliroot{kara} + tu) \\{\upshape= [One] does that, hence \pali{takkattu} (that-doer).}}
\example[0]{taṃ karotīti takkattā, tassa kattāti vā takkattā (ta + \paliroot{kara} + tu) \\{\upshape= [One] does that or does for that, hence \pali{takkattu} (that-doer).}}
\example[0]{bhojanaṃ dadātīti bhojanadātā, bhojanassa dātāti vā bhojanadātā (bhojana + \paliroot{dā} + tu) \\{\upshape= [One] gives food or gives for food, hence \pali{bhojanadātu} (food-giver).}}
\example[0]{karotīti kattā (\paliroot{kara} + tu) \\{\upshape= [One] does, hence \pali{kattu} (doer).}}
\example{saratīti saritā (\paliroot{sara} + tu) \\{\upshape= [One] remembers, hence \pali{saritu} (rememberer).}}
\transnote{The end product of \pali{tu}-paccaya has an irregular declensional paradigm. See, for example, \hyperref[sut:199]{Kacc 199}.}

\bigbullet{(4) With \pali{āvī}-paccaya}
\example{bhayaṃ passatīti bhayadassāvī (bhaya + \paliroot{disa} + āvī) \\{\upshape= [One] sees danger, hence \pali{bhayadassāvī} (danger-seer).}}

\head{528}{528, 577. visarujapadādito ṇa.}
\headtrans{After \pali{visa}, \pali{ruja}, \pali{pada} and so on, [there is] \pali{ṇa}-paccaya.}
\sutdef{visarujapadaiccevamādīhi dhātūhi ṇapaccayo hoti.}
\sutdeftrans{There is \pali{ṇa}-paccaya after the roots \pali{visa}, \pali{ruja}, \pali{pada} and so on.}
\example[0]{pavisatīti paveso (pa + \paliroot{visa} + ṇa) \\{\upshape= [One] enters, hence \pali{pavesa} (enterer).}}
\example[0]{rujatīti rogo (\paliroot{ruja} + ṇa) \\{\upshape= [It] hurts, hence \pali{roga} (disease).}}
\example[0]{uppajjatīti uppādo (u + \paliroot{pada} + ṇa) \\{\upshape= [It] arises, hence \pali{uppāda} (arising).}}
\example[0]{phusatīti phasso (\paliroot{phusa} + ṇa) \\{\upshape= [It] touches, hence \pali{phassa} (contact).}}
\example[0]{ucatīti oko (\paliroot{uca} + ṇa) \\{\upshape= [One] takes pleasure [in that], hence \pali{oka} (house).}}
\example[0]{bhavatīti bhāvo (\paliroot{bhū} + ṇa) \\{\upshape= [It] exists, hence \pali{bhāva} (state of being).}}
\example[0]{ayatīti āyo (ā + \paliroot{i} + ṇa) \\{\upshape= [It] comes, hence \pali{āya} (income/arival).}}
\example[0]{sammā bujjhatīti sambodho (saṃ + \paliroot{budha} + ṇa) \\{\upshape= [One] knows rightly, hence \pali{sambodha} (fully-awakening [one]).}}
\example{viharatīti vihāro (vi + \paliroot{hara} + ṇa) \\{\upshape= [One] lives [in], hence \pali{vihāra} (abode).}}
\transnote{For \pali{oka}, see \paliroot{uc} in MWD. For \pali{āya}, see also in MWD.}

\head{529}{529, 580. bhāve ca.}
\headtrans{In [the sense of] state, [there is \pali{ṇa}-paccaya] also.}
\sutdef{bhāvatthābhidheyye sabbadhātūhi ṇapaccayo hoti.}
\sutdeftrans{There is \pali{ṇa}-paccaya after all roots in denoting the sense of state.}
\example[0]{paccate, pacanaṃ vā pāko (\paliroot{paca} + ṇa) \\{\upshape= Cooking is done, or cooking [exists], hence \pali{pāka} (cooking).}}
\example[0]{cajate, cajanaṃ vā cāgo (\paliroot{caja} + ṇa) \\{\upshape= Abandoning is done, or abandoning [exists], hence \pali{cāga} (abandoning).}}
\example[0]{yāgo (\paliroot{yaja} + ṇa) \\{\upshape= sacrificing}}
\example[0]{yogo (\paliroot{yuja} + ṇa) \\{\upshape= applying}}
\example[0]{bhāgo (\paliroot{bhaja} + ṇa) \\{\upshape= dividing/portion}}
\example{paridāho (pari + \paliroot{daha} + ṇa) \\{\upshape= burning}}
\transnote{For \pali{paridāha}, also \pali{paridah}, see MWD.}

\head{530}{530, 584. kvi ca.}
\headtrans{[There is] \pali{kvi}-paccaya [after all roots] also.}
\sutdef{sabbadhātūhi kvipaccayo hoti.}
\sutdeftrans{There is \pali{kvi}-paccaya after all roots.}
\example[0]{sambhavatīti sambhū (saṃ + \paliroot{bhū} + kvi) \\{\upshape= [It] is united, hence \pali{sambhū}.}}
\example[0]{sayaṃ bhavatīti sayambhū (sayaṃ + \paliroot{bhū} + kvi) \\{\upshape= [One] exists by oneself, hence \pali{sayambhū} (the creator god).}}
\example[0]{visesena bhavatīti vibhū (vi + \paliroot{bhū} + kvi) \\{\upshape= [One] exists by superiority, hence \pali{vibhū} (supreme being).}}
\example[0]{bhujena gacchatīti bhujago (bhuja + \paliroot{gamu} + kvi) \\{\upshape= [It] goes by coil/bending, hence \pali{bhujaga} (snake).}}
\example{saṃ attānaṃ khanati, saṃ suṭṭhu khanatīti vā saṅkho (saṃ + \paliroot{khana} + kvi) \\{\upshape= [It] itself digs, or [it] digs well, hence \pali{saṅkha} (conch).}}
\transnote{The first example is dubious, see \pali{sambhū} and \pali{sambhava} in MWD. The second example is added by me from a Thai source. For \pali{vibhū}, see \pali{vibhava} in MWD.}

\head{531}{531, 589. dharādīhi rammo.}
\headtrans{After \pali{dhara} and so on, [there is] \pali{ramma}-paccaya.}
\sutdef{dharaiccevamādīhi dhātūhi rammapaccayo hoti.}
\sutdeftrans{There is \pali{ramma}-paccaya after the roots \pali{dhara} and so on.}
\example[0]{dharati tenāti dhammo (\paliroot{dhara} + ramma) \\{\upshape= [It] holds by that, hence \pali{dhamma} (characteristic/nature).}}
\example{karīyate tanti kammaṃ (\paliroot{kara} + ramma) \\{\upshape= [It] is done, hence \pali{kamma} (action).}}
\transnote{The roots are truncated because of \pali{ra}-anubandha, see \hyperref[sut:539]{Kacc 539} below.}

\head{532}{532, 590. tassīlādīsu ṇītvāvī ca.}
\headtrans{In the sense of habit of that and so on, [there are] \pali{ṇī}-, \pali{tu}-, \pali{āvī}-paccaya also.}
\sutdef{sabbehi dhātūhi tassīlādīsvatthesu ṇītuāvīiccete paccayā honti.}
\sutdeftrans{There are \pali{ṇī}-, \pali{tu}-, \pali{āvī}-paccaya and so on after all roots in the sense of \pali{tassīla} (habit of that) and so on.}
\example[0]{piyaṃ pasaṃsituṃ sīlaṃ yassa rañño, so hoti rājā piyapasaṃsī (piya + pa + \paliroot{saṃsa} + ṇī) \\{\upshape= Which king [has] a habit to praise [others] dearly, hence that king is \pali{piyapasaṃsī}.}}
\example[0]{brahmaṃ carituṃ sīlaṃ yassa puggalassa, so hoti puggalo brahmacārī (brahma + \paliroot{cara} + ṇī) \\{\upshape= Which person [has] a habit to behave divinely, hence that person is \pali{brahmacārī}.}}
\example[0]{pasayha pavattituṃ sīlaṃ yassa rañño, so hoti rājā pasayhapavattā (pasayha + pa + \paliroot{vatu} + tu) \\{\upshape= Which king [has] a habit to move forward forcefully, hence that king is \pali{pasayhapavattu}.}}
\example{bhayaṃ passituṃ sīlaṃ yassa samaṇassa, so hoti samaṇo bhayadassāvī (bhaya + \paliroot{disa} + āvī) \\{\upshape= Which ascetic [has] a habit to see danger, hence that acscetic is \pali{bhayadassāvī}.}}

\head{533}{533, 591. saddakudhacalamaṇḍattharudhādīhi yu.}
\headtrans{After [the sense of] \pali{sadda}, \pali{kudha}, \pali{cala}, \pali{maṇḍa}, and [the root] \pali{ruca} and so on, [there is] \pali{yu}-paccaya.}
\sutdef{saddakudhacalamaṇḍatthehi ca rucādīhi ca dhātūhi yupaccayo hoti tassīlādīsvatthesu.}
\sutdeftrans{There is \pali{yu}-paccaya after the roots having a sense of \pali{sadda}, \pali{kudha}, \pali{cala}, \pali{maṇḍa}, and the root \pali{ruca} and so on in the sense of \pali{tassīla} (habit of that) and so on.}
\example[0]{ghosanasīlo ghosano (ghosa + \paliroot{yu}) \\{\upshape= [One has] a habit of announcing, hence \pali{ghosana} (announcer).}}
\example[0]{bhāsanasīlo bhāsano (bhāsa + \paliroot{yu}) \\{\upshape= [One has] a habit of talking, hence \pali{bhāsana} (talker).}}
\example[0]{kodhano (\paliroot{kudha} + yu) \\{\upshape= a grumpy one}}
\example[0]{dosano (\paliroot{dusa} + yu) \\{\upshape= a wrathful one}}
\example[0]{calano (\paliroot{cala} + yu) \\{\upshape= a capricious one}}
\example[0]{kampano (\paliroot{kampa} + yu) \\{\upshape= a trembling one}}
\example[0]{phandano (\paliroot{phanda} + yu) \\{\upshape= a shivering one}}
\example[0]{maṇḍano (\paliroot{maṇḍa} + yu) \\{\upshape= a well-adorned one}}
\example[0]{vibhūsano (vi + \paliroot{bhūsa} + yu) \\{\upshape= a well-dressed one}}
\example[0]{rocano (\paliroot{ruca} + yu) \\{\upshape= a shining one}}
\example[0]{jotano (\paliroot{juta} + yu) \\{\upshape= a brilliant one}}
\example{vaḍḍhano (\paliroot{vaḍḍha} + yu) \\{\upshape= a prosperous one}}
\transnote{For the substitution of \pali{ana}, see \hyperref[sut:622]{Kacc 622}.}

\head{534}{534, 562. pārādigamimhā rū.}
\headtrans{After \pali{gamu} having \pali{pāra} in the front, [there is] \pali{rū}-paccaya.}
\sutdef{gamuiccetamhā dhātumhā pārasaddādimhā rūpaccayo hoti tassīlādīsvatthesu.}
\sutdeftrans{There is \pali{rū}-paccaya after the root \pali{gamu} having \pali{pāra} in the front in the sense of \pali{tassīla} (habit of that) and so on.}
\example{bhavapāraṃ gantuṃ sīlaṃ yassa purisassa, so hoti puriso bhavapāragū (bhavapāra + \paliroot{gamu} + rū) \\{\upshape= Which person [has] a habit to go to the other side of being, hence that person is \pali{bhavapāragū}.}}
\transnote{For the truncating effect of \pali{ra}-anubandha, see \hyperref[sut:539]{Kacc 539}.}

\head{535}{535, 593. bhikkhādito ca.}
\headtrans{After \pali{bhikkha} and so on, also [there is \pali{rū}-paccaya].}
\sutdef{bhikkhaiccevamādīhi dhātūhi rūpaccayo hoti tassīlādīsvatth\-esu.}
\sutdeftrans{There is \pali{rū}-paccaya after the root \pali{bhikkha} and so on in the sense of \pali{tassīla} (habit of that) and so on.}
\example[0]{bhikkhanasīlo yācanasīlo bhikkhu (\paliroot{bhikkha} + rū) \\{\upshape= [One has] a habit of begging, asking, hence \pali{bhikkhu} (mendicant).}}
\example{vijānanasīlo viññū (vi + \paliroot{ñā} + rū) \\{\upshape= [One has] knowledge normally, hence \pali{viññū} (wise one).}}

\head{536}{536, 594. hanatyādīnaṃ ṇuko.}
\headtrans{For \pali{hana} and so on, [there is \pali{ṇuka}-paccaya].}
\sutdef{hanatyādīnaṃ dhātūnaṃ ante ṇukapaccayo hoti tassīlādīsvatthesu.}
\sutdeftrans{There is \pali{ṇuka}-paccaya at the end of the roots \pali{hana} and so on in the sense of \pali{tassīla} (habit of that) and so on.}
\example[0]{āhananasīlo āghātuko (ā + \paliroot{hana} + ṇuka) \\{\upshape= [One has] a habit of killing, hence \pali{āghātuka} (killer).}}
\example{karaṇasīlo kāruko (\paliroot{kara} + ṇuka) \\{\upshape= [One has] a habit of doing, hence \pali{kāruka} (artisan).}}
\transnote{Because of \pali{ṇa}-paccaya, \pali{hana} can become \pali{ghāta}. See \hyperref[sut:591]{Kacc 591} (\pali{hanassa ghāto}).}

\head{537}{537, 566. nu niggahitaṃ padante.}
\headtrans{At the end of a term, \pali{nu} [becomes] niggahita.}
\sutdef{padante nukārāgamo niggahitamāpajjate.}
\sutdeftrans{The insertion of \pali{nu} at the end of a term becomes niggahita.}
\example{ariṃ dametīti arindamo (ari + \paliroot{dama} + a) \\{\upshape= [One] tames the enemy, hence \pali{arindama} (King Arindama, the enemy-tamer).}}
\transnote{For other examples, see \hyperref[sut:525]{Kacc 525}.}

\head{538}{538, 595. saṃhanāññāya vā ro gho.}
\headtrans{[There is] \pali{ra}-paccaya after \pali{saṃhana} or other. [Also \pali{hana} becomes] \pali{gha}.}
\sutdef{saṃpubbāya hanaiccetāya dhātuyā, aññāya vā dhātuyā rapaccayo, hanassa ca gho hoti.}
\sutdeftrans{[There is] \pali{ra}-paccaya after the root \pali{hana} having \pali{saṃ} in the front, or after other root. Also \pali{hana} becomes \pali{gha}.}
\example[0]{samaggaṃ kammaṃ samupagacchatīti saṅgho \\(saṃ + \paliroot{hana} + ra) \\{\upshape= [A group] approaches an action in concord, hence \pali{saṅgha}.}}
\example[0]{samantato nagarassa bāhire khaññatīti parikhā \\(pari + \paliroot{khanu} + ra) \\{\upshape= [People] dig [the ground] surrounding outside the city, hence \pali{parikhā} (ditch).}}
\example{antaṃ karotīti antako (anta + \paliroot{kara} + ra) \\{\upshape= [It] does the end, hence \pali{antaka} (death).}}
\transnote{For the truncation of the roots, see the next sutta. I have no idea why \pali{saṅgha} has a thing to do with `killing' (\paliroot{hana}), but the word itself is indeed old and was analyzed as such. See \pali{saṃgha} in MWD.}

\head{539}{539, 558. ramhi ranto rādi no.}
\headtrans{Because of \pali{ra}-paccaya, [there is] no \pali{ra} and so on at the end.}
\sutdef{ramhi paccaye pare sabbo dhātvanto rakārādi lopo hoti.}
\sutdeftrans{There is an elision of \pali{ra} and so on at the end of all roots because of \pali{ra}-paccaya behind.}
\example[0]{antako (anta + \paliroot{kara} + ra)}
\example[0]{pāragū (pāra + \paliroot{gamu} + rū)}
\example[0]{satthā (\paliroot{sāsa} + ratthu)}
\example{diṭṭho (\paliroot{disa} + riṭṭha)}
\transnote{I call the \pali{ra}-anubandha, last-syllable killer. It makes itself and the last syllable of the affected roots disappear. Not only \pali{ra}-paccaya, but also paccayas containing the \pali{ra}-anubandha, such as \pali{ramma}, \pali{rū}, etc. In the last example, \pali{diṭṭha} is in fact a product of \pali{ta}-paccaya, but the paccaya is substituted by \pali{riṭṭha} (see \hyperref[sut:572]{Kacc 572}).}

\head{540}{540, 545. bhāvakammesu tabbānīyā.}
\headtrans{In impersonal and passive meaning, [there are] \pali{tabba}- and \pali{anīya}-paccaya.}
\sutdef{bhāvakammaiccetesvatthesu tabbaanīyaiccete paccayā honti sabbadhātūhi.}
\sutdeftrans{There are \pali{tabba}- and \pali{anīya}-paccaya after all roots in the sense of \pali{bhāva} (impersonal passive) and \pali{kamma} (passive).}
\example[0]{bhavitabbaṃ, bhavanīyaṃ (\paliroot{bhū} + tabba/anīya) \\{\upshape= Being should be done [= A state should be made exist].}}
\example[0]{āsitabbaṃ, āsanīyaṃ (\paliroot{āsa} + tabba/anīya) \\{\upshape= Sitting should be done.}}
\example[0]{pajjitabbaṃ, pajjanīyaṃ (\paliroot{pada} + tabba/anīya) \\{\upshape= Reaching should be done/[Something] should be reached.}}
\example[0]{kattabbaṃ, karaṇīyaṃ (\paliroot{kara} + tabba/anīya) \\{\upshape= Making should be done/[Something] should be made.}}
\example{gantabbaṃ, gamanīyaṃ (\paliroot{gamu} + tabba/anīya) \\{\upshape= Going should be done/[Somewhere] should be gone to.}}
\transnote{Scholars sometimes call these verbal products \emph{future passive participle}, in the sense that they have an expectation in the action yet to be performed.}

\head{541}{541, 552. ṇyo ca.}
\headtrans{[There is] \pali{ṇya}-paccaya also [in impersonal and passive meaning].}
\sutdef{bhāvakammesu sabbadhātūhi ṇyapaccayo hoti.}
\sutdeftrans{There is \pali{ṇya}-paccaya after all roots in impersonal and passive meaning.}
\example[0]{kāriyaṃ (\paliroot{kara} + ṇya) \\{\upshape= [A thing] should be done.}}
\example[0]{jeyyaṃ (\paliroot{ji} + ṇya) \\{\upshape= [A thing] should be won.}}
\example{neyyaṃ (\paliroot{nī} + ṇya) \\{\upshape= [A thing] should be led.}}
\transnote{As marked by \pali{ca}, also \pali{teyya}-paccaya can be used.}
\example[0]{ñāteyyaṃ (\paliroot{ñā} + teyya) \\{\upshape= [A thing] should be known.}}
\example[0]{daṭṭheyyaṃ (\paliroot{disa} + teyya) \\{\upshape= [A thing] should be seen.}}
\example{patteyyaṃ (\paliroot{pada} + teyya) \\{\upshape= [A thing] should be attained.}}
\transnote{In Thai tradition, \pali{teyya} becomes \pali{tayya} instead.}

\head{542}{542, 557. karamhā ricca.}
\headtrans{After \pali{kara}, [there is] \pali{ricca}-paccaya.}
\sutdef{karaiccetamhā dhātumhā riccapaccayo hoti bhāvakammesu.}
\sutdeftrans{There is \pali{ricca}-paccaya after the root \pali{kara} in impersonal and passive meaning.}
\example{kiccaṃ (\paliroot{kara} + ricca) \\{\upshape= [A thing] should be done.}}
\transnote{This paccaya has \pali{ra}-anubandha (see \hyperref[sut:539]{Kacc 539}).}

\head{543}{543, 555. bhūto’bba.}
\headtrans{After \pali{bhū}, [there is] \pali{abba}-substitution.}
\sutdef{bhūiccetāya dhātuyā ṇyapaccayassa ūkārena saha abbādeso hoti bhāvakammesu.}
\sutdeftrans{There is a substitution of \pali{abba} together with \pali{ū} for \pali{ṇya}-paccaya after the root \pali{bhū} in impersonal and passive meaning.}
\example{bhabbo, bhabbaṃ (\paliroot{bhū} + ṇya) \\{\upshape= [A state] should be made exist.}}

\head{544}{544,~556.~vada\,mada\,gamu\,yuja\,garahā\,kārādīhi\ \ jja\,mma\,gga\,yh\,eyyā\ \ gāro\ \ vā.}
\headtrans{After \pali{vada}, \pali{mada}, \pali{gamu}, \pali{yuja}, \pali{garaha} and so on, [there are substitutions of] \pali{jja}, \pali{mma}, \pali{gga}, \pali{yha}, and \pali{eyya}, also \pali{gāro} [for \pali{garaha}].}
\sutdef{vadamadagamuyujagarahākārantaiccevamādīhi dhātūhi ṇyapaccayassa yathāsaṅkhyaṃ jjammaggayhaeyyādesā honti vā dhātvantena saha, garassaṃ ca gāro hoti bhāvakammesu.}
\sutdeftrans{There are substitutions of \pali{jja}, \pali{mma}, \pali{gga}, \pali{yha}, and \pali{eyya} for \pali{ṇya}-paccaya together with the root's ending after the roots \pali{vada}, \pali{mada}, \pali{gamu}, \pali{yuja}, \pali{garaha} and so on respectively in impersonal and passive meaning. Also \pali{gara} [of \pali{garaha}] becomes \pali{gāra}.}
\example[0]{vajjaṃ (\paliroot{vada} + ṇya) \\{\upshape= worthy to be said}}
\example[0]{majjaṃ (\paliroot{mada} + ṇya) \\{\upshape= worthy to make drunk [= intoxicant]}}
\example[0]{gammaṃ (\paliroot{gamu} + ṇya) \\{\upshape= worthy to go to}}
\example[0]{yoggaṃ (\paliroot{yuja} + ṇya) \\{\upshape= worthy to be applied}}
\example[0]{gārayhaṃ (\paliroot{garaha} + ṇya) \\{\upshape= worthy to be blamed}}
\example[0]{deyyaṃ (\paliroot{dā} + ṇya) \\{\upshape= worthy to be given}}
\example[0]{peyyaṃ (\paliroot{pā} + ṇya) \\{\upshape= worthy to be drunk}}
\example[0]{heyyaṃ (\paliroot{hā} + ṇya) \\{\upshape= worthy to be discarded}}
\example[0]{meyyaṃ (\paliroot{mā} + ṇya) \\{\upshape= worthy to be measured}}
\example{ñeyyaṃ (\paliroot{ñā} + ṇya) \\{\upshape= worthy to be known}}

\head{545}{545, 548. te kiccā.}
\headtrans{Those [should be called] \pali{kicca}-paccayas.}
\sutdef{ye paccayā tabbādayo riccantā, te kiccasaññāti veditabbā.}
\sutdeftrans{Which paccayas starting from \pali{tabba} to \pali{ricca} at the end [exist], those [paccayas] should be known as being called \pali{kicca}-paccayas.}
\transnote{These paccayas are \pali{tabba}, \pali{anīya}, \pali{ṇya}, \pali{teyya} (\pali{tayya}), and \pali{ricca}, which have impersonal and passive meaning.}

\head{546}{546, 562. aññe kita.}
\headtrans{Other [paccayas are called] \pali{kita}-paccayas.}
\sutdef{aññe paccayā kita eva saññā honti.}
\sutdeftrans{Other paccayas are called \pali{kita}-paccayas.}
\transnote{Still, as we shall see in those \pali{kita}-paccayas, some of them also have passive meaning.}

\head{547}{547, 596. nandādīhi yu.}
\headtrans{After \pali{nanda} and so on, [there is] \pali{yu}-paccaya.}
\sutdef{nandādīhi dhātūhi yupaccayo hoti bhāvakammesu.}
\sutdeftrans{There is \pali{yu}-paccaya after the root [such as] \pali{nanda} and so on in impersonal and passive meaning.}
\example[0]{nandīyate nandanaṃ, ninditabbaṃ vā nandanaṃ \\(\paliroot{nanda} + yu) \\{\upshape= Enjoying is done, or [it is] worthy to be enjoyed, hence \pali{nandana} (enjoyment).}}
\example[0]{gahaṇīyaṃ gahaṇaṃ (\paliroot{gaha} + yu) \\{\upshape= [It is] worthy to be grasped, hence \pali{gahaṇa} (grasping).}}
\example{caritabbaṃ caraṇaṃ (\paliroot{cara} + yu) \\{\upshape= [It is] worthy to be practised, hence \pali{caraṇa} (behavior).}}
\transnote{This paccaya can cause a confusion because it is substituted by \pali{ana} or \pali{aṇa} (see \hyperref[sut:549]{Kacc 549}). The substitution is described later in \hyperref[sut:622]{Kacc 622}.}

\head{548}{548, 597. kattukaraṇapadesesu ca.}
\headtrans{In [the meanings of] agent, cause, and location, [there is \pali{yu}-paccaya] also.}
\sutdef{kattukaraṇapadesaiccetesvatthesu ca yupaccayo hoti.}
\sutdeftrans{The \pali{yu}-paccaya is also in the meanings of agent, cause, and location.}

\bigbullet{(1) Agent}
\example{rajaṃ haratīti rajoharaṇaṃ (raja + \paliroot{hara} + yu) \\{\upshape= [It] takes away dust, hence \pali{rajoharaṇa} (dust removal, cleaning)}}

\bigbullet{(2) Cause}
\example{karoti tenāti karaṇaṃ (\paliroot{kara} + yu) \\{\upshape= [One] does by that, hence \pali{karaṇa} (cause, reason)}}

\bigbullet{(3) Location}
\example{tiṭṭhanti tasminti ṭhānaṃ (\paliroot{ṭhā} + yu) \\{\upshape= [They] stand on that, hence \pali{ṭhāna} (place, position, base)}}

\head{549}{549, 550. rahādito ṇa.}
\headtrans{After \pali{ra}, \pali{ha} and so on, [there is] \pali{ṇa}-substitution.}
\sutdef{rakārahakārādyantehi dhātūhi anādesassa nassa ṇo hoti.}
\sutdeftrans{The \pali{na} of \pali{ana}-substitution [for \pali{yu}-paccaya] becomes \pali{ṇa} after roots ending with \pali{ra}, \pali{ha} and so on.}
\example[0]{karoti tenāti karaṇaṃ (\paliroot{kara} + yu) \\{\upshape= [One] does by that, hence \pali{karaṇa} (cause, reason)}}
\example[0]{pūreti tenāti pūraṇaṃ (\paliroot{pūra} + yu) \\{\upshape= [One] fills by that, hence \pali{pūraṇa} (filler)}}
\example{gahaṇīyaṃ tenāti gahaṇaṃ (\paliroot{gaha} + yu) \\{\upshape= [It is] worthy to be grasped by that, hence \pali{gahaṇa} (grasper).}}

