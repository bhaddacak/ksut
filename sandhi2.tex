\section{Dutiyakaṇḍa}

\head{12}{12, 13. sarā sare lopaṃ.}
\headtrans{Vowels [get] elided, because of [the other] vowel.}
\sutdef{sarā kho sare pare lopaṃ papponti.}
\sutdeftrans{Vowels get elided, because of the other vowel.}
\example{yassindriyāni = yassa + indriyāni}
\transnote{I translate \pali{para sara} literally as `the other vowel.' You will see this a lot, as well as `the other consonant.' This means the latter/second/next/rear part to be put into connection.}

\head{13}{13, 15. vā paro asarūpā.}
\headtrans{Sometimes the other [vowel gets elided] from the dissimilar [preceding vowel].}
\sutdef{saramhā asarūpā paro saro lopaṃ pappoti vā.}
\sutdeftrans{From the [preceding] dissimilar vowel, the other vowel gets elided sometimes.}
\example[0]{cattārome = cattāro + ime}
\example{kinnumāva = kinnu + imā + eva}
\transnote{We can also render `from' (abl.) as `because of.' The particle \pali{vā} (or) here has the sense of `sometimes' or `this may also be the case.'}

\head{14}{14, 16. kvacāsavaṇṇaṃ lutte.}
\headtrans{[The other vowel becomes] \pali{asavaṇṇa}, when [the preceding vowel gets] elided.}
\sutdef{saro kho paro pubbasare lutte kvaci asavaṇṇaṃ pappoti.}
\sutdeftrans{When the preceding vowel gets elided, the other vowel becomes \pali{asavaṇṇa} in some places.}
\example[0]{nopeti = na + upeti}
\example{bandhusseva = bandhussa + iva}
\transnote{As shown by the examples, the technical term \pali{asavaṇṇa} vowels means the unpaired, i.e., \pali{e} and \pali{o}. As we will see \pali{kvaci} often later on, I render this as `in some places' and seldom `sometimes.' They are more or less the same. The word gives a sense of uncertainty.}

\head{15}{15, 17. dīghaṃ.}
\headtrans{[It becomes] elongated.}
\sutdef{saro kho paro pubbasare lutte kvaci dīghaṃ pappoti.}
\sutdeftrans{When the preceding vowel gets elided, the other vowel becomes elongated in some places.}
\example[0]{saddhīdha = saddhā + idha}
\example{cūbhayaṃ = ca + ubhayaṃ}

\head{16}{16, 18. pubbo ca.}
\headtrans{Also the preceding [vowel gets elongated]}
\sutdef{pubbo ca saro parasaralope kate kvaci dīghaṃ pappoti.}
\sutdeftrans{Also when the other vowel [are] made elided, the preceding vowel becomes elongated in some places.}
\example[0]{kiṃsūdha = kiṃ + su + idha}
\example{sādhūti = sādhu + iti}

\head{17}{17, 19. yamedantassādeso.}
\headtrans{[There is] \pali{ya-}substitution for \pali{e-}ending.}
\sutdef{ekārassa antabhūtassa sare pare kvaci yakārādeso hoti.}
\sutdeftrans{There is a substitution by \pali{ya} in some places for \pali{e} at the end [of the preceding term] because of the other vowel.}
\example[0]{myāyaṃ = me + ayaṃ}
\example[0]{tyāhaṃ = te + haṃ}
\example{tyāssa = te + assa}
\transnote{\pali{Yamedantassādeso} can be cut as \pali{yaṃ + e + antassa + ādeso}. The \pali{d} is an extra insertion (see \hyperref[sut:35]{Kacc 35}).}

\head{18}{18, 20. vamodudantānaṃ.}
\headtrans{[There is] \pali{va-}substitution for \pali{o-}ending and \pali{u-}ending.}
\sutdef{okārukārānaṃ antabhūtānaṃ sare pare kvaci vakārādeso hoti.}
\sutdeftrans{There is a substitution by \pali{va} in some places for \pali{o} and \pali{u} at the end [of the preceding term] because of the other vowel.}
\example[0]{khvassa = kho + assa}
\example[0]{svassa = so + assa}
\example[0]{bahvābādho = bahu + ābādho}
\example[0]{vatthvettha = vatthu + ettha}
\example{cakkhvāpāthamāgacchati = cakkhu + āpāthaṃ + āgacchati}
\transnote{\pali{Vamodudantānaṃ} can be cut as \pali{vaṃ + o + u + antānaṃ}.}

\head{19}{19, 22. sabbo caṃ ti.}
\headtrans{The whole \pali{ti} [becomes] \pali{ca}.}
\sutdef{sabbo icceso tisaddo byañjano sare pare kvaci cakāraṃ pappoti.}
\sutdeftrans{The whole \pali{ti} as a part of \pali{iti} becomes \pali{ca} in some places because of the other vowel.}
\example[0]{iccetaṃ = iti + etaṃ}
\example[0]{iccassa = iti + assa}
\example[0]{paccuttaritvā = pati + uttaritvā}
\example{paccāharati = pati + āharati}
\transnote{The whole \pali{ti} means the substitution is done upon both letters. As shown in the examples, this can be applied to other \pali{ti-}ending particles, such as \pali{pati} and also \pali{ati}.}

\head{20}{20, 27. do dhassa ca.}
\headtrans{Also \pali{da} for \pali{dha}.}
\sutdef{dhaiccetassa sare pare kvaci dakārādeso hoti.}
\sutdeftrans{There is a substitution by \pali{da} in some places for \pali{dha} because of the other vowel.}
\example{ekamidāhaṃ = ekaṃ + idha + ahaṃ}

\head{21}{21, 21. ivaṇṇo yaṃ navā.}
\headtrans{Sometimes \pali{i} and \pali{ī} [become] \pali{ya}.}
\sutdef{pubbo ivaṇṇo sare pare yakāraṃ pappoti navā.}
\sutdeftrans{The preceding \pali{i} and \pali{ī} become \pali{ya} sometimes because of the other vowel.}
\example[0]{paṭisanthāravuttyassa\footnote{In the CST corpus, it is \pali{paṭisanthāravutyassa}.} =  paṭisanthāravutti + assa}
\example{sabbāvittyānubhūyate\footnote{In the CST corpus, it is \pali{sabbāvityānubhūyate}.} = sabbā + vitti + anubhūyate}
\transnote{The vowels of \pali{i}-group (\pali{ivaṇṇa}) are \pali{i} and \pali{ī}.}

\head{22}{22, 28. evādissa ri pubbo ca rasso.}
\headtrans{[There is] \pali{ri} [for] the begining of \pali{eva}, also the shortened preceding [vowel].}
\sutdef{saramhā parassa evassa ekārassa ādissa rikāro hoti, pubbo ca saro rasso hoti navā.}
\sutdeftrans{There is [a substitution by] \pali{ri} from the vowel \pali{e}, the beginning of \pali{eva}, of the other [term]. Also the preceding vowel becomes shortened sometimes.}
\example[0]{yathariva = yathā + iva}
\example{tathariva = tathā + iva}

