\section{Tatiyakaṇḍa}

\head{161}{161, 244. tumhamhehi namākaṃ.}
\headtrans{After \pali{tumha} and \pali{amha}, [there is] \pali{ākaṃ}[-substitution] for \pali{naṃ}[-vibhatti].}
\sutdef{tehi tumhaamhehi naṃvacanassa ākaṃ hoti.}
\sutdeftrans{There is \pali{ākaṃ}[-substitution] for \pali{naṃ}-vibhatti behind those \pali{tumha} and \pali{amha}.}
\example[0]{tumhākaṃ = tumha + naṃ {\upshape (dat./gen.)}}
\example{amhākaṃ = amha + naṃ {\upshape (dat./gen.)}}

\head{162}{162, 237. vā yvappaṭhamo.}
\headtrans{Sometimes non-nominative \pali{yo}[-vibhatti] [becomes \pali{ākaṃ}].}
\sutdef{tehi tumhaamhehi yo appaṭhamo ākaṃ hoti vā.}
\sutdeftrans{Sometimes a non-nominative \pali{yo}[-vibhatti] behind those \pali{tumha} and \pali{amha} becomes \pali{ākaṃ}.}
\example[0]{tumhākaṃ = tumha + yo {\upshape (acc.)}}
\example[0]{tumhākaṃ passāmi = tumhe passāmi\\{\upshape= [I] see you (all).}}
\example[0]{amhākaṃ = amha + yo {\upshape (acc.)}}
\example{amhākaṃ passasi = amhe passasi\\{\upshape= [You] see us.}}

\head{163}{163, 240. sassaṃ.}
\headtrans{For \pali{sa}[-vibhatti], [there is] \pali{aṃ}-substitution.}
\sutdef{tehi humhaamhehi sassa vibhattissa aṃādeso hoti vā.}
\sutdeftrans{There is a substitution of \pali{aṃ} for \pali{sa}-vibhatti behind those \pali{tumha} and \pali{amha} sometimes.}
\example[0]{tumhaṃ = tumha + sa {\upshape (dat./gen.)}}
\example[0]{tumhaṃ dīyate = tava dīyate\\{\upshape= [A thing] is given to you. (dat.)}}
\example[0]{tumhaṃ pariggaho = tava pariggaho\\{\upshape= Your possession. (gen.)}}
\example[0]{amhaṃ = amha + sa {\upshape (dat./gen.)}}
\example[0]{amhaṃ dīyate = mama dīyate\\{\upshape= [A thing] is given to me. (dat.)}}
\example{amhaṃ pariggaho = mama pariggaho\\{\upshape= My possession. (gen.)}}

\head{164}{164, 200. sabbanāmakārate paṭhamo.}
\headtrans{Nominative [\pali{yo}-vibhatti] after \pali{a} of pronouns [becomes] \pali{e}.}
\sutdef{sabbesaṃ sabbanāmānaṃ akārato yo paṭhamo ettamāpajjate.}
\sutdeftrans{The nominative \pali{yo}[-vibhatti] after \pali{a} of all pronouns becomes \pali{e}.}
\example[0]{sabbe = sabba + yo}
\example[0]{ye = ya + yo}
\example[0]{te = ta + yo}
\example[0]{ke = ka + yo}
\example[0]{tumhe = tumha + yo}
\example[0]{amhe = amha + yo}
\example{ime = ima + yo}
\transnote{We can break \pali{sabbanāmakārate} to \pali{sabbanāma + naṃ + akāra + to + e}.}

\head{165}{165, 208. dvandaṭṭhā vā.}
\headtrans{After a copulative compound [of pronouns] sometimes.}
\sutdef{tasmā sabbanāmakārato dvandaṭṭhā yo paṭhamo ettamāpajjate vā.}
\sutdeftrans{The nominative \pali{yo}[-vibhatti] after \pali{a} of that pronominal copu\-lative-compound (\pali{dvandvasamāsa}) becomes \pali{e} sometimes.}
\example{katarakatame = katarakatama + yo (= katarakatamā)}

\head{166}{166, 209. nāññaṃ sabbanāmikaṃ.}
\headtrans{Not other pronominal form.}
\sutdef{sabbanāmikānaṃ dvandaṭṭhe nāññaṃ kāriyaṃ hoti.}
\sutdeftrans{Other pronominal form in copulative compound is unable to make.}
\example[0]{pubbāparānaṃ = pubbāpara + naṃ}
\example[0]{pubbuttarānaṃ = pubbuttara + naṃ}
\example{adharuttarānaṃ = adharuttara + naṃ}
\transnote{As shown in the examples, this can be understood that the previous rule (\pali{e}-substitution) is not applicable to other cases except those of \pali{yo}-vibhatti mentioned.}

\head{167}{167, 210. bahubbīhimhi ca.}
\headtrans{[Not] in atributive compound also.}
\sutdef{bahubbīhimhi ca samāse sabbanāmavidhānañca nāññaṃ kāriyaṃ hoti.}
\sutdeftrans{Other rendition of pronoun in an atributive compound (\pali{bahubbīhisamāsa}) is also unable to make.}
\example[0]{piyapubbāya = piyapubba + sa}
\example[0]{piyapubbānaṃ = piyapubba + naṃ}
\example[0]{piyapubbe = piyapubba + smiṃ}
\example{piyapubbassa = piyapubba + sa}
\transnote{This means that also in an atributive compound of pronouns, the \pali{e}-substitution is not applicable, except for the \pali{yo}-vibhatti mentioned.}

\head{168}{168, 203. sabbato naṃ saṃsānaṃ.}
\headtrans{After all [pronouns] \pali{naṃ}[-vibhatti] [becomes] \pali{saṃ} and \pali{sānaṃ}.}
\sutdef{sabbato sabbanāmato naṃvacanassa saṃsānaṃiccete ādesā honti.}
\sutdeftrans{There are substitutions of \pali{saṃ} and \pali{sānaṃ} for \pali{naṃ}-vibhatti after all pronouns.}
\example[0]{sabbesaṃ = sabba + naṃ {\upshape (m./nt.)}}
\example[0]{sabbesānaṃ = sabba + naṃ}
\example[0]{sabbāsaṃ = sabba + naṃ {\upshape (f.)}}
\example[0]{sabbāsānaṃ = sabba + naṃ}
\example[0]{yesaṃ = ya + naṃ {\upshape (m./nt.)}}
\example[0]{yesānaṃ = ya + naṃ}
\example[0]{yāsaṃ = ya + naṃ {\upshape (f.)}}
\example[0]{yāsānaṃ = ya + naṃ}
\example[0]{tesaṃ = ta + naṃ {\upshape (m./nt.)}}
\example[0]{tesānaṃ = ta + naṃ}
\example[0]{tāsaṃ = ta + naṃ {\upshape (f.)}}
\example[0]{tāsānaṃ = ta + naṃ}
\example[0]{kesaṃ = ka + naṃ {\upshape (m./nt.)}}
\example[0]{kesānaṃ = ka + naṃ}
\example[0]{kāsaṃ = ka + naṃ {\upshape (f.)}}
\example[0]{kāsānaṃ = ka + naṃ}
\example[0]{imesaṃ = ima + naṃ {\upshape (m./nt.)}}
\example[0]{imesānaṃ = ima + naṃ}
\example[0]{imāsaṃ = ima + naṃ {\upshape (f.)}}
\example[0]{imāsānaṃ = ima + naṃ}
\example[0]{amūsaṃ = amu + naṃ {\upshape (m./f./nt.)}}
\example{amūsānaṃ = amu + naṃ}

\head{169}{169, 117. rājassa rāju sunaṃhisu ca.}
\headtrans{For \pali{rāja}, [there is] \pali{rāju}-substitution because of \pali{su}, \pali{naṃ}, and \pali{hi}[-vibhatti] also.}
\sutdef{sabbasseva rājasaddassa rājuādeso hoti sunaṃhiiccetesu.}
\sutdeftrans{There is a substitution of \pali{rāju} for the whole \pali{rāja} because of \pali{su}, \pali{naṃ}, and \pali{hi}[-vibhatti].}
\example[0]{rājūsu = rāja + su}
\example[0]{rājūnaṃ = rāja + naṃ}
\example[0]{rājūhi = rāja + hi}
\example{rājūbhi = rāja + hi}

\head{170}{170, 220. sabbassimasse vā.}
\headtrans{For the whole \pali{ima}, [there is] \pali{e}-substitution sometimes.}
\sutdef{sabbasseva imasaddassa ekāro hoti vā sunaṃhiiccetesu.}
\sutdeftrans{There is a substitution of \pali{e} for the whole \pali{ima} sometimes because of \pali{su}, \pali{naṃ}, and \pali{hi}[-vibhatti].}
\example[0]{esu = ima + su (= imesu)}
\example[0]{esaṃ = ima + naṃ (= imesaṃ)}
\example[0]{ehi = ima + hi (= imehi)}
\example{ebhi = ima + hi (= imebhi)}

\head{171}{171, 219. animi nāmhi ca.}
\headtrans{[There are] \pali{ana}- and \pali{imi}-substitution [for \pali{ima}] because of \pali{nā}[-vibhatti] also.}
\sutdef{imasaddassa sabbasseva anaimiiccete ādesā honti nāmhi vibhattimhi.}
\sutdeftrans{There are substitutions of \pali{ana} and \pali{imi} for the whole \pali{ima} because of \pali{nā}-vibhatti.}
\example[0]{anena dhammadānena, sukhitā hotu sā pajā.\\{\upshape= By this giving of the Dhamma, may those people be happy.}}
\example{iminā buddhapūjena, patvāna amataṃ padaṃ.\\{\upshape= By this homage to the Buddha, [I] attained the deathless state.}}
\transnote{In the last example, it is better to treat the absolutive \pali{patvāna} as a finite verb to make the sense complete.}

\head{172}{172, 218. anapuṃsakassāyaṃ simhi.}
\headtrans{For non-neuter [\pali{ima}], [there is] \pali{ayaṃ}-substitution because of \pali{si}[-vibhatti].}
\sutdef{imasaddassa sabbasseva anapuṃsakassa ayaṃādeso hoti simhi vibhattimhi.}
\sutdeftrans{There is a substitution of \pali{ayaṃ} for the whole non-neuter \pali{ima} because of \pali{si}-vibhatti.}
\example[0]{ayaṃ = ima + si}
\example[0]{ayaṃ puriso {\upshape= this man (m.)}}
\example{ayaṃ itthī {\upshape= this woman (f.)}}

\head{173}{173, 223. amussa mo saṃ.}
\headtrans{For [non-neuter] \pali{amu}, \pali{ma} [becomes] \pali{sa}.}
\sutdef{amusaddassa anapuṃsakassa makāro sakāramāpajjate vā simhi vibhattimhi.}
\sutdeftrans{The \pali{ma} of non-neuter \pali{amu} becomes \pali{sa} sometimes because of \pali{si}-vibhatti.}
\example[0]{asu = amu + si}
\example[0]{asu rājā = amuko rājā {\upshape= the king over there (m.)}}
\example{asu itthī = amukā itthī {\upshape= the woman over there (f.)}}

\head{174}{174, 211. etatesaṃ to.}
\headtrans{For \pali{eta} and \pali{ta}, the \pali{ta}-letter [becomes \pali{sa}].}
\sutdef{etataiccetesaṃ anapuṃsakānaṃ takāro sakāramāpajjate simhi vibhattimhi.}
\sutdeftrans{The \pali{ta}-letter of non-neuter \pali{eta} and \pali{ta} becomes \pali{sa} because of \pali{si}-vibhatti.}
\example[0]{eso = eta + si {\upshape (m.)}}
\example[0]{so = ta + si {\upshape (m.)}}
\example[0]{eso puriso, so puriso}
\example[0]{esā = eta + si {\upshape (f.)}}
\example[0]{sā = ta + si {\upshape (f.)}}
\example{esā itthī, sā itthī}

\head{175}{175, 212. tassa vā nattaṃ sabbattha.}
\headtrans{For \pali{ta}-pronoun, sometimes [there is] \pali{na}-substitution in all [genders].}
\sutdef{tassa sabbanāmassa takārassa nattaṃ hoti vā sabbattha liṅgesu.}
\sutdeftrans{There is [a substitution of] \pali{na} for the \pali{ta}-letter of the pronoun \pali{ta} sometimes in all genders.}
\example[0]{nāya = ta + sa (= tāya) {\upshape (f.)}}
\example[0]{naṃ = ta + aṃ (= taṃ) {\upshape (m./f./nt.)}}
\example[0]{ne = ta + yo (= te) {\upshape (m.)}}
\example[0]{nesu = ta + su (= tesu) {\upshape (m./nt.)}}
\example[0]{namhi = ta + hi (= tamhi) {\upshape (m./nt.)}}
\example[0]{nāhi = ta + hi (= tāhi) {\upshape (f.)}}
\example{nābhi = ta + hi (= tābhi) {\upshape (f.)}}

\head{176}{176, 213. sasmāsmiṃsaṃsāsvattaṃ.}
\headtrans{Because of \pali{sa}, \pali{smā}, \pali{smiṃ}[-vibhatti], \pali{saṃ}, and \pali{sā}, [there is] \pali{a}-substitution.}
\sutdef{tassa sabbanāmassa takārassa sabbasseva attaṃ hoti vā sasmāsmiṃsaṃsāiccetesu sabbattha liṅgesu.}
\sutdeftrans{There is a substitution of \pali{a} for the whole \pali{ta}-letter of the pronoun \pali{ta} sometimes because of \pali{sa}, \pali{smā}, \pali{smiṃ}[-vibhatti], \pali{saṃ}, and \pali{sā} in all genders.}
\example[0]{assa = ta + sa (= tassa) {\upshape (m./nt.)}}
\example[0]{asmā = ta + smā (= tasmā) {\upshape (m./nt.)}}
\example[0]{asmiṃ = ta + smiṃ (= tasmiṃ) {\upshape (m./nt.)}}
\example[0]{assaṃ = ta + smiṃ (= tassaṃ) {\upshape (f.)}}
\example{assā = ta + sa (= tassā) {\upshape (f.)}}
\transnote{For \pali{saṃ}- and \pali{sā}-substitution, see (\hyperref[sut:179]{Kacc 179}).}

\head{177}{177, 221. imasaddassa ca.}
\headtrans{For \pali{ima} also, [there is \pali{a}-substitution].}
\sutdef{imasaddassa ca sabbasseva attaṃ hoti vā sasmāsmiṃsaṃsāiccetesu sabbattha liṅgesu.}
\sutdeftrans{There is also [a substitution of] \pali{a} for the whole \pali{ima} sometimes because of \pali{sa}, \pali{smā}, \pali{smiṃ}[-vibhatti], \pali{saṃ}, and \pali{sā} in all genders.}
\example[0]{assa = ima + sa (= imassa) {\upshape (m./nt.)}}
\example[0]{asmā = ima + smā (= imasmā) {\upshape (m./nt.)}}
\example[0]{asmiṃ = ima + smiṃ (= imasmiṃ) {\upshape (m./nt.)}}
\example[0]{assaṃ = ima + smiṃ (= imissaṃ) {\upshape (f.)}}
\example{assā = ima + sa (= imassā) {\upshape (f.)}}

\head{178}{178, 22. sabbato ko.}
\headtrans{After all [pronouns], [there is] \pali{ka}-insertion.}
\sutdef{sabbato sabbanāmato kakārāgamo hoti vā simhi vibhattimhi.}
\sutdeftrans{There is an insertion of \pali{ka} after all pronouns sometimes because of \pali{si}-vibhatti.}
\example[0]{sabbako = sabba + ka + si}
\example[0]{yako = ya + ka + si}
\example[0]{sako = ta + ka + si}
\example{amuko = amu + ka + si (= asuko)}

\head{179}{179, 204. ghapato smiṃsānaṃ saṃsā.}
\headtrans{After \pali{gha}- and \pali{pa}-alias, for \pali{smiṃ} and \pali{sa}[-vibhatti] [there are] \pali{saṃ}- and \pali{sā}-substitution [respectively].}
\sutdef{sabbato sabbanāmato ghapasaññato smiṃsaiccetesaṃ saṃ\-sāādesā honti vā yathāsaṅkhyaṃ.}
\sutdeftrans{There are substitutions of \pali{saṃ} and \pali{sā} for \pali{smiṃ} and \pali{sa}[-vibhatti] respectively sometimes after \pali{gha}-alias (feminine \pali{ā}) and \pali{pa}-alias (feminine \pali{i, ī, u, ū}) of all pronouns.}
\example[0]{sabbassaṃ = sabbā + smiṃ (= sabbāyaṃ)}
\example[0]{sabbassā = sabbā + sa (= sabbāya)}
\example[0]{imissaṃ = imā + smiṃ (= imāyaṃ)}
\example[0]{imissā = imā + sa (= imāya)}
\example[0]{amussaṃ = amu + smiṃ (= amuyaṃ)}
\example{amussā = amu + sa (= amuyā)}
\transnote{For the definition of \pali{pa}- and \pali{gha}-alias, see \hyperref[sut:59]{Kacc 59} and \hyperref[sut:60]{60}.}

\head{180}{180, 207. netāhi smimāyayā.}
\headtrans{No \pali{āya}- and \pali{yā}-substitution for \pali{smiṃ}[-vibhatti] after these [\pali{gha}- and \pali{pa}-pronouns].}
\sutdef{etehi sabbanāmehi ghapasaññehi smiṃvacanassa neva āya\-yāādesā honti.}
\sutdeftrans{There are no substitutions of \pali{āya} and \pali{yā} for \pali{smiṃ}-vibhatti after these \pali{gha}- and \pali{pa}-pronouns.}
\example[0]{etissaṃ = etā + smiṃ (= etāyaṃ)}
\example[0]{imissaṃ = imā + smiṃ (= imāyaṃ)}
\example{amussaṃ = amu + smiṃ (= amuyaṃ)}
\transnote{That is to say, according to this rule, there are no locative \pali{etāya}, \pali{imāya}, and \pali{amuyā}.}

\head{181}{181, 95. manogaṇādito smiṃnānamiā.}
\headtrans{After the \pali{mano}-group, for \pali{smiṃ} and \pali{nā}[-vibhatti], [there are] \pali{i}- and \pali{ā}-substitution.}
\sutdef{tasmā manogaṇādito smiṃnāiccetesaṃ ikāraākārādesā honti vā yathāsaṅkhyaṃ.}
\sutdeftrans{There are substitutions of \pali{i} and \pali{ā} for \pali{smiṃ} and \pali{nā}[-vibhatti] after the \pali{mano}-group.}
\example[0]{manasi = mana + smiṃ (= manasmiṃ)}
\example[0]{sirasi = sira + smiṃ (= sirasmiṃ)}
\example[0]{manasā = mana + nā (= manena)}
\example[0]{vacasā = vaca + nā (= vacena)}
\example[0]{sirasā = sira + nā (= sirena)}
\example[0]{sarasā = sara + nā (= sarena)}
\example[0]{tapasā = tapa + nā (= tapena)}
\example[0]{vayasā = vaya + nā (= vayena)}
\example[0]{yasasā = yasa + nā (= yasena)}
\example[0]{tejasā = teja + nā (= tejena)}
\example[0]{urasā = ura + nā (= urena)}
\example{thāmasā = thāma + nā (= thāmena)}

\head{182}{182, 97. sassa co.}
\headtrans{For \pali{sa}[-vibhatti], [there is] also \pali{o}[-substitution].}
\sutdef{tasmā manogaṇādito sassa ca okāro hoti.}
\sutdeftrans{There is also \pali{o}[-substitution] for \pali{sa}[-vibhatti] after the \pali{mano}-group.}
\example[0]{manaso = mana + sa (= manassa)}
\example[0]{thāmaso = thāma + sa (= thāmassa)}
\example{tapaso = tapa + sa (= tapassa)}

\head{183}{183, 48. etesamo lope.}
\headtrans{[The ending] of these [\pali{mana} and so on becomes] \pali{o}, when elided.}
\sutdef{etesaṃ manogaṇādīnaṃ anto ottamāpajjate vibhattilope kate.}
\sutdeftrans{When the vibhatti was elided, the ending of the \pali{mano}-group becomes \pali{o}.}
\example[0]{manomayaṃ {\upshape= constructed by the mind}}
\example[0]{ayomayaṃ {\upshape= made by iron}}
\example[0]{tejosamena {\upshape= equal to fire}}
\example[0]{tapoguṇena {\upshape= by the virtue of austerity}}
\example{siroruhena {\upshape= grown on the head}}
\transnote{This rule is applied for the \pali{mano}-group in compounds.}

\head{184}{184, 96. sa sare vāgamo.}
\headtrans{[There is] \pali{sa}-insertion because of vowel sometimes.}
\sutdef{eteheva manogaṇādīhi vibhattādese sare pare sakārāgamo hoti vā.}
\sutdeftrans{There is an insertion of \pali{sa} because of the vowel behind produced by vibhatti-substitution after the \pali{mano}-group sometimes.}
\example[0]{manasā = mana + sa}
\example[0]{vacasā = vaca + sa}
\example[0]{manasi = mana + smiṃ}
\example{vacasi = vaca + smiṃ}

\head{185}{185, 112. santasaddassa so bhe bo cante.}
\headtrans{For \pali{santa}, [there is] \pali{sa}-substitution and \pali{ba}-insertion at the end.}
\sutdef{sabbassa santasaddassa sakārādeso hoti bhakāre pare, ante ca bakārāgamo hoti.}
\sutdeftrans{There is a substitution of \pali{sa} for the whole \pali{santa} because of \pali{bha} behind, also an insertion of \pali{ba} at the end.}
\example[0]{sabbhi = santa + hi}
\example[0]{sabbhūto = santa + bhūta + si}
\example{sabbhāvo = santa + bhāva + si}
\transnote{In the first example, when \pali{santa} is followed by \pali{hi}, it becomes \pali{sa}, hence \pali{sa + hi}. Then \pali{ba} is inserted at the end, and \pali{hi} becomes \pali{bhi} (see \hyperref[sut:99]{Kacc 99}), yielding \pali{sabbhi}. Other examples can be understood in the same way.}

\head{186}{186, 107. simhi gacchantādīnaṃ ntasaddo aṃ.}
\headtrans{Because of \pali{si}[-vibhatti], \pali{nta} of \pali{gacchanta} and so on [becomes] \pali{aṃ}.}
\sutdef{simhi gacchantādīnaṃ ntasaddo amāpajjate vā.}
\sutdeftrans{The word \pali{nta} of \pali{gacchanta} and so on becomes \pali{aṃ} sometimes because of \pali{si}[-vibhatti].}
\example[0]{gacchaṃ = gacchanta + si (= gacchanto)}
\example[0]{mahaṃ = mahanta + si (= mahanto)}
\example[0]{caraṃ = caranta + si (= caranto)}
\example{khādaṃ = khādanta + si (= khādanto)}

\head{187}{187, 108. sesesu ntuva.}
\headtrans{Because of the remaining [vibhattis and paccayas], [\pali{nta} should be seen as] \pali{ntu}-paccaya.}
\sutdef{gacchantādīnaṃ ntasaddo ntuppaccayova daṭṭhabbo sesesu vibhattippaccayesu.}
\sutdeftrans{The \pali{nta} of \pali{gacchanta} and so on should be seen as \pali{ntu}-paccaya because of the remaining vibhattis and paccayas.}
\example[0]{gacchato = gacchanta + sa}
\example[0]{mahato = mahanta + sa}
\example[0]{gacchati = gacchanta + smiṃ}
\example[0]{mahati = mahanta + smiṃ}
\example[0]{gacchatā = gacchanta + nā}
\example{mahatā = mahanta + nā}
\transnote{For the comming of \pali{to}, \pali{ti}, and \pali{tā}, see \hyperref[sut:127]{Kacc 127}.}

\head{188}{188, 115. brahmattasakharājādito amānaṃ.}
\headtrans{After \pali{brahma}, \pali{atta}, \pali{sakha}, \pali{rāja}, and so on, \pali{aṃ}[-vibhatti] [becomes] \pali{ānaṃ}.}
\sutdef{brahmaattasakharājaiccevamādito aṃvacanassa ānaṃ hoti vā.}
\sutdeftrans{There is \pali{ānaṃ}[-substitution] for \pali{aṃ}-vibhatti sometimes after \pali{brahma}, \pali{atta}, \pali{sakha}, \pali{rāja}, and so on.}
\example[0]{brahmānaṃ = brahma + aṃ (= brahmaṃ)}
\example[0]{attānaṃ = atta + aṃ (= attaṃ)}
\example[0]{sakhānaṃ = sakha + aṃ (= sakhaṃ)}
\example{rājānaṃ = rāja + aṃ (= rājaṃ)}

\head{189}{189, 113. syā ca.}
\headtrans{Also \pali{si}[-vibhatti] [becomes] \pali{ā}.}
\sutdef{brahmaattasakharājaiccevamādito sivacanassa ā ca hoti.}
\sutdeftrans{There is \pali{ā}[-substitution] for \pali{si}-vibhatti also after \pali{brahma}, \pali{atta}, \pali{sakha}, \pali{rāja}, and so on.}
\example[0]{brahmā = brahma + si}
\example[0]{attā = atta + si}
\example[0]{sakhā = sakha + si}
\example[0]{rājā = rāja + si}
\example{ātumā = ātuma + si}

\head{190}{190, 114. yonamāno.}
\headtrans{For \pali{yo}[-vibhatti], [there is] \pali{āno}-substitution.}
\sutdef{brahmaattasakharājaiccevamādito yonaṃ ānoādeso hoti.}
\sutdeftrans{There is a substitution of \pali{āno} for \pali{yo}[-vibhatti] after \pali{brahma}, \pali{atta}, \pali{sakha}, \pali{rāja}, and so on.}
\example[0]{brahmāno = brahma + yo}
\example[0]{attāno = atta + yo}
\example[0]{sakhāno = sakha + yo}
\example[0]{rājāno = rāja + yo}
\example{ātumāno = ātuma + yo}

\head{191}{191, 130. sakhato cāyono.}
\headtrans{After \pali{sakha}, [there is] also \pali{āyo}- and \pali{no}-substitution.}
\sutdef{tasmā sakhato ca yonaṃ āyonoādesā honti.}
\sutdeftrans{There are substitutions of \pali{āyo} and \pali{no} for \pali{yo}[-vibhatti] also after that \pali{sakha}.}
\example[0]{sakhāyo = sakha + yo}
\example{sakhino = sakha + yo}

\head{192}{192, 135. smime.}
\headtrans{[For] \pali{smiṃ}[-vibhatti], [there is] \pali{e}-substitution.}
\sutdef{tasmā sakhato smiṃvacanassa ekāro hoti.}
\sutdeftrans{There is a substitution of \pali{e} for \pali{smiṃ}-vibhatti after that \pali{sakha}.}
\example{sakhe = sakha + smiṃ}

\head{193}{193, 122. brahmato gassa ca.}
\headtrans{After \pali{brahma}, for \pali{ga}-alias also [there is \pali{e}-substitution].}
\sutdef{tasmā brahmato gassa ca ekāro hoti.}
\sutdeftrans{There is \pali{e}[-substitution] also for \pali{ga}-alias (vocative \pali{si}-vibhatti) after that \pali{brahma}.}
\example{he brahme {\upshape (O Brahmā!)}}
\transnote{See \hyperref[sut:57]{Kacc 57} for the definition of \pali{ga}-alias.}

\head{194}{194, 131. sakhantassi nonānaṃsesu.}
\headtrans{For the ending of \pali{sakha}, [there is] \pali{i}-substitution because of \pali{no}, \pali{nā}, \pali{naṃ}, and \pali{sa}[-vibhatti].}
\sutdef{tassa sakhantassa ikāro hoti nonānaṃsaiccetesu.}
\sutdeftrans{There is [a substitution of] \pali{i} for the ending of that \pali{sakha} because of \pali{no}, \pali{nā}, \pali{naṃ}, and \pali{sa}[-vibhatti].}
\example[0]{sakhino = sakha + yo}
\example[0]{sakhinā = sakha + nā}
\example[0]{sakhīnaṃ = sakha + naṃ}
\example{sakhissa = sakha + sa}
\transnote{The \pali{no} mentioned is a product of \pali{yo}-vibhatti, described in \hyperref[sut:191]{Kacc 191}.}

\head{195}{195, 134. āro himhi vā.}
\headtrans{[There is] \pali{āra}-substitution [for \pali{sakha}] because of \pali{hi}[-vibhatti] sometimes.}
\sutdef{tassa sakhantassa āro hoti vā himhi vibhattimhi.}
\sutdeftrans{There is [a substitution of] \pali{āra} for the ending of that \pali{sakha} sometimes because of \pali{hi}-vibhatti.}
\example{sakhārehi = sakha + hi (= sakhehi)}

\head{196}{196, 133. sunamaṃsu vā.}
\headtrans{Because of \pali{su}, \pali{naṃ}, and \pali{aṃ}[-vibhatti], [there is \pali{āra}-substi\-tution for \pali{sakha}] sometimes.}
\sutdef{tassa sakhantassa āro hoti vā sunaṃaṃiccetesu.}
\sutdeftrans{There is [a substitution of] \pali{āra} for the ending of that \pali{sakha} sometimes because of \pali{su}, \pali{naṃ}, and \pali{aṃ}[-vibhatti].}
\example[0]{sakhāresu = sakha + su (= sakhesu)}
\example[0]{sakhārānaṃ = sakha + naṃ (= sakhīnaṃ)}
\example{sakhāraṃ = sakha + naṃ (= sakhaṃ)}

\head{197}{197, 125. brahmato tu smiṃ ni.}
\headtrans{After \pali{brahma}, [there is] \pali{ni}-substitution for \pali{smiṃ}[-vibhatti].}
\sutdef{tasmā brahmato smiṃvacanassa niādeso hoti.}
\sutdeftrans{There is a substitution of \pali{ni} for \pali{smiṃ}-vibhatti after that \pali{brahma}.}
\example{brahmani = brahma + smiṃ}
\transnote{As marked by \pali{tu}, there are other terms can be seen likewise, as follows:}
\example[0]{kammani = kamma + smiṃ}
\example[0]{cammani = camma + smiṃ}
\example{muddhani = muddha + smiṃ}

\head{198}{198, 123. uttaṃ sanāsu.}
\headtrans{[The ending of \pali{brahma} becomes] \pali{u} because of \pali{sa} and \pali{nā}[-vibhatti].}
\sutdef{tassa brahmasaddassa anto uttamāpajjate sanāiccetesu.}
\sutdeftrans{The ending of that \pali{brahma} becomes \pali{u} because of \pali{sa} and \pali{nā}[-vibhatti].}
\example[0]{brahmuno = brahma + sa}
\example{brahmunā = brahma + nā}

\head{199}{199, 158. satthupitādīnamā sismiṃ silopo ca.}
\headtrans{For \pali{satthu}, \pali{pitu}, and so on, [there is] \pali{ā}-substitution because of \pali{si}[-vibhatti]; the \pali{si} is elided also.}
\sutdef{satthupituādīnamanto āttamāpajjate sismiṃ, silopo ca hoti.}
\sutdeftrans{The ending of \pali{satthu}, \pali{pitu}, and so on becomes \pali{ā} because of \pali{si}[-vibhatti]; the \pali{si} is also elided.}
\example[0]{satthā = satthu + si}
\example[0]{pitā = pitu + si}
\example[0]{mātā = mātu + si}
\example[0]{bhātā = bhātu + si}
\example{kattā = kattu + si}

\head{200}{200, 159. aññesvārattaṃ.}
\headtrans{Because of other [vibhattis], [these is] \pali{āra}-substitution.}
\sutdef{satthupituādīnamanto aññesu vacanesu ārattamāpajjate.}
\sutdeftrans{The ending of \pali{satthu}, \pali{pitu}, and so on becomes \pali{āra} because of other vibhattis.}
\example[0]{satthāraṃ = satthu + aṃ}
\example[0]{pitaraṃ = pitu + aṃ}
\example[0]{mātaraṃ = mātu + aṃ}
\example[0]{bhātaraṃ = bhātu + aṃ}
\example[0]{kattāraṃ = kattu + aṃ}
\example[0]{satthārehi = satthu + hi}
\example[0]{pitarehi = pitu + hi}
\example[0]{mātarehi = mātu + hi}
\example[0]{bhātarehi = bhātu + hi}
\example{kattārehi = kattu + hi}
\transnote{For why \pali{āra} becomes \pali{ara}, see \hyperref[sut:209]{Kacc 209}.}

\head{201}{201, 163. vā naṃmhi.}
\headtrans{Sometimes because of \pali{naṃ}[-vibhatti].}
\sutdef{satthupituādīnamanto ārattamāpajjate vā naṃmhi vibhattimhi.}
\sutdeftrans{The ending of \pali{satthu}, \pali{pitu}, and so on becomes \pali{āra} sometimes because of \pali{naṃ}-vibhatti.}
\example[0]{satthārānaṃ = satthu + naṃ}
\example[0]{pitarānaṃ = pitu + naṃ}
\example[0]{mātarānaṃ = mātu + naṃ}
\example{bhātarānaṃ = bhātu + naṃ}

\head{202}{202, 164. satthunattañca.}
\headtrans{For \pali{satthu}, [there is] also \pali{a}-substitution [because of \pali{naṃ}-vibhatti].}
\sutdef{tassa satthusaddassa anto attamāpajjate vā naṃmhi vibhattimhi.}
\sutdeftrans{The ending of that \pali{satthu} becomes \pali{a} sometimes because of \pali{naṃ}-vibhatti.}
\example[0]{satthānaṃ = satthu + naṃ}
\example[0]{pitānaṃ = pitu + naṃ}
\example[0]{mātānaṃ = mātu + naṃ}
\example[0]{bhātānaṃ = bhātu + naṃ}
\example{kattānaṃ = kattu + naṃ}

\head{203}{203, 162. u sasmiṃ salopo ca.}
\headtrans{[There is] \pali{u}-substitution because of \pali{sa}[-vibhatti]; and \pali{sa} [is] elided.}
\sutdef{satthupituiccevamādīnamantassa uttaṃ hoti vā sasmiṃ, salopo ca.}
\sutdeftrans{There is [a substitution of] \pali{u} for the ending of \pali{satthu}, \pali{pitu}, and so on sometimes because of \pali{sa}[-vibhatti]; the \pali{sa} [is] also elided.}
\example[0]{satthu = satthu + sa (= satthussa, satthuno)}
\example[0]{pitu = pitu + sa (= pitussa, pituno)}
\example{bhātu = bhātu + sa (= bhātussa, bhātuno)}

\head{204}{204, 167. sakkamandhātādīnañca.}
\headtrans{Also [the ending of] \pali{sakkamandhātu} and so on [becomes \pali{u}].}
\sutdef{sakkamandhātuiccevamādīnamanto uttamāpajjate sasmiṃ, salopo ca hoti.}
\sutdeftrans{The ending of \pali{sakkamandhātu} and so on becomes \pali{u} because of \pali{sa}[-vibhatti]; the \pali{sa} is also elided.}
\example[0]{sakkamandhātu = sakkamandhātu + sa}
\example[0]{kattu = kattu + sa}
\example[0]{gantu = gantu + sa}
\example{dātu = dātu + sa}

\head{205}{205, 160. tato yonamo tu.}
\headtrans{After that [\pali{āra}, there is] \pali{o}-substitution for \pali{yo}[-vibhatti].}
\sutdef{tato ārādesato sabbesaṃ yonaṃ okārādeso hoti.}
\sutdeftrans{There is a substitution of \pali{o} for all \pali{yo}[-vibhatti] after that \pali{āra} produced.}
\example[0]{satthāro = satthu + yo}
\example[0]{pitaro = pitu + yo}
\example[0]{mātaro = mātu + yo}
\example[0]{bhātaro = bhātu + yo}
\example[0]{kattāro = kattu + yo}
\example{vattāro = vattu + yo}

\head{206}{206, 165. tato smimi.}
\headtrans{After that [\pali{āra}, there is] \pali{i}-substitution for \pali{smiṃ}[-vibhatti].}
\sutdef{tato ārādesato smiṃvacanassa ikārādeso hoti.}
\sutdeftrans{There is a substitution of \pali{i} for \pali{smiṃ}-vibhatti after that \pali{āra} produced.}
\example[0]{satthari = satthu + smiṃ}
\example[0]{pitari = pitu + smiṃ}
\example[0]{mātari = mātu + smiṃ}
\example[0]{dhītari = dhītu + smiṃ}
\example[0]{bhātari = bhātu + smiṃ}
\example[0]{kattari = kattu + smiṃ}
\example{vattari = vattu + smiṃ}

\head{207}{207, 161. nā ā.}
\headtrans{[For] \pali{nā}[-vibhatti], [there is] \pali{ā}-substitution.}
\sutdef{tato ārādesato nāvacanassa āādeso hoti.}
\sutdeftrans{There is a substitution of \pali{ā} for \pali{nā}-vibhatti after that \pali{āra} produced.}
\example[0]{satthārā = satthu + nā}
\example[0]{pitarā = pitu + nā}
\example[0]{mātarā = mātu + nā}
\example[0]{bhātarā = bhātu + nā}
\example[0]{dhītarā = dhītu + nā}
\example[0]{kattārā = kattu + nā}
\example{vattārā = vattu + nā}

\head{208}{208, 166. āro rassamikāre.}
\headtrans{The \pali{āra} [becomes] shortened because of \pali{i}.}
\sutdef{ārādeso rassamāpajjate ikāre pare.}
\sutdeftrans{The \pali{āra}-substitution becomes shortened because of \pali{i} behind.}
\example[0]{satthari = satthu + smiṃ}
\example[0]{pitari = pitu + smiṃ}
\example[0]{mātari = mātu + smiṃ}
\example[0]{dhītari = dhītu + smiṃ}
\example[0]{kattari = kattu + smiṃ}
\example{vattari = vattu + smiṃ}
\transnote{The \pali{i} mentioned is produced by \pali{smiṃ}-vibhatti, mentioned in \hyperref[sut:206]{Kacc 206}.}

\head{209}{209, 168. pitādīnamasimhi.}
\headtrans{For \pali{pitu} and so on, because of non-\pali{smiṃ}-vibhatti [\pali{āra} becomes shortened].}
\sutdef{pitādīnamārādeso rassamāpajjate asimhi vibhattimhi.}
\sutdeftrans{The \pali{āra}-substitution for \pali{pitu} and so on becomes shortened because of non-\pali{smiṃ}-vibhatti.}
\example[0]{pitarā = pitu + nā}
\example[0]{mātarā = mātu + nā}
\example[0]{bhātarā = bhātu + nā}
\example[0]{dhītarā = dhīta + nā}
\example[0]{pitaro = pita + yo}
\example[0]{mātaro = māta + yo}
\example[0]{bhātaro = bhāta + yo}
\example{dhītaro = dhīta + yo}

\head{210}{210, 239. tayātayīnaṃ takāro tvattaṃ vā.}
\headtrans{The \pali{ta}-letter of \pali{tayā} and \pali{tayi} [becomes] \pali{tva} sometimes.}
\sutdef{tayātayiiccetesaṃ takāro tvattamāpajjate vā.}
\sutdeftrans{The \pali{ta}-letter of \pali{tayā} and \pali{tayi} becomes \pali{tva} sometimes.}
\example[0]{tvayā = tumha + nā (= tayā)}
\example{tvayi = tumha + smiṃ (= tayi)}

