\section{Dutiyakaṇḍa}

\boxnote{Explanation on verbal derivatives.\\
\hspace{5mm}\bullet\ All paccayas producing verbal derivatives are listed in the table below, ordered by their first introduction.\\
\hspace{5mm}\bullet\ The Sutta column shows only the first one the paccaya is defined. For a more exhaustive list, see Table \ref{tab:kitapacc}.\\
\hspace{5mm}\bullet\ The columns Pres.\ (present meaning, future included), Past (past meaning), Act.\ (active meaning), and Pass.\ (passive meaning) show a variety of practical use cases. The absence of time can mean the products can be used regardless of time, or in all times (as often stated in the suttas).\\
\hspace{5mm}\bullet\ The last column shows that whether the products can complete a sentence like finite verbs.\\
\hspace{5mm}\bullet\ Parentheses show a possible application, not a normal use.
}

\setcounter{table}{1}
\begin{longtable}{%
		>{\itshape\raggedright\arraybackslash}p{0.15\linewidth}%
		>{\centering\arraybackslash}p{0.10\linewidth}%
		>{\centering\arraybackslash}p{0.08\linewidth}%
		>{\centering\arraybackslash}p{0.08\linewidth}%
		>{\centering\arraybackslash}p{0.08\linewidth}%
		>{\centering\arraybackslash}p{0.08\linewidth}%
		>{\centering\arraybackslash}p{0.08\linewidth}}
\caption{Verbal derivatives summarized}\label{tab:kitaverb}\\
\toprule
\upshape\bfseries\mbox{Paccaya} & \bfseries\mbox{Sutta} & \bfseries\mbox{Pres.} & \bfseries\mbox{Past} & \bfseries\mbox{Act.} & \bfseries\mbox{Pass.} & \bfseries\mbox{Finite} \\ \midrule
\endfirsthead
\multicolumn{7}{c}{\footnotesize\tablename\ \thetable: Verbal derivatives summarized (contd\ldots)}\\
\toprule
\upshape\bfseries\mbox{Paccaya} & \bfseries\mbox{Sutta} & \bfseries\mbox{Pres.} & \bfseries\mbox{Past} & \bfseries\mbox{Act.} & \bfseries\mbox{Pass.} & \bfseries\mbox{Finite} \\ \midrule
\endhead
\bottomrule
\ltblcontinuedbreak{7}
\endfoot
\bottomrule
\endlastfoot
%
anīya & \hyperref[sut:540]{540} & & & & \checkmark & \checkmark \\
tabba & \hyperref[sut:540]{540} & & & & \checkmark & \checkmark \\
ta & \hyperref[sut:555]{555} & (\checkmark) & \checkmark & \checkmark & \checkmark & \checkmark \\
tavantu & \hyperref[sut:555]{555} & & \checkmark & \checkmark & & \\
tāvī & \hyperref[sut:555]{555} & & \checkmark & \checkmark & & \\
tave & \hyperref[sut:561]{561} & & & \checkmark & \checkmark & \\
tuṃ & \hyperref[sut:561]{561} & & & \checkmark & \checkmark & \\
tuna & \hyperref[sut:564]{564} & (\checkmark) & \checkmark & \checkmark & \checkmark & (\checkmark) \\
tvā & \hyperref[sut:564]{564} & (\checkmark) & \checkmark & \checkmark & \checkmark & (\checkmark) \\
tvāna & \hyperref[sut:564]{564} & (\checkmark) & \checkmark & \checkmark & \checkmark & (\checkmark) \\
anta & \hyperref[sut:565]{565} & \checkmark & & \checkmark & & \\
māna & \hyperref[sut:565]{565} & \checkmark & & \checkmark & \checkmark & \\
\end{longtable}

\head{550}{550, 546. ṇādayo tekālikā.}
\headtrans{The \pali{ṇa}-paccaya and so on [are of] the three times.}
\sutdef{ṇādayo paccayā yupaccayantā tekālikāti veditabbā.}
\sutdeftrans{The paccayas starting from \pali{ṇa} and so on to \pali{yu} at the end should be known as related to the three times.}
\example[0]{kumbhaṃ karoti akāsi karissatīti kumbhakāro \\(kumbha + \paliroot{kara} + ṇa) \\{\upshape= [One] makes, made, and will make a pot, hence \pali{kumbhakāra} (potter).}}
\example{karoti akāsi karissati tenāti karaṇaṃ \\(kumbha + \paliroot{kara} + yu) \\{\upshape= [One] does, did, and will do by that, hence \pali{karaṇa} (cause).}}
\transnote{It is clear that the three times (\pali{tekālika}) mean present, past, and future.}

\head{551}{551, 598. saññāyaṃ dādhāto i.}
\headtrans{In [designating] a name after \pali{dā} and \pali{dhā}, [there is] \pali{i}-paccaya.}
\sutdef{saññāyamabhidheyyāyaṃ dādhāto ipaccayo hoti.}
\sutdeftrans{There is \pali{i}-paccaya after [the roots] \pali{dā} and \pali{dhā} in designating a name.}
\example[0]{paṭhamaṃ ādīyatīti ādi (ā + \paliroot{dā} + i) \\{\upshape= [It] is taken first, hence \pali{ādi} (beginning)}}
\example[0]{udakaṃ dadhātīti udadhi (udaka + \paliroot{dhā} + i) \\{\upshape= [It] upholds water, hence \pali{udadhi} (sea)}}
\example[0]{mahodakāni dadhātīti mahodadhi (mahā + udaka + \paliroot{dhā}~+~i) \\{\upshape= [It] upholds great [amount of] water, hence \pali{mahodadhi} (ocean)}}
\example[0]{vālāni dadhāti tasminti vāladhi (vāla + \paliroot{dhā} + i) \\{\upshape= [It] upholds hairs, hence \pali{vāladhi} (tail)}}
\example{sammā dhīyatīti sandhi (saṃ + \paliroot{dhā} + i) \\{\upshape= [It] is placed well, hence \pali{sandhi} (combination)}}

\head{552}{552, 609. ti kita cāsiṭṭhe.}
\headtrans{[There is] \pali{ti}-paccaya, also [other] \pali{kita}-paccayas in [the sense of] desire.}
\sutdef{saññāyamabhidheyyāyaṃ sabbadhātūhi tipaccayo hoti, kita ca āsiṭṭhe.}
\sutdeftrans{There is \pali{ti}-paccaya after all roots in designating a name, also [other] \pali{kita}-paccayas in [the sense of] desire.}
\example[0]{jino enaṃ bujjhatūti jinabuddhi (jina + \paliroot{budha} + ti) \\{\upshape= Let the Buddha know that, hence \pali{jinabuddhi} (knowledge for the Buddha).}}
\example[0]{dhanaṃ assa bhavatūti dhanabhūti (dhana + \paliroot{bhū} + ti) \\{\upshape= Let the wealth of him exist, hence \pali{dhanabhūti} (wealthy one).}}
\example[0]{bhavatūti bhūto (\paliroot{bhū} + ta) \\{\upshape= Let he exist, hence \pali{bhūta} (being).}}
\example[0]{dhammo enaṃ dadātūti dhammadinno (dhamma + \paliroot{dā} + ta) \\{\upshape= Let the nature gives that, hence \pali{dhammadinna} (one given by the nature).}}
\example{vaḍḍhatūti vaḍḍhamāno (\paliroot{vaḍḍha} + māna) \\{\upshape= Let him prosper, hence \pali{vaḍḍhamāna} (prospering one).}}
\transnote{As suggested by the definition, these examples are names of persons because they have a sense of wishing something.}

\head{553}{553, 599. itthiyamatiyavo vā.}
\headtrans{In [designating] a feminine word, [there are] \pali{a}-, \pali{ti}-, and \pali{yu}-paccaya sometimes.}
\sutdef{itthiyamabhidheyyāyaṃ sabbadhātūhi akāratiyuiccete paccayā honti vā.}
\sutdeftrans{There are \pali{a}-, \pali{ti}-, and \pali{yu}-paccaya sometimes after all roots in designating a feminine word.}
\example[0]{jīratītī jarā (\paliroot{jara} + a) \\{\upshape= [One] ages, hence \pali{jarā} (old age).}}
\example[0]{maññatīti mati (\paliroot{mana} + ti) \\{\upshape= [One] thinks, hence \pali{mati} (idea).}}
\example[0]{cetayatīti cetanā (\paliroot{cita} + yu) \\{\upshape= [One] intends, hence \pali{cetanā} (intention).}}
\example{vedayatīti vedanā (\paliroot{vida} + yu) \\{\upshape= [One] feels, hence \pali{vedanā} (feeling).}}

\head{554}{554, 601. karato ririya.}
\headtrans{After \pali{kara}, [there is] \pali{ririya}-paccaya.}
\sutdef{karato itthiyamanitthiyaṃ vā abhidheyyāyaṃ ririyapaccayo hoti vā.}
\sutdeftrans{There is \pali{ririya}-paccaya after [the root] \pali{kara} in designating a word, feminine or not.}
\example[0]{kattabbā kiriyā (\paliroot{kara} + ririya) \\{\upshape= [It is] worthy to be done, hence \pali{kiriyā} (action).}}
\example{karaṇīyaṃ kiriyaṃ (\paliroot{kara} + ririya) \\{\upshape= [It is] worthy to be done, hence \pali{kiriya} (action).}}
\transnote{The first \pali{ra} of the paccaya is anubandha causing the last syllable to be elided (see \hyperref[sut:539]{Kacc 539}). As shown by the examples, the word has two genders (feminine and neuter) in the same meaning. However, only \pali{kiriyā} does mean `verb' in grammatical sense.}

\head{555}{555, 612. atīte tatavantutāvī.}
\headtrans{In the past [meaning, there are] \pali{ta}-, \pali{tavantu}-, and \pali{tāvī}-paccaya.}
\sutdef{atīte kāle sabbadhātūhi tatavantutāvīiccete paccayā honti.}
\sutdeftrans{There are \pali{ta}-, \pali{tavantu}-, and \pali{tāvī}-paccaya after all roots in the past [meaning].}
\example[0]{huto, hutavā, hutāvī (\paliroot{hu} + ta/tavantu/tāvī) \\{\upshape= [One] sacrificed.}}
\example[0]{vusito, vusitavā, vusitāvī (\paliroot{vasa} + ta/tavantu/tāvī) \\{\upshape= [One] lived.}}
\example{bhutto, bhuttavā, bhuttāvī (\paliroot{bhuja} + ta/tavantu/tāvī) \\{\upshape= [One] ate.}}
\transnote{The products of these are past non-finite verbs. The words have to be applied with a nominal vibhatti in agreement with gender and number of their subject, for example, \pali{huto/hutā/ hutaṃ} (nominative singular m./f./nt.), \pali{hutā/hutāyo/hutāni} \linebreak(nominative plural m./f./nt.). Other cases can be rendered accordingly.}
\transnote{For terms ending with \pali{-tavantu}, they follow the paradigm of \pali{guṇavantu} (see \hyperref[sut:124]{Kacc 124}), thus \pali{hutavā/hutavatī/hutavaṃ} (nominative singular m./f./nt.). Do not be confused this with the absolutive \pali{hutvā}, which is completely a different form, used as an indeclinable.}
\transnote{For terms ending with \pali{-tāvī}, they have three genders, thus \pali{hutāvī/hutāvī/hutāvi} (nominative singular m./f./nt.).}
\transnote{Scholars often call the products of these as \emph{past participle}. The application is close but not exactly alike.}

\head{556}{556, 622. bhāvakammesu ta.}
\headtrans{In impersonal and passive [meaning, there is] \pali{ta}-paccaya.}
\sutdef{bhāvakammesu atīte kāle tapaccayo hoti sabbadhātūhi.}
\sutdeftrans{There is \pali{ta}-paccaya after all roots in the past impersonal and passive meaning.}
\example[0]{tassa gītaṃ (\paliroot{ge} + ta) \\{\upshape= The singing of that [person].}}
\example[0]{tassa naccaṃ, naṭṭaṃ (\paliroot{naṭa} + ta) \\{\upshape= The dancing of that [person].}}
\example[0]{tassa hasitaṃ (\paliroot{hasa} + ta) \\{\upshape= The laughing of that [person].}}
\example[0]{tena bhāsitaṃ (\paliroot{bhāsa} + ta) \\{\upshape= [It was] said by that [person].}}
\example{tena desitaṃ (\paliroot{disa} + ta) \\{\upshape= [It was] explained by that [person].}}
\transnote{It is better to see \pali{bhāva} here as a state of action.}

\head{557}{557, 606. budhagamāditthe kattari.}
\headtrans{In the sense of \pali{budha}, \pali{gamu} and so on, [there is \pali{ta}-paccaya] in active meaning.}
\sutdef{budhagamuiccevamādīhi dhātūhi tadatthe gamyamāne tapaccayo hoti kattari sabbakāle.}
\sutdeftrans{There is \pali{ta}-paccaya after the roots such as \pali{budha}, \pali{gamu} and so on in active meaning of all times.}
\example[0]{sabbe saṅkhatāsaṅkhate dhamme bujjhati abujjhi bujjhissatīti buddho (\paliroot{budha} + ta) \\{\upshape= [One] knows, knew, and will know all conditioned and unconditioned nature, hence \pali{buddha} (the Buddha).}}
\example[0]{saraṇaṅgato (saraṇa + \paliroot{gamu} + ta) \\{\upshape= a refuge-taking one}}
\example[0]{samathaṅgato (samatha + \paliroot{gamu} + ta) \\{\upshape= a peace-attaining one}}
\example[0]{amataṅgato (amata + \paliroot{gamu} + ta) \\{\upshape= an immortality-going one}}
\example{jānāti ajāni jānissatīti ñāto (\paliroot{ñā} + ta) \\{\upshape= [One] knows, knew, and will know, hence \pali{ñāta} (knower).}}
\transnote{In the definition, `\pali{tadatthe gamyamāne}' (being understood in that sense) looks redundant and adds nothing to the meaning, so I skipped it. For \pali{gamyamāna}, see MWD.}
\transnote{The products in this case are seemingly nominals regardless of time. But when the words are used as a verb, I supposed that the past meaning is normally expected.}

\head{558}{558, 602. jito ina sabbattha.}
\headtrans{After \pali{ji}, [there is] \pali{ina}-paccaya in all times.}
\sutdef{jiiccetāya dhātuyā inapaccayo hoti sabbakāle kattari.}
\sutdeftrans{There is \pali{ina}-paccaya after the root \pali{ji} in active meaning of all times.}
\example{pāpake akusale dhamme jināti ajini jinissatīti jino \\(\paliroot{ji} + ina) \\{\upshape= [One] overcomes, overcame, and will overcome the evil and unwholesome nature, hence \pali{jina} (the Buddha/Victor).}}

\head{559}{559, 603. supato ca.}
\headtrans{After \pali{supa} also, [there is \pali{ina}-paccaya].}
\sutdef{supaiccetāya dhātuyā inapaccayo hoti kattari, bhāve ca.}
\sutdeftrans{There is \pali{ina}-paccaya after the root \pali{supa} in the active meaning and an impersonal state.}
\example[0]{supatīti supinaṃ (\paliroot{supa} + ina) \\{\upshape= [One] dreams, hence \pali{supina} (dream).}}
\example{supīyate supinaṃ (\paliroot{supa} + ina) \\{\upshape= Dreaming is done, hence \pali{supina} (dreaming).}}

\head{560}{560, 604. īsaṃdusūhi kha.}
\headtrans{After \pali{īsaṃ}, \pali{du}, \pali{su} [nearby, there is] \pali{kha}-paccaya.}
\transnote{In a Thai source, this is `\pali{īsadusūhi kha}' instead.}
\sutdef{īsaṃdususaddādīhi sabbadhātūhi khapaccayo hoti.}
\sutdeftrans{There is \pali{kha}-paccaya after all roots having \pali{īsaṃ}, \pali{du}, \pali{su} and so on [nearby].}
\example[0]{īsassayo (īsaṃ + \paliroot{sī} + kha) \\{\upshape= a little sleep}}
\example[0]{dussayo (du + \paliroot{sī} + kha) \\{\upshape= a bad sleep}}
\example[0]{sussayo (su + \paliroot{sī} + kha) \\{\upshape= a good sleep}}
\example[0]{īsakkaraṃ (īsaṃ + \paliroot{kara} + kha) \\{\upshape= a little action}}
\example[0]{dukkaraṃ (du + \paliroot{kara} + kha) \\{\upshape= a difficult action}}
\example{sukaraṃ (su + \paliroot{kara} + kha) \\{\upshape= an easy action}}

\head{561}{561, 636. icchatthesu samānakattukesu tavetuṃ vā.}
\headtrans{In the sense of desire having the same subject, [there are] \pali{tave}- and \pali{tuṃ}-paccaya.}
\sutdef{icchatthesu samānakattukesu sabbadhātūhi tavetuṃiccete paccayā honti sabbakāle kattari.}
\sutdeftrans{There are \pali{tave}- and \pali{tuṃ}-paccaya after all roots in the sense of desire having the same subject, in the active meaning of all times.}
\example[0]{puññāni kātave (\paliroot{kara} + tave) icchati \\{\upshape= [One] desires to make merits.}}
\example{saddhammaṃ sotuṃ (\paliroot{su} + tuṃ) icchati \\{\upshape= [One] desires to hear the true teaching.}}
\transnote{These two verbal forms are used like indeclinables. Scholars call these \emph{infinitive}.}

\head{562}{562, 638. arahasakkādīsu ca.}
\headtrans{In [the senses of] worthiness and capableness also, [there is \pali{tuṃ}-paccaya].}
\sutdef{arahasakkādīsu ca atthesu sabbadhātūhi tuṃpaccayo hoti.}
\sutdeftrans{There is \pali{tuṃ} after all roots in the senses of worthiness and capableness also.}
\example[0]{ko taṃ nindituṃ (\paliroot{ninda} + tuṃ) arahati \\{\upshape= Who is worthy to insult that [person]?}}
\example{sakkā jetuṃ (\paliroot{ji} + tuṃ) dhanena vā \\{\upshape= [One is] able to win by wealth also.}}
\transnote{The last example is out of context. It is from Pabbatūpamasutta (S1 136) mentioning old age and death. The full line of it is ``\pali{Na cāpi mantayuddhena, sakkā jetuṃ dhanena vā}'' (Not also by spell [or] battle, or wealth can [one] defeat [old age and death]).}

\head{563}{563, 639. pattavacane alamatthesu ca.}
\headtrans{In expressing suitableness, in the sense of \pali{alaṃ}, also [there is \pali{tuṃ}-paccaya].}
\sutdef{pattavacane alamatthesu sabbadhātūhi tuṃpaccayo hoti.}
\sutdeftrans{There is \pali{tuṃ}-paccaya after all roots in the sense of \pali{alaṃ}, expressing suitableness.}
\example[0]{alameva dānāni dātuṃ (\paliroot{dā} + tuṃ) \\{\upshape= It is indeed suitable to give alms.}}
\example{alameva puññāni kātuṃ (\paliroot{kara} + tuṃ) \\{\upshape= It is indeed suitable to make merits.}}
\transnote{In Rūpa 639, \pali{pattavacana} is described as ``\pali{pattassa vacanaṃ pattavacanaṃ}'' which is of little help. To understand this, you have to see \pali{prāpta} in MWD. This word intended by \pali{patta} has one meaning as `proper' or `right.'}

\head{564}{564, 640. pubbakālekakattukānaṃ tunatvānatvā vā.}
\headtrans{For [roots] with the same subject in the preceding time, [there are] \pali{tuna}-, \pali{tvāna}-, and \pali{tvā}-paccaya sometimes.}
\sutdef{pubbakāle ekakattukānaṃ dhātūnaṃ tunatvānatvāiccete paccayā honti vā.}
\sutdeftrans{There are \pali{tuna}-, \pali{tvāna}-, and \pali{tvā}-paccaya sometimes for roots with the same subject in the preceding time.}
\example[0]{kātuna (\paliroot{kara} + tuna) kammaṃ gacchati \\{\upshape= Having done the action, [one] goes.}}
\example[0]{akātuna (na + \paliroot{kara} + tuna) puññaṃ kilissati \\{\upshape= Not having made merit, [one] suffers.}}
\example[0]{sattā sutvāna (\paliroot{su} + tvāna) dhammaṃ modanti \\{\upshape= Beings, having heard the teaching, rejoice.}}
\example[0]{ripuṃ jitvāna (\paliroot{ji} + tvāna) vasati \\{\upshape= Having defeated the enemy, [one] lives.}}
\example[0]{dhammaṃ sutvāna’ssa (\paliroot{su} + tvāna + assa) etadahosi \\{\upshape= Having heard the teaching, this [thought] happened to him.}}
\example[0]{ito sutvāna (\paliroot{su} + tvāna) amutra kathayanti \\{\upshape= Having heard here, [they] say there.}}
\example{sutvā (\paliroot{su} + tvā) jānissāma \\{\upshape= Having heard [that], [we] will know.}}
\transnote{The products of these three paccayas are mainly used to tell a sequence of events. They do not necessarily have only past meaning. In Thai tradition, we often use \pali{tūna} rather than \pali{tuna}. I also have amazed myself by not finding `\pali{ripu}' in Pāli dictionaries. See it in MWD.}
\transnote{Scholars call the products of these \emph{absolutive} or \emph{gerund}, among other names. None of the names is well fitted.}

\head{565}{565, 646. vattamāne mānantā.}
\headtrans{In the present time, [there are] \pali{māna}- and \pali{anta}-paccaya.}
\sutdef{vattamāne kāle sabbadhātūhi mānaantaiccete paccayā honti.}
\sutdeftrans{There are \pali{māna}- and \pali{anta}-paccaya after all roots in the present time.}
\example[0]{saramāno (\paliroot{sara} + māna) rodati \\{\upshape= Recalling, [one] cries.}}
\example{gacchanto (\paliroot{gamu} + anta) gaṇhāti \\{\upshape= Recalling, [one] cries.}}
\transnote{These verbals are called \emph{present participle} by scholars. The application looks similar.}

\head{566}{566, 574. sāsādīhi ratthu.}
\headtrans{After \pali{sāsa} and so on, [there is] \pali{ratthu}-paccaya.}
\sutdef{sāsaiccevamādīhi dhātūhi ratthupaccayo hoti.}
\sutdeftrans{There is \pali{ratthu}-paccaya after roots such as \pali{sāsa} and so on.}
\example{sāsatīti satthā (\paliroot{sāsa} + ratthu) \\{\upshape= [One] teaches, hence \pali{satthu} (teacher).}}
\transnote{This paccaya, and the followings, has \pali{ra}-anubandha, therefore the last syllable of the root is elided (see \hyperref[sut:539]{Kacc 539}). The products of this are declined irregularly (see \hyperref[sut:199]{Kacc 199}).}

\head{567}{567, 575. pātito ritu.}
\headtrans{After \pali{pā}, [there is] \pali{ritu}-paccaya.}
\sutdef{pāiccetāya dhātuyā ritupaccayo hoti.}
\sutdeftrans{There is \pali{ritu}-paccaya after the root \pali{pā}.}
\example{pāti puttanti pitā (\paliroot{pā} + ritu) \\{\upshape= [One] protects the child, hence \pali{pitu} (father).}}

\head{568}{568, 576. mānādīhi rātu.}
\headtrans{After \pali{māna} and so on, [there is] \pali{rātu}-paccaya, [\pali{ritu}-paccaya also].}
\sutdef{mānaiccevamādīhi dhātūhi rātupaccayo hoti, ritupaccayo ca.}
\sutdeftrans{There is \pali{rātu}-paccaya, \pali{ritu}-paccaya also, after roots such as \pali{māna} and so on.}
\example[0]{dhammena puttaṃ mānetīti mātā (\paliroot{māna} + rātu) \\{\upshape= [One] loves the child by nature, hence \pali{mātu} (mother).}}
\example[0]{pubbe bhāsatīti bhātā (\paliroot{bhāsa} + rātu) \\{\upshape= [One] speaks first, hence \pali{bhātu} (brother).}}
\example{mātāpitūhi dhārīyatīti dhītā (\paliroot{dhā} + ritu) \\{\upshape= [One] is held by parents, hence \pali{dhītu} (daughter).}}

\head{569}{569, 610. āgamā tuko.}
\headtrans{After \pali{āgama}, [there is] \pali{tuka}-paccaya.}
\sutdef{āiccādimhā gamito tukapaccayo hoti.}
\sutdeftrans{There is \pali{tuka}-paccaya after [the root] \pali{gamu} having \pali{ā} in the front.}
\example{āgacchatīti āgantuko (ā + \paliroot{gamu} + tuka) \\{\upshape= [One] comes, hence \pali{āgantuka} (guest).}}

\head{570}{570, 611. bhabbe ika.}
\headtrans{In [the sense of] capableness/likeliness, [there is] \pali{ika}-paccaya.}
\sutdef{gamuiccetamhā dhātumhā ikapaccayo hoti bhabbe.}
\sutdeftrans{There is \pali{ika}-paccaya after the root \pali{gamu} in [the sense of] capableness/likeliness.}
\example{gamissati gantuṃ bhabboti gamiko (\paliroot{gamu} + ika) \\{\upshape= [One] will go [or] is likely to go, hence \pali{gamika} (traveller).}}

