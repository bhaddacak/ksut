\section{Tatiyakaṇḍa}

\head{23}{23, 36. sarā pakati byañjane.}
\headtrans{Vowels [stay] unchanged because of [the other] consonant.}
\sutdef{sarā kho byañjane pare pakatirūpāni honti.}
\sutdeftrans{[The preceding] vowels retain the original form because of the other consonant.}
\example[0]{manopubbaṅgamā = mano + pubbaṅgamā}
\example[0]{pamādomaccuno = pamādo + maccuno}
\example{tiṇṇopāraṅgato = tiṇṇo + pāraṅgato}
\transnote{As shown in some corpus, the joining of terms in this case is unecessary.}

\head{24}{24, 35. sare kvaci.}
\headtrans{Because of [the other] vowel sometimes.}
\sutdef{sarā kho sare pare kvaci pakatirūpāni honti.}
\sutdeftrans{[The preceding] vowels retain the original form in some places because of the other vowel.}
\example{ko imaṃ = ko + imaṃ}
\transnote{Like the previous rule, the joining is unnecessary, even makes an ambiguous unit.}

\head{25}{25, 37. dīghaṃ.}
\headtrans{[Sometimes it becomes] elongated.}
\sutdef{sarā kho byañjane pare kvaci dīghaṃ papponti.}
\sutdeftrans{[The preceding] vowels become elongated in some places because of the other consonant.}
\example[0]{sammā dhammaṃ vipassato}
\example[0]{evaṃ gāme munī care}
\example{khantī paramaṃ tapo titikkhā}
\transnote{Examples in this sutta are unusual. It seems that their orginal form are \pali{sammādhammaṃ}, \pali{munīcare}, and \pali{khantīparamaṃ} respectively. When the text gets redacted the terms are separated for readability. The rule explains that the unexpected long vowels come from sandhi process.}

\head{26}{26, 38. rassaṃ.}
\headtrans{[Sometimes it becomes] shortened.}
\sutdef{sarā kho byañjane pare kvaci rassaṃ papponti.}
\sutdeftrans{[The preceding] vowels become shortened in some places because of the other consonant.}
\example[0]{bhovādi nāma so hoti}
\example{yathābhāvi guṇena so}
\transnote{The original form of the examples above could be \pali{bhovādīnāma} and \pali{yathābhāvīguṇena} respectively. See also the previous note.}

\head{27}{27, 39. lopañca tatrākāro.}
\headtrans{[Sometimes it gets] elided, and \pali{a} [is inserted] there.}
\sutdef{sarā kho byañjane pare kvaci lopaṃ papponti. tatra ca lope kate akārāgamo hoti.}
\sutdeftrans{[The preceding] vowels get elided in some places because of the other consonant. When made elided, also an insertion of \pali{a} is there.}
\example[0]{sa sīlavā = so sīlavā}
\example[0]{sa paññavā = so paññavā}
\example[0]{esa dhammo sanantano = eso dhammo sanantano}
\example[0]{sa ve kasāvamarahati = so ve kasāvamarahati}
\example[0]{sa mānakāmopi bhaveyya = so mānakāmopi bhaveyya}
\example{sa ve muni jātibhayaṃ adassi = so ve muni jātibhayaṃ adassi}

\head{28}{28, 40. para dvebhāvo ṭhāne.}
\headtrans{The other [consonant] becomes doubled, if suitable.}
\sutdef{saramhā parassa byañjanassa dvebhāvo hoti ṭhāne.}
\sutdeftrans{There is a reduplication of the other consonant from [the preceding] vowel, if suitable.}
\example[0]{idhappamādo = idha + pamādo}
\example[0]{pabbajjaṃ = pa + bajjaṃ}
\example[0]{cātuddasi = cātu + dasi}
\example[0]{pañcaddasi = pañca + dasi}
\example{abhikkantataro = abhi + kantataro}

\head{29}{29, 42. vagge ghosāghosānaṃ tatiyapaṭhamā.}
\headtrans{In the groups, [there is a reduplication of] the third and first [letter] for the voiced and voiceless.}
\sutdef{vagge kho pubbesaṃ byañjanānaṃ ghosāghosabhūtānaṃ sa\-ramhā yathāsaṅkhyaṃ tatiyapaṭhamakkharā dvebhāvaṃ gacchanti ṭhāne.}
\sutdeftrans{In the groups [of consonants], reduplication by the third and first letter go for the preceding voiced and voiceless consonants from vowel, if suitable.}
\example[0]{cajjhānapphalo = ca + jhānapphalo}
\example[0]{yatraṭṭhitaṃ = yatra + ṭhitaṃ}
\example{pabbatamuddhaniṭṭhito = pabbatamuddhani + ṭhito}
\transnote{This rule expects students to know some grammatical terms (see Table \ref{tab:letters} for better understanding), i.e., \pali{ghosa} (voiced, the 3rd and 4th) and \pali{aghosa} (voiceless, the 1st and 2nd), also \pali{dhanita} (aspirated, the 2nd and 4th) and \pali{sithila} (unaspirated, the 1st and 3rd). To put it simpler, we can say the \pali{dhanita} consonants are doubled by their \pali{sithila} counterpart in \pali{ghosa} and \pali{aghosa} group. That is to say, \pali{kha} gets doubled by \pali{ka} resulting in \pali{kkha}, and \pali{gha} gets doubled by \pali{ga} resulting in \pali{ggha}, for example (we will never see \pali{gkha} nor \pali{kgha}).}

