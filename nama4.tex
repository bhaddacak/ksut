\section{Catutthakaṇḍa}

\head{211}{211, 126. attanto hismimanattaṃ.}
\headtrans{The ending of \pali{atta}, because of \pali{hi}[-vibhatti], [becomes] \pali{ana}.}
\sutdef{tassa attano anto anattamāpajjate himhi vibhattimhi.}
\sutdeftrans{The ending of that \pali{atta} becomes \pali{ana} because of \pali{hi}-vibhatti.}
\example[0]{attanehi = atta + hi}
\example{attanebhi = atta + hi}

\head{212}{212, 129. tato smiṃ ni.}
\headtrans{After that [\pali{atta}], \pali{smiṃ}[-vibhatti] [becomes] \pali{ni}.}
\sutdef{tato attato smiṃvacanassa ni hoti.}
\sutdeftrans{There is [a substitution of] \pali{ni} for \pali{smiṃ}-vibhatti after that \pali{atta}.}
\example{attani = atta + smiṃ}

\head{213}{213, 127. sassa no.}
\headtrans{For \pali{sa}[-vibhatti], [there is] \pali{no}-substitution.}
\sutdef{tato attato sassa vibhattissa no hoti.}
\sutdeftrans{There is [a substitution of] \pali{no} for \pali{sa}-vibhatti after that \pali{atta}.}
\example{attano = atta + sa}

\head{214}{214, 128. smā nā.}
\headtrans{[For] \pali{smā}[-vibhatti], [there is] \pali{nā}-substitution.}
\sutdef{tato attato smā vacanassa nā hoti.}
\sutdeftrans{There is [a substitution of] \pali{nā} for \pali{smā}-vibhatti after that \pali{atta}.}
\example{attanā = atta + nā}

\head{215}{215, 141. jhalato ca.}
\headtrans{After \pali{jha}- and \pali{la}-alias, [\pali{smā} becomes \pali{nā}] also.}
\sutdef{jhalaiccetehi smāvacanassa nā hoti.}
\sutdeftrans{There is [a substitution of] \pali{nā} for \pali{smā}-vibhatti after \pali{jha}-alias (masculine \pali{i}-group) and \pali{la}-alias (masculine \pali{u}-group).}
\example[0]{agginā = aggi + smā}
\example[0]{daṇḍinā = daṇḍī + smā}
\example[0]{bhikkhunā = bhikkhu + smā}
\example{sayambhunā = sayambhū + smā}
\transnote{For the description of \pali{jha}- and \pali{la}-alias, see \hyperref[sut:58]{Kacc 58}.}

\head{216}{216, 180. ghapato smiṃ yaṃ vā.}
\headtrans{After \pali{gha}- and \pali{pa}-alias, \pali{smiṃ}[-vibhatti] [becomes] \pali{yaṃ} sometimes.}
\sutdef{tasmā ghapato smiṃvacanassa yaṃ hoti vā.}
\sutdeftrans{There is [a substitution of] \pali{yaṃ} for \pali{smiṃ}-vibhatti after \pali{gha}-alias (feminine \pali{ā}) and \pali{pa}-alias (feminine \pali{i}- and \pali{u}-group) sometimes.}
\example[0]{kaññāyaṃ = kaññā + smiṃ (= kaññāya)}
\example[0]{rattiyaṃ = ratti + smiṃ (= rattiyā)}
\example[0]{itthiyaṃ = itthī + smiṃ (= itthiyā)}
\example[0]{yāguyaṃ = yāgu + smiṃ (= yāguyā)}
\example{vadhuyaṃ = vadhū + smiṃ (= vadhuyā)}
\transnote{For the description of \pali{pa}- and \pali{gha}-alias, see \hyperref[sut:59]{Kacc 59} and \hyperref[sut:60]{60}.}

\head{217}{217, 199. yonaṃ ni napuṃsakehi.}
\headtrans{For \pali{yo}[-vibhatti], [there is] \pali{ni}-substitution after neuter nouns.}
\sutdef{sabbesaṃ yonaṃ ni hoti vā napuṃsakehi liṅgehi.}
\sutdeftrans{There is [a substitution of] \pali{ni} for all \pali{yo}-vibhattis sometimes after neuter nouns.}
\example[0]{aṭṭhīni = aṭṭhi + yo (= aṭṭhī)}
\example{āyūni = āyu + yo (= āyū)}

\head{218}{218, 196. ato niccaṃ.}
\headtrans{After \pali{a}-ending [of neuter terms], [there is] always \pali{ni}-substi\-tution [for \pali{yo}-vibhattis].}
\sutdef{akārantehi napuṃsakaliṅgehi yonaṃ ni hoti niccaṃ.}
\sutdeftrans{There is always [a substitution of] \pali{ni} for \pali{yo}-vibhattis after \pali{a}-ending neuter terms.}
\example[0]{yāni = ya + yo {\upshape (nom./acc.)}}
\example[0]{tāni = ta + yo {\upshape (nom./acc.)}}
\example[0]{kāni = ka + yo {\upshape (nom./acc.)}}
\example[0]{bhayāni = bhaya + yo {\upshape (nom./acc.)}}
\example{rūpāni = rūpa + yo {\upshape (nom./acc.)}}

\head{219}{219, 195. siṃ.}
\headtrans{[Always] \pali{si}[-vibhatti] [becomes] \pali{aṃ} [after \pali{a}-ending neuter terms].}
\sutdef{akārantehi napuṃsakaliṅgehi sivacanassa aṃ hoti niccaṃ.}
\sutdeftrans{There is always [a substitution of] \pali{aṃ} for \pali{si}-vibhatti after \pali{a}-ending neuter terms.}
\example[0]{sabbaṃ = sabba + si}
\example[0]{yaṃ = ya + si}
\example[0]{taṃ = ta + si}
\example[0]{kaṃ = ka + si}
\example{rūpaṃ = rūpa + si}

\head{220}{220, 74. sesato lopaṃ gasipi.}
\headtrans{After the rest, \pali{ga}-alias and \pali{si}[-vibhatti] [get] elided.}
\sutdef{tato niddiṭṭhehi liṅgehi sesato gasiiccete lopamāpajjante.}
\sutdeftrans{The \pali{ga}-alias (vocative \pali{si}) and \pali{si}[-vibhatti] after the remaining terms other than those explained become elided.}
\example[0]{bhoti itthi {\upshape (voc.)}}
\example[0]{sā itthī {\upshape (nom.)}}
\example[0]{bho daṇḍi {\upshape (voc.)}}
\example[0]{so daṇḍī {\upshape (nom.)}}
\example[0]{bho sattha {\upshape (voc.)}}
\example[0]{so satthā {\upshape (nom.)}}
\example[0]{bho rāja {\upshape (voc.)}}
\example{so rājā {\upshape (nom.)}}
\transnote{For the definition of \pali{ga}-alias, see \hyperref[sut:57]{Kacc 57}.}

\head{221}{221, 282. sabbāsamāvusopasagganipātādīhi ca.}
\headtrans{Also [there is elision of] all [vibhattis] after \pali{āvuso}, prefixes, particles, and so on.}
\sutdef{sabbāsaṃ vibhattīnaṃ ekavacanabahuvacanānaṃ paṭhamādutiyātatiyācatutthīpañcamīchaṭṭhīsattamīnaṃ lopo hoti, āvusoupasagganipātaiccevamādīhi ca.}
\sutdeftrans{Also there is elision of all singular and plural nominative, accusative, instrumental, dative, ablative, genitive, and locative vibhattis after \pali{āvuso}, prefixes, particles, and so on.}
\transnote{This rule potentially makes things confusing. In grammatical terms, we call all these \emph{indeclinables}. So, they are unaffected by vibhatti-operations. However, this sutta asserts that there are indeed such operations but they all get elided. Examples in this sutta are numerous. So, please read further in the text.}

\head{222}{222, 342. pumassa liṅgādīsu samāsesu.}
\headtrans{[There is elision of the ending] of \pali{puma} because of \pali{liṅga} and so on in compounds.}
\sutdef{pumaiccetassa anto lopamāpajjate liṅgādīsu parapadesu samā\-sesu.}
\sutdeftrans{The ending of \pali{puma} is elided because of succeeding \pali{liṅga} and so on in compounds.}
\example[0]{pulliṅgaṃ = puma + liṅgaṃ}
\example[0]{pumbhāvo = puma + bhāvo}
\example{puṅkokilo = puma + kokilo}
\transnote{For the examples, see also \hyperref[sut:82]{Kacc 82}, which has a different explanation.}

\head{223}{223, 188. aṃ yamīto pasaññato.}
\headtrans{[For] \pali{aṃ}[-vibhatti], [there is] \pali{yaṃ}-substitution after \pali{ī} of \pali{pa}-alias.}
\sutdef{aṃvacanassa yaṃ hoti vā īto pasaññato.}
\sutdeftrans{There is [a substitution of] \pali{yaṃ} for \pali{aṃ}-vibhatti sometimes after \pali{ī} of \pali{pa}-alias (feminine \pali{i}- and \pali{u}-group).}
\example{itthiyaṃ = itthī + aṃ (= itthiṃ)}
\transnote{For the definition of \pali{pa}-alias, see \hyperref[sut:59]{Kacc 59}.}

\head{224}{224, 153. naṃ jhato katarassā.}
\headtrans{[There is] \pali{naṃ}-substitution after shortened \pali{jha}-alias.}
\sutdef{tasmā jhato katarassā aṃvacanassa naṃ hoti.}
\sutdeftrans{There is [a substitution of] \pali{naṃ} for \pali{aṃ}-vibhatti after that shortened \pali{jha}-alias (masculine \pali{i}-group).}
\example[0]{daṇḍinaṃ = daṇḍī + aṃ}
\example{bhoginaṃ = bhogī + aṃ}
\transnote{For the definition of \pali{jha}-alias, see \hyperref[sut:58]{Kacc 58}.}

\head{225}{225, 151. yonaṃ no.}
\headtrans{For \pali{yo}[-vibhattis], [there is] \pali{no}-substitution [after shortened \pali{jha}-alias].}
\sutdef{sabbesaṃ yonaṃ jhato katarassā no hoti.}
\sutdeftrans{There is [a substitution of] \pali{no} for all \pali{yo}-vibhattis after that shortened \pali{jha}-alias.}
\example[0]{daṇḍino = daṇḍī + yo {\upshape (nom./acc.)}}
\example[0]{bhogino = bhogī + yo {\upshape (nom./acc.)}}
\example[0]{he daṇḍino {\upshape (voc.)}}
\example{he bhogino {\upshape (voc.)}}

\head{226}{226, 154. smiṃ ni.}
\headtrans{For \pali{smiṃ}[-vibhattis], [there is] \pali{ni}-substitution [after shortened \pali{jha}-alias].}
\sutdef{tasmā jhato katarassā smiṃvacanassa niādeso hoti.}
\sutdeftrans{There is a substitution of \pali{ni} for \pali{smiṃ}-vibhatti after that shortened \pali{jha}-alias.}
\example[0]{daṇḍini = daṇḍī + smiṃ}
\example{bhogini = bhogī + smiṃ}

\head{227}{227, 270. kissa ka ve ca.}
\headtrans{For \pali{kiṃ}, [there is] \pali{ka}-substitution because of \pali{va}[-paccaya] also.}
\sutdef{kimiccetassa ko ca hoti vapaccaye pare.}
\sutdeftrans{There is also [a substitution of] \pali{ka} for \pali{kiṃ} because of \pali{va}-paccaya behind.}
\example[0]{kva = kiṃ + va}
\example{kva gatosi tvaṃ devānaṃ piyatissa\\{\upshape= Where did you go, King Devānaṃ Piyatissa?}}

\head{228}{228, 272. ku hiṃhaṃsu ca.}
\headtrans{[There is] \pali{ku}-substitution [for \pali{kiṃ}] because of \pali{hiṃ} and \pali{haṃ}[-paccaya] also.}
\sutdef{kimiccetassa ku hoti hiṃhaṃiccetesu ca.}
\sutdeftrans{There is [a substitution of] \pali{ku} for \pali{kiṃ} because of \pali{hiṃ} and \pali{haṃ}[-paccaya] also.}
\example[0]{kuhiṃ = kiṃ + hiṃ {\upshape (where)}}
\example{kuhaṃ = kiṃ + haṃ}

\head{229}{229, 226. sesesu ca.}
\headtrans{Because of the remaining [vibhattis and paccayas] also.}
\sutdef{kimiccetassa ko hoti sesesu vibhattipaccayesu paresu.}
\sutdeftrans{There is [a substitution of] \pali{ka} for \pali{kiṃ} because of the remaining vibhattis and paccayas behind.}
\example[0]{ko pakāro = kathaṃ {\upshape (how)}}
\example{kaṃ pakāraṃ = kathaṃ}

\head{230}{230, 262. tratothesu ca.}
\headtrans{Because of \pali{tra}, \pali{to}, and \pali{tha}[-paccaya], [there is \pali{ku}-substitution] also}
\sutdef{kimiccetassa ku hoti tratothaiccetesu ca.}
\sutdeftrans{There is [a substitution of] \pali{ku} for \pali{kiṃ} because of \pali{tra}, \pali{to}, and \pali{tha}[-paccaya] also.}
\example[0]{kutra = kiṃ + tra}
\example[0]{kuto = kiṃ + to}
\example{kuttha = kiṃ + tha}

\head{231}{231, 263. sabbassetassakāro vā.}
\headtrans{For the whole \pali{eta}, [there is] \pali{a}-substitution sometimes [because of \pali{to} and \pali{tha}].}
\sutdef{sabbassa etasaddassa akāro hoti vā tothaiccetesu.}
\sutdeftrans{There is a substitution of \pali{a} for the whole \pali{eta} sometimes because of \pali{to} and \pali{tha}[-paccaya].}
\example[0]{ato = eta + to (= etto)}
\example{attha = eta + tha (= ettha)}

\head{232}{232, 267. tre niccaṃ.}
\headtrans{Because of \pali{tra}[-paccaya], [there is] always [\pali{a}-substitution for the whole \pali{eta}].}
\sutdef{sabbassa etasaddassa akāro hoti niccaṃ trapaccaye pare.}
\sutdeftrans{There is always a substitution of \pali{a} for the whole \pali{eta} because of \pali{tra}-paccaya behind.}
\example{atra = eta + tra.}

\head{233}{233, 264. e tothesu ca.}
\headtrans{[There is] \pali{e}-substitution [for \pali{eta}] because of \pali{to} and \pali{tha}[-paccaya] also.}
\sutdef{sabbassa etasaddassa ekāro hoti vā tothaiccetesu.}
\sutdeftrans{There is [a substitution of] \pali{e} for the whole \pali{eta} sometimes because of \pali{to} and \pali{tha}[-paccaya].}
\example[0]{etto = eta + to (= ato)}
\example{ettha = eta + tha (= attha)}

\head{234}{234, 265. imassi thaṃdānihatodhesu ca.}
\headtrans{For \pali{ima}, [there is] \pali{i}-substitution because of \pali{thaṃ}, \pali{dāni}, \pali{ha}, \pali{to}, and \pali{dha}[-paccaya] also.}
\sutdef{imasaddassa sabbasseva ikāro hoti thaṃdānihatodhaiccetesu.}
\sutdeftrans{There is [a substitution of] \pali{i} for the whole \pali{ima} because of \pali{thaṃ}, \pali{dāni}, \pali{ha}, \pali{to}, and \pali{dha}[-paccaya].}
\example[0]{itthaṃ = ima + thaṃ {\upshape (in this way, thus)}}
\example[0]{idāni = ima + dāni {\upshape (in this time, now)}}
\example[0]{iha = ima + ha {\upshape (in this place, here)}}
\example[0]{ito = ima + to {\upshape (from this, hence)}}
\example{idha = ima + dha {\upshape (in this place, here)}}

\head{235}{235, 281. a dhunāmhi ca.}
\headtrans{[There is] \pali{a}-substitution [for \pali{ima}] because of \pali{dhunā}[-paccaya] also.}
\sutdef{imasaddassa sabbasseva akāro hoti dhunāmhi paccaye pare.}
\sutdeftrans{There is [a substitution of] \pali{a} for the whole \pali{ima} because of \pali{dhunā}-paccaya behind.}
\example{adhunā = ima + dhunā {\upshape (now, recently)}}

\head{236}{236, 280. eta rahimhi.}
\headtrans{[There is] \pali{eta}-substitution [for \pali{ima}] because of \pali{rahi}[-paccaya].}
\sutdef{sabbasseva imasaddassa etādeso hoti rahimhi paccaye pare.}
\sutdeftrans{There is a substitution of \pali{eta} for the whole \pali{ima} because of \pali{rahi}-paccaya behind.}
\example{etarahi = ima + rahi {\upshape (now)}}

\head{237}{237, 176. itthiyamato āpaccayo.}
\headtrans{After the \pali{a}-letter in feminine [terms], [there is] \pali{ā}-paccaya.}
\sutdef{itthiyaṃ vattamānāya akārato āpaccayo hoti.}
\sutdeftrans{After the \pali{a}-letter occurring in feminine [terms], there is \pali{ā}-paccaya.}
\example[0]{sabbā = sabba + si}
\example[0]{yā = ya + si}
\example[0]{sā = sa + si}
\example[0]{kā = ka + si}
\example{katarā = katara + si}

\head{238}{238, 187. nadādito vā ī.}
\headtrans{After \pali{nada} or [non-\pali{nada}] and so on, [there is] \pali{ī}[-paccaya].}
\sutdef{nadādito vā anadādito vā itthiyaṃ vattamānāya īpaccayo hoti.}
\sutdeftrans{After \pali{nada} or non-\pali{nada} and so on occurring in feminine [nouns], there is \pali{ī}-paccaya.}
\example[0]{nadī = nada + ī}
\example[0]{mahī = maha + ī}
\example[0]{kumārī = kumāra + ī}
\example[0]{taruṇī = taruṇa + ī}
\example[0]{sakhī = sakha + ī}
\example{itthī = ittha + ī}

\head{239}{239, 190. ṇavaṇikaṇeyyaṇantuhi.}
\headtrans{After \pali{ṇava}, \pali{ṇika}, \pali{ṇeyya}, \pali{ṇa}, and \pali{ntu}[-paccaya], [there is \pali{ī}-paccaya].}
\sutdef{ṇavaṇikaṇeyyaṇantuiccetehi itthiyaṃ vattamānehi īpaccayo hoti.}
\sutdeftrans{After \pali{ṇava}, \pali{ṇika}, \pali{ṇeyya}, \pali{ṇa}, and \pali{ntu}[-paccaya] occurring in feminine [nouns], there is \pali{ī}-paccaya.}
\example[0]{māṇavī = manu + ṇava + ī}
\example[0]{paṇḍavī = paṇḍu + ṇava + ī}
\example[0]{nāvikī = nāvā + ṇika + ī}
\example[0]{venateyyī = vinatā + ṇeyya + ī}
\example[0]{kunteyyī = kuntā +ṇeyya + ī}
\example[0]{gotamī = gotama + ṇa + ī}
\example[0]{guṇavatī = guṇa + vantu + ī}
\example{sāmāvatī = sāmā + vantu + ī}
\transnote{These examples are female names produced by secondary derivation.}

\head{240}{240, 193. patibhikkhurājīkārantehi inī.}
\headtrans{After \pali{pati}, \pali{bhikkhu}, \pali{rāja}, and \pali{ī}-ending [nouns], [there is] \pali{inī}[-paccaya].}
\sutdef{patibhikkhurājīkārantehi itthiyaṃ vattamānehi inīpaccayo hoti.}
\sutdeftrans{After \pali{pati}, \pali{bhikkhu}, \pali{rāja}, and \pali{ī}-ending occurring in feminine [nouns], there is \pali{inī}-paccaya.}
\example[0]{gahapatānī = gahapati + inī}
\example[0]{bhikkhunī = bhikkhu + inī}
\example[0]{rājinī = rāja + inī}
\example[0]{hatthinī = hatthī + inī}
\example[0]{daṇḍinī = daṇḍī + inī}
\example[0]{medhāvinī = medhāvī + inī}
\example{tapassinī = tapassī + inī}

\head{241}{241, 191. ntussa tamīkāre.}
\headtrans{For \pali{ntu}[-paccaya], [there is] \pali{ta}-substitution because of \pali{ī}-paccaya.}
\sutdef{sabbasseva ntupaccayassa takāro hoti vā īkāre pare.}
\sutdeftrans{There is [a substitution of] \pali{ta} for the whole \pali{ntu}-paccaya sometimes because of \pali{ī}[-paccaya] behind.}
\example[0]{guṇavatī = guṇa + vantu + ī (= guṇavantī)}
\example[0]{kulavatī = kula + vantu + ī (= kulavantī)}
\example[0]{satimatī = sati + mantu + ī (= satimantī)}
\example[0]{mahatī = maha + ntu + ī (= mahantī)}
\example{gottamatī = gotta + mantu + ī (= gottamantī)}

\head{242}{242, 192. bhavato bhoto.}
\headtrans{For \pali{bhavanta}, [there is] \pali{bhota}-substitution.}
\sutdef{sabbasseva bhavantasaddassa bhotādeso hoti īkāre itthigate pare.}
\sutdeftrans{There is a substitution of \pali{bhota} for the whole \pali{bhavanta} because of the \pali{ī}[-paccaya] made feminine behind.}
\example[0]{bhoti = bhavanta + ī + si}
\example[0]{bhoti ayye}
\example[0]{bhoti kaññe}
\example{bhoti kharādiye}
\transnote{The examples show a vocative case of \pali{bhavatī}.}

\head{243}{243, 110. bho ge tu.}
\headtrans{[There is] \pali{bho}-substitution for \pali{ga}-alias.}
\sutdef{sabbasseva bhavantasaddassa bhoādeso hoti ge pare.}
\sutdeftrans{There is a substitution of \pali{bho} for the whole \pali{bhavanta} because of \pali{ga}-alias (vocative \pali{si}-vibhatti) behind.}
\example[0]{bho = bhavanta + si}
\example[0]{bho purisa}
\example[0]{bho aggi}
\example[0]{bho rāja}
\example[0]{bho sattha}
\example[0]{bho daṇḍi}
\example{bho sayambhu}
\transnote{For the definition of \pali{ga}-alias, see \hyperref[sut:57]{Kacc 57}. As marked by \pali{tu}, there are other forms of this as shown in the following examples. Moreover, \hyperref[sut:243a]{Kacc 243a} and \hyperref[sut:243b]{243b} below give us additional explanations.}
\example{bhonta, bhante, bhonto, bhadde, bhotā, bhoto}

\head{243a}{243a, 109. obhāvo kvaci yosu vakārassa.}
\headtrans{[There is] \pali{o}-substitution sometimes for \pali{va} [of \pali{bhavanta}] because of \pali{yo}[-vibhattis].}
\sutdef{bhavantaiccetassa vakārassa obhavo hoti kvaci yoiccetesu.}
\sutdeftrans{There is a substitution of \pali{o} for \pali{va} of \pali{bhavanta} in some places because of \pali{yo}[-vibhattis].}
\example[0]{bhonto = bhavanta + yo (= bhavanto)}
\example{imaṃ bhonto nisāmetha {\upshape (Sirs, [please] pay attention to this.)}}

\head{243b}{243b, 111. bhadantassa bhaddantabhante.}
\headtrans{For \pali{bhadanta}, [there are] \pali{bhaddanta}- and \pali{bhante}-substitution.}
\sutdef{sabbasseva bhadantasaddassa bhaddantabhanteiccete ādesā honti kvaci ge pare yosu ca.}
\sutdeftrans{There are substitutions of \pali{bhaddanta} and \pali{bhante} for the whole \pali{bhadanta} in some places because of \pali{ga}-alias and \pali{yo}-vibhattis.}
\example[0]{bhaddanta = bhadanta + si}
\example[0]{bhante = bhadanta + si}
\example{bhaddantā = bhadanta + yo}
\transnote{The products of this sutta are used for addressing.}

\head{244}{244, 72. akārapitādyantānamā.}
\headtrans{For the \pali{a}-letter and the ending of \pali{pitu} and so on, [there is] \pali{ā}-substitution.}
\sutdef{akāro ca pitādīnamanto ca āttamāpajjate ge pare.}
\sutdeftrans{The \pali{a}-letter and the ending of \pali{pitu} and so on become \pali{ā} because of \pali{ga}-alias behind.}
\example[0]{bho purisā (purisa + si)}
\example[0]{bho rājā (rāja + si)}
\example[0]{bho pitā (pitu + si)}
\example[0]{bho mātā (mātu + si)}
\example{bho satthā (satthu + si)}
\transnote{We can break this formula to \pali{akāra + pitu + ādi + anta + naṃ + ā}.}

\head{245}{245, 152. jhalapā rassaṃ.}
\headtrans{The \pali{jha}-, \pali{la}-, and \pali{pa}-alias [become] shortened.}
\sutdef{jhalapaiccete rassamāpajjante ge pare.}
\sutdeftrans{The \pali{jha}-alias (masculine \pali{i}-group), \pali{la}-alias (masculine \pali{u}-group), and \pali{pa}-alias (feminine \pali{i}- and \pali{u}-group) become shortened because of \pali{ga}-alias behind.}
\example[0]{bho daṇḍi (daṇḍī + si)}
\example[0]{bho sayambhu (sayambhū + si)}
\example[0]{bhoti itthi (itthī + si)}
\example{bhoti vadhu (vadhū + si)}
\transnote{For the definition of \pali{jha}-, \pali{la}-, and \pali{pa}-alias, see \hyperref[sut:58]{Kacc 58} and \hyperref[sut:59]{59}.}

\head{246}{246, 73. ākāro vā.}
\headtrans{The \pali{ā}-letter [becomes shortened] sometimes.}
\sutdef{ākāro rassamāpajjate vā ge pare.}
\sutdeftrans{The \pali{ā}-letter becomes shortened sometimes because of \pali{ga}-alias behind.}
\example[0]{bho rāja = bho rājā (rāja + si)}
\example[0]{bho atta = bho attā (atta + si)}
\example[0]{bho sakha = bho sakhā (sakha + si)}
\example{bho sattha = bho satthā (satthu + si)}

