\markboth{}{Abbreviations}
\clearpage
\phantomsection
\addcontentsline{toc}{chapter}{Abbreviations}
\setcounter{footnote}{0}
\chapter*{Abbreviations}

The first table contains abbreviations used for referencing to literary works.

\bigskip
\begin{longtable}[c]{@{}>{\raggedright\arraybackslash}p{0.17\linewidth}>{\raggedright\arraybackslash}p{0.78\linewidth}@{}}
\toprule
\bfseries\upshape \mbox{Abbrev.} & \bfseries\upshape Description \\ \midrule
\endfirsthead
\toprule
\bfseries\upshape \mbox{Abbrev.} & \bfseries\upshape Description \\ \midrule
\endhead
\bottomrule
\ltblcontinuedbreak{2}
\endfoot
\bottomrule
\endlastfoot
%%
CST & Chaṭṭha Saṅgāyana Tipiṭaka \\
DPD & Digital Pāḷi Dictionary\footnote{\url{dpdict.net}} \\
Kacc & Kaccāyanabyākaraṇaṃ \\
PNL & Pāli for New Learners\footnote{\url{bhaddacak.github.io/pnl}} \\
\end{longtable}

The second table is all about grammatical terms used in the book.

\bigskip
\begin{longtable}[c]{@{}>{\raggedright\arraybackslash}p{0.17\linewidth}>{\raggedright\arraybackslash}p{0.78\linewidth}@{}}
\toprule
\bfseries\upshape \mbox{Abbrev.} & \bfseries\upshape Description \\ \midrule
\endfirsthead
\toprule
\bfseries\upshape \mbox{Abbrev.} & \bfseries\upshape Description \\ \midrule
\endhead
\bottomrule
\ltblcontinuedbreak{2}
\endfoot
\bottomrule
\endlastfoot
%%
acc. & accusative case \\
dat. & dative case \\
f. & feminine \\
gen. & genitive case \\
ins. & instrumental case \\
loc. & locative case \\
m. & masculine \\
nom. & nominative case \\
nt. & neuter \\
voc. & vocative case \\
\end{longtable}

