\markboth{}{Abbreviations}
\clearpage
\phantomsection
\addcontentsline{toc}{chapter}{Abbreviations}
\setcounter{footnote}{0}
\chapter*{Abbreviations}\label{chap:abbrev}

The first table contains abbreviations used for referencing to literary works. Pāli texts related to the canon can be found in CSTR corpus, available in \textsc{Pāli\,Platform}.\footnote{\url{bhaddacak.github.io/paliplatform}} All essential Pāli dictionaries and grammatical textbooks can also be found in the program. To navigate with the numbering scheme of Saddanīti Dhātumālā, you only have to use the program.\footnote{Or use this web version of the book, \url{bhaddacak.github.io/sadddha}.}

In most explanations, references to Pāli dictionaries are not given, except in particular treatments. Sanskrit dictionaries are also used occasionally in case Pāli dictionaries give us little help. Three main Sanskrit lexicons are Macdonell's, Apte's, and Monier-Williams. The first two have a combined search at Digital Dictionaries of South Asia.\footnote{\url{dsal.uchicago.edu/dictionaries/sanskrit}} And all these can be found at Cologne Digital Sanskrit Dictionaries.\footnote{\url{www.sanskrit-lexicon.uni-koeln.de}}

\bigskip
\begin{longtable}[c]{@{}>{\raggedright\arraybackslash}p{0.17\linewidth}>{\raggedright\arraybackslash}p{0.78\linewidth}@{}}
\toprule
\bfseries\upshape \mbox{Abbrev.} & \bfseries\upshape Description \\ \midrule
\endfirsthead
\toprule
\bfseries\upshape \mbox{Abbrev.} & \bfseries\upshape Description \\ \midrule
\endhead
\bottomrule
\ltblcontinuedbreak{2}
\endfoot
\bottomrule
\endlastfoot
%%
Apte & Vaman Shivaram Apte's Practical Sanskrit-English Dictionary\footnote{\url{dsal.uchicago.edu/dictionaries/apte} (1957--59)} \\
CST & Chaṭṭha Saṅgāyana Tipiṭaka \\
CSTR & Chaṭṭha Saṅgāyana Tipiṭaka Restructured \\
DPD & Digital Pāḷi Dictionary\footnote{\url{dpdict.net}} \\
Kacc & Kaccāyanabyākaraṇaṃ \\
Kāt & Śarvavarman's Kātantra\footnote{The referencing scheme of this book corresponds to R.\,S.\,Saini's edition (\citealp{saini:katantra}).} \\
Khds & Khuddasikkhā \\
MD & Arthur Anthony Macdonell's Practical Sanskrit Dictionary \footnote{\url{dsal.uchicago.edu/dictionaries/macdonell} (1929)} \\
MWD & Monier-Williams Sanskrit-English Dictionary\footnote{\url{www.sanskrit-lexicon.uni-koeln.de/scans/MWScan/2020/web/index.php} (1899)} \\
Pāṇ & Pāṇini's Aṣṭādhyāyī\footnote{I use a digital version of Aṣṭādhyāyī from \url{sanskritdocuments.org/doc\_z\_misc\_major\_works/aShTAdhyAyI.html}.} \\
PNL & Pāli for New Learners\footnote{\url{bhaddacak.github.io/pnl}} \\
PTSD & The Pali Text Society's Pali-English Dictionary\footnote{\citealp{rhys:ptsd}} \\
Rūpa & Padarūpasiddhi \\
Sadd-Dhā & Saddanīti Dhātumālā \\
Snp-a & Suttanipāta-aṭṭhakathā (Buddhaghosa's Paramatthajotikā) \\
Sp & Buddhaghosa's Samantapāsādikā (the commentary to the Vinaya) \\
Vnl & Vinayālaṅkāla \\
Vnv & Vinayavichaya \\
Whit & \mbox{William Dwight Whitney's Sanskrit Grammar}\footnote{References to this book contain chapter and paragraph number, for example, Whit 1\S16. For an online source, see \url{en.wikisource.org/wiki/Sanskrit\_Grammar\_(Whitney)}.} \\
\end{longtable}

\bigskip
The second table is all about grammatical terms used in the book.

\bigskip
\begin{longtable}[c]{@{}>{\raggedright\arraybackslash}p{0.17\linewidth}>{\raggedright\arraybackslash}p{0.78\linewidth}@{}}
\toprule
\bfseries\upshape \mbox{Abbrev.} & \bfseries\upshape Description \\ \midrule
\endfirsthead
\toprule
\bfseries\upshape \mbox{Abbrev.} & \bfseries\upshape Description \\ \midrule
\endhead
\bottomrule
\ltblcontinuedbreak{2}
\endfoot
\bottomrule
\endlastfoot
%%
abl. & ablative case (pañcamī) \\
acc. & accusative case (dutiyā) \\
dat. & dative case (catutthī) \\
f. & feminine \\
gen. & genitive case (chaṭṭhī) \\
inf. & infinitive \\
ins. & instrumental case (tatiyā) \\
lit. & literally \\
loc. & locative case (sattamī) \\
m. & masculine \\
nom. & nominative case (paṭhamā) \\
nt. & neuter \\
Skt. & Sanskrit \\
voc. & vocative case (ālapana) \\
\end{longtable}

