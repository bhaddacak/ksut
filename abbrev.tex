\markboth{}{Abbreviations}
\clearpage
\phantomsection
\addcontentsline{toc}{chapter}{Abbreviations}
\setcounter{footnote}{0}
\chapter*{Abbreviations}

The first table contains abbreviations used for referencing to literary works. Pāli texts related to the canon can be found in CSTR corpus, available in \textsc{Pāli\,Platform}.\footnote{\url{bhaddacak.github.io/paliplatform}} Grammatical textbooks can also be found in the program. To navigate with the numbering scheme of Saddanīti Dhātumālā, you only have to use the program.\footnote{Or use this web version of the book, \url{bhaddacak.github.io/sadddha}.}

\bigskip
\begin{longtable}[c]{@{}>{\raggedright\arraybackslash}p{0.17\linewidth}>{\raggedright\arraybackslash}p{0.78\linewidth}@{}}
\toprule
\bfseries\upshape \mbox{Abbrev.} & \bfseries\upshape Description \\ \midrule
\endfirsthead
\toprule
\bfseries\upshape \mbox{Abbrev.} & \bfseries\upshape Description \\ \midrule
\endhead
\bottomrule
\ltblcontinuedbreak{2}
\endfoot
\bottomrule
\endlastfoot
%%
CST & Chaṭṭha Saṅgāyana Tipiṭaka \\
CSTR & Chaṭṭha Saṅgāyana Tipiṭaka Restructured \\
DPD & Digital Pāḷi Dictionary\footnote{\url{dpdict.net}} \\
Kacc & Kaccāyanabyākaraṇaṃ \\
Kāt & Śharvavarman's Kātantra\footnote{The referencing scheme of this book corresponds to the Julius Eggeling's edition (1874).} \\
Khds & Khuddasikkhā \\
Pāṇ & Pāṇini's Aṣṭādhyāyī\footnote{I use a digital version of Aṣṭādhyāyī at \url{sanskritdocuments.org/doc\_z\_misc\_major\_works/aShTAdhyAyI.html}.} \\
PNL & Pāli for New Learners\footnote{\url{bhaddacak.github.io/pnl}} \\
Rūpa & Padarūpasiddhi \\
Sadd-Dhā & Saddanīti Dhātumālā \\
Vnl & Vinayālaṅkāla \\
Vnv & Vinayavichaya \\
\end{longtable}

\newpage
The second table is all about grammatical terms used in the book.

\bigskip
\begin{longtable}[c]{@{}>{\raggedright\arraybackslash}p{0.17\linewidth}>{\raggedright\arraybackslash}p{0.78\linewidth}@{}}
\toprule
\bfseries\upshape \mbox{Abbrev.} & \bfseries\upshape Description \\ \midrule
\endfirsthead
\toprule
\bfseries\upshape \mbox{Abbrev.} & \bfseries\upshape Description \\ \midrule
\endhead
\bottomrule
\ltblcontinuedbreak{2}
\endfoot
\bottomrule
\endlastfoot
%%
abl. & ablative case (pañcamī) \\
acc. & accusative case (dutiyā) \\
dat. & dative case (catutthī) \\
f. & feminine \\
gen. & genitive case (chaṭṭhī) \\
ins. & instrumental case (tatiyā) \\
loc. & locative case (sattamī) \\
m. & masculine \\
nom. & nominative case (paṭhamā) \\
nt. & neuter \\
voc. & vocative case (ālapana) \\
\end{longtable}

