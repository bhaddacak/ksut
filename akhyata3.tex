\section{Tatiyakaṇḍa}

\head{458}{458, 461. kvacādivaṇṇānamekassarānaṃ dvebhāvo.}
\headtrans{In some places, the initial consonants [become] reduplicated with the same vowel.}
\sutdef{ādibhūtānaṃ vaṇṇānaṃ ekassarānaṃ kvaci dvebhāvo hoti.}
\sutdeftrans{There is a reduplication of the initial consonants with the same vowel in some places.}
\example[0]{titikkhati (\paliroot{tija} + kha + ti)}
\example[0]{jigucchati (\paliroot{gupa} + cha + ti)}
\example[0]{tikicchati (\paliroot{kita} + cha + ti)}
\example[0]{vīmaṃsati (\paliroot{māna} + sa + ti)}
\example[0]{bubhukkhati (\paliroot{bhuja} + kha + ti)}
\example[0]{pivāsati (\paliroot{pā} + sa + ti)}
\example[0]{daddallati (\paliroot{dala} + a + ti)}
\example[0]{dadāti (\paliroot{dā} + a + ti)}
\example[0]{jahāti (\paliroot{hā} + a + ti)}
\example{caṅkamati (\paliroot{kamu} + a + ti)}
\transnote{The reduplication can be caused by certain paccayas, such as \pali{kha}, \pali{cha}, and \pali{sa}, or it can be the nature of some roots as shown by the last four instances. And as you may see, the first consonant may undergo some change and the vowel may not be exactly duplicated (these will be explained subsequently). The root of \pali{daddallati} can be dubious. See further \pali{jājvalati} (from \paliroot{jval}) in MWD.}

\head{459}{459, 462. pubbo’bbhāso.}
\headtrans{The first [syllable of the reduplication is called] \pali{abbhāsa}.}
\sutdef{dvebhūtassa dhātussa yo pubbo, so abbhāsasañño hoti.}
\sutdeftrans{Which the first syllable of the reduplicated root [exists], it is called \pali{abbhāsa}.}
\example[0]{dadhāti (\paliroot{dhā} + a + ti)}
\example[0]{dadāti (\paliroot{dā} + a + ti)}
\example{babhūva (\paliroot{bhū} + a + a)}
\transnote{As stated by this sutta, the first syllables, i.e., \pali{da} and \pali{ba}, are called \pali{abbhāsa}. Normally in a perfect verb formation, like \pali{babhūva}, a reduplication occurs.}

\head{460}{460, 506. rasso.}
\headtrans{[The vowel in a reduplication becomes] shortened.}
\sutdef{abbhāse vattamānassa sarassa rasso hoti.}
\sutdeftrans{When a vowel occurring in a reduplication [it] becomes shortened.}
\example{dadhāti, jahāti}

\head{461}{461, 464. dutiyacatutthānaṃ paṭhamatatiyā.}
\headtrans{[There are insertions of] the first and third letter [of the character group] for the second and fourth.}
\sutdef{abbhāsagatānaṃ dutiyacatutthānaṃ paṭhamatatiyā honti.}
\sutdeftrans{There are insertions of the first and third letter [of the character group] for the second and fourth [respectively] of the reduplicated part.}
\example[0]{ciccheda (\paliroot{chida} + a + a)}
\example[0]{bubhukkhati (\paliroot{bhuja} + kha + ti)}
\example[0]{babhūva (\paliroot{bhū} + a + a)}
\example{dadhāti (\paliroot{dhā} + a + ti)}
\transnote{To put in another way, when a reduplication occurs, the \pali{sithila} (unaspirated) counterpart of the first syllable is duplicated, hence from the examples, \pali{ca} for \pali{cha}, \pali{ba} for \pali{bha}, and \pali{da} for \pali{dha}.}

\head{462}{462, 476. kavaggassa cavaggo.}
\headtrans{[When] \pali{ka}-group [occurring in a reduplication it becomes] \pali{ca}-group.}
\sutdef{abbhāse vattamānassa kavaggassa cavaggo hoti.}
\sutdeftrans{When \pali{ka}-group occurring in a reduplication [it] becomes \pali{ca}-group.}
\example[0]{cikicchati (\paliroot{kita} + cha + ti)}
\example[0]{jigucchati (\paliroot{gupa} + cha + ti)}
\example[0]{jighacchati (\paliroot{ghasa} + cha + ti)}
\example[0]{jigīsati (\paliroot{hara} + sa + ti)}
\example[0]{jaṅgamati (\paliroot{gamu} + a + ti)}
\example{caṅkamati (\paliroot{kamu} + a + ti)}
\transnote{The form of \pali{cikicchati} is dubious. We mostly see \pali{tikicchati} in this sense (to cure). Perhaps, it came from Skt.\ \pali{cikitsati} (See Apte).}

\head{463}{463, 532. mānakitānaṃ vatattaṃ vā.}
\headtrans{For \pali{māna} and \pali{kita}, [there are] \pali{va}- and \pali{ta}-substitution sometimes.}
\sutdef{mānakitaiccetesaṃ dhātūnaṃ abbhāsagatānaṃ vakāratakā\-rattaṃ hoti vā yathāsaṅkhyaṃ.}
\sutdeftrans{There are sometimes [substitutions of] \pali{va} and \pali{ta} for the reduplicated part of the roots \pali{māna} and \pali{kita} respectively.}
\example[0]{vīmaṃsati (\paliroot{māna} + sa + ti)}
\example{tikicchati (\paliroot{kita} + cha + ti)}

\head{464}{464, 504. hassa jo.}
\headtrans{[When] \pali{ha} [occurring in a reduplication, it becomes] \pali{ja}.}
\sutdef{abbhāse vattamānassa hakārassa jo hoti.}
\sutdeftrans{When \pali{ha} occurring in a reduplication, [it] becomes \pali{ja}.}
\example[0]{jahāti (\paliroot{hā} + a + ti)}
\example[0]{juhvati, juhoti (\paliroot{hu} + a + ti)}
\example{jahāra (\paliroot{hara} + a + )}

\head{465}{465, 463. antassivaṇṇākāro vā.}
\headtrans{The ending of the reduplicated part [becomes] \pali{i}-group, [also \pali{a}-group], sometimes.}
\sutdef{abbhāsassa antassa ivaṇṇo hoti, akāro vā.}
\sutdeftrans{The ending of the reduplicated part becomes \pali{i}-group, [also] \pali{a}-group, sometimes.}
\example[0]{jigucchati (\paliroot{gupa} + cha + ti)}
\example[0]{pivāsati (\paliroot{pā} + sa + ti)}
\example[0]{vīmaṃsati (\paliroot{māna} + sa + ti)}
\example[0]{jighacchati (\paliroot{ghasa} + cha + ti)}
\example[0]{babhūva (\paliroot{bhū} + a + a)}
\example{dadhāti (\paliroot{dhā} + a + ti)}

\head{466}{466, 489. niggahitañca.}
\headtrans{There in an insertion of niggahita also.}
\sutdef{abbhāsassa ante niggahitāgamo hoti vā.}
\sutdeftrans{There in an insertion of niggahita at the end of the reduplicated part sometimes.}
\example[0]{caṅkamati (\paliroot{kamu} + a + ti)}
\example[0]{cañcalati (\paliroot{cala} + a + ti)}
\example{jaṅgamati (\paliroot{gamu} + a + ti)}

\head{467}{467, 533. tato pāmānānaṃ vāmaṃ sesu.}
\headtrans{After that [reduplicated part, there are] \pali{vā}- and \pali{maṃ}-substi\-tution for \pali{pā} and \pali{māna} because of \pali{sa}-paccaya.}
\sutdef{tato abbhāsato pāmānaiccetesaṃ dhātūnaṃ vāmaṃiccete ād\-esā honti yathāsaṅkhyaṃ sapaccaye pare.}
\sutdeftrans{There are substitutions of \pali{vā} and \pali{maṃ} for the roots \pali{pā} and \pali{māna} respectively after that reduplicated part because of \pali{sa}-paccaya behind.}
\example[0]{pivāsati (\paliroot{pā} + sa + ti)}
\example{vīmaṃsati (\paliroot{māna} + sa + ti)}
\transnote{Here \pali{abbhāsato} makes a little confusion because it means the whole bunch of the reduplication before adding any paccaya, not exactly as described in \hyperref[sut:459]{Kacc 459}, in which only the front part is called \pali{abbhāsa}.}

\head{468}{468, 492. ṭhā tiṭṭho.}
\headtrans{[For] \pali{ṭhā}, [there is] \pali{tiṭṭha}-substitution.}
\sutdef{ṭhāiccetassa dhātussa tiṭṭhādeso hoti vā.}
\sutdeftrans{There is an substitution of \pali{tiṭṭha} for the root \pali{ṭhā} sometimes.}
\example[0]{tiṭṭhati (\paliroot{ṭhā} + a + ti)}
\example[0]{tiṭṭhatu (\paliroot{ṭhā} + a + tu)}
\example[0]{tiṭṭheyya (\paliroot{ṭhā} + a + eyya)}
\example{tiṭṭheyyuṃ (\paliroot{ṭhā} + a + eyyuṃ)}
\transnote{Also \pali{ṭhāti} can be seen but far less common.}

\head{469}{469, 494. pā pivo.}
\headtrans{[For] \pali{pā}, [there is] \pali{piva}-substitution.}
\sutdef{pāiccetassa dhātussa pivādeso hoti vā.}
\sutdeftrans{There is an substitution of \pali{piva} for the root \pali{pā} sometimes.}
\example[0]{pivati (\paliroot{pā} + a + ti)}
\example[0]{pivatu (\paliroot{pā} + a + tu)}
\example[0]{piveyya (\paliroot{pā} + a + eyya)}
\example{piveyyuṃ (\paliroot{pā} + a + eyyuṃ)}

\head{470}{470, 514. ñāssa jājaṃnā.}
\headtrans{[For] \pali{ñā}, [there are] \pali{jā}-, \pali{jaṃ}-, and \pali{nā}-substitution.}
\sutdef{ñāiccetassa dhātussa jājaṃnāādesā honti vā.}
\sutdeftrans{There are substitutions of \pali{jā}, \pali{jaṃ}, and \pali{nā} for the root \pali{ñā} sometimes.}
\example[0]{jānāti (\paliroot{ñā} + nā + ti)}
\example[0]{jāneyya, jāniyā, jaññā (\paliroot{ñā} + nā + eyya)}
\example{nāyati (\paliroot{ñā} + ya + ti)}
\transnote{As a passive form, \pali{nāyati} means ``Knowing is done'' or ``[Something] is known.'' The three of \pali{jāneyya, jāniyā,} and \pali{jaññā} are variations of its optative form, also \pali{jāne} can be used.}

\head{471}{471, 483. disassa passadissadakkhā vā.}
\headtrans{[For] \pali{disa}, [there are] \pali{passa}-, \pali{dissa}-, and \pali{dakkha}-substitution sometimes.}
\sutdef{disaiccetassa dhātussa passadissadakkhaiccete ādesā honti vā.}
\sutdeftrans{There are substitutions of \pali{passa}, \pali{dissa}, and \pali{dakkha} for the root \pali{disa} sometimes.}
\example[0]{passati, dissati, dakkhati (\paliroot{disa} + a + ti)}
\example{adakkha (\paliroot{disa} + a + a)}

\head{472}{472, 531. byañjanantassa co chapaccayesu ca.}
\headtrans{For the ending consonant [of a root, there is] \pali{ca}-substitution because of \pali{cha}-paccayas also.}
\sutdef{byañjanantassa dhātussa co hoti chapaccayesu paresu.}
\sutdeftrans{There is [a substitution of] \pali{ca} for the ending consonant of a root because of \pali{cha}-paccayas behind.}
\example[0]{jigucchati (\paliroot{gupa} + cha + ti)}
\example[0]{tikicchati (\paliroot{kita} + cha + ti)}
\example{jighacchati (\paliroot{ghasa} + cha + ti)}
\transnote{In the formula, \pali{chapaccayesu} (plural) implies that other paccayas in this group can also be the case.}

\head{473}{473, 529. ko khe ca.}
\headtrans{[There is] \pali{ka}-substitution bacause of \pali{kha}-paccaya also.}
\sutdef{byañjanantassa dhātussa ko hoti khapaccaye pare.}
\sutdeftrans{There is [a substitution of] \pali{ka} for the ending consonant of a root because of \pali{kha}-paccaya behind.}
\example[0]{titikkhati (\paliroot{tija} + kha + ti)}
\example{bubhukkhati (\paliroot{bhuja} + kha + ti)}

\head{474}{474, 535. harassa gī se.}
\headtrans{For \pali{hara}, [there is] \pali{gī}-substitution because of \pali{sa}-paccaya.}
\sutdef{haraiccetassa dhātussa sabbasseva gīādeso hoti sapaccaye pare.}
\sutdeftrans{There is a substitution of \pali{gī} for the whole root of \pali{hara} because of \pali{sa}-paccaya behind.}
\example{jigīsati (\paliroot{hara} + sa + ti)}

\head{475}{475, 465. brūbhūnamāhabhūvā parokkhāyaṃ.}
\headtrans{For \pali{brū} and \pali{bhū}, [there are] \pali{āha}- and \pali{bhūva}-substitution because of \pali{parokkhā}-vibhatti.}
\sutdef{brūbhūiccetesaṃ dhātūnaṃ āhabhūvaiccete ādesā honti yathā\-saṅkhyaṃ parokkhāyaṃ vibhattiyaṃ.}
\sutdeftrans{There are substitutions of \pali{āha} and \pali{bhūva} for the roots \pali{brū} and \pali{bhū} respectively because of \pali{parokkhā}-vibhatti (perfect).}
\example[0]{āha (\paliroot{brū} + a + a)}
\example[0]{āhu (\paliroot{brū} + a + u)}
\example[0]{babhūva (\paliroot{bhū} + a + a)}
\example{babhūvu (\paliroot{bhū} + a + u)}
\transnote{Normally, verbs in perfect tense get reduplicated, but \pali{āha} and \pali{āhu} are exceptional cases. However, other perfect verbs appear rarely in the Pāli scriptures.}

\head{476}{476, 442. gamissanto ccho vā sabbāsu.}
\headtrans{The ending of \pali{gamu} [becomes] \pali{ccha} sometimes because of all [paccayas and vibhattis].}
\sutdef{gamuiccetassa dhātussa anto makāro ccho hoti vā sabbāsu paccayavibhattīsu.}
\sutdeftrans{The ending \pali{ma} of the root \pali{gamu} becomes \pali{ccha} sometimes because of all paccayas and vibhattis.}
\example[0]{gacchamāno (\paliroot{gamu} + a + māna + si)}
\example[0]{gacchanto (\paliroot{gamu} + a + nta + si)}
\example[0]{gacchati (\paliroot{gamu} + a + ti)}
\example[0]{gameti (\paliroot{gamu} + a + ti)}
\example[0]{gacchatu (\paliroot{gamu} + a + tu)}
\example[0]{gametu (\paliroot{gamu} + a + ti)}
\example[0]{gaccheyya (\paliroot{gamu} + a + eyya)}
\example[0]{gameyya (\paliroot{gamu} + a + eyya)}
\example[0]{agacchā (\paliroot{gamu} + a + ā)}
\example[0]{agamā (\paliroot{gamu} + a + ā)}
\example[0]{agacchī (\paliroot{gamu} + a + ī)}
\example[0]{agamī (\paliroot{gamu} + a + ī)}
\example[0]{gacchissati (\paliroot{gamu} + a + ssati)}
\example[0]{gamissati (\paliroot{gamu} + a + ssati)}
\example[0]{agacchissā (\paliroot{gamu} + a + ssā)}
\example{agamissā (\paliroot{gamu} + a + ssā)}
\transnote{The first two examples (\pali{gacchamāno} and \pali{gacchanto}) are pre\-sent participle, formed by verbal \pali{kita} operation. These two forms, as well as other kita verbs, work with nominal vibhattis, not verbal. We will learn about this topic in the next chapter. Note that even though \pali{gameti} is valid, it far less common than \pali{gacchati}. For why it is \pali{gameti} not \pali{gamati}, see \hyperref[sut:510]{Kacc 510}.}

\head{477}{477, 479. vacassajjatanimhimakāro o.}
\headtrans{For \pali{a} of \pali{vaca}, because of \pali{ajjatanī}-vibhatti, [there is] \pali{o}-substi\-tution.}
\transnote{The bunch can be broken to \pali{vacassa + ajjatanimhi + akāro}. An extra \pali{ma} is inserted in the sandhi process.}
\sutdef{vacaiccetassa dhātussa akāro ottamāpajjate ajjatanimhi vi\-bh\-attimhi.}
\sutdeftrans{The letter \pali{a} of the root \pali{vaca} becomes \pali{o} because of \pali{ajjatanī}-vibhatti (aorist).}
\example[0]{avoca (\paliroot{vaca} + a + ā)}
\example{avocuṃ (\paliroot{vaca} + a + uṃ)}
\transnote{The first example uses an attanopada-vibhatti. In a Thai source, it is \pali{avoci (vaca + a + ī)} instead, by using parassapada. The long endings are often shortened. The preceding \pali{a} is called \emph{augment} (see \hyperref[sut:519]{Kacc 519}). Adding nothing to the meaning, but it is often seen.}

\head{478}{478, 438. akāro dīghaṃ himimesu.}
\headtrans{The letter \pali{a} [becomes] elongated because of \pali{hi}-, \pali{mi}-, and \pali{ma}-vibhatti.}
\sutdef{akāro dīghamāpajjate himimaiccetesu vibhattīsu.}
\sutdeftrans{The letter \pali{a} becomes elongated because of \pali{hi}-, \pali{mi}-, and \pali{ma}-vibhatti.}
\example[0]{gacchāhi (\paliroot{gamu} + a + hi)}
\example[0]{gacchāmi (\paliroot{gamu} + a + mi)}
\example[0]{gacchāma (\paliroot{gamu} + a + ma)}
\example{gacchāmhe (\paliroot{gamu} + a + mhe)}

\head{479}{479, 452. hi lopaṃ vā.}
\headtrans{The \pali{hi}-vibhatti [becomes] elided sometimes.}
\sutdef{hivibhatti lopamāpajjate vā.}
\sutdeftrans{The \pali{hi}-vibhatti becomes elided sometimes.}
\example[0]{gaccha, gacchāhi (\paliroot{gamu} + a + hi)}
\example[0]{gama, gamāhi (\paliroot{gamu} + a + hi)}
\example{gamaya, gamayāhi (\paliroot{gamu} + ya + hi)}
\transnote{In passive imperative form, \pali{gamaya} and \pali{ganayāhi} possibly means ``Allowing to go is done.'' Another equivalent form of these is \pali{gacchīyāhi}.}

\head{480}{480, 490. hotissarehohe bhavissantimhi ssassa ca.}
\headtrans{The vowel of \pali{hū} [becomes] \pali{eha}, \pali{oha}, and \pali{e} because of \pali{bhavissanti}-vibhatti. [There is an elision of] \pali{ssa} also.}
\sutdef{hūiccetassa dhātussa saro ehaohaettamāpajjate bhavissantimhi, ssassa ca lopo hoti vā.}
\sutdeftrans{The vowel of the root \pali{hū} becomes \pali{eha}, \pali{oha}, and \pali{e} because of \pali{bhavissanti}-vibhatti (future). There is also an elision of \pali{ssa} sometimes.}
\example[0]{hehiti, hohiti, heti, hehissati, hohissati, hessati \\(\paliroot{hū} + a + ssati)}
\example{hehinti, hohinti, henti, hehissanti, hohissanti, hessanti \\(\paliroot{hū} + a + ssanti)}

\head{481}{481, 524. karassa sapaccayassa kāho.}
\headtrans{For \pali{kara} together with paccaya, [there is] \pali{kāha}-substitution.}
\sutdef{karaiccetassa dhātussa sapaccayassa kāhādeso hoti vā bhavissantimhi vibhattimhi, sassa ca niccaṃ lopo hoti.}
\sutdeftrans{There is a substitution of \pali{kāha} for the root \pali{kara} together with paccaya sometimes because of \pali{bhavissanti}-vibhatti. There is always an elision of \pali{sa} as well.}
\example[0]{kāhati, kāhiti (\paliroot{kara} + o + ssati)}
\example[0]{kāhasi, kāhisi (\paliroot{kara} + o + ssasi)}
\example[0]{kāhāmi (\paliroot{kara} + o + ssāmi)}
\example{kāhāma (\paliroot{kara} + o + ssāma)}
\transnote{As shown by the examples, ``\pali{sassa ca niccaṃ lopo hoti}'' should be seen as ``\pali{ssassa} \ldots''.}

