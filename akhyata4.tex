\section{Catutthakaṇḍa}

\head{482}{482, 508. dādantassaṃ mimesu.}
\headtrans{The ending of \pali{dā} [becomes] \pali{aṃ} because of \pali{mi} and \pali{ma}\mbox{[-vibhatti]}.}
\sutdef{dāiccetassa dhātussa antassa aṃ hoti mimaiccetesu.}
\sutdeftrans{There is [a substitution of] \pali{aṃ} for the ending of the root \pali{dā} because of \pali{mi} and \pali{ma}[-vibhatti].}
\example[0]{dammi (\paliroot{dā} + a + mi)}
\example{damma (\paliroot{dā} + a + ma)}
\transnote{For why niggahita becomes \pali{m}, see \hyperref[sut:31]{Kacc 31}.}

\head{483}{483, 527. asaṃyogantassa vuddhi kārite.}
\headtrans{For [roots of] non-conjunct ending [there is] vuddhi-strength because of the \pali{kārita}-paccaya.}
\sutdef{asaṃyogantassa dhātussa kārite vuddhi hoti.}
\sutdeftrans{There is vuddhi-strength for roots of non-conjunct ending because of the \pali{kārita}-paccaya.}
\example[0]{kāreti (\paliroot{kara} + ṇe + ti)}
\example[0]{kārenti (\paliroot{kara} + ṇe + nti)}
\example[0]{kārayati (\paliroot{kara} + ṇaya + ti)}
\example[0]{kārayanti (\paliroot{kara} + ṇaya + nti)}
\example[0]{kārāpeti (\paliroot{kara} + ṇāpe + ti)}
\example[0]{kārāpenti (\paliroot{kara} + ṇāpe + nti)}
\example[0]{kārāpayati (\paliroot{kara} + ṇāpaya + ti)}
\example{kārāpayanti (\paliroot{kara} + ṇāpaya + nti)}
\transnote{Note that \pali{asaṃyoganta} is truncated from \pali{asaṃyoganta ādiby\-añjana}, ``the starting consonant with non-conjunct ending'' (see \hyperref[sut:400]{Kacc 400}). For \pali{kārita}-paccayas, see \hyperref[sut:438]{Kacc 438}. For vuddhi-strength, see also \hyperref[sut:405]{Kacc 405}.}

\head{484}{484, 542. ghaṭādīnaṃ vā.}
\headtrans{For \pali{ghaṭa} and so on sometimes, [there is vuddhi-strength].}
\sutdef{ghaṭādīnaṃ dhātūnaṃ asaṃyogantānaṃ vuddhi hoti vā kārite.}
\sutdeftrans{There is vuddhi-strength for roots of non-conjunct ending such as \pali{ghaṭa} and so on sometimes because of the \pali{kārita}-paccaya.}
\example[0]{ghāṭeti, ghaṭeti (\paliroot{ghaṭa} + ṇe + ti)}
\example[0]{ghāṭayati, ghaṭayati (\paliroot{ghaṭa} + ṇaya + ti)}
\example[0]{ghāṭāpeti, ghaṭāpeti (\paliroot{ghaṭa} + ṇāpe + ti)}
\example[0]{ghāṭāpayati, ghaṭāpayati (\paliroot{ghaṭa} + ṇāpaya + ti)}
\example[0]{gāmeti, gameti (\paliroot{gamu} + ṇe + ti)}
\example[0]{gāmayati, gamayati (\paliroot{gamu} + ṇaya + ti)}
\example[0]{gāmāpeti, gamāpeti (\paliroot{gamu} + ṇāpe + ti)}
\example{gāmāpayati, gamāpayati (\paliroot{gamu} + ṇāpaya + ti)}

\head{485}{485, 434. aññesu ca.}
\headtrans{Because of other [paccayas] also, [there is vuddhi-strength].}
\sutdef{aññesu ca paccayesu sabbesaṃ dhātūnaṃ asaṃyogantānaṃ vuddhi hoti.}
\sutdeftrans{There is vuddhi-strength for all roots of non-conjunct ending because of other paccayas also.}
\example[0]{jayati (\paliroot{ji} + a + ti)}
\example[0]{hoti (\paliroot{hū} + a + ti)}
\example{bhavati (\paliroot{bhū} + a + ti)}
\transnote{The final vuddhi forms can be a little unpredictable. For the examples, it is \pali{jayati} not \pali{jeti}, and \pali{hoti} not \pali{havati}, but \pali{bhavati} not \pali{bhoti}.}

\head{486}{486, 543. guhadusānaṃ dīghaṃ.}
\headtrans{[The vowel of] \pali{guha} and \pali{dusa} [becomes] elongated [because of the \pali{kārita}-paccaya].}
\sutdef{guhadusaiccetesaṃ dhātūnaṃ saro dīghamāpajjate kārite.}
\sutdeftrans{The vowel of the roots \pali{guha} and \pali{dusa} becomes elongated because of the \pali{kārita}-paccaya.}
\example[0]{gūhayati (\paliroot{guha} + ṇaya + ti)}
\example{dūsayati (\paliroot{dusa} + ṇaya + ti)}

\head{487}{487, 478. vacavasavahādīnamukāro vassa ye.}
\headtrans{[There is] \pali{u}-substitution for \pali{va} of \pali{vaca}, \pali{vasa}, \pali{vaha} and so on because of \pali{ya}-paccaya.}
\sutdef{vacavasavahaiccevamādīnaṃ dhātūnaṃ vakārassa ukāro hoti yapaccaye pare.}
\sutdeftrans{There is [a substitution of] \pali{u} for \pali{va} of the roots \pali{vaca}, \pali{vasa}, \pali{vaha} and so on because of \pali{ya}-paccaya behind.}
\example[0]{uccate (\paliroot{vaca} + ya + te)}
\example[0]{vuccati (\paliroot{vaca} + ya + ti)}
\example[0]{vussati (\paliroot{vasa} + ya + ti)}
\example{vuyhati (\paliroot{vaha} + ya + ti)}
\transnote{For why \pali{vuyhati}, see the next sutta.}

\head{488}{488, 481. havipariyayo lo vā.}
\headtrans{[There is] a reversal of \pali{ha} [and \pali{ya} because of \pali{ya}-paccaya, and it becomes] \pali{la} sometimes.}
\sutdef{hakārassa vipariyayo hoti yapaccaye pare, yapaccayassa ca lo hoti vā.}
\sutdeftrans{There is a reversal of \pali{ha} [and \pali{ya}] because of \pali{ya}-paccaya. There is also \pali{la}[-substitution] for \pali{ya}-paccaya sometimes.}
\example[0]{vulhati (\paliroot{vaha} + ya + ti)}
\example{vuyhati (\paliroot{vaha} + ya + ti)}

\head{489}{489, 519. gahassa ghe ppe.}
\headtrans{For \pali{gaha}, [there is] \pali{ghe}-substitution because of \pali{ppa}.}
\sutdef{gahaiccetassa dhātussa sabbassa ghekāro hoti ppapaccaye pare.}
\sutdeftrans{There is [a substitution of] \pali{ghe} for the whole root of \pali{gaha} because of \pali{ppa}-paccaya behind.}
\example{gheppati (\paliroot{gaha} + ppa + ti)}

\head{490}{490, 518. halopo ṇhāmhi.}
\headtrans{[There is] an elision of \pali{ha} [of the root \pali{gaha}] because of \pali{ṇhā}-paccaya.}
\sutdef{gahaiccetassa dhātussa hakārassa lopo hoti ṇhāmhi paccaye pare.}
\sutdeftrans{There is an elision of \pali{ha} of the root \pali{gaha} because of \pali{ṇhā}-paccaya behind.}
\example{gaṇhāti (\paliroot{gaha} + ṇhā + ti)}

\head{491}{491, 523. karassa kāsattamajjatanimhi.}
\headtrans{For \pali{kara}, [there is] \pali{kāsa}-substitution because of \pali{ajjatanī}-vibhatti.}
\sutdef{karaiccetassa dhātussa sabbassa kāsattaṃ hoti vā ajjatanimhi vibhattimhi.}
\sutdeftrans{There is [a substitution of] \pali{kāsa} for the whole root of \pali{kara} sometimes because of \pali{ajjatanī}-vibhatti.}
\example[0]{akāsi = akari (\paliroot{kara} + o + ī)}
\example{akāsuṃ = akaruṃ (\paliroot{kara} + o + uṃ)}

\head{492}{492, 499. asasmā mimānaṃ mhimhā’ntalopo ca.}
\headtrans{[There are] \pali{mhi} and \pali{mha}[-substitution] for \pali{mi} and \pali{ma} after \pali{asa}, an elision of the ending also.}
\sutdef{asaiccetāya dhātuyā mimaiccetesaṃ vibhattīnaṃ mhimhādesā honti vā, dhātvantassa lopo ca.}
\sutdeftrans{There are substitutions of \pali{mhi} and \pali{mha} for \pali{mi}- and \pali{ma}-vibhatti after the root \pali{asa} sometimes. [There is] also an elision of the root's ending.}
\example[0]{amhi = asmi (\paliroot{asa} + a + mi)}
\example{amha = asma (\paliroot{asa} + a + ma)}

\head{493}{493, 498. thassa tthattaṃ.}
\headtrans{For \pali{tha} [after \pali{asa}, there is] \pali{ttha}[-substitution].}
\sutdef{asaiccetāya dhātuyā thassa vibhattissa tthattaṃ hoti, dhātvantassa lopo ca.}
\sutdeftrans{There is [a substitution of] \pali{ttha} for \pali{tha}-vibhatti after the root \pali{asa}. [There is] also an elision of the root's ending.}
\example{attha (\paliroot{asa} + a + tha)}

\head{494}{494, 495. tissa tthittaṃ.}
\headtrans{For \pali{ti} [after \pali{asa}, there is] \pali{tthi}[-substitution].}
\sutdef{asaiccetāya dhātuyā tissa vibhattissa tthittaṃ hoti, dhātvantassa lopo ca.}
\sutdeftrans{There is [a substitution of] \pali{tthi} for \pali{ti}-vibhatti after the root \pali{asa}. [There is] also an elision of the root's ending.}
\example{atthi (\paliroot{asa} + a + ti)}

\head{495}{495, 500. tussa tthuttaṃ.}
\headtrans{For \pali{tu} [after \pali{asa}, there is] \pali{tthu}[-substitution].}
\sutdef{asaiccetāya dhātuyā tussa vibhattissa tthuttaṃ hoti, dhātvantassa lopo ca.}
\sutdeftrans{There is [a substitution of] \pali{tthu} for \pali{tu}-vibhatti after the root \pali{asa}. [There is] also an elision of the root's ending.}
\example{atthu (\paliroot{asa} + a + tu)}

\head{496}{496, 497. simhi ca.}
\headtrans{Because of \pali{si}-vibhatti, also [there is an elision of the ending of \pali{asa}].}
\sutdef{asasseva dhātussa simhi vibhattimhi antassa lopo ca hoti.}
\sutdeftrans{There is also an elision of the ending of the root \pali{asa} because of \pali{si}-vibhatti.}
\example[0]{asi (\paliroot{asa} + a + si)}
\example{ko nu tvamasi mārisa {\upshape (Who are you, sir?)}}

\head{497}{497, 477. labhasmā īiṃnaṃ tthatthaṃ.}
\headtrans{After \pali{labha}, [there are] \pali{ttha}- and \pali{tthaṃ}-substitution for \pali{ī}- and \pali{iṃ}-vibhatti.}
\sutdef{labhaiccetāya dhātuyā īiṃnaṃ vibhattīnaṃ tthatthaṃādesā honti, dhātvantassa lopo ca.}
\sutdeftrans{There are substitutions of \pali{ttha} and \pali{tthaṃ} for \pali{ī}- and \pali{iṃ}-vibhatti after the root \pali{labha}. [There is] also an elision of the root's ending.}
\example[0]{alattha (\paliroot{labha} + a + ī)}
\example{alatthaṃ (\paliroot{labha} + a + iṃ)}

\head{498}{498, 480. kusasmā dī cchi.}
\headtrans{For \pali{ī} [after \pali{kusa}, there is] \pali{cchi}[-substitution].}
\transnote{The \pali{d} in the formula is an extra insertion.}
\sutdef{kusaiccetāya dhātuyā īvibhattissa cchi hoti, dhātvantassa lopo ca.}
\sutdeftrans{There is [a substitution of] \pali{cchi} for \pali{ī}-vibhatti after the root \pali{kusa}. [There is] also an elision of the root's ending.}
\example{akkocchi (\paliroot{kusa} + a + ī)}

\head{499}{499, 507. dādhātussa dajjaṃ.}
\headtrans{For the root \pali{dā}, [there is] \pali{dajja}[-substitution].}
\sutdef{dāiccetassa dhātussa sabbassa dajjādeso hoti vā.}
\sutdeftrans{There is a substitution of \pali{dajja} for the whole root of \pali{dā} sometimes.}
\example[0]{dajjāmi = dadāmi (\paliroot{dā} + a + mi)}
\example{dajjeyya = dadeyya (\paliroot{dā} + a + eyya)}

\head{500}{500, 486. vadassa vajjaṃ.}
\headtrans{For \pali{vada}, [there is] \pali{vajja}[-substitution].}
\sutdef{vadaiccetassa dhātussa sabbassa vajjādeso hoti vā.}
\sutdeftrans{There is a substitution of \pali{vajja} for the whole root of \pali{vada} sometimes.}
\example[0]{vajjāmi = vadāmi (\paliroot{vada} + a + mi)}
\example{vajjeyya = vadeyya (\paliroot{vada} + a + eyya)}

\head{501}{501, 443. gamissa ghammaṃ.}
\headtrans{For \pali{gamu}, [there is] \pali{ghamma}[-substitution].}
\sutdef{gamuiccetassa dhātussa sabbassa ghammādeso hoti vā.}
\sutdeftrans{There is a substitution of \pali{ghamma} for the whole root of \pali{gamu} sometimes.}
\example[0]{ghammatu (\paliroot{gamu} + a + tu)}
\example[0]{ghammāhi (\paliroot{gamu} + a + hi)}
\example{ghammāmi (\paliroot{gamu} + a + mi)}

\head{502}{502,~493.~yamhi\ \ dā\,dhā\,mā\,ṭhā\,hā\,pā\,maha\,mathādīnamī.}
\headtrans{Because of \pali{ya}-paccaya, [the ending of] \pali{dā}, \pali{dhā}, \pali{mā}, \pali{ṭhā}, \pali{hā}, \pali{pā}, \pali{maha}, \pali{matha}, and so on becomes \pali{ī}.}
\sutdef{yamhi paccaye pare dādhāmāṭhāhāpāmahamathaiccevamādīnaṃ dhātūnaṃ anto īkāramāpajjate.}
\sutdeftrans{Because of \pali{ya}-paccaya behind, the ending of [these] roots, i.e., \pali{dā}, \pali{dhā}, \pali{mā}, \pali{ṭhā}, \pali{hā}, \pali{pā}, \pali{maha}, \pali{matha}, and so on, becomes \pali{ī}.}
\example[0]{dīyati (\paliroot{dā} + ya + ti)}
\example[0]{dhīyati (\paliroot{dhā} + ya + ti)}
\example[0]{mīyati (\paliroot{mā} + ya + ti)}
\example[0]{ṭhīyati (\paliroot{ṭhā} + ya + ti)}
\example[0]{hīyati (\paliroot{hā} + ya + ti)}
\example[0]{pīyati (\paliroot{pā} + ya + ti)}
\example[0]{mahīyati (\paliroot{maha} + ya + ti)}
\example{mathīyati (\paliroot{matha} + ya + ti)}

\head{503}{503, 485. yajassādissi.}
\headtrans{The beginning of \pali{yaja} [becomes] \pali{i} [because of \pali{ya}-paccaya].}
\sutdef{yajaiccetassa dhātussa ādissa ikārādeso hoti yapaccaye pare.}
\sutdeftrans{There is a substitution of \pali{i} for the beginning of the root \pali{yaja} because of \pali{ya}-paccaya behind.}
\example[0]{ijjate (\paliroot{yaja} + ya + te)}
\example{ijjate mayā buddho {\upshape (The Buddha is worshipped by me.)}}

\head{504}{504, 470. sabbato uṃ iṃsu.}
\headtrans{After all [roots], \pali{uṃ}-vibhatti [becomes] \pali{iṃsu}.}
\sutdef{sabbehi dhātūhi uṃvibhattissa iṃsuādeso hoti.}
\sutdeftrans{There is a substitution of \pali{iṃsu} for \pali{uṃ}-vibhatti after all roots.}
\example[0]{upasaṅkamiṃsu (upa + saṃ + \paliroot{kamu} + a + uṃ)}
\example{nisīdiṃsu (ni + \paliroot{sī} + a + uṃ)}

\head{505}{505, 482. jaramarānaṃ jīrajiyyamiyyā vā.}
\headtrans{[There are substitutions of] \pali{jīra}, \pali{jiyya}, and \pali{miyya} for \pali{jara} and \pali{mara} sometimes.}
\sutdef{jaramaraiccetesaṃ dhātūnaṃ jīrajiyyamiyyādesā honti vā.}
\sutdeftrans{There are substitutions of \pali{jīra}, \pali{jiyya}, and \pali{miyya} for the roots \pali{jara} and \pali{mara} sometimes.}
\example[0]{jīrati (\paliroot{jara} + a + ti)}
\example[0]{jīranti (\paliroot{jara} + a + nti)}
\example[0]{jiyyati (\paliroot{jara} + a + ti)}
\example[0]{jiyyanti (\paliroot{jara} + a + nti)}
\example[0]{miyyati (\paliroot{mara} + a + ti)}
\example{miyyanti (\paliroot{mara} + a + nti)}

\head{506}{506, 496. sabbatthā’sassādilopo ca.}
\headtrans{Because of all [vibhattis and paccayas], the beginning of \pali{asa} [becomes] elided also.}
\sutdef{sabbattha vibhattipaccayesu asaiccetassa dhātussa ādissa lopo hoti vā.}
\sutdeftrans{There is an elision of the beginning of the root \pali{asa} because of all vibhattis and paccayas.}
\example[0]{siyā (\paliroot{asa} + a + eyya)}
\example[0]{santi (\paliroot{asa} + a + nti)}
\example[0]{sante (\paliroot{asa} + nta + smiṃ)}
\example{samāno (\paliroot{asa} + māna + si)}

\head{507}{507, 501. asabbadhātuke bhū.}
\headtrans{Because of non-all-roots [vibhattis, there is] \pali{bhū}[-substitution] [for \pali{asa}].}
\sutdef{asasseva dhātussa bhū hoti vā asabbadhātuke.}
\sutdeftrans{There is [a substitution of] \pali{bhū} for the [whole of] root \pali{asa} sometimes because of non-all-roots [vibhattis].}
\example[0]{bhavissati (\paliroot{asa} + a + ssati)}
\example{bhavissanti (\paliroot{asa} + a + ssanti)}
\transnote{In \hyperref[sut:431]{Kacc 431}, \pali{sabbadhātuka}-vibhattis are defined. Hence, the \pali{asabbadhātuka}-vibhattis (inapplicable to all roots) are those of parokkhā, ajjatanī, bhavissanti, and kālātipatti. Why we do not just simply see the instances as products of \paliroot{bhū} is still mysterious to me.}

\head{508}{508, 515. eyyassa ñāto iyāñā.}
\headtrans{For \pali{eyya} after \pali{ñā}, [there are] \pali{iyā}- and \pali{ñā}-substitution.}
\sutdef{eyyassa vibhattissa ñāiccetāya dhātuyā parassa iyāñāādesā honti vā.}
\sutdeftrans{There are substitutions of \pali{iyā} and \pali{ñā} for \pali{eyya}-vibhatti after the root \pali{ñā} sometimes.}
\example[0]{jāniyā (\paliroot{ñā} + nā + eyya)}
\example{jaññā (\paliroot{ñā} + nā + eyya)}
\transnote{Other forms of this are \pali{jāneyya} and \pali{jāne}. For how come the letter \pali{ja}, see \hyperref[sut:470]{Kacc 470} (\pali{ñāssa jājaṃnā}).}

\head{509}{509, 516. nāssa lopo yakārattaṃ.}
\headtrans{[There is] an elision of \pali{nā} [after \pali{ñā}]. [It may also become] \pali{ya}.}
\sutdef{ñāiccetāya dhātuyā parassa nāpaccayassa lopo hoti vā, yakārattañca.}
\sutdeftrans{There is an elision of \pali{nā}-paccaya after the root \pali{ñā} sometimes. [It may become] \pali{ya} also.}
\example[0]{jaññā (\paliroot{ñā} + nā + eyya)}
\example{nāyati (\paliroot{ñā} + nā + ti)}

\head{510}{510, 487. lopañcettamakāro.}
\headtrans{The \pali{a}-paccaya [becomes] elided and \pali{e}.}
\sutdef{akārapaccayo lopamāpajjate, ettañca hoti vā.}
\sutdeftrans{The \pali{a}-paccaya becomes elided, and becomes \pali{e} sometimes.}
\example[0]{vajjemi, vademi (\paliroot{vada} + a + mi)}
\example{vajjāmi, vadāmi (\paliroot{vada} + a + mi)}

\head{511}{511, 521. uttamokāro.}
\headtrans{The \pali{o}-paccaya [becomes] \pali{u}.}
\sutdef{okārapaccayo uttamāpajjate vā.}
\sutdeftrans{The \pali{o}-paccaya becomes \pali{u} sometimes.}
\example{kurute (\paliroot{kara} + o + te)}

\head{512}{512, 522. karassākāro ca.}
\headtrans{The letter \pali{a} of \pali{kara} also [becomes \pali{u}].}
\sutdef{karaiccetassa dhātussa akāro uttamāpajjate vā.}
\sutdeftrans{The letter \pali{a} of the root \pali{kara} becomes \pali{u} sometimes.}
\example[0]{kurute (\paliroot{kara} + o + te)}
\example{kubbanti (\paliroot{kara} + o + nti)}

\head{513}{513, 435. o ava sare.}
\headtrans{[The ending] \pali{o} [becomes] \pali{ava} because of vowel.}
\sutdef{okārassa dhātvantassa sare pare avādeso hoti.}
\sutdeftrans{There is a substitution of \pali{ava} for the ending \pali{o} of a root because of the vowel behind.}
\example[0]{cavati (\paliroot{cu} + a + ti)}
\example{bhavati (\paliroot{bhū} + a + ti)}

\head{514}{514, 491. e aya.}
\headtrans{[The ending] \pali{e} [becomes] \pali{aya}.}
\sutdef{ekārassa dhātvantassa sare pare ayādeso hoti.}
\sutdeftrans{There is a substitution of \pali{aya} for the ending \pali{e} of a root because of the vowel behind.}
\example[0]{nayati (\paliroot{nī} + a + ti)}
\example{jayati (\paliroot{ji} + a + ti)}
\transnote{Why does the sutta mention \pali{e} after all? It has something to do with vuddhi-strength. See \hyperref[sut:485]{Kacc 485}.}

\head{515}{515, 541. te āvāyā kārite.}
\headtrans{Those [\pali{o} and \pali{e} become] \pali{āva} and \pali{āya} because of the \pali{kārita}-paccayas.}
\sutdef{te oeiccete āvaāyādese pāpuṇanti kārite.}
\sutdeftrans{Those \pali{o} and \pali{e} become \pali{āva} and \pali{āya} because of the \pali{kārita}-paccayas.}
\example[0]{lāveti (\paliroot{lu} + ṇe + ti)}
\example{nāyeti (\paliroot{nī} + ṇe + ti)}

\head{516}{516, 466. ikārāgamo asabbadhātukamhi.}
\headtrans{[There is] an insertion of \pali{i} after the \pali{asabbadhātuka}-vibhatti.}
\sutdef{sabbasmiṃ asabbadhātukamhi ikārāgamo hoti.}
\sutdeftrans{There is an insertion of \pali{i} after the all \pali{asabbadhātuka}-vibhatti.}
\example[0]{gamissati (\paliroot{gamu} + a + ssati)}
\example[0]{karissati (\paliroot{kara} + o + ssati)}
\example[0]{labhissati (\paliroot{labha} + a + ssati)}
\example{pacissati (\paliroot{paca} + a + ssati)}
\transnote{For \pali{asabbadhātuka}-vibhatti, see \hyperref[sut:507]{Kacc 507}.}

\head{517}{517,~488.~kvaci~dhātuvibhattipaccayānaṃ~dīghaviparītāde\-salopāgamā~ca.}
\headtrans{In certain unexplained [points/instances], elongation, deformation, substitution, elision, and insertion upon roots, vibhattis, and paccayas [should be done] also.}
\sutdef{idha ākhyāte aniddiṭṭhesu sādhanesu kvaci dhātuvibhattipaccayānaṃ dīghaviparītādesalopāgamaiccetāni kāriyāni jinavacanānurūpāni kātabbāni.}
\sutdeftrans{In this Ākhyāta [section], in certain unexplained [points/instances] the operations of elongation, deformation, substitution, elision, and insertion upon roots, vibhattis, and paccayas should be done conforming to the Buddha's words.}
\example[0]{jāyati (\paliroot{jana} + ya + ti)}
\example[0]{kareyya (\paliroot{kara} + o + eyya)}
\example[0]{jāniyā (\paliroot{ñā} + nā + eyya)}
\example[0]{siyā (\paliroot{asa} + a + eyya)}
\example[0]{kare (\paliroot{kara} + o + eyya)}
\example[0]{gacche (\paliroot{gamu} + a + eyya)}
\example[0]{jaññā (\paliroot{ñā} + nā + eyya)}
\example[0]{vakkhetha (\paliroot{vaca} + a + tha)}
\example[0]{dakkhetha (\paliroot{disa} + a + tha)}
\example[0]{dicchati (\paliroot{dā} + sa + ti)}
\example[0]{agacchi (\paliroot{gamu} + a + ī)}
\example[0]{agacchuṃ (\paliroot{gamu} + a + uṃ)}
\example[0]{ahosi (\paliroot{hū} + a + ī)}
\example{ahesuṃ (\paliroot{hū} + a + uṃ)}
\transnote{Some of the examples in fact have already been explained.}

\head{518}{518, 446. attanopadāni parassapadattaṃ.}
\headtrans{[In some places], \pali{attanopada}[-vibhattis] become \pali{parassapada}.}
\sutdef{attanopadāni kvaci parassapadattamāpajjante.}
\sutdeftrans{In some places, \pali{attanopada}[-vibhattis] become \pali{parassapada}.}
\example[0]{vuccati (\paliroot{vaca} + ya + te)}
\example[0]{labbhati (\paliroot{labha} + ya + te)}
\example{paccati (\paliroot{paca} + ya + te)}

\head{519}{519, 457. akārāgamo hiyyattanīajjatanīkālātipattīsu.}
\headtrans{[In some places, there is] an insertion of \pali{a} because of \pali{hiyyattanī}-, \pali{ajjatanī}-, and \pali{kālātipatti}-vibhatti.}
\sutdef{kvaci akārāgamo hoti hiyyattanīajjatanīkālātipattiiccetāsu vibhattīsu.}
\sutdeftrans{In some places, there is an insertion of \pali{a} because of \pali{hiyyattanī}-, \pali{ajjatanī}-, and \pali{kālātipatti}-vibhatti.}
\example[0]{agamā (\paliroot{gamu} + a + ā)}
\example[0]{agamī (\paliroot{gamu} + a + ī)}
\example{agamissā (\paliroot{gamu} + a + ssā)}
\transnote{This \pali{a} in the front is called \emph{augment} by scholars.}

\head{520}{520, 502. brūto ī timhi.}
\headtrans{After \pali{brū}, [there is] \pali{i}-insertion because of \pali{ti}.}
\sutdef{brūiccetāya dhātuyā īkārāgamo hoti timhi vibhattimhi.}
\sutdeftrans{There is an insertion of \pali{ī} after the root \pali{brū} because of \pali{ti}-vibhatti.}
\example{bravīti (\paliroot{brū} + a + ti)}

\head{521}{521, 425. dhātussanto lopo’nekasarassa.}
\headtrans{The ending [vowel] of a root [becomes] elided because of multiple vowels.}
\sutdef{dhātussa anto kvaci lopo hoti anekasarassa.}
\sutdeftrans{The ending [vowel] of a root becomes elided in some places because of multiple vowels.}
\example[0]{gacchati (\paliroot{gamu} + a + ti)}
\example[0]{sarati (\paliroot{sara} + a + ti)}
\example{marati (\paliroot{mara} + a + ti)}
\transnote{This means, for example, when \paliroot{gamu} (having two vowels) is used, it becomes \paliroot{gam} first. Then it becomes \pali{gacch} by \hyperref[sut:476]{Kacc 476}. Finally, we get \pali{gacchati}. Other roots can be seen likewise with simpler procedure.}

\head{522}{522, 476. isuyamūnamanto ccho vā.}
\headtrans{The ending of \pali{isu} and \pali{yamu} [becomes] \pali{ccha} sometimes.}
\sutdef{isuyamuiccetesaṃ dhātūnaṃ anto ccho hoti vā.}
\sutdeftrans{The ending of roots \pali{isu} and \pali{yamu} becomes \pali{ccha} sometimes.}
\example[0]{icchati (\paliroot{isu} + a + ti)}
\example{niyacchati (ni + \paliroot{yamu} + a + ti)}
\transnote{For \pali{niyacchati}, see MWD under \pali{ni +} \paliroot{yam}, but also under \pali{ni +} \paliroot{gam}.}

\head{523}{523, 526. kāritānaṃ ṇo lopaṃ.}
\headtrans{The \pali{ṇa} of the \pali{kārita}-paccayas [becomes] elided.}
\sutdef{kāritaiccetesaṃ paccayānaṃ ṇo lopamāpajjate.}
\sutdeftrans{The \pali{ṇa}[-anubandha] of the \pali{kārita}-paccayas becomes elided.}
\example[0]{kāreti (\paliroot{kara} + ṇe + ti)}
\example[0]{kārayati (\paliroot{kara} + ṇaya + ti)}
\example[0]{kārāpeti (\paliroot{kara} + ṇāpe + ti)}
\example{kārāpayati (\paliroot{kara} + ṇāpaya + ti)}

\newpage
\setcounter{table}{11}
\begin{longtable}{%
		>{\raggedright\arraybackslash}p{0.25\linewidth}%
		>{\raggedright\arraybackslash}p{0.45\linewidth}}
\caption{Roots having particular treatments}\label{tab:roots-irr}\\
\toprule
\upshape\bfseries Root & \bfseries Suttas \\ \midrule
\endfirsthead
\multicolumn{2}{c}{\footnotesize\tablename\ \thetable: Roots having particular treatments (contd\ldots)}\\
\toprule
\upshape\bfseries Root & \bfseries Suttas \\ \midrule
\endhead
\bottomrule
\ltblcontinuedbreak{2}
\endfoot
\bottomrule
\endlastfoot
%
\paliroot{asa} & \hyperref[sut:492]{492}--\hyperref[sut:496]{496}, \hyperref[sut:506]{506}, \hyperref[sut:507]{507} \\
\paliroot{isu} & \hyperref[sut:522]{522} \\
\paliroot{kara} & \hyperref[sut:481]{481}, \hyperref[sut:491]{491}, \hyperref[sut:512]{512} \\
\paliroot{kita} & \hyperref[sut:463]{463} \\
\paliroot{kusa} & \hyperref[sut:498]{498} \\
\paliroot{gamu} & \hyperref[sut:476]{476}, \hyperref[sut:501]{501} \\
\paliroot{gaha} & \hyperref[sut:489]{489}, \hyperref[sut:490]{490} \\
\paliroot{guha} & \hyperref[sut:486]{486} \\
\paliroot{jara} & \hyperref[sut:505]{505} \\
\paliroot{ñā} & \hyperref[sut:470]{470}, \hyperref[sut:508]{508}, \hyperref[sut:509]{509} \\
\paliroot{ṭhā} & \hyperref[sut:468]{468} \\
\paliroot{dā} & \hyperref[sut:482]{482}, \hyperref[sut:499]{499} \\
\paliroot{disa} & \hyperref[sut:471]{471} \\
\paliroot{dusa} & \hyperref[sut:486]{486} \\
\paliroot{pā} & \hyperref[sut:467]{467}, \hyperref[sut:469]{469} \\
\paliroot{brū} & \hyperref[sut:475]{475}, \hyperref[sut:520]{520} \\
\paliroot{bhū} & \hyperref[sut:475]{475} \\
\paliroot{mara} & \hyperref[sut:505]{505} \\
\paliroot{māna} & \hyperref[sut:463]{463}, \hyperref[sut:467]{467} \\
\paliroot{yaja} & \hyperref[sut:503]{503} \\
\paliroot{yamu} & \hyperref[sut:522]{522} \\
\paliroot{vaca} & \hyperref[sut:477]{477}, \hyperref[sut:487]{487} \\
\paliroot{vada} & \hyperref[sut:500]{500} \\
\paliroot{vasa} & \hyperref[sut:487]{487} \\
\paliroot{vaha} & \hyperref[sut:487]{487} \\
\paliroot{hara} & \hyperref[sut:474]{474} \\
\paliroot{hū} & \hyperref[sut:480]{480} \\
\end{longtable}


