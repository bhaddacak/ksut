\chapter{Samāsa}

\section{Sattamakaṇḍa}
\raggedbottom

This section of Kaccāyana is all about compounds. There are categories of how to put words together into one unit, mostly following Sanskrit grammar and terminology. That is the main content of this part.

One thing new learners should know at this stage is that sandhi and samāsa might look similar in combining words together but they are radically different. Sandhi is phonetic or euphonic combination. It has nothing to do with the meaning of the components. Samāsa, in contrast, is grammatical combination. So, the forms of terms before they get combined affect the meaning of the whole unit.

There are a handful of samāsa types you will learn in due course. The main types are summarized in Table \ref{tab:samasatype}. Note that the definition of \pali{tappurisa} covers \pali{kammadhāraya}, \pali{digu}, and \pali{na-tappurisa}. Therefore the three are subtypes of determinative compound. All these are indexed in Table \ref{tab:samasaindex} at the end of this chapter.

{\footnotesize
\begin{longtable}{%
		@{}>{\raggedright\arraybackslash}p{0.2\linewidth}%
		>{\raggedright\arraybackslash}p{0.17\linewidth}%
		>{\raggedright\arraybackslash}p{0.5\linewidth}@{}}
\caption{Types of compounds}\label{tab:samasatype}\\
\toprule
\bfseries Pāli & \bfseries English & \bfseries Key feature \\ \midrule
\endfirsthead
\multicolumn{3}{c}{\tablename\ \thetable: Types of compounds (contd\ldots)}\\
\toprule
\bfseries Pāli & \bfseries English & \bfseries Key feature \\ \midrule
\endhead
\bottomrule
\ltblcontinuedbreak{3}
\endfoot
\bottomrule
\endlastfoot
%
dvanda & copulative & A collection of nouns in the same case, normally connected with \pali{ca}. \\
tappurisa & \mbox{determinative} & The pre-combined terms are related by different cases. \\
\mbox{kammadhāraya} & appositional & The pre-combined terms have the same case and signify the same thing. \\
digu & numerical & The first component is a number. \\
na-tappurisa & -- & The first component is the negative particle \pali{na}. \\
bahubbīhi & attributive & The unit signifies other thing outside itself, often embedded with other types, and can be very long. \\
abyayībhāva & adverbial & The first component is an indeclinable, so the unit is also used as such. \\
\end{longtable}
}

\head{316}{316, 331. nāmānaṃ samāso yuttattho.}
\headtrans{The combined meaning of nouns [is] \pali{samāsa}.}
\transnote{This sutta came from Kāt 2.259.}
\sutdef{tesaṃ nāmānaṃ payujjamānapadatthānaṃ yo yuttattho, so samāsasañño hoti.}
\sutdeftrans{Which [unit is] a combined meaning of those nouns having related meanings, that [unit] is named \pali{samāsa} (compound).}
\example[0]{kathinassa dussaṃ = kathinadussaṃ {\upshape (kathina's cloth)}}
\example[0]{āgantukassa bhattaṃ = āgantukabhattaṃ {\upshape (food for guest)}}
\example[0]{jīvitindriyaṃ = jīvitaṃ ca indriyaṃ {\upshape (vitality)}}
\example[0]{samaṇabrāhmaṇā = samaṇo ca brāhmaṇo {\upshape (ascetic and brahmin)}}
\example[0]{sāriputtamoggallānā = sāriputto ca moggallāno {\upshape (Sāriputta and Moggallāna)}}
\example{brāhmaṇagahapatikā = brāhmaṇo ca gahapatiko {\upshape (brahmin and householder)}}
\transnote{As you see in the examples, there are kinds of Pāli compounds we will learn later on. This first sutta just shows the basic idea of large units constructed by multiple words.}

\head{317}{317, 332. tesaṃ vibhattiyo lopā ca.}
\headtrans{[There are] elisions of vibhattis of those [compounds] also.}
\sutdef{tesaṃ yuttatthānaṃ samāsānaṃ vibhattiyo lopā ca honti.}
\sutdeftrans{There are also elisions of vibhattis of those compounds having meanings combined.}
\example[0]{kathinassa dussaṃ = kathinadussaṃ {\upshape (kathina's cloth)}}
\example{āgantukassa bhattaṃ = āgantukabhattaṃ {\upshape (food for guest)}}
\transnote{As shown in the same examples again, the genitive vibhatti in the preceding part of these compounds is elided. For the vi\-bh\-atti of the whole units, it depends on the function (= kāraka) of the compounds.}
\transnote{In some cases, however, vibhatis are retained, as shown in the followings.}
\example[0]{yo pabhaṃ karotīti, so pabhaṅkaro}
\example[0]{yo amataṃ dadotīti, so amatandado}
\example[0]{yo medhaṃ karotīti, so medhaṅkaro}
\example{yo dīpaṃ karotīti, so dīpaṅkaro}

\boxnote{Simple analytic sentence of a compound.\\
\hspace{5mm}\bullet\ When a compound is unpacked or restored its pre-combined components, we should get a comprehensible sentence corresponding to the meaning of the compound. We call this \emph{analytic sentence} (\pali{viggahavākya}) of the compound.\\
\hspace{5mm}\bullet\ If a certain vibhatti gets elided in the compound, in the analytic sentence it is restored. For example, ``\pali{kathinassa dussaṃ iti kathinadussaṃ hoti}'' is for \pali{kathinadussaṃ}.\\
\hspace{5mm}\bullet\ More often, the verb `to be' is left out and \pali{iti} is in sandhi form, hence ``\pali{kathinassa dussan'ti kathinadussaṃ}.''\\
\hspace{5mm}\bullet\ Likewise, for \pali{dīpaṅkaro} it is ``\pali{yo dīpaṃ karotī'ti so dīpaṅkaro}'' (Which person makes support, that person [is named] dīpaṅkara). In this case, the nominal vi\-bh\-atti is retained and changed. For the change of \pali{ṃ} to the last letter of consonant groups, see \hyperref[sut:31]{Kacc 31}.\\
\hspace{5mm}\bullet\ Still, many examples given in the text have the simplest form without \pali{iti}, e.g., \pali{tayo lokā tilokaṃ} (three-world).
}

\head{318}{318, 333. pakati cassa sarantassa.}
\headtrans{[There are] original [forms] of that [nominal base] ending with a vowel, [when vibhattis got elided].}
\sutdef{luttāsu vibhattīsu assa sarantassa liṅgassa pakatirūpāni honti.}
\sutdeftrans{When vibhattis [got] elided, there are original forms of that nominal base ending with a vowel.}
\example[0]{cakkhuṃ ca sotaṃ = cakkhusotaṃ}
\example[0]{mukhaṃ ca nāsikaṃ = mukhanāsikaṃ}
\example[0]{rañño putto = rājaputto}
\example{rañño puriso = rājapuriso}
\transnote{This sutta complements the previous one. That is to say, when vibhattis get elided, the terms also return to their original form. In the examples, \pali{cakkhu}, \pali{mukha}, and \pali{rāja} are all in their original form.}

\head{319}{319, 330. upasagganipātapubbako abyayībhāvo.}
\headtrans{[A compound having] a preceding prefix or particle [is] \pali{abya\-yībhāva}.}
\sutdef{upasagganipātapubbako samāso abyayībhāvasañño hoti.}
\sutdeftrans{A compound having a preceding prefix or particle is named \pali{abyayībhāva} (adverbial compound).}
\example[0]{nagarassa samīpe pavattati kathā iti upanagaraṃ\\{\upshape= [Which] speech happens in a nearby area of city, [that speech is called] \pali{upanagara} (near-to-city).}}
\example[0]{darathānaṃ abhāvo niddarathaṃ\\{\upshape= The absence of anxieties [is called] \pali{niddaratha} (anxiety-free).}}
\example[0]{makasānaṃ abhāvo nimmakasaṃ\\{\upshape= The absence of mosquitos [is called] \pali{nimmakasa} (mosquito-free).}}
\example[0]{vuḍḍhānaṃ paṭipāṭi yathāvuḍḍhaṃ\\{\upshape= The order of seniority [is called] \pali{yathāvuḍḍha} (in-seniority-order).}}
\example[0]{ye ye vuḍḍhā yathāvuḍḍhaṃ\\{\upshape= Everyone of the elderly [is called] \pali{yathāvuḍḍha}.}}
\example[0]{jīvassa yattako paricchedo yāvajīvaṃ\\{\upshape= Which limit [is] as long as the life, [that limit is called] \pali{yāvajīva} (life-long).}}
\example[0]{cittamadhikicca pavattanti te dhammāti adhicittaṃ\\{\upshape= [Which conditions] happen relating to the mind, those conditions [are called] \pali{adhicitta} (mind-related).}}
\example[0]{pabbatassa tiro tiropabbataṃ\\{\upshape= Outside a mountain [is called] \pali{tiropabbata} (extra-mountain).}}
\example[0]{sotassa pati pavattati nāvā iti patisotaṃ\\{\upshape= [Which] boat moves against a stream, [that boat is called] \pali{patisota} (anti-stream).}}
\example{pāsādassa anto antopāsādaṃ\\{\upshape= Inside a mansion [is called] \pali{antopāsāda} (intra-mansion).}}
\transnote{This kind of compounds has an indeclinable as its front part. The examples given are all worth studying closely. Learning from these can help you realize that these compounds are normally used as modifiers. Thus this type of compounds is called `\emph{adverbial}.'}
\transnote{One noteworthy point is on \pali{yathā} in the 4th and 5th examples. It has two meaning, first `in order' (\pali{paṭipāṭi}), and second `repetition' (\pali{vicchā}).}

\head{320}{320, 335. so napuṃsakaliṅgo.}
\headtrans{That [adverbial compound should be seen as] neuter.}
\transnote{This sutta was taken from Kāt 2.273 (\pali{sa napuṃsakaliṅgaṃ syāt}).}
\sutdef{so abyayībhāvasamāso napuṃsakaliṅgova daṭṭhabbo.}
\sutdeftrans{That adverbial compound should be seen only as neuter.}
\example[0]{kumārīsu adhikicca pavattati kathā iti adhikumāri\\{\upshape= [Which] talk happens concerning [in] girls, [that talk is] \mbox{\pali{adhikumāri}} (girl-concerning).}}
\example[0]{vadhuyā samīpe pavattati kathā iti upavadhu\\{\upshape= [Which] talk happens nearby a girl, [that talk is] \pali{upavadhu} (near-to-girl).}}
\example{gaṅgāya samīpe pavattati kathā iti upagaṅgaṃ\\{\upshape= [Which] talk happens nearby a river, [that talk is] \pali{upagaṅga} (near-to-river).}}

\head{321}{321, 349. digussekattaṃ.}
\headtrans{[There is] a singularity of \pali{digu-samāsa}.}
\sutdef{digussa samāsassa ekattaṃ hoti, napuṃsakaliṅgattañca.}
\sutdeftrans{There is a singularity of \pali{digu-samāsa} (numerical compound), [it is] neuter as well.}
\example[0]{tayo lokā tilokaṃ {\upshape(the three-world)}}
\example[0]{tayo daṇḍā tidaṇḍaṃ {\upshape(the three-stick)}}
\example[0]{tīṇi nayanāni tinayanaṃ {\upshape(the three-eye)}}
\example[0]{tayo siṅgā tisiṅgaṃ {\upshape(the three-horn)}}
\example[0]{catasso disā catuddisaṃ {\upshape(the four-direction)}}
\example{pañca indriyāni pañcindriyaṃ {\upshape(the five-faculty)}}
\transnote{This kind of compound has a number as one component. See its definition in \hyperref[sut:325]{Kacc 325}.}

\head{322}{322,~359.~tathā~~dvande~~pāṇi\,tūriya\,yogga\,senaṅga\,khudda\-jantuka\,vividhaviruddha\,visabhāgatthādīnañca.}
\headtrans{Likewise in \pali{dvanda-samāsa}, [an application of words such as] living being, music, farming equipment, parts of army, small aminals, pair of enmity, various different meaning, and so on [is singular and neuter].}
\sutdef{tathā dvande samāse pāṇitūriyayoggasenaṅgakhuddajantukavividhaviruddhavisabhāgatthaiccevamādīnaṃ ekattaṃ hoti, napuṃsakaliṅgattañca.}
\sutdeftrans{Likewise, there is a singularity of [an application of words such as] living being, music, farming equipment, parts of army, small aminals, pair of enmity, various different meaning, and so on in \pali{dvanda-samāsa} (copulative compound), [it is] neuter as well.}
\transnote{Except the 3rd and 7th usage, this sutta has a combination of Pāṇini's sūtras, i.e., Pāṇ 2.4.2, 2.4.8, and 2.4.9.}

\bigbullet{(1) Living being's organs}
\example[0]{cakkhu ca sotañca cakkhusotaṃ\\{\upshape= eye and ear}}
\example[0]{mukhañca nāsikā ca mukhanāsikaṃ\\{\upshape= mouth and nose}}
\example{chavimaṃsañca lohitañca chavimaṃsalohitaṃ\\{\upshape= skin, flesh, and blood}}

\bigbullet{(2) Music}
\example[0]{saṅkho ca paṇavo ca saṅkhapaṇavaṃ\\{\upshape= conch and drum}}
\example[0]{gītañca vāditañca gītavāditaṃ\\{\upshape= singing and playing}}
\example{daddari ca ḍiṇḍimo ca daddariḍiṇḍimaṃ\\{\upshape= daddari and ḍiṇḍima drum}}

\bigbullet{(3) Farming equipment}
\example[0]{phālo ca pācanañca phālapācanaṃ\\{\upshape= plow and goad}}
\example{yugañca naṅgalañca yuganaṅgalaṃ\\{\upshape= yoke and plow}}

\bigbullet{(4) Parts of army}
\example[0]{asi ca cammañca asicammaṃ\\{\upshape= sword and leather armor}}
\example[0]{dhanu ca kalāpo ca dhanukalāpaṃ\\{\upshape= bow and quiver}}
\example[0]{hatthī ca asso ca hatthiassaṃ\\{\upshape= elephant and horse}}
\example{ratho ca pattiko ca rathapattikaṃ\\{\upshape= chariot and foot soldier}}

\bigbullet{(5) Small animals}
\example[0]{ḍaṃsā ca makasā ca ḍaṃsamakasaṃ\\{\upshape= gadflies and mosquitos}}
\example[0]{kuntho ca kipilliko ca kunthakipillikaṃ\\{\upshape= ant and termite}}
\example{kīṭo ca sarīsapo ca kīṭasarīsapaṃ\\{\upshape= insect and reptile}}

\bigbullet{(6) Pair of eternal enmity}
\example[0]{ahi ca nakulo ca ahinakulaṃ\\{\upshape= snake and mongoose}}
\example[0]{biḷāro ca mūsiko ca biḷāramūsikaṃ\\{\upshape= cat and mouse}}
\example{kāko ca ulūko ca kākolūkaṃ\\{\upshape= crow and owl}}

\bigbullet{(7) Various different meaning}
\example[0]{sīlañca paññāṇañca sīlapaññāṇaṃ\\{\upshape= morality and wisdom}}
\example[0]{samatho ca vipassanā ca samathavipassanaṃ\\{\upshape= tranquility and insight meditation}}
\example{vijjā ca caraṇañca vijjācaraṇaṃ\\{\upshape= knowledge and behavior}}
\transnote{Note that even though the compounds signify multiple things, the words are treated as one unit, hence a neuter noun.}

\head{323}{323,~360.~vibhāsā~rukkhatiṇapasudhanadhaññajanapadā\-dīnañca.}
\headtrans{Sometimes also [an application of words such as types or names of] tree, grass, livestock, wealth, grain, province, and so on [is singular and neuter].}
\transnote{This sutta was partly taken from Pāṇ 2.4.12.}
\sutdef{rukkhatiṇapasudhanadhaññajanapadaiccevamādīnaṃ vibhā\-sā ekattaṃ hoti, napuṃsakaliṅgattañca dvande samāse.}
\sutdeftrans{There is a singularity of [an application of words such as types or names of] tree, grass, livestock, wealth, grain, province, and so on sometimes, and it is neuter, in copulative compound.}
\example[0]{assattho ca kapītano ca assatthakapītanaṃ/assatthakatanā\\{\upshape= holy fig tree and pipal tree}}
\example[0]{usīrañca bīraṇañca usīrabīraṇaṃ/usīrabīraṇā\\{\upshape= usīra and bīraṇa fragrant grass}}
\example[0]{ajo ca eḷako ca ajeḷakaṃ/ajeḷakā\\{\upshape= goat and ram}}
\example[0]{hiraññañca suvaṇṇañca hiraññasuvaṇṇaṃ/hiraññasuvaṇṇā\\{\upshape= silver and gold}}
\example[0]{sāli ca yavo ca sāliyavaṃ/sāliyavā\\{\upshape= wheat and barley}}
\example[0]{kāsī ca kosalā ca kāsikosalaṃ/kāsikosalā\\{\upshape= Kāsī and Kosala}}

\head{324}{324, 339. dvipade tulyādhikaraṇe kammadhārayo.}
\headtrans{Because of the two terms denoting the same thing, [the compound is named] \pali{kammadhāraya}.}
\transnote{This sutta was taken from Kāt 2.263. It is similar to Pāṇ 1.2.42, but \pali{samānādhikaraṇa} is used instead of \pali{tulyādhikaraṇa}.}
\sutdef{dve padāni tulyādhikaraṇāni yadā samasyante, tadā so samāso kammadhārayasañño hoti.}
\sutdeftrans{When two terms denoting the same thing are joined together, then that compound is named \pali{kammadhāraya} (appositional).\footnote{In some textbook, this is called `descriptive determinative' (\citealp[p.~131]{collins:grammar}).}}
\example[0]{mahanto ca so puriso cāti mahāpuriso\\{\upshape= [It is] great and it [is] also a man, hence \pali{mahāpurisa} (great man).}}
\example[0]{kaṇho ca so sappo cāti kaṇhasappo\\{\upshape= [It is] black and it [is] also a snake, hence \pali{kaṇhasappa} (black snake/cobra).}}
\example[0]{nīlañca taṃ uppalañcāti nīluppalaṃ\\{\upshape= [It is] blue and it [is] also a lotus, hence \pali{nīluppala} (blue lotus).}}
\example[0]{lohitañca taṃ candanañcāti lohitacandanaṃ\\{\upshape= [It is] red and it [is] also sandal wood, hence \pali{lohitacandana} (red sandal wood).}}
\example[0]{brāhmaṇī ca sā dārikā cāti brāhmaṇadārikā\\{\upshape= [It is] a brahmin and it [is] also a girl, hence \pali{brāhmaṇadārikā} (brahmin girl).}}
\example{khattiyā ca sā kaññā cāti khattiyakaññā\\{\upshape= [It is] of warrior-caste and it [is] also a girl, hence \pali{khattiyakaññā} (warrior-caste girl).}}
\transnote{The key word here is \pali{tulyādhikaraṇa} (or \pali{samānādhikaraṇa} in Pāṇinian terminology). The term literally means those having an equal basis or support. So, the two terms in this compound signify the same entity. We call this \emph{apposition} in English.}

\head{325}{325, 348. saṅkhyāpubbo digu.}
\headtrans{[A compound having] a number in the front [is named] \pali{digu}.}
\sutdef{saṅkhyāpubbo kammadhārayasamāso digusañño hoti.}
\sutdeftrans{An appositional compound having a number in the front is named \pali{digu} (numerical).}
\transnote{This definition was taken from Pāṇ 2.1.52 or Kāt 2.264.}
\example[0]{tīṇi malāni timalaṃ {\upshape (the three impurities)}}
\example[0]{tīṇi phalāni tiphalaṃ {\upshape (the three fruits)}}
\example[0]{tayo lokā tilokaṃ {\upshape (the three worlds)}}
\example[0]{tayo daṇḍā tidaṇḍaṃ {\upshape (the three sticks)}}
\example[0]{catasso disā catuddisaṃ {\upshape (the four directions)}}
\example[0]{pañca indriyāni pañcindriyaṃ {\upshape (the five faculties)}}
\example{satta godāvariyo sattagodāvaraṃ {\upshape (the seven rivers of Godāvarī)}}
\transnote{The name \pali{digu} (Skt.\ \pali{dvigu}) literally means `two cows.' It is technically used only for this type of compound. The whole unit is neuter, as stated in \hyperref[sut:321]{Kacc 321}.}

\head{326}{326, 341. ubhe tappurisā.}
\headtrans{Both [\pali{digu}- and \pali{kammadhāraya}-samāsa are called] \pali{tappurisa}.}
\sutdef{ubhe digukammadhārayasamāsā tappurisasaññā honti.}
\sutdeftrans{Both \pali{digu}- and \pali{kammadhāraya}-samāsa are called \pali{tappurisa}.}
\transnote{This sutta was taken from Kāt 2.265, and the definition is found in Kāt 2.266. Pāṇini did not define the term but used as a heading (\pali{adhikāra}) in Pāṇ 2.1.22. The term is defined as \pali{tassa puriso tappuriso} (Rūpa 341). The meaning `his man' does little help. So, we normally use this technical term without translating it. However, it is called `dependent determinative' by scholars.\footnote{\citealp{collins:grammar}, p.~131}}
\example{na brāhmaṇo abrāhmaṇo\\{\upshape [One is] not a brahmin, hence \pali{abrāhmaṇa} (non-brahmin).}}
\transnote{This example is not relevant to the definition, but it is \pali{tappurisa} anyway. The example came directly from Kāśikā on Pāṇ 2.2.6. This compound is called \pali{nañtatpuruṣa} in Sanskrit. Other examples are skipped, but they should be seen likewise. See also \hyperref[sut:333]{Kacc 333} below.}

\head{327}{327, 351. amādayo parapadebhi.}
\headtrans{[The vibhattis such as] \pali{aṃ} and so on [joining with] other terms [produce \pali{tappurisa-samāsa}].}
\sutdef{tā amādayo nāmehi parapadebhi yadā samasyante, tadā so samāso tappurisasañño hoti.}
\sutdeftrans{When [terms with] those [vibhattis such as] \pali{aṃ} and so on are joined with other nominal terms, then that compound is named \pali{tappurisa}.}
\transnote{This definition probably came from Kāt 2.266.}
\bigbullet{(1) With accusatives}
\example[0]{bhūmiṃ gato bhūmigato\\{\upshape= [One] went on the earth, hence \pali{bhūmigata} (earth-goer).}}
\example[0]{sabbarattiṃ sobhaṇo sabbarattisobhaṇo \\{\upshape= [One is] beautiful all night, hence \pali{sabbarattisobhaṇa} (all-night beautiful one).}}
\example{apāyaṃ gato apāyagato\\{\upshape= [One] went to hell, hence \pali{apāyagata} (hell-goer).}}
\bigbullet{(2) With instrumentals}
\example[0]{issarena kataṃ issarakataṃ \\{\upshape= [A thing was] created by God, hence \pali{issarakata} (created by God).}}
\example{sallena viddho sallaviddho\\{\upshape= [One] was hit by an arrow, hence \pali{sallaviddha} (hit-by-arrow one).}}
\bigbullet{(3) With datives}
\example[0]{kathinassa dussaṃ kathinadussaṃ \\{\upshape= A cloth for the kathina, hence \pali{kathinadussa} (kathina-cloth).}}
\example{āgantukassa bhattaṃ āgantukabhattaṃ \\{\upshape= Food for guest, hence \pali{āgantukabhatta} (guest-food).}}
\bigbullet{(4) With ablatives}
\example[0]{methunā apeto methunāpeto\\{\upshape= [One] abstained from sexual intercourse, hence \pali{methunāpeta} (sex-abstaining one).}}
\example{corā bhayaṃ corabhayaṃ\\{\upshape= Danger/fear from thief, hence \pali{corabhaya} (danger/fear from thief).}}
\bigbullet{(5) With genitives}
\example[0]{rañño putto rājaputto\\{\upshape= A son of the king, hence \pali{rājaputta} (prince).}}
\example{dhaññānaṃ rāsi dhaññarāsi\\{\upshape= A heap of grains, hence \pali{dhaññarāsi} (grain-heap).}}
\bigbullet{(6) With locatives}
\example[0]{rūpe saññā rūpasaññā\\{\upshape= Perception/recognition in form, hence \pali{rūpasaññā} (perception/regonition of form).}}
\example{saṃsāre dukkhaṃ saṃsāradukkhaṃ \\{\upshape= Suffering in the cyclic existence, hence \pali{saṃsāradukkha} (cyclic-existence-suffering).}}
\transnote{To put it another way, tappurisa-samāsa is a compound in which one component declines by a vibhatti from the second (accusative case) onward to the seventh (locative case). If kammadhāraya and digu are counted as tappurisa, as stated in the previous sutta, the first vibhatti (nominative case) can also be the case.}

\head{328}{328, 352. aññapadatthesu bahubbīhi.}
\headtrans{[The compound of nouns] in the meaning of other terms [is] \pali{bahubbīhi}.}
\sutdef{aññesaṃ padānaṃ atthesu dve nāmāni bahūni nāmāni yadā samasyante, tadā so samāso bahubbīhi sañño hoti.}
\sutdeftrans{When two nouns [or] several nouns in the meaning of other terms are joined together, then that compound is named \pali{bahubbīhi} (attributive).}
\transnote{The definition looks similar to Kāt 2.267.}

\bigbullet{(1) Accusative attributive}
\example{āgatā samaṇā imaṃ saṅghārāmaṃ soyaṃ āgatasamaṇo\\{\upshape= Ascetics have come to [which] this monastery, that this [mona\-stery is called] \pali{āgatasamaṇa} (ascetic-having-come [monastery]).}}
\transnote{Let us pause and try to digest this first. The main idea of \pali{bahubbīhi} (lit. ``many rice grains'') compound is that its meaning (the signified) is not in the compound but in other terms (aññapada). Therefore, \pali{āgatasamaṇa} here does not mean ``ascetic who has come,'' but it describes an atttribute of another thing, i.e., a monastery in this case (\pali{imaṃ saṅghārāmaṃ} which does not appear in the compound). The use of other terms in other cases can be understood in the same way.}

\boxnote{The context can tell whether a compound is tappurisa or bahubbīhi.\\
\hspace{5mm}\bullet\ As you might see from the above example, \pali{āgata\-samaṇo} can means ``an ascetic who has come'' in a certain sentence, e.g., \pali{so āgatasamaṇo hoti}. By this use, the compound is tappurisa (or kammadhāraya in this case).\\
\hspace{5mm}\bullet\ If the term is used in \pali{so āgatasamaṇaṃ [ārāmaṃ] gacchati}, it means ``He goes to [a temple that] an ascetic has come.'' Then the term is bahubbīhi because it signifies the place (the entity outside the compound).
}

\bigbullet{(2) Instrumental attributive}
\example{jitāni indriyāni anena samaṇena soyaṃ jitindriyo\\{\upshape= The faculties were defeated by [which] this ascetic, that this [ascetic is called] \pali{jitindriya} (faculty-defeating [ascetic]).}}

\bigbullet{(3) Dative attributive}
\example{dinno suṅko yassa rañño soyaṃ dinnasuṅko\\{\upshape= Tax was given to which king, that this [king is called] \pali{dinnasuṅka} (tax-given [king]).}}

\bigbullet{(4) Ablative attributive}
\example{niggatā janā asmā gāmā soyaṃ niggatajano\\{\upshape= People departed from [which] this village, that this [village is called] \pali{niggatajana} (people-departed [village]).}}

\bigbullet{(5) Genitive attributive}
\example{chinno hatto yassa purisassa soyaṃ chinnahattho\\{\upshape= A hand of which man was amputated, that this [man is called] \pali{chinnahattha} (hand-amputated [man]).}}

\bigbullet{(6) Locative attributive}
\example{sampannāni sassāni yasmiṃ janapade soyaṃ sampannasasso\\{\upshape= Crops have fully grown in which district, that this [district is called] \pali{sampannasassa} ([district] with fully grown crops).}}

\bigbullet{(7) Metaphorical attributive}
\example[0]{nigrodhaparimaṇḍalo iva parimaṇḍalo yassa rājakumārassa soyaṃ nigrodhaparimaṇḍalo\\{\upshape= The circumference of which prince [is like] the circumference of a banyan tree, that this [prince is called] \pali{nigrodhaparimaṇḍala} ([prince] with circumference of banyan tree [=? protective/dependable]).}}
\example[0]{cakkhubhūto iva bhūto yo bhagavā soyaṃ cakkhubhūto\\{\upshape= Which the Blessed One existed [is] like the existing of eyes, that this [the Blessed One is called] \pali{cakkhubhūta} ([the Blessed One] who is like the existing of eyes).}}
\example{suvaṇṇavaṇṇo viya vaṇṇo yassa bhagavato soyaṃ suvaṇṇavaṇṇo\\{\upshape= The complexion of which the Blessed One [is] like gold color, that this [the Blessed One is called] \pali{suvaṇṇavaṇṇo} ([the Blessed One] with golden-complexion).}}
\transnote{The compounds used in the above examples are described as the followings: \pali{nigrodhassa parimaṇḍalo nigrodhaparimaṇḍalo}, \pali{cakkhuno bhūto cakkhubhūto}, and \pali{suvaṇṇassa vaṇṇo suvaṇṇa\-vaṇṇo}. These are \pali{tappurisa} by themselves, but when denoting other things, they become \pali{bahubbīhi}.}

\bigbullet{(8) An analysis of \pali{sayaṃpatitapaṇṇapupphaphalavāyutoyāhārā}}
\example[0]{paṇṇañca pupphañca phalañca paṇṇapupphaphalāni\\{\upshape= Leaf, flower, and fruit [are] \pali{paṇṇapupphaphala}.}}
\example[0]{sayameva patitāni sayaṃpatitāni\\{\upshape= [Things] just fell by themselves [are] \pali{sayaṃpatita}.}}
\example[0]{sayaṃpatitāni ca tāni paṇṇapupphaphalāni ceti sayaṃpatitapaṇṇapupphaphalāni\\{\upshape= Those [are] leaf, flower, and fruit, also fell by themselves, hence \pali{sayaṃpatitapaṇṇapupphaphala} (self-fallen leaf, flower, and fruit).}}
\example[0]{vāyu ca toyañca vāyutoyāni\\{\upshape= Wind and water [are] \pali{vāyutoya}.}}
\example[0]{sayaṃpatitapaṇṇapupphaphalāni ca vāyutoyāni ca sayaṃpatitapaṇṇapupphaphalavāyutoyāni\\{\upshape= [Those are] self-fallen leaf, flower, and fruit, also wind and water, hence \pali{sayaṃpatitapaṇṇapupphaphalavāyutoya}.}}
\example{sayaṃpatitapaṇṇapupphaphalavāyutoyāni āhārā yesaṃ te sayaṃpatitapaṇṇapupphaphalavāyutoyāhārā\\{\upshape= The self-fallen leaf, flower, and fruit, as well as wind and water, [are] food of which [sages], those [sages are called] \pali{sayaṃpatitapaṇṇapupphaphalavāyutoyāhāra} (subsisting oneself with self-fallen leaf, flower, and fruit, as well as wind and water).}}
\transnote{When analyzed as such, the compound is specifically named \pali{tulyādhikaraṇapahubbīhi} having \pali{dvanda} and \pali{kammadhāraya} inside (\pali{dvandakammadhārayagabbha}). It is \pali{tulyādhikaraṇa} by the fact that \pali{sayaṃpatitapaṇṇapupphaphalavāyutoyāni} and \pali{āhārā} take the same case.}
\transnote{If the compound is instead analyzed as \pali{sayaṃpatitapaṇṇapupphaphalavāyutoyehi āhārā yesaṃ te sayaṃpatitapaṇṇapupphaphalavāyutoyāhārā}, it is called \pali{bhinnādhikaraṇabahubbīhi} by the fact that the two terms mentioned above take different cases.}

\bigbullet{(9) An analysis of \pali{nānādumapatitapupphavāsitasānu}}
\example[0]{nānāpakārā dumā nānādumā\\{\upshape= Various trees, hence \pali{nānādumā}.}}
\example[0]{nānādumehi patitāni nānādumapatitāni\\{\upshape= [Things] fell from various trees, hence \pali{nānādumapatita}.}}
\example[0]{nānādumapatitāni ca tāni pupphāni cāti nānādumapatitapupphāni\\{\upshape= [Those] fell from various trees and they [are] also flowers, hence \pali{nānādumapatitapuppha}.}}
\example[0]{nānādumapatitapupphehi vāsitā nānādumapatitapupphavā\-sitā\\{\upshape= [Places] perfumed by/from flowers fallen from various trees [are] \pali{nānādumapatitapupphavāsita}.}}
\example{nānādumapatitapupphavāsitā sānū yassa pabbatarājassa so\-yaṃ nānādumapatitapupphavāsitasānu\\{\upshape= Of which grand mountain the plateaus perfumed by/from flowers fallen from various trees, that this [grand mountain is called] \pali{nānādumapatitapupphavāsitasānu}.}}
\transnote{When analyzed as such, the compound is called \pali{tulyādhikara\-ṇabahubbīhi} having \pali{kammadhāraya} and \pali{tappurisa} inside. Another variation of analysis should be understood as mentioned in the previous example.}

\bigbullet{(10) An analysis of \pali{byālambambudharabinducumbitakūṭo}}
\example[0]{ambuṃ dhāretīti ambudharo\\{\upshape= [Which thing] carries water, hence [that thing is] \pali{ambu\-dhara} (water-carrier, rain-cloud).}}
\example[0]{vividhā ālambo byālambo\\{\upshape= [A thing] hanging down in various ways [is] \pali{byālamba}.}}
\example[0]{byālambo ca so ambudharo cāti byālambambudharo\\{\upshape= [The thing is] hanging down in various ways and it [is] also a rain-cloud, hence \pali{byālambambudhara}.}}
\example[0]{byālambambudharassa bindū byālambambudharabindū\\{\upshape= Drops of a rain-cloud hanging down in various ways [are] \pali{byālambambudharabinda}.}}
\example[0]{byālambambudharabindūhi cumbito byālambambudharabinducumbito\\{\upshape= Being kissed by drops of a rain-cloud hanging down in various ways [is] \pali{byālambambudharabinducumbita}.}}
\example{byālambamburebinducumbito kūṭo yassa pabbatarājassa so\-yaṃ byālambambudharabinducumbitakūṭo\\{\upshape= Which grand mountain [has] the peak kissed by drops of a rain-cloud hanging down in various ways, that this [grand mountain is called] \pali{byālambambudharabinducumbitakūṭa}.}}
\transnote{When analyzed as such, the compound is called \pali{tulyādhikara\-ṇabahubbīhi} having \pali{kammadhāraya} and \pali{tappurisa} inside.}

\bigbullet{(11) An analysis of \pali{amitabalaparakkamajuti}}
\example[0]{na mitā amitā\\{\upshape= [Things were] not measured, hence \pali{amita} (immeasurable).}}
\example[0]{balañca parakkamo ca juti ca balaparakkamajutiyo\\{\upshape= Strength, effort, and radiance [are] \pali{balaparakkamajuti}.}}
\example{amitā balaparakkamajutiyo yassa soyaṃ amitabalaparakkamajuti\\{\upshape= Which [person has] immeasurable strength, effort, and radiance, that this [person is called] \pali{amitabalaparakkamajuti}.}}
\transnote{When analyzed as such, the compound is called \pali{tulyādhikara\-ṇabahubbīhi} having \pali{kammadhāraya} and \pali{dvanda} inside.}

\bigbullet{(12) An analysis of \pali{pīṇorakkhaṃsabāhu}}
\example[0]{uro ca akkhañca aṃso ca bāhu ca urakkhaṃsabāhavo\\{\upshape= Chest, collarbone, shoulder, and arm [are] \pali{urakkhaṃsabāhava}.\footnote{Having the same meaning of `axle,' \pali{akkha} (Skt.\ \pali{akṣa}) also means `the collar-bone' (see MWD).}}}
\example{pīṇā urakkhaṃsabāhavo yassa bhagavato soyaṃ pīṇorakkhaṃ\-sabāhu\\{\upshape= Which the Blessed One [has] full chest, collarbone, shoulder, and arm, that this [the Blessed One is called] \pali{pīṇorakkhaṃsa\-bāhu}.}}
\transnote{When analyzed as such, the compound is called \pali{tulyādhikara\-ṇabahubbīhi} having \pali{dvanda} inside.}

\bigbullet{(13) An analysis of \pali{pīṇagaṇḍavadanathanūrujaghana}}
\example[0]{gaṇḍo ca vadanañca thano ca ūru ca jaghanañca gaṇḍavada\-nathanūrujaghanā\\{\upshape= Cheek, face, breast, thigh, and loin [are] \pali{gaṇḍavadanathanū\-rujaghana}.}}
\example{pīṇā gaṇḍavadanathanūrujaghanā yassā sāyaṃ pīṇagaṇḍava\-danathanūrujaghanā\\{\upshape= Which [girl has] full cheek, face, breast, thigh, and loin, that this [girl is called] \pali{pīṇagaṇḍavadanathanūrujaghanā}.}}
\transnote{When analyzed as such, the compound is called \pali{tulyādhikara\-ṇabahubbīhi} having \pali{dvanda} inside.}

\bigbullet{(14) An analysis of \pali{pavarasurāsuragaruḍamanujabhujagagandhabbamakuṭakūṭacumbitaselasaṅghaṭṭitacaraṇa}}
\example[0]{surā ca asurā ca garuḍā ca manujā ca bhujagā ca gandhabbā ca surāsuragaruḍamanujabhujagagandhabbā\\{\upshape= Gods, demons, garudas, human beings, serpents, and heavenly musicians [are] \pali{surāsuragaruḍamanujabhujagagandhabba}.}}
\example[0]{pavarā ca te surāsuragaruḍamanujabhujagagandhabbā cāti pavarasurāsuragaruḍamanujabhujagagandhabbā\\{\upshape= [Those are] excellent and they [are] also gods, demons, etc., hence \pali{pavarasurāsuragaruḍamanujabhujagagandhabbā}.}}
\example[0]{pavarasurāsuragaruḍamanujabhujagagandhabbānaṃ ma\-kuṭāni pavarasurāsuragaruḍamanujabhujagagandhabbama\-kuṭāni\\{\upshape= The crowns of excellent gods, demons, etc., [are] \pali{pavarasurāsuragaruḍamanujabhujagagandhabbamakuṭa}.}}
\example[0]{pavarasurāsuragaruḍamanujabhujagagandhabbamakuṭānaṃ kūṭāni pavarasurāsuragaruḍamanujabhujagagandhabbama\-kuṭakūṭāni\\{\upshape= The tops of the crowns of excellent gods, demons, etc., [are] \pali{pavarasurāsuragaruḍamanujabhujagagandhabbamakuṭakūṭa}.}}
\example[0]{pavarasurāsuragaruḍamanujabhujagagandhabbamakuṭakū-\\ṭesu cumbitā pavarasurāsuragaruḍamanujabhujagagandhab\-bamakuṭakūṭacumbitā\\{\upshape= Softly touched on the tops \ldots\ [are] \pali{pavarasurāsuragaruḍamanujabhujagagandhabbamakuṭakūṭacumbita}.\footnote{For \pali{cumbita}, see Apte.}}}
\example[0]{pavarasurāsuragaruḍamanujabhujagagandhabbamakuṭakū-\\ṭacumbitā ca te selā cāti pavarasurāsuragaruḍamanujabhujagagandhabbamakuṭakūṭacumbitaselā\\{\upshape=  [Those were] softly touched on the tops \ldots\ and they are also gems, hence \pali{pavarasurāsuragaruḍamanujabhujagagandhabbamakuṭakūṭacumbitasela}.}}
\example[0]{pavarasurāsuragaruḍamanujabhujagagandhabbamakuṭakū-\\ṭacumbitaselehi saṅghaṭṭitā pavarasurāsuragaruḍamanujabhu-\\jagagandhabbamakuṭakūṭacumbitaselasaṅghaṭṭitā\\{\upshape= Touched by the gems softly touched on the tops \ldots\ [is] \pali{pavara\-surāsuragaruḍamanujabhujagagandhabbamakuṭakūṭacumbi-\\taselasaṅghaṭṭita}.}}
\example{pavarasurāsuragaruḍamanujabhujagagandhabbamakuṭakū\-ṭacumbitaselasaṅghaṭṭitā caraṇā yassa tathāgatassa soyaṃ pa\-varasurāsuragaruḍamanujabhujagagandhabbamakuṭakūṭa-\\cumbitaselasaṅghaṭṭitacaraṇo\\{\upshape= Which the Blessed One [has] the walkings touched by the gems \ldots, that this [the Blessed One is called] \pali{pavarasurāsuraga\-ruḍamanujabhujagagandhabbamakuṭakūṭacumbitaselasaṅ-\\ghaṭṭitacaraṇa}.}}
\transnote{When analyzed as such, the compound is called \pali{tulyādhikara\-ṇabahubbīhi} having \pali{dvanda}, \pali{kammadhāraya}, and \pali{tappurisa} inside.}

\bigbullet{(15) An analysis of \pali{catuddisa}}
\example{catasso disā yassa soyaṃ catuddiso\\{\upshape= Which [person has] four directions, that this [person is called] \pali{catuddisa} (having four directions [=? well-known])}}

\bigbullet{(16) An analysis of \pali{pañcacakkhu}}
\example{pañca cakkhūni yassa tathāgatassa soyaṃ pañcacakkhu\\{\upshape= Which the Blessed One [has] five eyes, that this [the Blessed One is called] \pali{pañcacakkhu}.}}

\bigbullet{(17) An analysis of \pali{dasabala}}
\example{dasa balāni yassa soyaṃ dasabalo\\{\upshape= Which [person has] ten strengths, that this [person is called] \pali{dasabala}.}}

\bigbullet{(18) An analysis of \pali{anantañāṇa}}
\example[0]{nassa anto anantaṃ\\{\upshape= One has no limit, hence \pali{anata} (limitless).}}
\example{anantaṃ ñāṇaṃ yassa tathāgatassa soyaṃ anantañāṇo\\{\upshape= Which the Blessed One [has] limitless insight, that this [the Blessed One is called] \pali{anantañāṇa}.}}

\bigbullet{(19) An analysis of \pali{amitaghanasarīra}}
\example[0]{na mitaṃ amitaṃ\\{\upshape= [A thing was] not measured, hence \pali{amita} (immeasurable).}}
\example[0]{ghanaṃ eva sarīraṃ ghanasarīraṃ\\{\upshape= [It is] solid [and it is also] body, hence \pali{ghanasarīra}.}}
\example{amitaṃ ghanasarīraṃ yassa tathāgatassa soyaṃ amitagha\-nasarīro\\{\upshape= Which the Blessed One [has] immeasurable solid body, that this [the Blessed One is caleld] \pali{amitaghanasarīra}.}}

\bigbullet{(20) An analysis of \pali{amitabalaparakkamapatta}}
\example[0]{na mitā amitā\\{\upshape= immeasurable}}
\example[0]{balañca parakkamo ca balaparakkamā\\{\upshape= strength and effort}}
\example[0]{amitā eva balaparakkamā amitabalaparakkamā\\{\upshape= immeasurable strength and effort}}
\example{amitabalaparakkamā pattā yena soyaṃ amitabalaparakkamapatto\\{\upshape= The immeasurable strength and effort were attained by which [person], that this [person is called] \pali{amitabalaparakkamapatta}.}}
\transnote{When analyzed as such, the compound is called \pali{tulyādhikara\-ṇabahubbīhi} having \pali{kammadhāraya} and \pali{dvanda} inside.}

\bigbullet{(21) An analysis of \pali{mattabhamaragaṇacumbitavikasitapupphavallināgarukkhopasobhitakandara}}
\example[0]{mattā eva bhamarā mattabhamarā\\{\upshape= [They were] intoxicated [and they are] bees, hence \pali{mattabhamara} (intoxicated bee).}}
\example[0]{mattabhamarānaṃ gaṇā mattabhamaragaṇā\\{\upshape= groups of intoxicated bees}}
\example[0]{mattabhamaragaṇehi cumbitāni mattabhamaraṇacumbitāni\\{\upshape= kissed by groups of intoxicated bees}}
\example[0]{vikasitāni eva pupphāni vikasitapupphāni\\{\upshape= blooming flowers}}
\example[0]{mattabhamaragaṇacumbitāni vikasitapupphāni yesaṃ teti mattabhamaragaṇacumbitavikasitapupphā\\{\upshape= Which [plants have] blooming flowers kissed by groups of intoxicated bees, those [plants are called] \pali{mattabhamaragaṇacum\-bitavikasitapuppha}.}}
\example[0]{valli ca nāgarukkho ca mallināgarukkhā\\{\upshape= creeper and iron-wood tree}}
\example[0]{mattabhamaragaṇacumbitavikasitapupphā ca te vallināga\-rukkhā cāti mattabhamaragaṇacumbitavikasitapupphavallināgarukkhā\\{\upshape= creeper and iron-wood tree having blooming flowers \ldots}}
\example[0]{mattabhamaragaṇacumbitavikasitapupphavallināgarukkhe-\\hi
upasobhitāni mattabhamaragaṇacumbitavikasitapupphaval\-lināgarukkhopasobhitāni\\{\upshape= beautiful with creepers and iron-wood trees \ldots}}
\example{mattabhamaragaṇacumbitavikasitapupphavallināgarukkho-\\pasobhitāni kandarāni yassa pabbatarājassa soyaṃ mattabha-\\maragaṇacumbitavikasitapupphavallināgarukkhopasobhita-\\kandaro\\{\upshape= Which grand mountain [has] grottos beautiful with creepers \ldots, that this [grand mountain is called] \pali{mattabhamaragaṇacumbitavikasitapupphavallināgarukkhopasobhitakandara}.}}
\transnote{When analyzed as such, the compound is called \pali{tulyādhikara\-ṇabahubbīhi} having \pali{dvanda}, \pali{kammadhāraya}, and \pali{tappurisa} inside.}

\bigbullet{(22) An analysis of \pali{nānārukkhatiṇapatitapupphopasobhitakandara}}
\example[0]{rukkho ca tiṇañca rukkhatiṇāni\\{\upshape= tree and grass}}
\example[0]{nānā pakārāni eva rukkhatiṇāni nānārukkhatiṇāni\\{\upshape= various kinds of trees and grasses}}
\example[0]{nānārukkhatiṇehi patitāni nānārukkhatiṇapatitāni\\{\upshape= fallen from various kinds of trees and grasses}}
\example[0]{nānārukkhatiṇapatitāni ca tāni pupphāni cāti nānārukkhati\-ṇapatitapupphāni\\{\upshape= flowers fallen from various kinds of trees and grasses}}
\example[0]{nānārukkhatiṇapatitapupphehi upasobhitāni nānārukkhati\-ṇapatitapupphopasobhitāni\\{\upshape= beautiful with flowers fallen from various kinds of trees and grasses}}
\example{nānārukkhatiṇapatitapupphopasobhitāni kandarāni yassa pabbatarājassa soyaṃ nānārukkhatiṇapatitapupphopasobhitakandaro\\{\upshape= Which grand mountain [has] grottos beautiful with flowers \ldots, that this [grand mountain is called] \pali{nānārukkhatiṇapatitapupphopasobhitakandara}.}}
\transnote{When analyzed as such, the compound is called \pali{tulyādhikara\-ṇabahubbīhi} having \pali{dvanda}, \pali{kammadhāraya}, and \pali{tappurisa} inside.}

\bigbullet{(23) An analysis of \pali{nānāmusalaphālapabbatatarukaliṅgara\-saradhanugadāsitomara}}
\example[0]{musalo ca phālo ca pabbato ca taru ca kaliṅgaro ca saro ca dhanu ca gadā ca asi ca tomaro ca musalaphālapabbatataruka\-liṅgarasaradhanugadāsitomarā\\{\upshape= pestle, plowshare, mountain, tree, log, arrow, bow, iron bar, sword, and spear}}
\example[0]{nānā pakārā eva musalaphālapabbatatarukaliṅgarasara-\\dhanugadāsitomarā nānāmusalaphālapabbatatarukaliṅgara-\\saradhanugadāsitomarā\\{\upshape= various kinds of pestle, etc.}}
\example{nānāmusalaphālapabbatatarukaliṅgarasaradhanugadāsito\-marā hatthesu yesaṃ te nānāmusalaphālapabbatatarukaliṅga\-rasaradhanugadāsitomarahatthā\\{\upshape= Which [people have] various kinds of pestle, etc., in hands, they [are called] \pali{nānāmusalaphālapabbatatarukaliṅgarasara\-dhanugadāsitomarahattha}.}}
\transnote{When analyzed as such, the compound is called \pali{tulyādhikara\-ṇabahubbīhi} having \pali{dvanda} and \pali{kammadhāraya} inside.}

\head{329}{329, 357. nāmānaṃ samuccayo dvando.}
\headtrans{The combination of nouns [of one vibhatti is] \pali{dvanda}.}
\sutdef{nāmānaṃ ekavibhattikānaṃ yo samuccayo, so dvandasañño hoti.}
\sutdeftrans{Which combination is of one-vibhatti nouns, that [combination] is called \pali{dvanda}.}
\example[0]{candimā ca sūriyo ca candimasūriyā\\{\upshape= the moon and the sun}}
\example[0]{samaṇo ca brāhmaṇo ca samaṇabrāhmaṇā\\{\upshape= ascetic and brahmin}}
\example[0]{sāriputto ca moggallāno ca sāriputtamoggallānā\\{\upshape= Sāriputta and Moggallāna}}
\example[0]{brāhmaṇo ca gahapatiko ca brāhmaṇagahapatikā\\{\upshape= brahmin and householder}}
\example[0]{yamo ca varuṇo ca yamavaruṇā\\{\upshape= Yama and Varuṇa}}
\example{kuvero ca vāsavo ca kuveravāsavā\\{\upshape= Kuvera and Vāsava}}

\head{330}{330, 340. mahataṃ mahā tulyādhikaraṇe pade.}
\headtrans{Because of terms denoting the same thing, \pali{mahanta} [becomes] \pali{mahā}.}
\sutdef{tesaṃ mahantasaddānaṃ mahāādeso hoti tulyādhikaraṇe pade.}
\sutdeftrans{There is a substitution of \pali{mahā} for those \pali{mahanta} because of terms denoting the same thing.}
\example[0]{mahanto ca so puriso cāti mahāpuriso\\{\upshape= great man}}
\example[0]{mahantī ca sā devī cāti mahādevī\\{\upshape= great queen}}
\example{mahantañca taṃ balañcāti mahābalaṃ\\{\upshape= great strength}}

\head{331}{331, 353. itthiyaṃ bhāsitapumitthī pumāva ce.}
\headtrans{Because of a feminine term, if [it was] mentioned [formerly] as masculine, [it is] masculine.}
\sutdef{itthiyaṃ tulyādhikaraṇe pade ce bhāsitapumitthī pumāva daṭṭhabbā.}
\sutdeftrans{Because of a feminine term denoting the same thing, if [that] feminine term [was] mentioned [formerly] as masculine, [it] should be seen as masculine.}
\example[0]{dīghā jaṅghā yassa soyaṃ dīghajaṅgho\\{\upshape= Which [person has] a long leg, that this [person is called] \pali{dīghajaṅgha} (long-legged).}}
\example[0]{kalyāṇabhariyo {\upshape ([One] having lovely wive)}}
\example{pahūtapañño {\upshape ([One] having many insights)}}
\transnote{This sutta is straightforward. Once a feminine word forming a compound with an adjective or another apposition, then it is used to modify or mention an entity. The final compound is treated as masculine if the modified or mentioned entity is masculine. This is a \pali{bahubbīhi} compound. Hence, for example, \pali{dīghajaṅghā kaññā} (long-legged girl) is a normal case.}

\head{332}{332, 343. kammadhārayasaññe ca.}
\headtrans{Because of appositional compound also, [feminine is seen as masculine].}
\sutdef{kammadhārayasaññe ca samāse itthiyaṃ tulyādhikaraṇe pade pubbe bhāsitapumitthī ce, pumāva daṭṭhabbā.}
\sutdeftrans{When an appositional compound [is] feminine, denoting the same thing, if [that] feminine [term was] mentioned formerly as masculine, [it] should be seen as masculine.}
\example[0]{brāhmaṇadārikā\\{\upshape= a brahmin girl}}
\example{khattiyakaññā\\{\upshape= a warrior girl}}
\transnote{This means, from the examples, \pali{brāhmaṇa} and \pali{khattiya} are retained their masculine form. So, we do not use \pali{brāhmaṇīdārikā} nor \pali{khattiyākaññā}, so to speak.}

\head{333}{333, 344. attaṃ nassa tappurise.}
\headtrans{[There is] \pali{a} for \pali{na} because of tappurisa compound.}
\sutdef{nassa padassa tappurise uttarapade attaṃ hoti.}
\sutdeftrans{There is \pali{a} for the word (particle) \pali{na} because of tappurisa term behind.}
\example[0]{na brāhmaṇo abrāhmaṇo {\upshape (a non-brahmin)}}
\example[0]{avasalo {\upshape (a non-outcaste)}}
\example[0]{abhikkhu {\upshape (a non-monk)}}
\example[0]{apañcavassaṃ {\upshape (a non-five-rain)}}
\example{apañcagavaṃ {\upshape (a non-five-cow)}}
\transnote{This compound may be called \pali{na-tappurisa} after its Sanskrit name. See \hyperref[sut:326]{Kacc 326}.}

\head{334}{334, 345. sare an.}
\headtrans{Because of vowel, [there is] \pali{an}-substitution.}
\sutdef{nassa padassa tappurise anaādeso hoti sare pare.}
\sutdeftrans{There is a substitution of \pali{ana} for the word (particle) \pali{na} in tappurisa compound because of a vowel behind.}
\example[0]{na asso anasso {\upshape (a non-horse)}}
\example[0]{anissaro {\upshape (a non-ruler)}}
\example{anariyo {\upshape (a non-noble one)}}

\head{335}{335, 346. kada kussa.}
\headtrans{[There is] \pali{kada}-substitution for \pali{ku} [because of a vowel behind].}
\sutdef{kuiccetassa kada hoti sare pare.}
\sutdeftrans{There is [a substitution of] \pali{kada} for \pali{ku} because of a vowel behind.}
\example[0]{kucchitaṃ annaṃ kadannaṃ\\{\upshape= Disgusting boiled rice [is] \pali{kadanna}.}}
\example{kucchitaṃ asanaṃ kadassanaṃ\\{\upshape= Disgusting food [is] \pali{kadassana}.}}
\transnote{In Rūpa 346, this sutta is \pali{kad kussa}. In a Sinhala source, it is \pali{kadaṃ kussa}. I follow a Thai source here.}

\head{336}{336, 347. kā’ppatthesu ca.}
\headtrans{Also \pali{kā}-substitution [for \pali{ku}] in the senses of smallness.}
\sutdef{kuiccetassa kā hoti appatthesu ca.}
\sutdeftrans{There is [a substitution of] \pali{kā} for \pali{ku} because of the senses of smallness.}
\example[0]{kālavaṇaṃ {\upshape (little salt)}}
\example{kāpupphaṃ {\upshape (a small flower)}}

\head{337}{337, 350. kvaci samāsantagatānamakāranto.}
\headtrans{In some places, the ending of noun at the end of the compound [becomes] \pali{a}.}
\sutdef{samāsantagatānaṃ nāmānamanto saro kvaci akāro hoti.}
\sutdeftrans{The ending vowel of noun at the end of the compound becomes \pali{a} in some places.}
\example[0]{devānaṃ rājā devarājo/devarājā {\upshape (the king of gods)}}
\example[0]{devānaṃ sakhā devasakho/devasakhā {\upshape (the friend of gods)}}
\example[0]{pañca ahāni pañcāhaṃ {\upshape (the five-day)}}
\example[0]{pañcagavaṃ {\upshape (the five-cow)}}
\example[0]{chattupāhanaṃ {\upshape (the umbrella-and-sandal)}}
\example[0]{upasaradaṃ {\upshape (the time near autumn)}}
\example[0]{visālakkho {\upshape (the wide-eyed one)}}
\example{vimukho {\upshape (the one with deformed face)}}

\head{338}{338, 356. nadimhā ca.}
\headtrans{After \pali{nadī} also [there is \pali{ka}-paccaya at the end].}
\sutdef{nadimhā ca kapaccayo hoti samāsante.}
\sutdeftrans{There is also \pali{ka}-paccaya after \pali{nadī} at the end of the compound.}
\example{bahū nadiyo yasmiṃ soyanti bahunadiko\\{\upshape= There are many rivers in which place, that this [place is called] \pali{bahunadika} (having many rivers).}}
\transnote{This rule can be applied to other similar words as well, for example, \pali{bahukantiko} (one having many pleasures), \pali{bahunāriko} (one having many women).}

\head{339}{339, 358. jāyāya tudaṃjāni patimhi.}
\headtrans{[There are] \pali{tudaṃ}- and \pali{jāni}-substitution for \pali{jāyā} because of \pali{pati}.}
\sutdef{jāyāiccetāya tudaṃjāniiccete ādesā honti patimhi pare.}
\sutdeftrans{There are substitutions of \pali{tudaṃ} and \pali{jāni} for \pali{jāyā} because of \pali{pati} behind.}
\example[0]{tudaṃpatī {\upshape (the wive-and-husband)}}
\example{jānipatī {\upshape (the wive-and-husband)}}

\head{340}{340, 355. dhanumhā ca.}
\headtrans{After \pali{dhanu} also, [there is \pali{ā}-paccaya].}
\sutdef{dhanumhā ca āpaccayo hoti samāsante.}
\sutdeftrans{There is \pali{ā}-paccaya also after \pali{dhanu} at the end of the compound.}
\example{gāṇḍīvadhanvā {\upshape (Arjuna's bow)}}
\transnote{This sutta came from Pāṇ 5.4.132, and the example came from Kāśikā on this sūtra. All these seem to have nothing to do with Buddhist texts.}

\head{341}{341, 336. aṃ vibhattīnamakārantā abyayībhāvā.}
\headtrans{[There is] \pali{aṃ}-substitution for vibhattis having \pali{a}-ending after abyayībhāva compound.}
\sutdef{tasmā akārantā abyayībhāvasamāsā parāsaṃ vibhattīnaṃ kvaci aṃ hoti.}
\sutdeftrans{After that \pali{a}-ending of an abyayībhāva compound, there is \pali{aṃ}-substitution in some places for those vibhattis behind.}
\transnote{See examples in \hyperref[sut:319]{Kacc 319}. This sutta looks similar to Kāt 2.207. We can understand this simply in this way. When using an adverbial compound ending with \pali{a}, we make it end with \pali{aṃ}.}

\head{342}{342, 337. saro rasso napuṃsake.}
\headtrans{The vowel [of an adverbial compound] becomes shortened because of neuter.}
\sutdef{napuṃsake vattamānassa abyayībhāvasamāsassa liṅgassa saro rasso hoti.}
\sutdeftrans{The vowel of an adverbial compound occuring in neuter becomes shortened.}
\example[0]{kumārīsu adhikicca pavattati kathā iti adhikumāri\\{\upshape= A speech occurs concerning girls, hence \pali{adhikumāri} (girl-concerning [speech]).}}
\example{upavadhu {\upshape (near-to-girl [speech])}}
\transnote{This sutta came from Kāt 2.258. As stated in \hyperref[sut:320]{Kacc 320} that adverbial compounds are neuter, hence they normally end with a short vowel. In \pali{a}-ending terms, we can use it by \pali{aṃ}-substitution (\hyperref[sut:341]{Kacc 341}). How to put \pali{i}- and \pali{u}-ending terms in a sentence is still in question.}

\head{343}{343, 338. aññasmā lopo ca.}
\headtrans{After other [non-\pali{a}-ending adverbial compound, there is] an elision [of vibhattis] also.}
\sutdef{aññasmā abyayībhāvasamāsā anakārantā parāsaṃ vibhattīnaṃ lopo ca hoti.}
\sutdeftrans{There is also an elision of vibhattis behind other non-\pali{a}-ending adverbial compound.}
\example{adhitthi/adhikumāri {\upshape (girl-concerning)}}
\transnote{By this rule, it seems to mean that when this kind of words is used, it goes like an indeclinable one. I can not find such a use in the main Pāli texts.}

\setcounter{table}{1}
\begin{longtable}{%
		>{\raggedright\arraybackslash}p{0.3\linewidth}%
		>{\raggedright\arraybackslash}p{0.4\linewidth}}
\caption{Index of compounds}\label{tab:samasaindex}\\
\toprule
\bfseries Item & \bfseries Suttas \\ \midrule
\endfirsthead
\multicolumn{2}{c}{\tablename\ \thetable: Index of compounds (contd\ldots)}\\
\toprule
\bfseries Item & \bfseries Suttas \\ \midrule
\endhead
\bottomrule
\ltblcontinuedbreak{2}
\endfoot
\bottomrule
\endlastfoot
%
basic ideas & \hyperref[sut:316]{316}, \hyperref[sut:317]{317}, \hyperref[sut:318]{318}, \hyperref[sut:337]{337} \\
dvanda & \hyperref[sut:322]{322}, \hyperref[sut:323]{323}, \hyperref[sut:329]{329}, \hyperref[sut:339]{339} \\
tappurisa & \hyperref[sut:326]{326}, \hyperref[sut:327]{327} \\
kammadhāraya & \hyperref[sut:324]{324}, \hyperref[sut:330]{330}, \hyperref[sut:331]{331}, \hyperref[sut:332]{332}, \hyperref[sut:335]{335}, \hyperref[sut:336]{336}, \hyperref[sut:338]{338}, \hyperref[sut:340]{340} \\
digu & \hyperref[sut:321]{321}, \hyperref[sut:325]{325} \\
na-tappurisa & (\hyperref[sut:326]{326}), \hyperref[sut:333]{333}, \hyperref[sut:334]{334} \\
bahubbīhi & \hyperref[sut:328]{328} \\
abyayībhāva & \hyperref[sut:319]{319}, \hyperref[sut:320]{320}, \hyperref[sut:341]{341}, \hyperref[sut:342]{342}, \hyperref[sut:343]{343} \\
mahā- & \hyperref[sut:330]{330} \\
kada- & \hyperref[sut:335]{335} \\
kā- & \hyperref[sut:336]{336} \\
-ka (nadī) & \hyperref[sut:338]{338} \\
tudaṃ/jāni & \hyperref[sut:339]{339} \\
dhanu & \hyperref[sut:340]{340} \\
\end{longtable}

