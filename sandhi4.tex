\section{Catutthakaṇḍa}

\head{30}{30, 58. aṃ byañjane niggahitaṃ.}
\headtrans{Niggahita [becomes] \pali{aṃ} because of [the other] consonant.}
\sutdef{niggahitaṃ kho byañjane pare aṃ iti hoti.}
\sutdeftrans{Niggahita becomes \pali{aṃ} because of the other consonant.}
\example[0]{evaṃ vutte}
\example{taṃ sādhūti paṭissuṇitvā}
\transnote{The original form of the examples could be \pali{evaṃvutte} and \pali{taṃsādhūti}. This is a form of \pali{pakatisandhi} (retaining the original).}

\head{31}{31, 49. vaggantaṃ vā vagge.}
\headtrans{[Niggahita becomes] the last letter of the group sometimes because of the group.}
\sutdef{vaggabhūte byañjane pare niggahitaṃ kho vaggantaṃ vā pappoti.}
\sutdeftrans{Because of the other consonant in groups, niggahita becomes the last letter of the group sometimes.}
\example[0]{tanniccutaṃ = taṃ + niccutaṃ}
\example[0]{dhammañcare = dhammaṃ + care}
\example[0]{santantassa = santaṃ + tassa}
\example[0]{taṅkāruṇikaṃ = taṃ + kāruṇikaṃ}
\example{evaṅkho = evaṃ + kho}

\head{32}{32, 50. ehe ñaṃ.}
\headtrans{Because of \pali{e} or \pali{ha}, [niggahita becomes] \pali{ña}.}
\sutdef{ekārahakāre pare niggahitaṃ kho ñakāraṃ pappoti vā.}
\sutdeftrans{Because of \pali{e} or \pali{ha} in the other [term], niggahita becomes \pali{ña} in some places}
\example[0]{paccattaññeva = paccattaṃ + eva}
\example[0]{taññevettha = taṃ + eva + ettha}
\example[0]{evañhi = evaṃ + hi}
\example{tañhi = taṃ + hi}

\head{33}{33, 50. sa ye ca.}
\headtrans{Because of together with \pali{ya} also.}
\sutdef{niggahitaṃ kho yakāre pare saha yakārena ñakāraṃ pappoti vā.}
\sutdeftrans{Niggahita together with \pali{ya} becomes \pali{ña} sometimes because of the rear \pali{ya}.}
\example[0]{saññogo = saṃ + yogo}
\example{saññuttaṃ = saṃ + yuttaṃ}

\head{34}{34, 52. madā sare.}
\headtrans{Substitution of \pali{ma} and \pali{da} because of [the other] vowel.}
\sutdef{niggahitassa kho sare pare makāradakārādesā honti vā.}
\sutdeftrans{There is a substitution of \pali{ma} and \pali{da} for niggahita sometimes because of the other vowel.}
\example[0]{tamahaṃ = taṃ + ahaṃ}
\example{etadavoca = etaṃ + avoca}

\head{35}{35, 34. ya va ma da na ta ra lā cāgamā.}
\headtrans{[There are] insertions of \pali{ya}, \pali{va}, \pali{ma}, \pali{da}, \pali{na}, \pali{ta}, \pali{ra}, and \pali{la} also.}
\sutdef{sare pare yakāro vakāro makāro dakāro nakāro takāro rakāro lakāro ime āgamā honti vā.}
\sutdeftrans{There are insertions of \pali{ya}, \pali{va}, \pali{ma}, \pali{da}, \pali{na}, \pali{ta}, \pali{ra}, and \pali{la} sometimes because of the other vowel.}
\example[0]{nayimassa = na + imassa}
\example[0]{yathayidaṃ = yathā + idaṃ}
\example[0]{vudikkhati = vudikkhati}
\example[0]{lahumessati = lahu + essati}
\example[0]{garumessati = garu + essati}
\example[0]{kasāmiva = kasā + iva}
\example[0]{sammadaññā = samma + aññā}
\example[0]{manasādaññā = manasā + aññā}
\example[0]{attadatthamabhiññāya = atta + attha + abhiññāya}
\example[0]{ciraṃnāyati = ciraṃ + āyati}
\example[0]{itonāyati = ito + āyati}
\example[0]{yasmātiha = yasmā + iha}
\example[0]{ajjatagge = ajja + agge}
\example[0]{sabbhireva = sabbhi + eva}
\example[0]{āraggeriva = āragge + iva}
\example[0]{sāsaporiva = sāsapo + iva}
\example[0]{chaḷabhiññā = cha + abhiññā}
\example{saḷāyatanaṃ = cha + āyatanaṃ}
\transnote{In the formula, precisely it should be \pali{yavamadamataraḷā cāgamā}. For \pali{saḷāyatanaṃ}, see also Kacc 374.}

\head{36}{36, 47. kvaci o byañjane.}
\headtrans{In some places, [there is] an \pali{o}-insertion because of [the other] consonant.}
\sutdef{byañjane pare kvaci okārāgamo hoti.}
\sutdeftrans{There is an insertion of \pali{o} in some places because of the other consonant.}
\example{parosahassaṃ = para + sahassaṃ}

\head{37}{37, 57. niggahitañca.}
\headtrans{Niggahita-insertion also.}
\sutdef{niggahitañcāgamo hoti sare vā byañjane vā pare kvaci.}
\sutdeftrans{There is an insertion of niggahita in some places because of the other vowel or consonant.}
\example[0]{cakkhuṃudapādi = cakkhu + udapādi}
\example[0]{avaṃsiro = ava + siro}
\example[0]{yāvañcidha = yāva + ca + idha}
\example[0]{aṇuṃthūlāni = aṇu + thūlāni}
\example{manopubbaṅgamā = manopubba + gamā}

\head{38}{38, 53. kvaci lopaṃ.}
\headtrans{In some places, [niggahita gets] elided.}
\sutdef{niggahitaṃ kho sare pare kvaci lopaṃ pappoti.}
\sutdeftrans{Niggahita gets elided in some places because of the other vowel.}
\example[0]{tāsāhaṃ = tāsaṃ + ahaṃ}
\example{vidūnaggamiti = vidūnaṃ + aggamiti}

\head{39}{39, 54. byañjane ca.}
\headtrans{[Niggahita gets elided] because of consonant also.}
\sutdef{niggahitaṃ kho byañjane pare kvaci lopaṃ pappoti.}
\sutdeftrans{Niggahita gets elided in some places because of the other consonant.}
\example[0]{ariyasaccānadassanaṃ = ariyasaccānaṃ + adassanaṃ}
\example{buddhānasāsanaṃ = buddhānaṃ + sāsanaṃ}

\head{40}{40, 55. paro vā saro.}
\headtrans{The other vowel [gets elided] sometimes.}
\sutdef{niggahitamhā paro saro lopaṃ pappoti vā.}
\sutdeftrans{The other vowel behind niggahita gets elided sometimes.}
\example[0]{abhinandunti = abhinanduṃ + iti}
\example{uttattaṃva = uttattaṃ + eva}

\head{41}{41, 56. byañjano ca visaññogo.}
\headtrans{[A double] consonant becomes undoubled also.}
\sutdef{niggahitamhā parasmiṃ sare lutte yadi byañjano sasaññogo visaññogo hoti.}
\sutdeftrans{When the other vowel gets elided, if there is a double-consonant, [make it] undoubled.}
\example[0]{evaṃsa = evaṃ + assa}
\example{pupphaṃsā = pupphaṃ + assā}

