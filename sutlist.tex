\markboth{}{Kaccāyana's suttas}
\cleardoublepage
\phantomsection
\addcontentsline{toc}{chapter}{Kaccāyana's suttas}
\chapter*{Kaccāyana's suttas}

1, 1. attho akkharasaññāto.\hfill \pageref{sut:1}\par \noindent
2, 2. akkharāpādayo ekacattālīsaṃ.\hfill \pageref{sut:2}\par \noindent
3, 3. tatthodantā sarā aṭṭha.\hfill \pageref{sut:3}\par \noindent
4, 4. lahumattā tayo rassā.\hfill \pageref{sut:4}\par \noindent
5, 5. aññe dīghā.\hfill \pageref{sut:5}\par \noindent
6, 8. sesā byañjanā.\hfill \pageref{sut:6}\par \noindent
7, 9. vaggā pañcapañcaso mantā.\hfill \pageref{sut:7}\par \noindent
8, 10. aṃ iti niggahitaṃ.\hfill \pageref{sut:8}\par \noindent
9, 11. parasamaññā payoge.\hfill \pageref{sut:9}\par \noindent
10, 12. pubbamadhoṭhita massaraṃ sarena viyojaye.\hfill \pageref{sut:10}\par \noindent
11, 14. naye paraṃ yutte.\hfill \pageref{sut:11}\par \noindent
12, 13. sarā sare lopaṃ.\hfill \pageref{sut:12}\par \noindent
13, 15. vā paro asarūpā.\hfill \pageref{sut:13}\par \noindent
14, 16. kvacāsavaṇṇaṃ lutte.\hfill \pageref{sut:14}\par \noindent
15, 17. dīghaṃ.\hfill \pageref{sut:15}\par \noindent
16, 18. pubbo ca.\hfill \pageref{sut:16}\par \noindent
17, 19. yamedantassādeso.\hfill \pageref{sut:17}\par \noindent
18, 20. vamodudantānaṃ.\hfill \pageref{sut:18}\par \noindent
19, 22. sabbo caṃ ti.\hfill \pageref{sut:19}\par \noindent
20, 27. do dhassa ca.\hfill \pageref{sut:20}\par \noindent
21, 21. ivaṇṇo yaṃ navā.\hfill \pageref{sut:21}\par \noindent
22, 28. evādissa ri pubbo ca rasso.\hfill \pageref{sut:22}\par \noindent
23, 26. sarā pakati byañjane.\hfill \pageref{sut:23}\par \noindent
24, 35. sare kvaci.\hfill \pageref{sut:24}\par \noindent
25, 37. dīghaṃ.\hfill \pageref{sut:25}\par \noindent
26, 38. rassaṃ.\hfill \pageref{sut:26}\par \noindent
27, 39. lopañca tatrākāro.\hfill \pageref{sut:27}\par \noindent
28, 40. para dvebhāvo ṭhāne.\hfill \pageref{sut:28}\par \noindent
29, 42. vagge ghosāghosānaṃ tatiyapaṭhamā.\hfill \pageref{sut:29}\par \noindent
30, 58. aṃ byañjane niggahitaṃ.\hfill \pageref{sut:30}\par \noindent
31, 49. vaggantaṃ vā vagge.\hfill \pageref{sut:31}\par \noindent
32, 50. ehe ñaṃ.\hfill \pageref{sut:32}\par \noindent
33, 50. sa ye ca.\hfill \pageref{sut:33}\par \noindent
34, 52. madā sare.\hfill \pageref{sut:34}\par \noindent
35, 34. ya\,va\,ma\,da\,na\,ta\,ra\,lā cāgamā.\hfill \pageref{sut:35}\par \noindent
36, 47. kvaci o byañjane.\hfill \pageref{sut:36}\par \noindent
37, 57. niggahitañca.\hfill \pageref{sut:37}\par \noindent
38, 53. kvaci lopaṃ.\hfill \pageref{sut:38}\par \noindent
39, 54. byañjane ca.\hfill \pageref{sut:39}\par \noindent
40, 55. paro vā saro.\hfill \pageref{sut:40}\par \noindent
41, 56. byañjano ca visaññogo.\hfill \pageref{sut:41}\par \noindent
42, 32. go sare puthassāgamo kvaci.\hfill \pageref{sut:42}\par \noindent
43, 33. pāssa canto rasso.\hfill \pageref{sut:43}\par \noindent
44, 24. abbho abhi.\hfill \pageref{sut:44}\par \noindent
45, 25. ajjho adhi.\hfill \pageref{sut:45}\par \noindent
46, 26. te na vā ivaṇṇe.\hfill \pageref{sut:46}\par \noindent
47, 23. atissa cantassa.\hfill \pageref{sut:47}\par \noindent
48, 43. kvaci paṭi patissa.\hfill \pageref{sut:48}\par \noindent
49, 44. puthassu byañjane.\hfill \pageref{sut:49}\par \noindent
50, 45. o avassa.\hfill \pageref{sut:50}\par \noindent
51, 59. anupadiṭṭhānaṃ vuttayogato.\hfill \pageref{sut:51}\par \noindent
52, 60. jinavacanayuttaṃ hi.\hfill \pageref{sut:52}\par \noindent
53, 61. liṅgañca nippajjate.\hfill \pageref{sut:53}\par \noindent
54, 62. tato ca vibhattiyo.\hfill \pageref{sut:54}\par \noindent
55, 63. si yo, aṃ yo, nā hi, sa naṃ, smā hi, sa naṃ, smiṃ su.\par \noindent
\hfill \pageref{sut:55}\par \noindent
56, 64. tadanuparodhena.\hfill \pageref{sut:56}\par \noindent
57, 71. ālapane si ga sañño.\hfill \pageref{sut:57}\par \noindent
58, 29. ivaṇṇuvaṇṇā jhalā.\hfill \pageref{sut:58}\par \noindent
59, 182. te itthikhyā po.\hfill \pageref{sut:59}\par \noindent
60, 177. ā gho.\hfill \pageref{sut:60}\par \noindent
61, 86. sāgamo se.\hfill \pageref{sut:61}\par \noindent
62, 206. saṃsāsvekavacanesu ca.\hfill \pageref{sut:62}\par \noindent
63, 217. etimāsami.\hfill \pageref{sut:63}\par \noindent
64, 216. tassā vā.\hfill \pageref{sut:64}\par \noindent
65, 215. tato sassa ssāya.\hfill \pageref{sut:65}\par \noindent
66, 205. gho rassaṃ.\hfill \pageref{sut:66}\par \noindent
67, 229. no ca dvādito naṃmhi.\hfill \pageref{sut:67}\par \noindent
68, 184. amā pato smiṃsmānaṃ vā.\hfill \pageref{sut:68}\par \noindent
69, 186. ādito o ca.\hfill \pageref{sut:69}\par \noindent
70, 30. jhalānamiyuvā sare vā.\hfill \pageref{sut:70}\par \noindent
71, 505. yavakārā ca.\hfill \pageref{sut:71}\par \noindent
72, 185. pasaññassa ca.\hfill \pageref{sut:72}\par \noindent
73, 174. gāva se.\hfill \pageref{sut:73}\par \noindent
74, 169. yosu ca.\hfill \pageref{sut:74}\par \noindent
75, 170. avamhi ca.\hfill \pageref{sut:75}\par \noindent
76, 171. āvassu vā.\hfill \pageref{sut:76}\par \noindent
77, 175. tato namaṃ patimhālutte ca samāse.\hfill \pageref{sut:77}\par \noindent
78, 3. o sare ca.\hfill \pageref{sut:78}\par \noindent
79, 46. tabbiparītūpapade byañjane ca.\hfill \pageref{sut:79}\par \noindent
80, 173. goṇa naṃmhi vā.\hfill \pageref{sut:80}\par \noindent
81, 172. suhināsu ca.\hfill \pageref{sut:81}\par \noindent
82, 149. aṃmo niggahitaṃ jhalapehi.\hfill \pageref{sut:82}\par \noindent
83, 67. saralopo’mādesapaccayādimhi saralope tu pakati.\hfill \pageref{sut:83}\par \noindent
84, 144. agho rassamekavacanayosvapi ca.\hfill \pageref{sut:84}\par \noindent
85, 150. na sismimanapuṃsakāni.\hfill \pageref{sut:85}\par \noindent
86, 227. ubhādito naminnaṃ.\hfill \pageref{sut:86}\par \noindent
87, 231. iṇṇamiṇṇannaṃ tīhi saṅkhyāhi.\hfill \pageref{sut:87}\par \noindent
88, 147. yosu katanikāralopesu dīghaṃ.\hfill \pageref{sut:88}\par \noindent
89, 87. sunaṃhisu ca.\hfill \pageref{sut:89}\par \noindent
90, 252. pañcādīnamattaṃ.\hfill \pageref{sut:90}\par \noindent
91, 194. patissinīmhi.\hfill \pageref{sut:91}\par \noindent
92, 100. ntussanto yosu ca.\hfill \pageref{sut:92}\par \noindent
93, 106. sabbassa vā aṃsesu.\hfill \pageref{sut:93}\par \noindent
94, 105. simhi vā.\hfill \pageref{sut:94}\par \noindent
95, 145. aggissini.\hfill \pageref{sut:95}\par \noindent
96, 148. yosvakatarasso jho.\hfill \pageref{sut:96}\par \noindent
97, 156. vevosu lo ca.\hfill \pageref{sut:97}\par \noindent
98, 186. mātulādīnamānattamīkāre.\hfill \pageref{sut:98}\par \noindent
99, 81. smāhismiṃnaṃ mhābhimhi vā.\hfill \pageref{sut:99}\par \noindent
100, 214. na timehi katākārehi.\hfill \pageref{sut:100}\par \noindent
101, 80. suhisvakāro e.\hfill \pageref{sut:101}\par \noindent
102, 202. sabbanāmānaṃ naṃmhi ca.\hfill \pageref{sut:102}\par \noindent
103, 79. ato nena.\hfill \pageref{sut:103}\par \noindent
104, 66. so.\hfill \pageref{sut:104}\par \noindent
105. so vā.\hfill \pageref{sut:105}\par \noindent
106, 313. dīghorehi.\hfill \pageref{sut:106}\par \noindent
107, 69. sabbayonīnamāe.\hfill \pageref{sut:107}\par \noindent
108, 90. smāsmiṃnaṃ vā.\hfill \pageref{sut:108}\par \noindent
109, 304. āya catutthekavacanassa tu.\hfill \pageref{sut:109}\par \noindent
110, 201. tayo neva ca sabbanāmehi.\hfill \pageref{sut:110}\par \noindent
111, 179. ghato nādīnaṃ.\hfill \pageref{sut:111}\par \noindent
112, 183. pato yā.\hfill \pageref{sut:112}\par \noindent
113, 132. sakhato gasse vā.\hfill \pageref{sut:113}\par \noindent
114, 178. ghate ca.\hfill \pageref{sut:114}\par \noindent
115, 181. na ammādito.\hfill \pageref{sut:115}\par \noindent
116, 157. akatarassā lato yvālapanassa vevo.\hfill \pageref{sut:116}\par \noindent
117, 124. jhalato sassa no vā.\hfill \pageref{sut:117}\par \noindent
118, 146. ghapato ca yonaṃ lopo.\hfill \pageref{sut:118}\par \noindent
119, 155. lato vokāro ca.\hfill \pageref{sut:119}\par \noindent
120, 243. amhassa mamaṃ savibhattissa se.\hfill \pageref{sut:120}\par \noindent
121, 233. mayaṃ yomhi paṭhame.\hfill \pageref{sut:121}\par \noindent
122, 99. ntussa nto.\hfill \pageref{sut:122}\par \noindent
123, 103. ntassa se vā.\hfill \pageref{sut:123}\par \noindent
124, 98. ā simhi.\hfill \pageref{sut:124}\par \noindent
125, 198. aṃ napuṃsake.\hfill \pageref{sut:125}\par \noindent
126, 101. avaṇṇā ca ge.\hfill \pageref{sut:126}\par \noindent
127, 102. totitā sasmiṃnāsu.\hfill \pageref{sut:127}\par \noindent
128, 104. naṃmhi taṃ vā.\hfill \pageref{sut:128}\par \noindent
129, 222. imassidamaṃsisu napuṃsake.\hfill \pageref{sut:129}\par \noindent
138, 225. amussāduṃ.\hfill \pageref{sut:138}\par \noindent
131. itthipumanapuṃsakasaṅkhyaṃ.\hfill \pageref{sut:131}\par \noindent
132, 228. yosu dvinnaṃ dve ca.\hfill \pageref{sut:132}\par \noindent
133, 230. ticatunnaṃ tisso catasso tayo cattāro tīṇi cattāri.\hfill \pageref{sut:133}\par \noindent
134, 251. pañcādīnamakāro.\hfill \pageref{sut:134}\par \noindent
135, 118. rājassa rañño rājino se.\hfill \pageref{sut:135}\par \noindent
136, 119. raññaṃ naṃmhi vā.\hfill \pageref{sut:136}\par \noindent
137, 116. nāmhi raññā vā.\hfill \pageref{sut:137}\par \noindent
138, 121. smiṃmhi raññe rājini.\hfill \pageref{sut:138}\par \noindent
139, 245. tumhākaṃ tayimayi.\hfill \pageref{sut:139}\par \noindent
140, 232. tvamahaṃ simhi ca.\hfill \pageref{sut:140}\par \noindent
141, 241. tavamama se.\hfill \pageref{sut:141}\par \noindent
142, 242. tuyhaṃmayhañca.\hfill \pageref{sut:142}\par \noindent
143, 235. taṃmamaṃmhi.\hfill \pageref{sut:143}\par \noindent
144, 234. tavaṃmamañca navā.\hfill \pageref{sut:144}\par \noindent
145, 238. nāmhi tayāmayā.\hfill \pageref{sut:145}\par \noindent
146, 236. tumhassa tuvaṃtvamaṃmhi.\hfill \pageref{sut:146}\par \noindent
147, 246. padato dutiyācatutthīchaṭṭhīsu vono.\hfill \pageref{sut:147}\par \noindent
148, 247. temekavacanesu ca.\hfill \pageref{sut:148}\par \noindent
149, 148. na aṃmhi.\hfill \pageref{sut:149}\par \noindent
150, 249. vā tatiye ca.\hfill \pageref{sut:150}\par \noindent
151, 250. bahuvacanesu vono.\hfill \pageref{sut:151}\par \noindent
152, 236. pumantassā simhi.\hfill \pageref{sut:152}\par \noindent
153, 138. amālapanekavacane.\hfill \pageref{sut:153}\par \noindent
154. samāse ca vibhāsā.\hfill \pageref{sut:154}\par \noindent
155, 137. yosvāno.\hfill \pageref{sut:155}\par \noindent
156, 142. āne smiṃmhi vā.\hfill \pageref{sut:156}\par \noindent
157, 140. hivibhattimhi ca.\hfill \pageref{sut:157}\par \noindent
158, 143. susmimā vā.\hfill \pageref{sut:158}\par \noindent
159, 139. u nāmhi ca.\hfill \pageref{sut:159}\par \noindent
160, 167. a kammantassa ca.\hfill \pageref{sut:160}\par \noindent
161, 244. tumha’mhehi namākaṃ.\hfill \pageref{sut:161}\par \noindent
162, 237. vā yvappaṭhamo.\hfill \pageref{sut:162}\par \noindent
163, 240. sassaṃ.\hfill \pageref{sut:163}\par \noindent
164, 200. sabbanāma’kārate paṭhamo.\hfill \pageref{sut:164}\par \noindent
165, 208. dvandaṭṭhā vā.\hfill \pageref{sut:165}\par \noindent
166, 209. nāññaṃ sabbanāmikaṃ.\hfill \pageref{sut:166}\par \noindent
167, 210. bahubbīhimhi ca.\hfill \pageref{sut:167}\par \noindent
168, 203. sabbato naṃ saṃsānaṃ.\hfill \pageref{sut:168}\par \noindent
169, 117. rājassa rāju sunaṃhisu ca.\hfill \pageref{sut:169}\par \noindent
170, 220. sabbassimasse vā.\hfill \pageref{sut:170}\par \noindent
171, 219. animi nāmhi ca.\hfill \pageref{sut:171}\par \noindent
172, 218. anapuṃsakassāyaṃ simhi.\hfill \pageref{sut:172}\par \noindent
173, 223. amussa mo saṃ.\hfill \pageref{sut:173}\par \noindent
174, 211. etatesaṃ to.\hfill \pageref{sut:174}\par \noindent
175, 212. tassa vā nattaṃ sabbattha.\hfill \pageref{sut:175}\par \noindent
176, 213. sasmāsmiṃsaṃsāsvattaṃ.\hfill \pageref{sut:176}\par \noindent
177, 221. imasaddassa ca.\hfill \pageref{sut:177}\par \noindent
178, 22. sabbato ko.\hfill \pageref{sut:178}\par \noindent
179, 204. ghapato smiṃsānaṃ saṃsā.\hfill \pageref{sut:179}\par \noindent
180, 207. netāhi smimāyayā.\hfill \pageref{sut:180}\par \noindent
181, 95. manogaṇādito smiṃnānamiā.\hfill \pageref{sut:181}\par \noindent
182, 97. sassa co.\hfill \pageref{sut:182}\par \noindent
183, 48. etesamo lope.\hfill \pageref{sut:183}\par \noindent
184, 96. sa sare vāgamo.\hfill \pageref{sut:184}\par \noindent
185, 112. santasaddassa so bhe bo cante.\hfill \pageref{sut:185}\par \noindent
186, 107. simhi gacchantādīnaṃ ntasaddo aṃ.\hfill \pageref{sut:186}\par \noindent
187, 108. sesesu ntuva.\hfill \pageref{sut:187}\par \noindent
188, 115. brahmattasakharājādito amānaṃ.\hfill \pageref{sut:188}\par \noindent
189, 113. syā ca.\hfill \pageref{sut:189}\par \noindent
190, 114. yonamāno.\hfill \pageref{sut:190}\par \noindent
191, 130. sakhato cāyono.\hfill \pageref{sut:191}\par \noindent
192, 135. smime.\hfill \pageref{sut:192}\par \noindent
193, 122. brahmato gassa ca.\hfill \pageref{sut:193}\par \noindent
194, 131. sakhantassi nonānaṃsesu.\hfill \pageref{sut:194}\par \noindent
195, 134. āro himhi vā.\hfill \pageref{sut:195}\par \noindent
196, 133. sunamaṃsu vā.\hfill \pageref{sut:196}\par \noindent
197, 125. brahmato tu smiṃ ni.\hfill \pageref{sut:197}\par \noindent
198, 123. uttaṃ sanāsu.\hfill \pageref{sut:198}\par \noindent
199, 158. satthupitādīnamā sismiṃ silopo ca.\hfill \pageref{sut:199}\par \noindent
200, 159. aññesvārattaṃ.\hfill \pageref{sut:200}\par \noindent
201, 163. vā naṃmhi.\hfill \pageref{sut:201}\par \noindent
202, 164. satthunattañca.\hfill \pageref{sut:202}\par \noindent
203, 162. u sasmiṃ salopo ca.\hfill \pageref{sut:203}\par \noindent
204, 167. sakkamandhātādīnañca.\hfill \pageref{sut:204}\par \noindent
205, 160. tato yonamo tu.\hfill \pageref{sut:205}\par \noindent
206, 165. tato smimi.\hfill \pageref{sut:206}\par \noindent
207, 161. nā ā.\hfill \pageref{sut:207}\par \noindent
208, 166. āro rassamikāre.\hfill \pageref{sut:208}\par \noindent
209, 168. pitādīnamasimhi.\hfill \pageref{sut:209}\par \noindent
210, 239. tayātayīnaṃ takāro tvattaṃ vā.\hfill \pageref{sut:210}\par \noindent
211, 126. attanto hismimanattaṃ.\hfill \pageref{sut:211}\par \noindent
212, 329. tato smiṃ ni.\hfill \pageref{sut:212}\par \noindent
213, 127. sassa no.\hfill \pageref{sut:213}\par \noindent
214, 128. smā nā.\hfill \pageref{sut:214}\par \noindent
215, 141. jhalato ca.\hfill \pageref{sut:215}\par \noindent
216, 180. ghapato smiṃ yaṃ vā.\hfill \pageref{sut:216}\par \noindent
217, 199. yonaṃ ni napuṃsakehi.\hfill \pageref{sut:217}\par \noindent
218, 196. ato niccaṃ.\hfill \pageref{sut:218}\par \noindent
219, 196. siṃ.\hfill \pageref{sut:219}\par \noindent
220, 74. sesato lopaṃ gasipi.\hfill \pageref{sut:220}\par \noindent
221, 282. sabbāsamāvusopasagganipātādīhi ca.\hfill \pageref{sut:221}\par \noindent
222, 342. pumassa liṅgādīsu samāsesu.\hfill \pageref{sut:222}\par \noindent
223, 188. aṃ yamīto pasaññato.\hfill \pageref{sut:223}\par \noindent
224, 153. naṃ jhato katarassā.\hfill \pageref{sut:224}\par \noindent
225, 151. yonaṃ no.\hfill \pageref{sut:225}\par \noindent
226, 154. smiṃ ni.\hfill \pageref{sut:226}\par \noindent
227, 270. kissa ka ve ca.\hfill \pageref{sut:227}\par \noindent
228, 272. ku hiṃhaṃsu ca.\hfill \pageref{sut:228}\par \noindent
229, 226. sesesu ca.\hfill \pageref{sut:229}\par \noindent
230, 262. tratothesu ca.\hfill \pageref{sut:230}\par \noindent
231, 263. sabbassetassakāro vā.\hfill \pageref{sut:231}\par \noindent
232, 267. tre niccaṃ.\hfill \pageref{sut:232}\par \noindent
233, 264. e tothesu ca.\hfill \pageref{sut:233}\par \noindent
234, 265. imassi thaṃdānihatodhesu ca.\hfill \pageref{sut:234}\par \noindent
235, 281. a dhunāmhi ca.\hfill \pageref{sut:235}\par \noindent
236, 280. eta rahimhi.\hfill \pageref{sut:236}\par \noindent
237, 176. itthiyamato āpaccayo.\hfill \pageref{sut:237}\par \noindent
238, 187. nadādito vā ī.\hfill \pageref{sut:238}\par \noindent
239, 190. ṇavaṇikaṇeyyaṇantuhi.\hfill \pageref{sut:239}\par \noindent
240, 193. patibhikkhurājīkārantehi inī.\hfill \pageref{sut:240}\par \noindent
241, 191. ntussa tamīkāre.\hfill \pageref{sut:241}\par \noindent
242, 192. bhavato bhoto.\hfill \pageref{sut:242}\par \noindent
243, 110. bho ge tu.\hfill \pageref{sut:243}\par \noindent
243a, 109. obhāvo kvaci yosu vakārassa.\hfill \pageref{sut:243a}\par \noindent
243b, 111. bhadantassa bhaddantabhante.\hfill \pageref{sut:243b}\par \noindent
244, 72. akārapitādyantānamā.\hfill \pageref{sut:244}\par \noindent
245, 152. jhalapā rassaṃ.\hfill \pageref{sut:245}\par \noindent
246, 73. ākāro vā.\hfill \pageref{sut:246}\par \noindent
247, 261. tvādayo vibhattisaññāyo.\hfill \pageref{sut:247}\par \noindent
248, 260. kvaci to pañcamyatthe.\hfill \pageref{sut:248}\par \noindent
249, 266. tratha sattamiyā sabbanāmehi.\hfill \pageref{sut:249}\par \noindent
250, 268. sabbato dhi.\hfill \pageref{sut:250}\par \noindent
251, 269. kiṃsmā vo.\hfill \pageref{sut:251}\par \noindent
252, 271. hiṃ haṃ hiñcanaṃ.\hfill \pageref{sut:252}\par \noindent
253, 273. tamhā ca.\hfill \pageref{sut:253}\par \noindent
254, 274. imasmā hadhā ca.\hfill \pageref{sut:254}\par \noindent
255, 275. yato hiṃ.\hfill \pageref{sut:255}\par \noindent
256. kāle.\hfill \pageref{sut:256}\par \noindent
257, 279. kiṃsabbaññekayakuhi dādācanaṃ.\hfill \pageref{sut:257}\par \noindent
258, 278. tamhā dāni ca.\hfill \pageref{sut:258}\par \noindent
259, 279. imasmā rahidhunādāni ca.\hfill \pageref{sut:259}\par \noindent
260, 277. sabbassa so dāmhi vā.\hfill \pageref{sut:260}\par \noindent
261, 369. avaṇṇo ye lopañca.\hfill \pageref{sut:261}\par \noindent
262, 391. vuḍḍhassa jo iyiṭṭhesu.\hfill \pageref{sut:262}\par \noindent
263, 392. pasatthassa so ca.\hfill \pageref{sut:263}\par \noindent
264, 393. antikassa nedo.\hfill \pageref{sut:264}\par \noindent
265, 394. bāḷhassa sādho.\hfill \pageref{sut:265}\par \noindent
266, 395. appassa kaṇa.\hfill \pageref{sut:266}\par \noindent
267, 396. yuvānañca.\hfill \pageref{sut:267}\par \noindent
268, 397. vantumantuvīnañca lopo.\hfill \pageref{sut:268}\par \noindent
269, 401. yavataṃ\ \ ta\,la\,ṇa\,dakārānaṃ\ \ byañjanāni\par \noindent
\hspace{15mm} ca\,la\,ña\,jakārattaṃ.\hfill \pageref{sut:269}\par \noindent
270, 120. amha\,tumha\,ntu\,rāja\,brahm\,atta\,sakha\,satthu-\par \noindent
\hspace{15mm} pitādīhi\ \ smā\ \ nāva.\hfill \pageref{sut:270}\par \noindent
271, 88, 308. yasmādapeti bhayamādatte vā tadapādānaṃ.\par \noindent
\hfill \pageref{sut:271}\par \noindent
272, 309. dhātunāmānamupasaggayogādvīsvapi ca.\hfill \pageref{sut:272}\par \noindent
273, 310. rakkhaṇatthānamicchitaṃ.\hfill \pageref{sut:273}\par \noindent
274, 311. yena vā’dassanaṃ.\hfill \pageref{sut:274}\par \noindent
275,~312.~dūr\,antik\,addha\,kāla\,nimmāna\,tvālopa\,disāyoga-\par \noindent
\hspace{15mm} vibhatt\,ārappayoga\,suddhap\,pamocana\,hetu\,vivit-\par \noindent
\hspace{15mm} tap\,pamāṇa\,pubbayoga\,bandhana\,guṇavacana\,pañ-\par \noindent
\hspace{15mm} ha\,kathana\,thok\,ākattūsu\ \ ca.\hfill \pageref{sut:275}\par \noindent
276, 84, 302. yassa dātukāmo rocate dhārayate vā taṃ\par \noindent
\hspace{15mm} sampadānaṃ.\hfill \pageref{sut:276}\par \noindent
277,~303.~silāgha\,hanu\,ṭhā\,sapa\,dhāra\,piha\,kudha\,duh\,iss-\par \noindent
\hspace{15mm} osūya\,rādh\,ikkha\,paccāsuṇa\,anupatigiṇa\,pubba-\par \noindent
\hspace{15mm} kattā\,rocanattha\,tadattha\,tumatth\,ālamattha-\par \noindent
\hspace{15mm} maññān\,ādar\,appāṇini\ \ gatyatthakammani\ \ āsīs-\par \noindent
\hspace{15mm} attha\,sammuti\,bhiyya\,sattamyatthesu\ \ ca.\hfill \pageref{sut:277}\par \noindent
278, 93, 320. yodhāro tamokāsaṃ.\hfill \pageref{sut:278}\par \noindent
279, 82, 292. yena vā kayirate taṃ karaṇaṃ.\hfill \pageref{sut:279}\par \noindent
280, 75, 285. yaṃ karoti taṃ kammaṃ.\hfill \pageref{sut:280}\par \noindent
281, 77, 294. yo karoti sa kattā.\hfill \pageref{sut:281}\par \noindent
282, 295. yo kāreti sa hetu.\hfill \pageref{sut:282}\par \noindent
283, 91, 316. yassa vā pariggaho taṃ sāmī.\hfill \pageref{sut:283}\par \noindent
284, 65, 283. liṅgatthe paṭhamā.\hfill \pageref{sut:284}\par \noindent
285, 70. ālapane ca.\hfill \pageref{sut:285}\par \noindent
286, 83, 291. karaṇe tatiyā.\hfill \pageref{sut:286}\par \noindent
287, 299. sahādiyoge ca.\hfill \pageref{sut:287}\par \noindent
288, 78, 293. kattari ca.\hfill \pageref{sut:288}\par \noindent
289, 297. hetvatthe ca.\hfill \pageref{sut:289}\par \noindent
290, 298. sattamyatthe ca.\hfill \pageref{sut:290}\par \noindent
291, 299. yenaṅgavikāro.\hfill \pageref{sut:291}\par \noindent
292, 300. visesane ca.\hfill \pageref{sut:292}\par \noindent
293, 85, 301. sampadāne catutthī.\hfill \pageref{sut:293}\par \noindent
294, 305. namoyogādīsvapi ca.\hfill \pageref{sut:294}\par \noindent
295, 89, 307. apādāne pañcamī.\hfill \pageref{sut:295}\par \noindent
296, 314. kāraṇatthe ca.\hfill \pageref{sut:296}\par \noindent
297, 76, 284. kammatthe dutiyā.\hfill \pageref{sut:297}\par \noindent
298, 287. kāladdhānamaccantasaṃyoge.\hfill \pageref{sut:298}\par \noindent
299, 288. kammappavacanīyayutte.\hfill \pageref{sut:299}\par \noindent
300, 286. gati\,buddhi\,bhuja\,paṭha\,hara\,kara\,sayādīnaṃ\par \noindent
\hspace{15mm}\ \ kārite\ \ vā.\hfill \pageref{sut:300}\par \noindent
301, 92, 315. sāmismiṃ chaṭṭhī.\hfill \pageref{sut:301}\par \noindent
302, 94, 319. okāse sattamī.\hfill \pageref{sut:302}\par \noindent
303, 321. sām\,issar\,ādhipati\,dāyāda\,sakkhī\,patibhū\,pasuta-\par \noindent
\hspace{15mm} kusalehi\ \ ca.\hfill \pageref{sut:303}\par \noindent
304, 322. niddhāraṇe ca.\hfill \pageref{sut:304}\par \noindent
305, 323. anādare ca.\hfill \pageref{sut:305}\par \noindent
306, 289. kvaci dutiyā chaṭṭhīnamatthe.\hfill \pageref{sut:306}\par \noindent
307, 290. tatiyāsattamīnañca.\hfill \pageref{sut:307}\par \noindent
308, 317. chaṭṭhī ca.\hfill \pageref{sut:308}\par \noindent
309, 318. dutiyāpañcamīnañca.\hfill \pageref{sut:309}\par \noindent
310, 324. kammakaraṇanimittatthesu sattamī.\hfill \pageref{sut:310}\par \noindent
311, 325. sampadāne ca.\hfill \pageref{sut:311}\par \noindent
312, 326. pañcamyatthe ca.\hfill \pageref{sut:312}\par \noindent
313, 327. kālabhāvesu ca.\hfill \pageref{sut:313}\par \noindent
314, 328. upa’dhādhikissaravacane.\hfill \pageref{sut:314}\par \noindent
315, 329. maṇḍitussukkesu tatiyā.\hfill \pageref{sut:315}\par \noindent
316, 331. nāmānaṃ samāso yuttattho.\hfill \pageref{sut:316}\par \noindent
317, 332. tesaṃ vibhattiyo lopā ca.\hfill \pageref{sut:317}\par \noindent
318, 333. pakati cassa sarantassa.\hfill \pageref{sut:318}\par \noindent
319, 330. upasagganipātapubbako abyayībhāvo.\hfill \pageref{sut:319}\par \noindent
320, 335. so napuṃsakaliṅgo.\hfill \pageref{sut:320}\par \noindent
321, 349. digussekattaṃ.\hfill \pageref{sut:321}\par \noindent
322,~359.~tathā\ \ dvande\ \ pāṇi\,tūriya\,yogga\,senaṅga\,khudda-\par \noindent
\hspace{15mm} jantuka\,vividhaviruddha\,visabhāgatthādīnañca.\hfill \pageref{sut:322}\par \noindent
323,~360.~vibhāsā rukkhatiṇapasudhanadhaññajanapadā-\par \noindent
\hspace{15mm} dīnañca.\hfill \pageref{sut:323}\par \noindent
324, 339. dvipade tulyādhikaraṇe kammadhārayo.\hfill \pageref{sut:324}\par \noindent
325, 348. saṅkhyāpubbo digu.\hfill \pageref{sut:325}\par \noindent
326, 341. ubhe tappurisā.\hfill \pageref{sut:326}\par \noindent
327, 351. amādayo parapadebhi.\hfill \pageref{sut:327}\par \noindent
328, 352. aññapadatthesu bahubbīhi.\hfill \pageref{sut:328}\par \noindent
329, 357. nāmānaṃ samuccayo dvando.\hfill \pageref{sut:329}\par \noindent
330, 340. mahataṃ mahā tulyādhikaraṇe pade.\hfill \pageref{sut:330}\par \noindent
331, 353. itthiyaṃ bhāsitapumitthī pumāva ce.\hfill \pageref{sut:331}\par \noindent
332, 343. kammadhārayasaññe ca.\hfill \pageref{sut:332}\par \noindent
333, 344. attaṃ nassa tappurise.\hfill \pageref{sut:333}\par \noindent
334, 345. sare an.\hfill \pageref{sut:334}\par \noindent
335, 346. kada kussa.\hfill \pageref{sut:335}\par \noindent
336, 347. kā’ppatthesu ca.\hfill \pageref{sut:336}\par \noindent
337, 350. kvaci samāsantagatānamakāranto.\hfill \pageref{sut:337}\par \noindent
338, 356. nadimhā ca.\hfill \pageref{sut:338}\par \noindent
339, 358. jāyāya tudaṃjāni patimhi.\hfill \pageref{sut:339}\par \noindent
340, 355. dhanumhā ca.\hfill \pageref{sut:340}\par \noindent
341, 336. aṃ vibhattīnamakārantā abyayībhāvā.\hfill \pageref{sut:341}\par \noindent
342, 337. saro rasso napuṃsake.\hfill \pageref{sut:342}\par \noindent
343, 338. aññasmā lopo ca.\hfill \pageref{sut:343}\par \noindent
344, 361. vā ṇa’pacce.\hfill \pageref{sut:344}\par \noindent
345, 366. ṇāyanaṇāna vacchādito.\hfill \pageref{sut:345}\par \noindent
346, 367. ṇeyyo kattikādīhi.\hfill \pageref{sut:346}\par \noindent
347, 368. ato ṇi vā.\hfill \pageref{sut:347}\par \noindent
348, 371. ṇavopakvādīhi.\hfill \pageref{sut:348}\par \noindent
349, 372. ṇera vidhavādito.\hfill \pageref{sut:349}\par \noindent
350, 373. yena vā saṃsaṭṭhaṃ tarati carati vahati ṇiko.\hfill \pageref{sut:350}\par \noindent
351,~374.~tamadhīte tenakatādi sannidhānaniyogasippa-\par \noindent
\hspace{15mm} bhaṇḍajīvikatthesu ca.\hfill \pageref{sut:351}\par \noindent
352, 376. ṇa rāgā tassedamaññatthesu ca.\hfill \pageref{sut:352}\par \noindent
353, 378. jātādīnamimiyā ca.\hfill \pageref{sut:353}\par \noindent
354, 379. samūhatthe kaṇaṇā.\hfill \pageref{sut:354}\par \noindent
355, 380. gāmajanabandhusahāyādīhi tā.\hfill \pageref{sut:355}\par \noindent
356, 381. tadassa ṭhānamiyo ca.\hfill \pageref{sut:356}\par \noindent
357, 382. upamatthāyitattaṃ.\hfill \pageref{sut:357}\par \noindent
358, 383. tannissitatthe lo.\hfill \pageref{sut:358}\par \noindent
359, 384. ālu tabbahule.\hfill \pageref{sut:359}\par \noindent
360, 387. ṇyattatā bhāve tu.\hfill \pageref{sut:360}\par \noindent
361, 388. ṇa visamādīhi.\hfill \pageref{sut:361}\par \noindent
362, 389. ramaṇīyādito kaṇ.\hfill \pageref{sut:362}\par \noindent
363, 390. visese taratamisikiyiṭṭhā.\hfill \pageref{sut:363}\par \noindent
364, 398. tadassatthīti vī ca.\hfill \pageref{sut:364}\par \noindent
365, 399. tapādito sī.\hfill \pageref{sut:365}\par \noindent
366, 400. daṇḍādito ikaī.\hfill \pageref{sut:366}\par \noindent
367, 401. madhvādito ro.\hfill \pageref{sut:367}\par \noindent
368, 402. guṇādito vantu.\hfill \pageref{sut:368}\par \noindent
369, 403. satyādīhi mantu.\hfill \pageref{sut:369}\par \noindent
370, 405. saddhādito ṇa.\hfill \pageref{sut:370}\par \noindent
371, 404. āyussukārāsa mantumhi.\hfill \pageref{sut:371}\par \noindent
372, 385. tappakativacane mayo.\hfill \pageref{sut:372}\par \noindent
373, 406. saṅkhyāpūraṇe mo.\hfill \pageref{sut:373}\par \noindent
374, 408. sa chassa vā.\hfill \pageref{sut:374}\par \noindent
375, 412. ekādito dasassī.\hfill \pageref{sut:375}\par \noindent
376, 257. dase so niccañca.\hfill \pageref{sut:376}\par \noindent
377. ante niggahitañca.\hfill \pageref{sut:377}\par \noindent
378, 414. ti ca.\hfill \pageref{sut:378}\par \noindent
379, 258. la darānaṃ.\hfill \pageref{sut:379}\par \noindent
380, 255. vīsatidasesu bā dvissa tu.\hfill \pageref{sut:380}\par \noindent
381, 254. ekādito dassa ra saṅkhyāne.\hfill \pageref{sut:381}\par \noindent
382, 259. aṭṭhādito ca.\hfill \pageref{sut:382}\par \noindent
383, 253. dvekaṭṭhānamākāro vā.\hfill \pageref{sut:383}\par \noindent
384, 407. catucchehī thaṭhā.\hfill \pageref{sut:384}\par \noindent
385, 409. dvitīhi tiyo.\hfill \pageref{sut:385}\par \noindent
386, 410. tiye dutāpi ca.\hfill \pageref{sut:386}\par \noindent
387,~411.~tesamaḍḍhūpapadena aḍḍhuḍḍhadivaḍḍha-\par \noindent
\hspace{15mm} diyaḍḍhaḍḍhatiyā.\hfill \pageref{sut:387}\par \noindent
388, 68. sarūpānamekasesvasakiṃ.\hfill \pageref{sut:388}\par \noindent
389,~413.~gaṇane dasassa dviticatupañcachasattaaṭṭha-\par \noindent
\hspace{15mm} navakānaṃ vīticattārapaññāchasattāsanavā\par \noindent
\hspace{15mm} yosu, yonañcīsamāsaṃṭhiritītuti.\hfill \pageref{sut:389}\par \noindent
390,~256.~catūpapadassa lopo tuttarapadādicassa cucopi\par \noindent
\hspace{15mm} navā.\hfill \pageref{sut:390}\par \noindent
391, 423. yadanupapannā nipātanā sijjhanti.\hfill \pageref{sut:391}\par \noindent
392, 418. dvādito ko’nekattheca.\hfill \pageref{sut:392}\par \noindent
393,~415.~dasadasakaṃ sataṃ dasakānaṃ sataṃ\par \noindent
\hspace{15mm} sahassañca yomhi.\hfill \pageref{sut:393}\par \noindent
394, 416. yāva taduttari dasaguṇitañca.\hfill \pageref{sut:394}\par \noindent
395, 417. sakanāmehi.\hfill \pageref{sut:395}\par \noindent
396, 363. tesaṃ ṇo lopaṃ.\hfill \pageref{sut:396}\par \noindent
397, 420. vibhāge dhā ca.\hfill \pageref{sut:397}\par \noindent
398, 421. sabbanāmehi pakāravacane tu thā.\hfill \pageref{sut:398}\par \noindent
399, 422. kimimehi thaṃ.\hfill \pageref{sut:399}\par \noindent
400, 364. vuddhādisarassa vā’saṃyogantassa saṇe ca.\hfill \pageref{sut:400}\par \noindent
401, 375. mā yūnamāgamo ṭhāne.\hfill \pageref{sut:401}\par \noindent
402, 377. āttañca.\hfill \pageref{sut:402}\par \noindent
403,~354.~kvacādimajjhuttarānaṃ dīgharassā paccayesu\par \noindent
\hspace{15mm} ca.\hfill \pageref{sut:403}\par \noindent
404, 370. tesu vuddhilopāgamavikāraviparītādesā ca.\hfill \pageref{sut:404}\par \noindent
405, 365. ayuvaṇṇānañcāyo vuddhi.\hfill \pageref{sut:405}\par \noindent
406, 429. atha pubbāni vibhattīnaṃ cha parassapadāni.\hfill \pageref{sut:406}\par \noindent
407, 439. parānyattanopadāni.\hfill \pageref{sut:407}\par \noindent
408, 431. dve dve paṭhamamajjhimuttamapurisā.\hfill \pageref{sut:408}\par \noindent
409, 441. sabbesamekābhidhāne paro puriso.\hfill \pageref{sut:409}\par \noindent
410, 432. nāmamhi payujjamānepi tulyādhikaraṇe\par \noindent
\hspace{15mm} paṭhamo.\hfill \pageref{sut:410}\par \noindent
411, 436. tumhe majjhimo.\hfill \pageref{sut:411}\par \noindent
412, 437. amhe uttamo.\hfill \pageref{sut:412}\par \noindent
413, 427. kāle.\hfill \pageref{sut:413}\par \noindent
414, 428. vattamānā paccuppanne.\hfill \pageref{sut:414}\par \noindent
415, 451. āṇatyāsiṭṭhe’nuttakāle pañcamī.\hfill \pageref{sut:415}\par \noindent
416, 454. anumatiparikappatthesu sattamī.\hfill \pageref{sut:416}\par \noindent
417, 460. apaccakkhe parokkhātīte.\hfill \pageref{sut:417}\par \noindent
418, 456. hiyyopabhuti paccakkhe hiyyattanī.\hfill \pageref{sut:418}\par \noindent
419, 469. samīpe’jjatanī.\hfill \pageref{sut:419}\par \noindent
420, 471. māyoge sabbakāle ca.\hfill \pageref{sut:420}\par \noindent
421, 473. anāgate bhavissantī.\hfill \pageref{sut:421}\par \noindent
422, 475. kriyā’tipanne’tīte kālātipatti.\hfill \pageref{sut:422}\par \noindent
423, 426. vattamānā ti anti, si tha, mi ma, te ante,\par \noindent
\hspace{15mm} se vhe, e mhe.\hfill \pageref{sut:423}\par \noindent
424, 450. pañcamī tu antu, hi tha, mi ma, taṃ antaṃ,\par \noindent
\hspace{15mm} ssu vho, e āmase.\hfill \pageref{sut:424}\par \noindent
425, 453. sattamī eyya eyyuṃ, eyyāsi eyyātha, eyyāmi\par \noindent
\hspace{15mm} eyyāma, etha eraṃ, etho eyyāvho, eyyaṃ\par \noindent
\hspace{15mm} eyyāmhe.\hfill \pageref{sut:425}\par \noindent
426, 459. parokkhā a u, e ttha, aṃ mha, ttha re, ttho\par \noindent
\hspace{15mm} vho, iṃ mhe.\hfill \pageref{sut:426}\par \noindent
427, 455. hiyyattanī ā ū, o ttha, aṃ mhā, ttha tthuṃ,\par \noindent
\hspace{15mm} se vhaṃ, iṃ mhase.\hfill \pageref{sut:427}\par \noindent
428, 468. ajjatanī ī uṃ, o ttha, iṃ mhā, ā ū, se vhaṃ,\par \noindent
\hspace{15mm} aṃ mhe.\hfill \pageref{sut:428}\par \noindent
429, 472. bhavissantī ssati ssanti, ssasi ssatha, ssāmi\par \noindent
\hspace{15mm} ssāma, ssate ssante, ssase ssavhe, ssaṃ\par \noindent
\hspace{15mm} ssāmhe.\hfill \pageref{sut:429}\par \noindent
430, 373. kālātipatti ssā ssaṃsu, sse ssatha, ssaṃ\par \noindent
\hspace{15mm} ssāmhā, ssatha ssisu, ssase ssavhe, ssiṃ\par \noindent
\hspace{15mm} ssāmhase.\hfill \pageref{sut:430}\par \noindent
431, 458. hiyyattanī sattamī pañcamī vattamānā sabba-\par \noindent
\hspace{15mm} dhātukaṃ.\hfill \pageref{sut:431}\par \noindent
432, 362. dhātuliṅgehi parā paccayā.\hfill \pageref{sut:432}\par \noindent
433, 528. tija\,gupa\,kita\,mānehi\ \ kha\,cha\,sā\ \ vā.\hfill \pageref{sut:433}\par \noindent
434, 534. bhuja\,ghasa\,hara\,su\,pādīhi\ \ tumicchatthesu.\hfill \pageref{sut:434}\par \noindent
435, 536. āya nāmato kattūpamānādācāre.\hfill \pageref{sut:435}\par \noindent
436, 537. īyūpamānā ca.\hfill \pageref{sut:436}\par \noindent
437, 538. nāmamhātticchatthe.\hfill \pageref{sut:437}\par \noindent
438, 540. dhātūhi ṇe ṇaya ṇāpe ṇāpayā kāritāni hetvatthe.\par \noindent
\hspace{15mm}\hfill \pageref{sut:438}\par \noindent
439, 539. dhāturūpe nāmasmā ṇayo ca.\hfill \pageref{sut:439}\par \noindent
440, 445. bhāvakammesu yo.\hfill \pageref{sut:440}\par \noindent
441, 447. tassa cavaggayakāravakārattaṃ sadhātvantassa.\par \noindent
\hspace{15mm}\hfill \pageref{sut:441}\par \noindent
442, 448. ivaṇṇāgamo vā.\hfill \pageref{sut:442}\par \noindent
443, 449. pubbarūpañca.\hfill \pageref{sut:443}\par \noindent
444, 501. tathā kattari ca.\hfill \pageref{sut:444}\par \noindent
445, 433. bhūvādito a.\hfill \pageref{sut:445}\par \noindent
446, 509. rudhādito niggahitapubbañca.\hfill \pageref{sut:446}\par \noindent
447, 510. divādito yo.\hfill \pageref{sut:447}\par \noindent
448, 512. svādito ṇu ṇā uṇā ca.\hfill \pageref{sut:448}\par \noindent
449, 513. kiyādito nā.\hfill \pageref{sut:449}\par \noindent
450, 517. gahādito ppaṇhā.\hfill \pageref{sut:450}\par \noindent
451, 520. tanādito oyirā.\hfill \pageref{sut:451}\par \noindent
452, 525. curādito ṇeṇayā.\hfill \pageref{sut:452}\par \noindent
453, 444. attanopadāni bhāve ca kammani.\hfill \pageref{sut:453}\par \noindent
454, 440. kattari ca.\hfill \pageref{sut:454}\par \noindent
455, 530. dhātuppaccayehi vibhattiyo.\hfill \pageref{sut:455}\par \noindent
456, 430. kattari parassapadaṃ.\hfill \pageref{sut:456}\par \noindent
457, 424. bhūvādayo dhātavo.\hfill \pageref{sut:457}\par \noindent
458, 461. kvacādivaṇṇānamekassarānaṃ dvebhāvo.\hfill \pageref{sut:458}\par \noindent
459, 462. pubbo’bbhāso.\hfill \pageref{sut:459}\par \noindent
460, 506. rasso.\hfill \pageref{sut:460}\par \noindent
461, 464. dutiyacatutthānaṃ paṭhamatatiyā.\hfill \pageref{sut:461}\par \noindent
462, 476. kavaggassa cavaggo.\hfill \pageref{sut:462}\par \noindent
463, 532. mānakitānaṃ vatattaṃ vā.\hfill \pageref{sut:463}\par \noindent
464, 504. hassa jo.\hfill \pageref{sut:464}\par \noindent
465, 463. antassivaṇṇākāro vā.\hfill \pageref{sut:465}\par \noindent
466, 489. niggahitañca.\hfill \pageref{sut:466}\par \noindent
467, 533. tato pāmānānaṃ vāmaṃ sesu.\hfill \pageref{sut:467}\par \noindent
468, 462. ṭhā tiṭṭho.\hfill \pageref{sut:468}\par \noindent
469, 494. pā pivo.\hfill \pageref{sut:469}\par \noindent
470, 514. ñāssa jājaṃnā.\hfill \pageref{sut:470}\par \noindent
471, 483. disassa passadissadakkhā vā.\hfill \pageref{sut:471}\par \noindent
472, 531. byañjanantassa co chapaccayesu ca.\hfill \pageref{sut:472}\par \noindent
473, 529. ko khe ca.\hfill \pageref{sut:473}\par \noindent
474, 535. harassa gī se.\hfill \pageref{sut:474}\par \noindent
475, 565. brūbhūnamāhabhūvā parokkhāyaṃ.\hfill \pageref{sut:475}\par \noindent
476, 442. gamissanto ccho vā sabbāsu.\hfill \pageref{sut:476}\par \noindent
477, 479. vacassajjatanimhimakāro o.\hfill \pageref{sut:477}\par \noindent
478, 438. akāro dīghaṃ himimesu.\hfill \pageref{sut:478}\par \noindent
479, 452. hi lopaṃ vā.\hfill \pageref{sut:479}\par \noindent
480, 490. hotissarehohe bhavissantimhi ssassa ca.\hfill \pageref{sut:480}\par \noindent
481, 524. karassa sapaccayassa kāho.\hfill \pageref{sut:481}\par \noindent
482, 508. dādantassaṃ mimesu.\hfill \pageref{sut:482}\par \noindent
483, 527. asaṃyogantassa vuddhi kārite.\hfill \pageref{sut:483}\par \noindent
484, 542. ghaṭādīnaṃ vā.\hfill \pageref{sut:484}\par \noindent
485, 434. aññesu ca.\hfill \pageref{sut:485}\par \noindent
486, 543. guhadusānaṃ dīghaṃ.\hfill \pageref{sut:486}\par \noindent
487, 478. vacavasavahādīnamukāro vassa ye.\hfill \pageref{sut:487}\par \noindent
488, 481. havipariyayo lo vā.\hfill \pageref{sut:488}\par \noindent
489, 519. gahassa ghe ppe.\hfill \pageref{sut:489}\par \noindent
490, 518. halopo ṇhāmhi.\hfill \pageref{sut:490}\par \noindent
491, 523. karassa kāsattamajjatanimhi.\hfill \pageref{sut:491}\par \noindent
492, 499. asasmā mimānaṃ mhimhā’ntalopo ca.\hfill \pageref{sut:492}\par \noindent
493, 498. thassa tthattaṃ.\hfill \pageref{sut:493}\par \noindent
494, 495. tissa tthittaṃ.\hfill \pageref{sut:494}\par \noindent
495, 500. tussa tthuttaṃ.\hfill \pageref{sut:495}\par \noindent
496, 497. simhi ca.\hfill \pageref{sut:496}\par \noindent
497, 477. labhasmā ī iṃnaṃ ttha tthaṃ.\hfill \pageref{sut:497}\par \noindent
498, 480. kusasmā dī cchi.\hfill \pageref{sut:498}\par \noindent
499, 507. dādhātussa dajjaṃ.\hfill \pageref{sut:499}\par \noindent
500, 486. vadassa vajjaṃ.\hfill \pageref{sut:500}\par \noindent
501, 443. gamissa ghammaṃ.\hfill \pageref{sut:501}\par \noindent
502,~493.~yamhi\ \ dā\,dhā\,mā\,ṭhā\,hā\,pā\,maha\,mathādīnamī.\hfill \pageref{sut:502}\par \noindent
503, 485. yajassādissi.\hfill \pageref{sut:503}\par \noindent
504, 470. sabbato uṃ iṃsu.\hfill \pageref{sut:504}\par \noindent
505, 482. jaramarānaṃ jīrajiyyamiyyā vā.\hfill \pageref{sut:505}\par \noindent
506, 496. sabbatthā’sassādilopo ca.\hfill \pageref{sut:506}\par \noindent
507, 501. asabbadhātuke bhū.\hfill \pageref{sut:507}\par \noindent
508, 515. eyyassa ñāto iyāñā.\hfill \pageref{sut:508}\par \noindent
509, 516. nāssa lopo yakārattaṃ.\hfill \pageref{sut:509}\par \noindent
510, 487. lopañcettamakāro.\hfill \pageref{sut:510}\par \noindent
511, 521. uttamokāro.\hfill \pageref{sut:511}\par \noindent
512, 522. karassākāro ca.\hfill \pageref{sut:512}\par \noindent
513, 435. o ava sare.\hfill \pageref{sut:513}\par \noindent
514, 491. e aya.\hfill \pageref{sut:514}\par \noindent
515, 541. te āvāyā kārite.\hfill \pageref{sut:515}\par \noindent
516, 466. ikārāgamo asabbadhātukamhi.\hfill \pageref{sut:516}\par \noindent
517, 488. kvaci dhātuvibhattipaccayānaṃ dīghaviparītā-\par \noindent
\hspace{15mm} desalopāgamā ca.\hfill \pageref{sut:517}\par \noindent
518, 446. attanopadāni parassapadattaṃ.\hfill \pageref{sut:518}\par \noindent
519, 457. akārāgamo hiyyattanīajjatanīkālātipattīsu.\hfill \pageref{sut:519}\par \noindent
520, 502. brūto ī timhi.\hfill \pageref{sut:520}\par \noindent
521, 425. dhātussanto lopo’nekasarassa.\hfill \pageref{sut:521}\par \noindent
522, 476. isuyamūnamanto ccho vā.\hfill \pageref{sut:522}\par \noindent
523, 526. kāritānaṃ ṇo lopaṃ.\hfill \pageref{sut:523}\par \noindent
524, 561. dhātuyā kammādimhi ṇo.\hfill \pageref{sut:524}\par \noindent
525, 565. saññāyama nu.\hfill \pageref{sut:525}\par \noindent
526, 567. pure dadā ca iṃ.\hfill \pageref{sut:526}\par \noindent
527, 568. sabbato ṇvutvāvī vā.\hfill \pageref{sut:527}\par \noindent
528, 577. visarujapadādito ṇa.\hfill \pageref{sut:528}\par \noindent
529, 580. bhāve ca.\hfill \pageref{sut:529}\par \noindent
530, 584. kvi ca.\hfill \pageref{sut:530}\par \noindent
531, 589. dharādīhi rammo.\hfill \pageref{sut:531}\par \noindent
532, 590. tassīlādīsu ṇītvāvī ca.\hfill \pageref{sut:532}\par \noindent
533, 591. saddakudhacalamaṇḍattharudhādīhi yu.\hfill \pageref{sut:533}\par \noindent
534, 562. pārādigamimhā rū.\hfill \pageref{sut:534}\par \noindent
535, 593. bhikkhādito ca.\hfill \pageref{sut:535}\par \noindent
536, 594. hanatyādīnaṃ ṇuko.\hfill \pageref{sut:536}\par \noindent
537, 566. nu niggahitaṃ padante.\hfill \pageref{sut:537}\par \noindent
538, 595. saṃhanāññāya vā ro gho.\hfill \pageref{sut:538}\par \noindent
539, 558. ramhi ranto rādi no.\hfill \pageref{sut:539}\par \noindent
540, 545. bhāvakammesu tabbānīyā.\hfill \pageref{sut:540}\par \noindent
541, 552. ṇyo ca.\hfill \pageref{sut:541}\par \noindent
542, 557. karamhā ricca.\hfill \pageref{sut:542}\par \noindent
543, 555. bhūto’bba.\hfill \pageref{sut:543}\par \noindent
544,~556.~vada\,mada\,gamu\,yuja\,garahā\,kārādīhi\ \ jja\,mma-\par \noindent
\hspace{15mm} gga\,yh\,eyyā\ \ gāro\ \ vā.\hfill \pageref{sut:544}\par \noindent
545, 548. te kiccā.\hfill \pageref{sut:545}\par \noindent
546, 562. aññe kita.\hfill \pageref{sut:546}\par \noindent
547, 596. nandādīhi yu.\hfill \pageref{sut:547}\par \noindent
548, 597. kattukaraṇapadesesu ca.\hfill \pageref{sut:548}\par \noindent
549, 550. rahādito ṇa.\hfill \pageref{sut:549}\par \noindent
550, 546. ṇādayo tekālikā.\hfill \pageref{sut:550}\par \noindent
551, 598. saññāyaṃ dā dhāto i.\hfill \pageref{sut:551}\par \noindent
552, 609. ti kita cāsiṭṭhe.\hfill \pageref{sut:552}\par \noindent
553, 599. itthiyamatiyavo vā.\hfill \pageref{sut:553}\par \noindent
554, 601. karato ririya.\hfill \pageref{sut:554}\par \noindent
555, 612. atīte tatavantutāvī.\hfill \pageref{sut:555}\par \noindent
556, 622. bhāvakammesu ta.\hfill \pageref{sut:556}\par \noindent
557, 606. budhagamāditthe kattari.\hfill \pageref{sut:557}\par \noindent
558, 602. jito ina sabbattha.\hfill \pageref{sut:558}\par \noindent
559, 603. supato ca.\hfill \pageref{sut:559}\par \noindent
560, 604. īsaṃdusūhi kha.\hfill \pageref{sut:560}\par \noindent
561, 636. icchatthesu samānakattukesu tavetuṃ vā.\hfill \pageref{sut:561}\par \noindent
562, 638. arahasakkādīsu ca.\hfill \pageref{sut:562}\par \noindent
563, 639. pattavacane alamatthesu ca.\hfill \pageref{sut:563}\par \noindent
564, 640. pubbakālekakattukānaṃ tunatvānatvā vā.\hfill \pageref{sut:564}\par \noindent
565, 646. vattamāne mānantā.\hfill \pageref{sut:565}\par \noindent
566, 574. sāsādīhi ratthu.\hfill \pageref{sut:566}\par \noindent
567, 575. pātito ritu.\hfill \pageref{sut:567}\par \noindent
568, 576. mānādīhi rātu.\hfill \pageref{sut:568}\par \noindent
569, 610. āgamā tuko.\hfill \pageref{sut:569}\par \noindent
570, 611. bhabbe ika.\hfill \pageref{sut:570}\par \noindent
571, 624. paccayādaniṭṭhā nipātanā sijjhanti.\hfill \pageref{sut:571}\par \noindent
572, 625. sāsadisato tassa riṭṭho ca.\hfill \pageref{sut:572}\par \noindent
573, 626. sādisantapucchabhanjahansādīhi ṭṭho.\hfill \pageref{sut:573}\par \noindent
574, 613. vasato uṭṭha.\hfill \pageref{sut:574}\par \noindent
575, 614. vassa vā vu.\hfill \pageref{sut:575}\par \noindent
576, 607. dhaḍhabhahehi dhaḍhā ca.\hfill \pageref{sut:576}\par \noindent
577, 628. bhanjato ggo ca.\hfill \pageref{sut:577}\par \noindent
578, 560. bhujādīnamanto no dvi ca.\hfill \pageref{sut:578}\par \noindent
579, 629. vaca vāvu.\hfill \pageref{sut:579}\par \noindent
580, 630. gupādīnañca.\hfill \pageref{sut:580}\par \noindent
581, 616. tarādīhi iṇṇo.\hfill \pageref{sut:581}\par \noindent
582, 631. bhidādito innaannaīṇā vā.\hfill \pageref{sut:582}\par \noindent
583, 617. susapacasakato kkhakkā ca.\hfill \pageref{sut:583}\par \noindent
584, 618. pakkamādīhi nto ca.\hfill \pageref{sut:584}\par \noindent
585, 619. janādīnamā timhi ca.\hfill \pageref{sut:585}\par \noindent
586, 600. gamakhanahanaramādīnamanto.\hfill \pageref{sut:586}\par \noindent
587, 632. rakāro ca.\hfill \pageref{sut:587}\par \noindent
588, 620. ṭhāpānamiī ca.\hfill \pageref{sut:588}\par \noindent
589, 621. hantehi ho hassa ḷo vā adahanahānaṃ.\hfill \pageref{sut:589}\par \noindent
590, 579. ṇamhi ranjassa jo bhāvakaraṇesu.\hfill \pageref{sut:590}\par \noindent
591, 544. hanassa ghāto.\hfill \pageref{sut:591}\par \noindent
592, 503. vadho vā sabbattha.\hfill \pageref{sut:592}\par \noindent
593, 564. ākārantānamāyo.\hfill \pageref{sut:593}\par \noindent
594, 582. purasamupaparīhi karotissa khakharā vā\par \noindent
\hspace{15mm} tapaccayesu ca.\hfill \pageref{sut:594}\par \noindent
595, 637. tavetunādīsu kā.\hfill \pageref{sut:595}\par \noindent
596, 551. gamakhanahanādīnaṃ tuṃtabbādīsu na.\hfill \pageref{sut:596}\par \noindent
597, 641. sabbehi tunādīnaṃ yo.\hfill \pageref{sut:597}\par \noindent
598, 643. canantehi raccaṃ.\hfill \pageref{sut:598}\par \noindent
599, 644. disā svānasvāntalopo ca.\hfill \pageref{sut:599}\par \noindent
600,~645.~ma\,ha\,da\,bhehi\ \ mma\,yha\,jja\,bbha\,ddhā\ \ ca.\hfill \pageref{sut:600}\par \noindent
601, 334. taddhitasamāsakitakā nāmaṃ vā tavetunādīsu\par \noindent
\hspace{15mm} ca.\hfill \pageref{sut:601}\par \noindent
602, 6. dumhi garu.\hfill \pageref{sut:602}\par \noindent
603, 7. dīgho ca.\hfill \pageref{sut:603}\par \noindent
604, 684. akkharehi kāra.\hfill \pageref{sut:604}\par \noindent
605, 647. yathāgamamikāro.\hfill \pageref{sut:605}\par \noindent
606, 642. dadhantato yo kvaci.\hfill \pageref{sut:606}\par \noindent
607, 578. niggahita saṃyogādi no.\hfill \pageref{sut:607}\par \noindent
608, 623. sabbattha ge gī.\hfill \pageref{sut:608}\par \noindent
609, 484. sadassa sīdattaṃ.\hfill \pageref{sut:609}\par \noindent
610, 627. yajassa sarassi ṭṭhe.\hfill \pageref{sut:610}\par \noindent
611, 608. hacatutthānamantānaṃ do dhe.\hfill \pageref{sut:611}\par \noindent
612, 615. ḍo ḍhakāre.\hfill \pageref{sut:612}\par \noindent
613, 583. gahassa ghara ṇe vā.\hfill \pageref{sut:613}\par \noindent
614, 581. dahassa do ḷaṃ.\hfill \pageref{sut:614}\par \noindent
615, 586. dhātvantassa lopo kvimhi.\hfill \pageref{sut:615}\par \noindent
616, 587. vidante ū.\hfill \pageref{sut:616}\par \noindent
617, 633. namakarānamantānaṃ niyuttatamhi.\hfill \pageref{sut:617}\par \noindent
618, 571. na kagattaṃ cajā ṇvumhi.\hfill \pageref{sut:618}\par \noindent
619, 573. karassa ca tattaṃ tusmiṃ.\hfill \pageref{sut:619}\par \noindent
620, 549. tuṃtunatabbesu vā.\hfill \pageref{sut:620}\par \noindent
621, 553. kāritaṃ viya ṇānubandho.\hfill \pageref{sut:621}\par \noindent
622, 570. anakā yuṇvūnaṃ.\hfill \pageref{sut:622}\par \noindent
623, 554. kagā cajānaṃ.\hfill \pageref{sut:623}\par \noindent
624, 563. kattari kita.\hfill \pageref{sut:624}\par \noindent
625, 605. bhāvakammesu kiccaktakhatthā.\hfill \pageref{sut:625}\par \noindent
626, 634. kammani dutiyāyaṃ kto.\hfill \pageref{sut:626}\par \noindent
627, 652. khyādīhi man ma ca to vā.\hfill \pageref{sut:627}\par \noindent
628, 653. samādīhi thamā.\hfill \pageref{sut:628}\par \noindent
629, 569. gahassupadhasse vā.\hfill \pageref{sut:629}\par \noindent
630, 654. masussa sussa ccharaccherā.\hfill \pageref{sut:630}\par \noindent
631, 655. āpubbacarassa ca.\hfill \pageref{sut:631}\par \noindent
632, 656. alakalasalehi layā.\hfill \pageref{sut:632}\par \noindent
633, 657. yāṇalāṇā.\hfill \pageref{sut:633}\par \noindent
634, 658. mathissa thassa lo ca.\hfill \pageref{sut:634}\par \noindent
635, 559. pesātisaggapattakālesu kiccā.\hfill \pageref{sut:635}\par \noindent
636, 659. avassakā’dhamiṇesu ṇī ca.\hfill \pageref{sut:636}\par \noindent
637. arahasakkādīhi tuṃ.\hfill \pageref{sut:637}\par \noindent
638, 660. vajādīhi pabbajjādayo nippajjante.\hfill \pageref{sut:638}\par \noindent
639, 585. kvilopo ca.\hfill \pageref{sut:639}\par \noindent
640. sacajānaṃ kagā ṇānubandhe.\hfill \pageref{sut:640}\par \noindent
641, 572. nudādīhi yuṇvūnamanānanākānanakā\par \noindent
\hspace{15mm} sakāritehi ca.\hfill \pageref{sut:641}\par \noindent
642, 588. iyatamakiesānamantassaro dīghaṃ kvaci\par \noindent
\hspace{15mm} disassa guṇaṃ do raṃ sakkhī ca.\hfill \pageref{sut:642}\par \noindent 
643, 635. bhyādīhi matibudhipūjādīhi ca kto.\hfill \pageref{sut:643}\par \noindent
644,~661.~vepu\,sī\,dava\,vamu\,ku\,dā\,bhū\,hvādīhi\ \ thu\,ttima-\par \noindent
\hspace{15mm} ṇimā\ \ nibbatte.\hfill \pageref{sut:644}\par \noindent
645, 662. akkose namhāni.\hfill \pageref{sut:645}\par \noindent
646, 419. ekādito sakissa kkhattuṃ.\hfill \pageref{sut:646}\par \noindent
647,~663.~sunassunass\,oṇa\,vān\,uvān\,ūn\,unakh\,un\,ā\,nā.\hfill \pageref{sut:647}\par \noindent
648, 664. taruṇassa susu ca.\hfill \pageref{sut:648}\par \noindent
649, 665. yuvassuvassuvuvānunūnā.\hfill \pageref{sut:649}\par \noindent
650, 651. kāle vattamānātīte ṇvādayo.\hfill \pageref{sut:650}\par \noindent
651, 647. bhavissati gamādīhi ṇīghiṇ.\hfill \pageref{sut:651}\par \noindent
652, 648. kriyāyaṃ ṇvutavo.\hfill \pageref{sut:652}\par \noindent
653, 306. bhāvavācimhi catutthī.\hfill \pageref{sut:653}\par \noindent
654, 649. kammani ṇo.\hfill \pageref{sut:654}\par \noindent
655, 650. sese ssaṃntumānānā.\hfill \pageref{sut:655}\par \noindent
656, 666. chadādīhi tatraṇa.\hfill \pageref{sut:656}\par \noindent
657, 667. vadādīhi ṇitto gaṇe.\hfill \pageref{sut:657}\par \noindent
658, 668. midādīhi ttitiyo.\hfill \pageref{sut:658}\par \noindent
659, 669. usuranjadaṃsānaṃ daṃsassa daḍḍho ḍhaṭhā\par \noindent
\hspace{15mm} ca.\hfill \pageref{sut:659}\par \noindent
660, 670. sūvusānamūvusānamato tho ca.\hfill \pageref{sut:660}\par \noindent
661, 671. ranjudādīhi dhadiddakirā kvaci jadalopo ca.\hfill \pageref{sut:661}\par \noindent
662, 672. paṭito hissa heraṇhīraṇ.\hfill \pageref{sut:662}\par \noindent
663, 673. kaḍyādīhi ko.\hfill \pageref{sut:663}\par \noindent
664, 674. khādāmagamānaṃ khandha’ndhagandhā.\hfill \pageref{sut:664}\par \noindent
665, 675. paṭādīhyalaṃ.\hfill \pageref{sut:665}\par \noindent
666, 676. puthassa puthupathāmo vā.\hfill \pageref{sut:666}\par \noindent
667, 677. sasvādīhi tudavo.\hfill \pageref{sut:667}\par \noindent
668, 678. cyādīhi īvaro.\hfill \pageref{sut:668}\par \noindent
669, 679. munādīhi ci.\hfill \pageref{sut:669}\par \noindent
670, 680. vidādīhyūro.\hfill \pageref{sut:670}\par \noindent
671, 681. hanādīhi ṇunutavo.\hfill \pageref{sut:671}\par \noindent
672, 682. kuṭādīhi ṭho.\hfill \pageref{sut:672}\par \noindent
673, 683. manupūrasuṇādīhi ussanusisā.\hfill \pageref{sut:673}\par \noindent
