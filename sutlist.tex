\markboth{}{Kaccāyana's suttas}
\cleardoublepage
\phantomsection
\addcontentsline{toc}{chapter}{Kaccāyana's suttas}
\chapter*{Kaccāyana's suttas}


1, 1. attho akkharasaññāto.\hfill \pageref{sut:1}\par \noindent
2, 2. akkharāpādayo ekacattālīsaṃ.\hfill \pageref{sut:2}\par \noindent
3, 3. tatthodantā sarā aṭṭha.\hfill \pageref{sut:3}\par \noindent
4, 4. lahumattā tayo rassā.\hfill \pageref{sut:4}\par \noindent
5, 5. aññe dīghā.\hfill \pageref{sut:5}\par \noindent
6, 8. sesā byañjanā.\hfill \pageref{sut:6}\par \noindent
7, 9. vaggā pañcapañcaso mantā.\hfill \pageref{sut:7}\par \noindent
8, 10. aṃ iti niggahitaṃ.\hfill \pageref{sut:8}\par \noindent
9, 11. parasamaññā payoge.\hfill \pageref{sut:9}\par \noindent
10, 12. pubbamadhoṭhita massaraṃ sarena viyojaye.\hfill \pageref{sut:10}\par \noindent
11, 14. naye paraṃ yutte.\hfill \pageref{sut:11}\par \noindent
12, 13. sarā sare lopaṃ.\hfill \pageref{sut:12}\par \noindent
13, 15. vā paro asarūpā.\hfill \pageref{sut:13}\par \noindent
14, 16. kvacāsavaṇṇaṃ lutte.\hfill \pageref{sut:14}\par \noindent
15, 17. dīghaṃ.\hfill \pageref{sut:15}\par \noindent
16, 18. pubbo ca.\hfill \pageref{sut:16}\par \noindent
17, 19. yamedantassādeso.\hfill \pageref{sut:17}\par \noindent
18, 20. vamodudantānaṃ.\hfill \pageref{sut:18}\par \noindent
19, 22. sabbo caṃ ti.\hfill \pageref{sut:19}\par \noindent
20, 27. do dhassa ca.\hfill \pageref{sut:20}\par \noindent
21, 21. ivaṇṇo yaṃ navā.\hfill \pageref{sut:21}\par \noindent
22, 28. evādissa ri pubbo ca rasso.\hfill \pageref{sut:22}\par \noindent
23, 26. sarā pakati byañjane.\hfill \pageref{sut:23}\par \noindent
24, 35. sare kvaci.\hfill \pageref{sut:24}\par \noindent
25, 37. dīghaṃ.\hfill \pageref{sut:25}\par \noindent
26, 38. rassaṃ.\hfill \pageref{sut:26}\par \noindent
27, 39. lopañca tatrākāro.\hfill \pageref{sut:27}\par \noindent
28, 40. para dvebhāvo ṭhāne.\hfill \pageref{sut:28}\par \noindent
29, 42. vagge ghosāghosānaṃ tatiyapaṭhamā.\hfill \pageref{sut:29}\par \noindent
30, 58. aṃ byañjane niggahitaṃ.\hfill \pageref{sut:30}\par \noindent
31, 49. vaggantaṃ vā vagge.\hfill \pageref{sut:31}\par \noindent
32, 50. ehe ñaṃ.\hfill \pageref{sut:32}\par \noindent
33, 50. sa ye ca.\hfill \pageref{sut:33}\par \noindent
34, 52. madā sare.\hfill \pageref{sut:34}\par \noindent
35, 34. ya va ma da na ta ra lā cāgamā.\hfill \pageref{sut:35}\par \noindent
36, 47. kvaci o byañjane.\hfill \pageref{sut:36}\par \noindent
37, 57. niggahitañca.\hfill \pageref{sut:37}\par \noindent
38, 53. kvaci lopaṃ.\hfill \pageref{sut:38}\par \noindent
39, 54. byañjane ca.\hfill \pageref{sut:39}\par \noindent
40, 55. paro vā saro.\hfill \pageref{sut:40}\par \noindent
41, 56. byañjano ca visaññogo.\hfill \pageref{sut:41}\par \noindent
42, 32. go sare puthassāgamo kvaci.\hfill \pageref{sut:42}\par \noindent
43, 33. pāssa canto rasso.\hfill \pageref{sut:43}\par \noindent
44, 24. abbho abhi.\hfill \pageref{sut:44}\par \noindent
45, 25. ajjho adhi.\hfill \pageref{sut:45}\par \noindent
46, 26. te na vā ivaṇṇe.\hfill \pageref{sut:46}\par \noindent
47, 23. atissa cantassa.\hfill \pageref{sut:47}\par \noindent
48, 43. kvaci paṭi patissa.\hfill \pageref{sut:48}\par \noindent
49, 44. puthassu byañjane.\hfill \pageref{sut:49}\par \noindent
50, 45. o avassa.\hfill \pageref{sut:50}\par \noindent
51, 59. anupadiṭṭhānaṃ vuttayogato.\hfill \pageref{sut:51}\par \noindent
52, 60. jinavacanayuttaṃ hi.\par \noindent
53, 61. liṅgañca nippajjate.\par \noindent
54, 62. tato ca vibhattiyo.\par \noindent
55, 63. si yo, aṃ yo, nā hi, sa naṃ, smā hi, sa naṃ, smiṃ su.\par \noindent
56, 64. tadanuparodhena.\par \noindent
57, 71. ālapane si ga sañño.\par \noindent
58, 29. ivaṇṇuvaṇṇā jhalā.\par \noindent
59, 182. te itthikhyā po.\par \noindent
60, 177. ā gho.\par \noindent
61, 86. sāgamo se.\par \noindent
62, 206. saṃsāsvekavacanesu ca.\par \noindent
63, 217. etimāsami.\par \noindent
64, 216. tassā vā.\par \noindent
65, 215. tato sassa ssāya.\par \noindent
66, 205. gho rassaṃ.\par \noindent
67, 229. no ca dvādito naṃmhi.\par \noindent
68, 184. amā pato smiṃsmānaṃ vā.\par \noindent
69, 186. ādito o ca.\par \noindent
70, 30. jhalānamiyuvā sare vā.\par \noindent
71, 505. yavakārā ca.\par \noindent
72, 185. pasaññassa ca.\par \noindent
73, 174. gāva se.\par \noindent
74, 169. yosu ca.\par \noindent
75, 170. avamhi ca.\par \noindent
76, 171. āvassu vā.\par \noindent
77, 175. tato namaṃ patimhā lutte ca samāse.\par \noindent
78, 3. osare ca.\par \noindent
79, 46. tabbiparītūpapade byañjane ca.\par \noindent
80, 173. goṇa naṃmhi vā.\par \noindent
81, 172. suhināsu ca.\par \noindent
82, 149. aṃmo niggahitaṃ jhalapehi.\par \noindent
83, 67. saralopo’ mādesa paccayādimhi saralope tu pakati.\par \noindent
84, 144. aghorassamekavacanayosvapi ca.\par \noindent
85, 150. na sismimanapuṃsakāni.\par \noindent
86, 227. ubhādito naminnaṃ.\par \noindent
87, 231. iṇṇamiṇṇannaṃ tīhi saṅkhyāhi.\par \noindent
88, 147. yosu katanikāralopesu dīghaṃ.\par \noindent
89, 87. sunaṃhisu ca.\par \noindent
90, 252. pañcādīnamattaṃ.\par \noindent
91, 194. patissinīmhi.\par \noindent
92, 100. ntussanto yosu ca.\par \noindent
93, 106. sabbassa vā aṃsesu.\par \noindent
94, 105. simhi vā.\par \noindent
95, 145. aggissini.\par \noindent
96, 148. yosvakatarasso jho.\par \noindent
97, 156. vevosu lo ca.\par \noindent
98, 186. mātulādīnamānattamīkāre.\par \noindent
99, 81. smāhismiṃnaṃmhābhimhivā.\par \noindent
100, 214. na timehi katākārehi.\par \noindent
101, 80. suhisvakāro e.\par \noindent
102, 202. sabbanāmānaṃ naṃmhi ca.\par \noindent
103, 79. ato nena.\par \noindent
104, 66. so.\par \noindent
105,.. so vā.\par \noindent
106, 313. dīghorehi.\par \noindent
107, 69. sabbayonīnamāe.\par \noindent
108, 90. smāsmiṃnaṃ vā.\par \noindent
109, 304. āya catutthekavacanassa tu.\par \noindent
110, 201. tayo neva ca sabbanāmehi.\par \noindent
111, 179. ghato nādīnaṃ.\par \noindent
112, 183. pato yā.\par \noindent
113, 132. sakhato gasse vā.\par \noindent
114, 178. ghate ca.\par \noindent
115, 181. na ammādito.\par \noindent
116, 157. akatarassā lato yvālapanassa vevo.\par \noindent
117, 124. jhalato sassano vā.\par \noindent
118, 146. ghapato ca yonaṃ lopo.\par \noindent
119, 155. lato vokāro ca.\par \noindent
120, 243. amhassa mamaṃ sapibhattissa se.\par \noindent
121, 233. mayaṃ yomhi paṭhame.\par \noindent
122, 99. ntussa nto.\par \noindent
123, 103. ntassa se vā.\par \noindent
124, 98. ā simhi.\par \noindent
125, 198. aṃ napuṃsake.\par \noindent
126, 101. avaṇṇā ca ge.\par \noindent
127, 102. to ti tā sa smiṃ nāsu.\par \noindent
128, 104. naṃmhi taṃ vā.\par \noindent
129, 222. imassidamaṃsisu napuṃsake.\par \noindent
138, 225. amussāduṃ.\par \noindent
131,.. itthipumanapuṃsakasaṅkhyaṃ.\par \noindent
132, 228. yosu dvinnaṃ dve ca.\par \noindent
133, 230. ti catunnaṃ tisso catasso tayo cattāro tīṇi cattāri.\par \noindent
134, 251. pañcādīnamakāro.\par \noindent
135, 118. rājassa rañño rājino se.\par \noindent
136, 119. raññaṃ naṃmhi vā.\par \noindent
137, 116. nāmhiraññā vā.\par \noindent
138, 121. smiṃmhi raññe rājini.\par \noindent
139, 245. tumhākaṃ tayimayi.\par \noindent
140, 232. tvamahaṃ simhi ca.\par \noindent
141, 241. tava mamase.\par \noindent
142, 242. tuyhaṃ mayhañca.\par \noindent
143, 235. taṃ mamaṃmhi.\par \noindent
144, 234. tavaṃ mamañca navā.\par \noindent
145, 238. nāmhī tayā mayā.\par \noindent
146, 236. tumhassa tuvaṃ tvamaṃmhi.\par \noindent
147, 246. padato dutiyā catutthī chaṭṭhīsu vono.\par \noindent
148, 247. temekavacanesu ca.\par \noindent
149, 148. na aṃmhi.\par \noindent
150, 249. vā tatiye ca.\par \noindent
151, 250. bahuvacanesu vono.\par \noindent
152, 236. pumantassā simhi.\par \noindent
153, 138. amālapanekavacane.\par \noindent
154,.. samāse ca vibhāsā.\par \noindent
155, 137. yosvāno.\par \noindent
156, 142. āne smiṃmhi vā.\par \noindent
157, 140. hivibhattimhi ca.\par \noindent
158, 143. susmimā vā.\par \noindent
156, 139. unāmhi ca.\par \noindent
160, 167. a kammantassa ca.\par \noindent
161, 244. tumha’mhehi namākaṃ.\par \noindent
162, 237. vā yvappaṭhamo.\par \noindent
163, 240. sassaṃ.\par \noindent
164, 200. sabbanāma’kārate paṭhamo.\par \noindent
165, 208. dvandaṭṭhā vā.\par \noindent
166, 209. nāññaṃ sabbanāmikaṃ.\par \noindent
167, 210. bahubbīhimhi ca.\par \noindent
168, 203. sabbato naṃ saṃ sānaṃ.\par \noindent
169, 117. rājassa rāju sunaṃhisu ca.\par \noindent
170, 220. sabbassimasse vā.\par \noindent
171, 219. animi nāmhi ca.\par \noindent
172, 218. anapuṃsakassāyaṃ simhi.\par \noindent
173, 223. amussa mo saṃ.\par \noindent
174, 211. etatesaṃ to.\par \noindent
175, 212. tassa vā nattaṃ sabbattha.\par \noindent
176, 213. sasmāsmiṃsaṃsāsvattaṃ.\par \noindent
177, 221. imasaddassa ca.\par \noindent
178, 22. sabbato ko.\par \noindent
179, 204. yapato smiṃsānaṃ saṃsā.\par \noindent
180, 207. netāhi smimāya yā.\par \noindent
181, 95. manogaṇādito smiṃnānamiā.\par \noindent
182, 97. sassa co.\par \noindent
183, 48. etesamo lope.\par \noindent
184, 96. sa sare vāgamo.\par \noindent
185, 112. santasaddassa so te bo cante.\par \noindent
186, 107. simhigacchantādīnaṃ ntasaddo aṃ.\par \noindent
187, 108. sesesu ntuva.\par \noindent
188, 115. brahmatta sakha rājādito amānaṃ.\par \noindent
189, 113. syā ca.\par \noindent
190, 114. yonamāno.\par \noindent
191, 130. sakhato cāyo no.\par \noindent
192, 135. smime.\par \noindent
193, 122. brahmato gassa ca.\par \noindent
194, 131. sakhantassi no nā naṃ sesu.\par \noindent
195, 134. āro himhi vā.\par \noindent
196, 133. sunamaṃsu vā.\par \noindent
197, 125. brahmato tu smiṃ ni.\par \noindent
198, 123. uttaṃ sanāsu.\par \noindent
199, 158. satthupitādīnamā sismiṃ silopo ca.\par \noindent
200, 159. aññesvārattaṃ.\par \noindent
201, 163. vā naṃmhi.\par \noindent
202, 164. satthunattañca.\par \noindent
203, 162. usasmiṃ salopo ca.\par \noindent
204, 167. sakkamandhātādīnañca.\par \noindent
205, 160. tato yonamo tu.\par \noindent
206, 165. tato smimi.\par \noindent
207, 161. nā ā.\par \noindent
208, 166. āro rassamikāre.\par \noindent
209, 168. pitādīnamasimhi.\par \noindent
210, 239. tayātayīnaṃ takāro tvattaṃ vā.\par \noindent
211, 126. attanto hismi’manattaṃ.\par \noindent
212, 329. tato smiṃni.\par \noindent
213, 127. sassa no.\par \noindent
214, 128. smā nā.\par \noindent
215, 141. jhalato ca.\par \noindent
216, 180. ghapato smiṃ yaṃ vā.\par \noindent
217, 199. yonaṃ ni napuṃsakehi.\par \noindent
218, 196. ato niccaṃ.\par \noindent
219, 196. siṃ.\par \noindent
220, 74. sesato lopaṃ gasipi.\par \noindent
221, 282. sabbāsamāvuso pasagganipātādīhi ca.\par \noindent
222, 342. pumassa liṅgādīsu samāsesu.\par \noindent
223, 188. aṃ yamīto pasaññato.\par \noindent
224, 153. naṃ jhato katarassā.\par \noindent
225, 151. yonaṃ no.\par \noindent
226, 154. smiṃ ni.\par \noindent
227, 270. kissa kave ca.\par \noindent
228, 272. ku hiṃhaṃsu ca.\par \noindent
229, 226. sesesu ca.\par \noindent
230, 262. tratothesu ca.\par \noindent
231, 263. sabbassetassa,kāro vā.\par \noindent
232, 267. tre niccaṃ.\par \noindent
233, 264. e tothesu ca.\par \noindent
234, 265. imassi thaṃ dāni ha to dhesu ca.\par \noindent
235, 281. adhunāmhi ca.\par \noindent
236, 280. eta rahimhi.\par \noindent
237, 176. itthiyamato āpaccayo.\par \noindent
238, 187. nadādito vā ī.\par \noindent
239, 190. ṇava ṇika ṇeyya ṇa ntuhi.\par \noindent
240, 193. pati bhikkhurājīkārantehi inī.\par \noindent
241, 191. ntussa tamīkāre.\par \noindent
242, 192. bhavato bhoto.\par \noindent
243, 110. bho ge tu.\par \noindent
244, 72. akārapitādyantānamā.\par \noindent
245, 152. jhalapā rassaṃ.\par \noindent
246, 73. ākāro vā.\par \noindent
247, 261. tvādayo vibhattisaññāyo.\par \noindent
248, 260. kvaci to pañcamyatthe.\par \noindent
249, 266. tra tha sattamiyā sabbanāmehi.\par \noindent
250, 268. sabbato dhi.\par \noindent
251, 269. kiṃ smā vo.\par \noindent
252, 271. hiṃ haṃ hiñcanaṃ.\par \noindent
253, 273. tamhā ca.\par \noindent
254, 274. imasmā ha dhā ca.\par \noindent
255, 275. yato hiṃ.\par \noindent
256,.. kāle.\par \noindent
257, 279. kiṃsabbaññekayakuhidādācanaṃ.\par \noindent
258, 278. tamhā dāni ca.\par \noindent
259, 279. imasmā rahi dhunā dāni ca.\par \noindent
260, 277. sabbassa so dāmhi vā.\par \noindent
261, 369. avaṇṇo ye lopañca.\par \noindent
262, 391. vuḍḍhassa jo iyiṭṭhesu.\par \noindent
263, 392. pasatthassa so ca.\par \noindent
264, 393. antikassa nedo.\par \noindent
265, 394. bāḷhassa sādho.\par \noindent
266, 395. appassa kaṇa.\par \noindent
267, 396. yuvānañca.\par \noindent
268, 397. vantumantu vīnañca lopo.\par \noindent
269, 401. yavataṃ ta la ṇa dakārānaṃ byañjanāni ca la ña ja kārattaṃ.\par \noindent
270, 120. amha tumhanturāja brahmatta sakhasatthu pitādīhismā nāva.\par \noindent
271, 88, 308. yasmā dapeti bhayamādatte vā tadapādānaṃ.\par \noindent
272, 309. dhātunā mānamupasaggayogādvīsvapi ca.\par \noindent
273, 310. rakkhaṇatthānamicchitaṃ.\par \noindent
274, 311. yena vā’ dassanaṃ.\par \noindent
275, 312. dūranti kaddha kāla nimmāna tvālopadisāyogavibhattārappayogasuddhappamocana hetu vivittappamāṇa pubbayogabandhana guṇavacana pañha kathana thokākattūsu ca.\par \noindent
276, 84, 302. yassa dātukāmo rocate dhārayate vā taṃ sampadānaṃ.\par \noindent
277, 303. silāgha hanu ṭhā sapa dhāra piha kudha duhissosūya rādhikkha paccāsuṇa anupatigiṇa pubbakattārocanattha tadattha tumatthālamattha maññānādarappāṇini, gatyatthakammani, āsīsattha sammuti bhiyya sattamyatthesu ca.\par \noindent
278, 93, 320. yodhāro tamokāsaṃ.\par \noindent
279, 82, 292. yena vā kayirate taṃ karaṇaṃ.\par \noindent
280, 75, 285. yaṃ karoti taṃ kammaṃ.\par \noindent
281, 77, 294. yo karoti sa kattā.\par \noindent
282, 295. yo kāreti sa hetu.\par \noindent
283, 91, 316. yassa vā pariggaho taṃ sāmī.\par \noindent
284, 65, 283. liṅgatthe paṭhamā.\par \noindent
285, 70. ālapane ca.\par \noindent
286, 83, 291. karaṇe tatiyā.\par \noindent
287, 299. sahādiyoge ca.\par \noindent
288, 78, 293. kattari ca.\par \noindent
289, 297. hetvatthe ca.\par \noindent
290, 298. sattamyatthe ca.\par \noindent
291, 299. yenaṅgavikāro.\par \noindent
292, 300. visesane ca.\par \noindent
293, 85, 301. sampadāne catutthī.\par \noindent
294, 305. namoyogādīsvapi ca.\par \noindent
295, 89, 307. apādāne pañcamī.\par \noindent
296, 314. kāraṇatthe ca.\par \noindent
297, 76, 284. kammatthe dutiyā.\par \noindent
298, 287. kāladdhānamaccantasaṃyoge.\par \noindent
299, 288. kammappavadhanīyayutte.\par \noindent
300, 286. gati buddhi bhuja paṭha hara kara sayādīnaṃ kārite vā.\par \noindent
301, 92, 315. sāmismiṃ chaṭṭhī.\par \noindent
302, 94, 319. okāse sattamī.\par \noindent
303, 321. sāmissarādhipati dāyāda sakkhīpatibhū pasutakusalehi ca.\par \noindent
304, 322. niddhāraṇe ca.\par \noindent
305, 323. anādare ca.\par \noindent
306, 289. kvaci dutiyā chaṭṭhīnamatthe.\par \noindent
307, 290. tatiyāsattamīnañca.\par \noindent
308, 317. chaṭṭhī ca.\par \noindent
309, 318. dutiyāpañcamīnañca.\par \noindent
310, 324. kamma karaṇa nimittatthesu sattamī.\par \noindent
311, 325. sampadāne ca.\par \noindent
312, 326. pañcamyatthe ca.\par \noindent
313, 327. kālabhāvesu ca.\par \noindent
314, 328. upa’jhādhikissaravacane.\par \noindent
315, 329. maṇḍitu’ssukkesu tatiyā.\par \noindent
316, 331. nāmānaṃ samāso yuttattho.\par \noindent
317, 332. tesaṃ vibhattiyo lopā ca.\par \noindent
318, 333. pakati cassa sarantassa.\par \noindent
319, 330. upasagganipātapubbako abyayībhāvo.\par \noindent
320, 335. so napuṃsakaliṅgo.\par \noindent
321, 349. digussekattaṃ.\par \noindent
322, 359. tathā dvande pāṇi tūriya yoggasenaṅgakhuddajantuka vividhaviruddha visabhāgatthādīnañca.\par \noindent
323, 360. vibhāsā rukkha tiṇa pasukha na dhañña janapadādīnañca.\par \noindent
324, 339. dvipade tulyādhikaraṇe kammadhārayo.\par \noindent
325, 348. saṅkhyāpubbo digu.\par \noindent
326, 341. ubhe tappurisā.\par \noindent
327, 351. amādayo parapadebhi.\par \noindent
328, 352. aññapadatthesu bahubbīhi.\par \noindent
329, 357. nāmānaṃ samuccayo dvando.\par \noindent
330, 340. mahataṃ mahā tulyādhikaraṇe pade.\par \noindent
331, 353. itthiyaṃ bhāsitapumitthī pumāva ce.\par \noindent
332, 343. kammadhārayasaññe ca.\par \noindent
333, 344. attaṃ nassa tappurise.\par \noindent
334, 345. sare an.\par \noindent
335, 346. kada kussa.\par \noindent
336, 347. kā’ppatthesu ca.\par \noindent
337, 350. kvaci samāsantagatānamakāranto.\par \noindent
338, 356. nadimhā ca.\par \noindent
339, 358. jāyāya tudaṃ jāni patimhi.\par \noindent
340, 355. dhanumhā ca.\par \noindent
341, 336. aṃ vibhattīnamakārantā abyayībhāvā.\par \noindent
342, 337. saro rasso napuṃsake.\par \noindent
343, 338. aññasmā lopo ca.\par \noindent
344, 361. vāṇa’pacce.\par \noindent
345, 366. ṇāyana ṇāna vacchādito.\par \noindent
346, 367. ṇeyyo kattikādīhi.\par \noindent
347, 368. ato ṇi vā.\par \noindent
348, 371. ṇavo’ pakvādīhi.\par \noindent
349, 372. ṇera vidhavādito.\par \noindent
350, 373. yena vā saṃsaṭṭhaṃ tarati carati vahati ṇiko.\par \noindent
351, 374. tamadhīte tenakatādi sannidhāna niyoga sippa bhaṇḍa jīvikatthesu ca.\par \noindent
352, 376. ṇa rāgā tasse damaññatthesu ca.\par \noindent
353, 378. jātādīnamimiyā ca.\par \noindent
354, 379. samūhatthe kaṇa ṇā.\par \noindent
355, 380. gāma jana bandhu sahāyādīhitā.\par \noindent
356, 381. tadassa ṭhānamiyo ca.\par \noindent
357, 382. upamatthāyitattaṃ.\par \noindent
358, 383. tannissitatthe lo.\par \noindent
359, 384. ālu tabbahule.\par \noindent
360, 387. ṇya tta tā bhāve tu.\par \noindent
361, 388. ṇa visamādīhi.\par \noindent
362, 389. ramaṇīyādito kaṇ.\par \noindent
363, 390. visese taratamisikiyiṭṭhā.\par \noindent
364, 398. tadassatthīti vī ca.\par \noindent
365, 399. tapādito sī.\par \noindent
366, 400. daṇḍādito ikaī.\par \noindent
367, 401. madhvādito ro.\par \noindent
368, 402. guṇādito vantu.\par \noindent
369, 403. satyādīhi mantu.\par \noindent
370, 405. saddhādito ṇa.\par \noindent
371, 404. āyussukārāsa mantumhi.\par \noindent
372, 385. tappakativacane mayo.\par \noindent
373, 406. saṅkhyāpūraṇe mo.\par \noindent
374, 408. sa chassavā.\par \noindent
375, 412. ekādito dasassī.\par \noindent
376, 257. dase so niccañca.\par \noindent
377,.. ante niggahitañca.\par \noindent
378, 414. ti ca.\par \noindent
379, 258. la darānaṃ.\par \noindent
380, 255. vīsati dasesu bā dvissa tu.\par \noindent
381, 254. ekādito dassa ra saṅkhyāne.\par \noindent
382, 259. aṭṭhādito ca.\par \noindent
383, 253. dvekaṭṭhānamākāro vā.\par \noindent
384, 407. catucchehī tha ṭhā.\par \noindent
385, 409. dvitīhi tiyo.\par \noindent
386, 410. tiye dutāpi ca.\par \noindent
387, 411. tesamaḍḍhūpapadena aḍḍhuḍḍha divaḍḍha diyaḍḍhaḍḍhatiyā.\par \noindent
388, 68. sarūpānamekasesvasakiṃ.\par \noindent
389, 413. gaṇane dasassa dviticatupañcachasattaaṭṭhanavakānaṃ vī ti cattāra paññā cha sattāsanavā yosu, yonañcīsamāsaṃ ṭhi ri tī tuti.\par \noindent
390, 256. catūpapadassa lopo tuttarapadādicassa cucopi navā.\par \noindent
391, 423. yadanupapannā nipātanā sijjhanti.\par \noindent
392, 418. dvādito ko’nekattheca.\par \noindent
393, 415. dasadasakaṃ sataṃ dasakānaṃ sataṃ sahassañca yomhi.\par \noindent
394, 416. yāva taduttari dasaguṇitañca.\par \noindent
395, 417. sakanāmehi.\par \noindent
396, 363. tesaṃ ṇo lopaṃ.\par \noindent
397, 420. vibhāge dhā ca.\par \noindent
398, 421. sabbanāmehi pakāravacane tu thā.\par \noindent
399, 422. kimimehi thaṃ.\par \noindent
400, 364. vuddhādisarassa vā’saṃyogantassa saṇe ca.\par \noindent
401, 375. māyūnamāgamo ṭhāne.\par \noindent
402, 377. āttañca.\par \noindent
403, 354. kvacādimajjhuttarānaṃ dīgharassā paccayesu ca.\par \noindent
404, 370. tesu vuddhilopāgamavikāraviparītādesā ca.\par \noindent
405, 365. ayuvaṇṇānañcāyo vuddhi.\par \noindent
406, 429. atha pubbāni vibhattīnaṃ cha parassapadāni.\par \noindent
407, 439. parāṇyattanopadāni.\par \noindent
408, 431. dve dve paṭhama majjhimuttamapurisā.\par \noindent
409, 441. sabbesamekābhidhāne paro puriso.\par \noindent
410, 432. nāmamhi payujjamānepi tulyādhikaraṇe paṭhamo.\par \noindent
411, 436. tumhe majjhimo.\par \noindent
412, 437. amhe uttamo.\par \noindent
413, 427. kāle.\par \noindent
414, 428. vattamānā paccuppanne.\par \noindent
415, 451. āṇatyā siṭṭhe’nuttakāle pañcamī.\par \noindent
416, 454. anumatiparikappatthesu sattamī.\par \noindent
417, 460. apaccakkhe parokkhātīte.\par \noindent
418, 456. hiyyopabhuti paccakkhe hiyyattanī.\par \noindent
419, 469. samīpe’jjatanī.\par \noindent
420, 471. māyoge sabbakāle ca.\par \noindent
421, 473. anāgate bhavissantī.\par \noindent
422, 475. kriyātipanne’tīte kālātipatti.\par \noindent
423, 426. vattamānā ti anti, si tha, mi ma, te ante, se vhe, e mhe.\par \noindent
424, 450. pañcamī tu antu, hi tha, mi ma, taṃ antaṃ, ssu vho, e āmase.\par \noindent
425, 453. sattamī eyya eyyuṃ, eyyā si eyyā tha, eyyāmi eyyāma, etha eraṃ, etho eyyāvho, eyyaṃ eyyāmhe.\par \noindent
426, 459. parokkhā a u e ttha, aṃ mha, ttha re, ttho vho, iṃ mhe.\par \noindent
427, 455. hiyyattanī ā ū, o ttha, aṃ mhā, ttha tthuṃ, se vhaṃ, iṃ mhase.\par \noindent
428, 468. ajjatanī ī uṃ, o ttha, iṃ mhā, ā ū, se vhaṃ, aṃ mhe.\par \noindent
429, 472. bhavissantī ssati ssanti, ssasi ssatha, ssāmi ssāma, ssate ssante, ssase ssavhe, ssaṃ ssāmhe.\par \noindent
430, 373. kālātipatti ssā ssaṃsu, sse ssatha, ssaṃ ssāmhā, ssatha ssisu, ssase ssavhe, ssiṃ ssāmhase.\par \noindent
431, 458. hiyyattanī sattamī pañcamī vattamānā sabbadhātukaṃ.\par \noindent
432, 362. dhātuliṅgehi parā paccayā.\par \noindent
433, 528. tija gupa kita māne hi kha cha sā vā.\par \noindent
434, 534. bhuja ghasa hara su pādīhi tumicchatthesu.\par \noindent
435, 536. āya nāmato kattūpamānādācāre.\par \noindent
436, 537. īyūpamānā ca.\par \noindent
437, 538. nāmamhā’tticchatthe.\par \noindent
438, 540. dhātūhi ṇe ṇaya ṇāpe ṇāpayā kāritāni hetvatthe.\par \noindent
439, 539. dhāturūpe nāmasmā ṇayo ca.\par \noindent
440, 445. bhāvakammesu yo.\par \noindent
441, 447. tassa cavaggayakāravakārattaṃ sadhātvantassa.\par \noindent
442, 448. ivaṇṇāgamo vā.\par \noindent
443, 449. pubbarūpañca.\par \noindent
444, 501. tathā kattari ca.\par \noindent
445, 433. bhūvādito a.\par \noindent
446, 509. rudhādito niggahitapubbañca.\par \noindent
447, 510. divādito yo.\par \noindent
448, 512. svādito ṇu ṇā uṇā ca.\par \noindent
449, 513. kiyādito nā.\par \noindent
450, 517. gahādito ppa ṇhā.\par \noindent
451, 520. tanādito oyirā.\par \noindent
452, 525. curādito, ṇe ṇayā.\par \noindent
453, 444. attanopadāni bhāve ca kammani.\par \noindent
454, 440. kattari ca.\par \noindent
455, 530. dhātuppaccayehi vibhattiyo.\par \noindent
456, 430. kattari parassapadaṃ.\par \noindent
457, 424. bhūvādayo dhātavo.\par \noindent
458, 461. kvacādivaṇṇānamekassarānaṃ dvebhāvo.\par \noindent
459, 462. pubbo’bbhāso.\par \noindent
460, 506. rasso.\par \noindent
461, 464. dutiyacatutthānaṃ paṭhamatatiyā.\par \noindent
462, 476. kavaggassa cavaggo.\par \noindent
463, 532. mānakitānaṃ vatattaṃ vā.\par \noindent
464, 504. hassa jo.\par \noindent
465, 463. antassivaṇṇākāro vā.\par \noindent
466, 489. niggahitañca.\par \noindent
467, 533. tato pāmānānaṃ vā maṃ sesu.\par \noindent
468, 462. ṭhā tiṭṭho.\par \noindent
469, 494. pā pivo.\par \noindent
470, 514. ñāssa jā jaṃ nā.\par \noindent
471, 483. disassa passa dissa dakkhā vā.\par \noindent
472, 531. byañjanantassa co chapaccayesu ca.\par \noindent
473, 529. ko khe ca.\par \noindent
474, 535. harassa gī se.\par \noindent
475, 565. brūbhūnamāhabhūvā parokkhāyaṃ.\par \noindent
476, 442. gamissanto ccho vā sabbāsu.\par \noindent
477, 479. vacassajjatanimhi makāro o.\par \noindent
478, 438. akāro dīghaṃ himimesu.\par \noindent
479, 452. hi lopaṃ vā.\par \noindent
480, 490. hotissarehohe bhavissantimhi ssassa ca.\par \noindent
481, 524. karassa sapaccayassa kāho.\par \noindent
482, 508. dādantassaṃ mimesu.\par \noindent
483, 527. asaṃyogantassa vuddhi kārite.\par \noindent
484, 542. ghaṭādīnaṃ vā.\par \noindent
485, 434. aññesu ca.\par \noindent
486, 543. guha dusānaṃ dīghaṃ.\par \noindent
487, 478. vaca vasa vahādīnamukāro vassa ye.\par \noindent
488, 481. ha vipariyayo lo vā.\par \noindent
489, 519. gahassa ghe ppe.\par \noindent
490, 518. halo po ṇhāmhi.\par \noindent
491, 523. karassa kāsattamajjatanimhi.\par \noindent
492, 499. asasmā mimānaṃ mhimhā’ ntalopo ca.\par \noindent
493, 498. thassa tthattaṃ.\par \noindent
494, 495. tissa tthittaṃ.\par \noindent
495, 500. tussa tthuttaṃ.\par \noindent
496, 497. simhi ca.\par \noindent
497, 477. labhasmā ī iṃnaṃ ttha tthaṃ.\par \noindent
498, 480. kusasmā dī cchi.\par \noindent
499, 507. dādhātussa dajjaṃ.\par \noindent
500, 486. vadassa vajjaṃ.\par \noindent
501, 443. gamissa ghammaṃ.\par \noindent
502, 493. yamhi dā dhā mā ṭhā hā pā mahamathādīnamī.\par \noindent
503, 485. yajassādissi.\par \noindent
504, 470. sabbato uṃ iṃsu.\par \noindent
505, 482. jara marānaṃ jīra jiyya miyyā vā.\par \noindent
506, 496. sabbatthā’ sassādilopo ca.\par \noindent
507, 501. asabbadhātuke bhū.\par \noindent
508, 515. eyyassa ñāto iyā ñā.\par \noindent
509, 516. nāssa lopo yakārattaṃ.\par \noindent
510, 487. lopañcettamakāro.\par \noindent
511, 521. uttamokāro.\par \noindent
512, 522. karassākāro ca.\par \noindent
513, 435. o ava sare.\par \noindent
514, 491. e aya.\par \noindent
515, 541. te āvāyā kārite.\par \noindent
516, 466. ikārāgamo asabbadhātukamhi.\par \noindent
517, 488. kvaci dhātuvibhattipaccayānaṃ dīghaviparītādesalopāgamā ca.\par \noindent
518, 446. attanopadāni parassapadattaṃ.\par \noindent
519, 457. akārāgamo hiyyattanīajjatanīkālātipattīsu.\par \noindent
520, 502. brūto ī timhi.\par \noindent
521, 425. dhātussanto lopo’nekasarassa.\par \noindent
522, 476. isuyamūnamanto ccho vā.\par \noindent
523, 526. kāritānaṃ ṇo lopaṃ.\par \noindent
524, 561. dhātuyā kammādimhi ṇo.\par \noindent
525, 565. saññāyama nu.\par \noindent
526, 567. pure dadā ca iṃ.\par \noindent
527, 568. sabbato ṇvu tvāvī vā.\par \noindent
528, 577. visa ruja padādito ṇa.\par \noindent
529, 580. bhāve ca.\par \noindent
530, 584. kvi ca.\par \noindent
531, 589. dharādīhi rammo.\par \noindent
532, 590. tassīlādīsu ṇītvāvī ca.\par \noindent
533, 591. sadda ku dha cala maṇḍattharudhādīhi yu.\par \noindent
534, 562. pārādigamimhā rū.\par \noindent
535, 593. bhikkhādito ca.\par \noindent
536, 594. hanatyādīnaṃ ṇuko.\par \noindent
537, 566. nu niggahitaṃ padante.\par \noindent
538, 595. saṃhanāññāya vā ro gho.\par \noindent
539, 558. ramhi ranto rādi no.\par \noindent
540, 545. bhāvakammesu tabbānīyā.\par \noindent
541, 552. ṇyo ca.\par \noindent
542, 557. karamhā ricca.\par \noindent
543, 555. bhūto’bba.\par \noindent
544, 556. vada mada gamu yuja garahākārādīhi jja mma gga yheyyā gāro vā.\par \noindent
545, 548. te kiccā.\par \noindent
546, 562. aññe kita.\par \noindent
547, 596. nandādīhi yu.\par \noindent
548, 597. kattukaraṇapadesesu ca.\par \noindent
549, 550. rahādito ṇa.\par \noindent
550, 546. ṇādayo tekālikā.\par \noindent
551, 598. saññāyaṃ dā dhāto i.\par \noindent
552, 609. ti kita cāsiṭṭhe.\par \noindent
553, 599. itthiyamatiyavo vā.\par \noindent
554, 601. karato ririya.\par \noindent
555, 612. atīte tatavantutāvī.\par \noindent
556, 622. bhāvakammesu ta.\par \noindent
557, 606. budhagamāditthe kattari.\par \noindent
558, 602. jito ina sabbattha.\par \noindent
559, 603. supato ca.\par \noindent
560, 604. īsaṃdusūhi kha.\par \noindent
561, 636. icchatthesu samānakattukesu tave tuṃ vā.\par \noindent
562, 638. arahasakkādīsu ca.\par \noindent
563, 639. pattavacane alamatthesu ca.\par \noindent
564, 640. pubbakāle’ kakattukānaṃ tuna tvāna tvā vā.\par \noindent
565, 646. vattamāne mānantā.\par \noindent
566, 574. sāsādīhi ratthu.\par \noindent
567, 575. pātito ritu.\par \noindent
568, 576. mānādīhi rātu.\par \noindent
569, 610. āgamā tuko.\par \noindent
570, 611. bhabbe ika.\par \noindent
571, 624. paccayādaniṭṭhā nipātanā sijjhanti.\par \noindent
572, 625. sāsa disato tassa riṭṭho ca.\par \noindent
573, 626. sādisanta puccha bhanja hansādīhi ṭṭho.\par \noindent
574, 613. vasato uṭṭha.\par \noindent
575, 614. vassa vā vu.\par \noindent
576, 607. dhaḍhabhahehi dhaḍhā ca.\par \noindent
577, 628. bhanjato ggo ca.\par \noindent
578, 560. bhujādīnamanto no dvi ca.\par \noindent
579, 629. vaca vāvu.\par \noindent
580, 630. gupādīnañca.\par \noindent
581, 616. tarādīhi iṇṇo.\par \noindent
582, 631. bhidādito inna anna īṇā vā.\par \noindent
583, 617. susa paca sakato kkha kkā ca.\par \noindent
584, 618. pakkamādīhi nto ca.\par \noindent
585, 619. janādīnamā timhi ca.\par \noindent
586, 600. gama khana hana ramādīnamanto.\par \noindent
587, 632. rakāro ca.\par \noindent
588, 620. ṭhāpānami ī ca.\par \noindent
589, 621. hantehi ho hassa ḷo vā adahanahānaṃ.\par \noindent
590, 579. ṇamhiranjassa jo bhāvakaraṇesu.\par \noindent
591, 544. hanassa ghāto.\par \noindent
592, 503. vadho vā sabbattha.\par \noindent
593, 564. ākārantānamāyo.\par \noindent
594, 582. pura samupa parīhi karotissa kha kharā vā tapaccayesu ca.\par \noindent
595, 637. tave tunādīsu kā.\par \noindent
596, 551. gama khana hanādīnaṃ tuṃ tabbādīsu na.\par \noindent
597, 641. sabbehi tunādīnaṃ yo.\par \noindent
598, 643. canantehi raccaṃ.\par \noindent
599, 644. disā svāna svāntalopo ca.\par \noindent
600, 645. mahadabhehi mma yha jja bbha ddhā ca.\par \noindent
601, 334. taddhitasamāsakitakā nāmaṃ vā’ tave tunādīsu ca.\par \noindent
602, 6. dumhi garu.\par \noindent
603, 7. dīgho ca.\par \noindent
604, 684. akkharehi kāra.\par \noindent
605, 647. yathāgamamikāro.\par \noindent
606, 642. dadhantato yo kvaci.\par \noindent
607, 578. niggahita saṃyogādi no.\par \noindent
608, 623. sabbattha ge gī.\par \noindent
609, 484. sadassa sīdattaṃ.\par \noindent
610, 627. yajassa sarassi ṭṭhe.\par \noindent
611, 608. hacatutthānamantānaṃ do dhe.\par \noindent
612, 615. ḍo ḍhakāre.\par \noindent
613, 583. gahassa ghara ṇe vā.\par \noindent
614, 581. dahassa do ḷaṃ.\par \noindent
615, 586. dhātvantassa lopo kvimhi.\par \noindent
616, 587. vidante ū.\par \noindent
617, 633. namakarānamantānaṃ niyuttatamhi.\par \noindent
618, 571. na kagattaṃ cajā ṇvumhi.\par \noindent
619, 573. karassa ca tattaṃ tusmiṃ.\par \noindent
620, 549. tuṃ tuna tabbesu vā.\par \noindent
621, 553. kāritaṃ viya ṇānubandho.\par \noindent
622, 570. anakā yu-ṇvūnaṃ.\par \noindent
623, 554. kagā cajānaṃ.\par \noindent
624, 563. kattari kit.\par \noindent
625, 605. bhāvakammesu kicca kta khatthā.\par \noindent
626, 634. kammani dutiyāyaṃ kto.\par \noindent
627, 652. khyādīhi man ma ca to vā.\par \noindent
628, 653. samādīhi tha mā.\par \noindent
629, 569. gahassu’padhasse vā.\par \noindent
630, 654. masussa sussa cchara ccherā.\par \noindent
631, 655. āpubbacarassa ca.\par \noindent
632, 656. ala kala salehi layā.\par \noindent
633, 657. yāṇa lāṇā.\par \noindent
634, 658. mathissa thassa lo ca.\par \noindent
635, 559. pesātisaggapattakālesukiccā.\par \noindent
636, 659. avassakā’dhamiṇesu ṇī ca.\par \noindent
637,.. arahasakkādīhi tuṃ.\par \noindent
638, 660. vajādīhipabbajjādayo nippajjante.\par \noindent
639, 585. kvilopo ca.\par \noindent
640,.. sacajānaṃ kagā ṇānubandhe.\par \noindent
641, 572. nudādīhi yuṇvūnamanānanākānanakā sakāritehi ca.\par \noindent
642, 588. i ya ta ma ki e sāna’mantassaro dīghaṃ kvaci disassa guṇaṃ do raṃ sakkhī ca.\par \noindent
643, 635. bhyādīhi mati budhi pūjādīhi ca kto.\par \noindent
644, 661. vepu sī dava vamu ku dā bhū hvādīhi thu ttima ṇimā nibbatte.\par \noindent
645, 662. akkose namhāni.\par \noindent
646, 419. ekādito sakissa kkhattuṃ.\par \noindent
647, 663. sunassunassoṇa vānuvānūnunakhunānā.\par \noindent
648, 664. taruṇassa susu ca.\par \noindent
649, 665. yuvassuvassuvuvānunūnā.\par \noindent
650, 651. kāle vattamānātīte ṇvādayo.\par \noindent
651, 647. tavissati gamādīhi ṇī ghiṇ.\par \noindent
652, 648. kriyāyaṃ ṇvu tavo.\par \noindent
653, 306. bhāvavācimhi catutthī.\par \noindent
654, 649. kammani ṇo.\par \noindent
655, 650. sese ssaṃ ntumānānā.\par \noindent
656, 666. chavādīti tatraṇa.\par \noindent
657, 667. vadādīhi ṇitto gaṇe.\par \noindent
658, 668. midādīhi ttitiyo.\par \noindent
659, 669. usu ranja daṃsānaṃ daṃsassa daḍḍhoḍha ṭhā ca.\par \noindent
660, 670. sūvusānamūvusānamato tho ca.\par \noindent
661, 671. ranjudādīhi dha didda kirā kvaci jadalopo ca.\par \noindent
662, 672. paṭito hissa heraṇ hīraṇ.\par \noindent
663, 673. kaḍyādīhi ko.\par \noindent
664, 674. khādāmagamānaṃ khandha’ndha gandhā.\par \noindent
665, 675. paṭādīhyalaṃ.\par \noindent
666, 676. puthassa puthupathāmo vā.\par \noindent
667, 677. sasvādīhi tudavo.\par \noindent
668, 678. cyādīhi īvaro.\par \noindent
669, 679. munādīhi ci.\par \noindent
670, 680. vidādīhyūro.\par \noindent
671, 681. hanādīhi ṇunutavo.\par \noindent
672, 682. kuṭādīhi ṭho.\par \noindent
673, 683. manu pūra suṇādīhi ussa nusisā.\par \noindent
