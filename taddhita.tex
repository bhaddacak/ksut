\chapter{Taddhita}

This part is about what scholars call \emph{secondary derivation}. It is \emph{secondary} in the sense that the formation of a new word is derived from another nominal base rather than a root. We will learn \emph{primary derivation} (kita) in the book of kibbidhāna.

The word \pali{taddhita} itself has a mysterious origin, and it is really an old grammatical term, already used by Pāṇini. It seems to be described by him as \pali{tasmai hitam} (=? setting up for that), used as a heading in Pāṇ 5.1.5.

The main task of this topic is to recognize a number of paccayas that give derivatives new meaning. All paccayas are listed in Table \ref{tab:tadpacc}.

An important operation related to the process is vuddhi-strength transformation, mostly caused by paccayas with \pali{ṇa}. The definition of this can be found toward the end in \hyperref[sut:400]{Kacc 400} and \hyperref[sut:405]{405}.

Note that when an example is shown with its declined form, for example, in nominative case, when it is decomposed, the case vibhatti (e.g., \pali{si}) is left out to make it less distracting.

{
\begin{longtable}{%
		>{\raggedright\arraybackslash}p{0.2\linewidth}%
		>{\raggedright\arraybackslash}p{0.4\linewidth}}
\caption{Paccayas for secondary devivation}\label{tab:tadpacc}\\
\toprule
\bfseries Paccaya & \bfseries Sutta \\ \midrule
\endfirsthead
\multicolumn{2}{c}{\footnotesize\tablename\ \thetable: Paccayas for secondary devivation (contd\ldots)}\\
\toprule
\bfseries Paccaya & \bfseries Sutta \\ \midrule
\endhead
\bottomrule
\ltblcontinuedbreak{2}
\endfoot
\bottomrule
\endlastfoot
%
āyitatta & \hyperref[sut:357]{357} \\
ālu & \hyperref[sut:359]{359} \\
ika & \hyperref[sut:366]{366}, \hyperref[sut:353]{353} \\
iṭṭha & \hyperref[sut:363]{363} \\
ima & \hyperref[sut:353]{353} \\
iya & \hyperref[sut:353]{353}, \hyperref[sut:356]{356}, \hyperref[sut:363]{363} \\
isika & \hyperref[sut:363]{363} \\
ī & \hyperref[sut:366]{366}, \hyperref[sut:375]{375} \\
īya\footnote{This is found in a Thai source.} & \hyperref[sut:356]{356} \\
ka & \hyperref[sut:392]{392} \\
kaṇ & \hyperref[sut:362]{362} \\
kaṇa & \hyperref[sut:354]{354} \\
kiya & \hyperref[sut:353]{353} \\
ṭha & \hyperref[sut:384]{384} \\
ṇa & \hyperref[sut:344]{344}, \hyperref[sut:352]{352}, \hyperref[sut:354]{354}, \hyperref[sut:361]{361}, \hyperref[sut:370]{370} \\
ṇava & \hyperref[sut:348]{348} \\
ṇāna & \hyperref[sut:345]{345} \\
ṇāyana & \hyperref[sut:345]{345} \\
ṇi & \hyperref[sut:347]{347} \\
ṇika & \hyperref[sut:350]{350}, \hyperref[sut:351]{351} \\
ṇeyya & \hyperref[sut:346]{346} \\
ṇera & \hyperref[sut:349]{349} \\
ṇya & \hyperref[sut:360]{360} \\
tama & \hyperref[sut:363]{363} \\
tara & \hyperref[sut:363]{363} \\
tā & \hyperref[sut:355]{355}, \hyperref[sut:360]{360} \\
tiya & \hyperref[sut:385]{385}, \hyperref[sut:386]{386} \\
tta & \hyperref[sut:360]{360} \\
ttana & \hyperref[sut:360]{360} \\
tha & \hyperref[sut:384]{384} \\
thatthā & \hyperref[sut:398]{398} \\
thaṃ & \hyperref[sut:399]{399} \\
thā & \hyperref[sut:398]{398} \\
dhā & \hyperref[sut:397]{397} \\
ma & \hyperref[sut:373]{373} \\
mantu & \hyperref[sut:369]{369} \\
maya & \hyperref[sut:372]{372} \\
ra & \hyperref[sut:367]{367} \\
la & \hyperref[sut:358]{358} \\
vantu & \hyperref[sut:368]{368} \\
vī & \hyperref[sut:364]{364} \\
sa & \hyperref[sut:364]{364} \\
sī & \hyperref[sut:365]{365} \\
so & \hyperref[sut:397]{397} \\
\end{longtable}
}

\section{Aṭṭhamakaṇḍa}
\raggedbottom

\head{344}{344, 361. vā ṇa’pacce.}
\headtrans{Sometimes, [there is] \pali{ṇa}[-paccaya] in [the sense of] offspring.}
\transnote{This sutta came from Kāt 2.288.}
\sutdef{ṇapaccayo hoti vā tassāpaccamiccetasmiṃ atthe.}
\sutdeftrans{There is \pali{ṇa}-paccaya sometimes in the sense of one's offspring.}
\example[0]{vasiṭṭhassa apaccaṃ vāsiṭṭho (vasiṭṭha + ṇa)\\{\upshape= [He is] the offspring of Vasiṭṭha, hence \pali{Vāsiṭṭho}.}}
\example[0]{vasiṭṭhassa apaccaṃ vāsiṭṭhī\\{\upshape= [She is] the offspring of Vasiṭṭha, hence \pali{Vāsiṭṭhī}.}}
\example{vasiṭṭhassa apaccaṃ vāsiṭṭhaṃ\\{\upshape= [This family is] the offspring of Vasiṭṭha, hence \pali{Vāsiṭṭhaṃ}.}}
\transnote{Once the base has been derived, it can be of three genders depending on the signified. Other examples in the text can be understood in the same way. New learners should know that \pali{ṇa}-paccaya causes the first syllable to be in \pali{vuddhi} strength, hence \pali{vasiṭṭha} becomes \pali{vāsiṭṭha}. Other paccayas having this \pali{ṇa}-anubandha should be seen in the same way. This rule is described in \hyperref[sut:400]{Kacc 400} and \hyperref[sut:405]{405} in detail}

\boxnote{Simple analytic sentence of a secondary derivative.\\
\hspace{5mm}\bullet\ To understand clearly what a final term of the derivation really means, we have to see its analytic sentence (\pali{viggahavākya}).\\
\hspace{5mm}\bullet\ The simplest construction of the analysis can be seen in the examples above, for instance, \pali{vasiṭṭhassa apaccaṃ vāsiṭṭho}.\\
\hspace{5mm}\bullet\ To make it more understandable, or when you prefer clarity over brevity, you can put it in this way:\\
\hspace{7mm}$\triangleright$\ \pali{yo puriso vasiṭṭhassa apaccaṃ iti so vāsiṭṭho}\\
\hspace{7mm}$\triangleright$\ \pali{yā itthī vasiṭṭhassa apaccaṃ iti sā vāsiṭṭhī}\\
\hspace{7mm}$\triangleright$\ \pali{yaṃ kulaṃ vasiṭṭhassa apaccaṃ iti taṃ vāsiṭṭhaṃ}
}

\head{345}{345, 366. ṇāyanaṇāna vacchādito.}
\headtrans{[There are] \pali{ṇāyana}- and \pali{ṇāna}-paccaya after [the clan of] \pali{Vaccha} and so on.}
\sutdef{tasmā vacchādito gottagaṇato ṇāyanaṇānapaccayā honti vā tassāpaccamiccetasmiṃ atthe.}
\sutdeftrans{There are \pali{ṇāyana}- and \pali{ṇāna}-paccaya after that clan of \pali{Vaccha} and so on sometimes in the sense of one's offspring.}
\example[0]{vacchassa apaccaṃ vacchāyano/vacchāyanī/vacchāyanaṃ (vaccha + ṇāyana)}
\example[0]{vacchassa apaccaṃ vacchāno/vacchānī/vacchānaṃ (vaccha + ṇāna)}
\example[0]{sakaṭassa apaccaṃ sākaṭāyano/sākaṭāyanī/sākaṭāyanaṃ (sa\-kaṭa + ṇāyana)}
\example[0]{sakaṭassa apaccaṃ sākaṭāno/sākaṭānī/sākaṭānaṃ (sakaṭa + ṇāna)}
\example{kaccāyano/kaccāyanī/kaccāyanaṃ (kacca + ṇāyana)}
\example{kaccāno/kaccānī/kaccānaṃ (kacca + ṇāna)}
\transnote{Examples given in the text are numerous. They are derived straightforwardly in the same manner. Note that it is not all about proper noun denoting a clan's name. It can be a noun having the sense of group, e.g., \pali{corāyano, corāno, corāyanī, corānī, corāyanaṃ, corānaṃ} (one of the clan of thief).}

\newpage
\boxnote{What is \pali{anubandha}?\\
\hspace{5mm}\bullet\ Literally, \pali{anubandha} means ``a thing tagged behind.'' The term has purely technical use.\\
\hspace{5mm}\bullet\ Even though the term is used in \hyperref[sut:621]{Kacc 621} and \hyperref[sut:640]{Kacc 640}, it has never been defined clearly. In Rūpa 553, we have a short definition as \pali{anubandho appayogī} (anubandha is a thing not applied). This idea came from Kāt 3.435 (\pali{yo 'nubandho 'prayogī}).\\
\hspace{5mm}\bullet\ To put it in a more understandable way, anubandha is an operator that does something but itself gets elided, not visible after the operation (see \hyperref[sut:396]{Kacc 396}).\\
\hspace{5mm}\bullet\ A marked example is \pali{ṇa}. When it operates, it causes vuddhi strength to the target, but it is not part of the product.\\
\hspace{5mm}\bullet\ Also note that when \pali{ṇa}-paccaya leaves \pali{ṇa} in the final product without making anything vuddhi, it does not count as anubandha.
}

\head{346}{346, 367. ṇeyyo kattikādīhi.}
\headtrans{[There is] \pali{ṇeyya}-paccaya after [the clans of] \pali{Kattikā} and so on.}
\sutdef{tehi gottagaṇehi kattikādīhi ṇeyyapaccayo hoti vā tassāpaccamiccetasmiṃ atthe.}
\sutdeftrans{There is \pali{ṇeyya}-paccaya after those clans of \pali{Kattikā} and so on sometimes in the sense of one's offspring.}
\example[0]{kattikāya apaccaṃ kattikeyyo (kattikā + ṇeyya)}
\example[0]{venateyyo (vinatā + ṇeyya)}
\example[0]{rohiṇeyyo (rohiṇī + ṇeyya)}
\example[0]{gaṅgeyyo (gaṅgā + ṇeyya)}
\example[0]{kaddameyyo (kaddamā + ṇeyya)}
\example[0]{nādeyyo (nādī + ṇeyya)}
\example[0]{āleyyo (ālī + ṇeyya)}
\example[0]{āheyo (āhī + ṇeyya)}
\example[0]{kāmeyyo (kāmī + ṇeyya)}
\example[0]{soceyyo (suci + ṇeyya)}
\example[0]{sāleyyo (sālā + ṇeyya)}
\example[0]{bāleyyo (bālā + ṇeyya)}
\example[0]{māleyyo (mālā + ṇeyya)}
\example{kāleyyo (kalā + ṇeyya)}
\transnote{Note that the ancestor in this case is of a female's name, and the product of the derivation is the name of a son.}

\head{347}{347, 368. ato ṇi vā.}
\headtrans{After \pali{a}, [there is] \pali{ṇi}-paccaya sometimes.}
\sutdef{tasmā akārato ṇipaccayo hoti vā tassāpaccamiccetasmiṃ atthe.}
\sutdeftrans{There is \pali{ṇi}-paccaya after that [clan of] \pali{a}-ending sometimes in the sense of one's offspring.}
\example[0]{dakkhassa apaccaṃ dakkhi (dakkha + ṇi)}
\example[0]{duṇassa apaccaṃ doṇi (duṇa + ṇi)}
\example[0]{vāsavi (vāsava + ṇi)}
\example[0]{sakyaputti (sakyaputta + ṇi)}
\example[0]{nāṭaputti (nāṭaputta + ṇi)}
\example[0]{dāsaputti (dāsaputta + ṇi)}
\example[0]{dāsavi (dāsava + ṇi)}
\example[0]{vāruṇi (varuṇa + ṇi)}
\example[0]{gaṇḍi (gaṇḍa + ṇi)}
\example[0]{bāladevi (baladeva + ṇi)}
\example[0]{pāvaki (pāvaka + ṇi)}
\example[0]{jenadatti (jinadatta + ṇi)}
\example[0]{buddhi (buddha + ṇi)}
\example[0]{dhammi (dhamma + ṇi)}
\example[0]{saṅghi (saṅgha + ṇi)}
\example[0]{kappi (kappa + ṇi)}
\example{anuruddhi (anuruddha + ṇi)}

\head{348}{348, 371. ṇavopakvādīhi.}
\headtrans{[There is] \pali{ṇava}-paccaya after \pali{upaka} and so on.}
\sutdef{upakuiccevamādīhi ṇavapaccayo hoti vā tassāpaccamiccetasmiṃ atthe.}
\sutdeftrans{There is \pali{ṇava}-paccaya after \pali{upaku} and so on sometimes in the sense of one's offspring.}
\example[0]{upakussa apaccaṃ opakavo (upaka + ṇava)}
\example[0]{manuno apaccaṃ mānavo (manū + ṇava)}
\example[0]{bhaggussa apaccaṃ bhaggavo (bhaggu + ṇava)}
\example[0]{paṇḍussa apaccaṃ paṇḍavo (paṇḍu + ṇava)}
\example{bahussa apaccaṃ bāhavo (bahu + ṇava)}

\head{349}{349, 372. ṇera vidhavādito.}
\headtrans{[There is] \pali{ṇera}-paccaya after \pali{vidhava} and so on.}
\sutdef{tasmā vidhavādito ṇerapaccayo hoti vā tassāpaccamiccetasmiṃ atthe.}
\sutdeftrans{There is \pali{ṇera}-paccaya after that \pali{vidhava} and so on sometimes in the sense of one's offspring.}
\example[0]{vidhavāya apaccaṃ vedhavero (vidhava + ṇera)}
\example[0]{bandhukiyā apaccaṃ bandhukero (bandhukī + ṇera)}
\example[0]{samaṇassa apaccaṃ sāmaṇero (samaṇa + ṇera)}
\example[0]{sāmaṇerī/sāmaṇeraṃ (samaṇa + ṇera)}
\example[0]{nāḷikero/nāḷikerī/nāḷikeraṃ (nāḷika + ṇera)}

\head{350}{350, 373. yena vā saṃsaṭṭhaṃ tarati carati vahati ṇiko.}
\headtrans{[In the sense of] to be mixed, to cross, to travel, and to carry with something, [there is] \pali{ṇika}-paccaya.}
\sutdef{yena vā saṃsaṭṭhaṃ, yena vā tarati, yena vā carati, yena vā vahati iccetesvatthesu ṇikapaccayo hoti vā.}
\sutdeftrans{There is \pali{ṇika}-paccaya sometimes in these senses: to be mixed with something, to cross [a river] with something, to travel with something, and to carry with something.}

\bigbullet{(1) To be mixed with something}
\example[0]{tilena saṃsaṭṭhaṃ bhojanaṃ telikaṃ (tila + ṇika)\\{\upshape= Food mixed with sesame seeds, hence \pali{telika} (sesame-mixed).}}
\example[0]{goḷikaṃ (guḷa + ṇika)\\{\upshape= sugar-mixed}}
\example{ghātikaṃ (ghata + ṇika)\\{\upshape= ghee-mixed}}

\bigbullet{(2) To cross with something}
\example[0]{nāvāya taratīti nāviko (nāva + ṇika)\\{\upshape= [One] crosses with a boat, hence \pali{nāvika} (boat-crosser).}}
\example{oḷumpiko (uḷumpa + ṇika)\\{\upshape= one crossing with a raft}}

\bigbullet{(3) To travel with something}
\example[0]{sakaṭena caratīti sākaṭiko (sakaṭa + ṇika)\\{\upshape= [One] travels with a cart, hence \pali{sākaṭika} (cart-traveller).}}
\example[0]{pattiko (patta + ṇika)\\{\upshape= a footman-army (infantry)\footnote{For \pali{patta}, see \pali{pattin} in MWD.}}}
\example[0]{daṇḍiko (daṇḍī + ṇika)\\{\upshape= a traveller with a stick}}
\example[0]{dhammiko (dhamma + ṇika)\\{\upshape= a dhamma-practitioner}}
\example{pādiko (pāda + ṇika)\\{\upshape= a foot-traveller (pedestrian)}}

\bigbullet{(4) To carry with something}
\example[0]{sīsena vahatīti sīsiko (sīsa + ṇika)\\{\upshape= [One] carries [things] with the head, hence \pali{sīsika} (head-carrier).}}
\example[0]{aṃsena vahatīti aṃsiko (aṃsa + ṇika)\\{\upshape= [One] carries [things] with the shoulder, hence \pali{aṃsika} (shou\-lder-carrier).}}
\example[0]{khandhiko (khandha + ṇika)\\{\upshape= back-carrier}}
\example{aṅguliko (aṅgula + ṇika)\\{\upshape= finger-carrier}}

\head{351}{351,~374.~tamadhīte~tenakatādi~sannidhānaniyogasippa\-bhaṇḍajīvikatthesu ca.}
\headtrans{In the senses of [one] learns that, [action] done by that, there [thing is] stored, that [one] engaged in, that one's skill, that one's commodity, and that one's livelihood, also [there is \pali{ṇika}-paccaya].}
\sutdef{tamadhīte, tenakatādiatthe, tamhi sannidhānā, tattha niyutto, tamassa sippaṃ, tamassa bhaṇḍaṃ, tamassa jīvikaṃ iccetesvatthesu ca ṇikapaccayo hoti vā.}
\sutdeftrans{There is \pali{ṇika}-paccaya in these senses, i.e., [one] learns that, [action] done by that, there [thing is] stored, that [one] engaged in, that one's skill, that one's commodity, and that one's livelihood.}

\bigbullet{(1) One learns something}
\example[0]{vinayamadhīte venayiko (vinaya + ṇika)\\{\upshape= [One] learns the Vinaya, hence \pali{venayika} (Vinaya-learner).}}
\example[0]{suttantiko (suttanta + ṇika)\\{\upshape= a Suttanta-learner}}
\example[0]{ābhidhammiko (abhidhamma + ṇika)\\{\upshape= an Abhidhamma-learner}}
\example{veyyākaraṇiko (byākaraṇa + ṇika)\\{\upshape= a grammar-learner/grammarian}}
\transnote{The verb \pali{adhīyati (adhi + $\surd{i}$)} means `to study.' The form \mbox{\pali{adhīte}} is dubious, but \pali{adhīyate} looks more grammatical. However, I found \pali{adhīte} is used in Pāṇ 4.2.59, but the applied paccaya is different.}

\bigbullet{(2) An action done by something}
\example[0]{kāyena kataṃ kammaṃ kāyikaṃ (kāya + ṇika)\\{\upshape= An action was done by the body, hence \pali{kāyika} (bodily-done [action])}}
\example[0]{vācasikaṃ (vaca + ṇika)\\{\upshape= verbally-done [action]}}
\example{mānasikaṃ (mana + ṇika)\\{\upshape= mentally-done [action]}}
\transnote{Since \pali{mana} and \pali{vaca} belong to the \pali{mano}-group, they usually have \pali{sa}-insertion (\hyperref[sut:184]{Kacc 184}).}

\bigbullet{(3) A place where something is stored}
\example[0]{sarīre sannidhānā vedanā sārīrikā (sarīra + ṇika)\\{\upshape= The feeling stored in the body, hence \pali{sārīrika} (bodily-stored [feeling]).}}
\example{mānasikā (mana + ṇika)\\{\upshape= mentally-stored [feeling]}}

\bigbullet{(4) A thing that someone engaged in}
\example[0]{dvāre niyutto dovāriko (dvāra + ṇika)\\{\upshape= [One] was appointed at the gate, hence \pali{dovārika} (gate-keeper).}}
\example[0]{bhaṇḍāgāriko (bhaṇḍāgāra + ṇika)\\{\upshape= warehouse-keeper}}
\example[0]{nāgariko (nagara + ṇika)\\{\upshape= town-dweller}}
\example{nāvakammiko (navakamma + ṇika)\\{\upshape= construction-supervisor}}
\transnote{The term \pali{nāvakammika} is not found in use, but \pali{navakammika} is more common. See also \pali{navakarmika} in MWD.}

\bigbullet{(5) A skillful ability of someone}
\example[0]{vīṇā assa sippaṃ veṇiko (vīṇā + ṇika)\\{\upshape= A lute [is] a skill of that [person], hence \pali{veṇika} (lutist).}}
\example[0]{pāṇaviko (paṇava + ṇika)\\{\upshape= small drum artist}}
\example[0]{modiṅgiko (mudiṅga + ṇika)\\{\upshape= two-sided drum artist}}
\example{vaṃsiko (vaṃsa + ṇika)\\{\upshape= flutist}}

\bigbullet{(6) A commodity of someone}
\example[0]{gandho assa bhaṇḍaṃ gandhiko (gandha + ṇika)\\{\upshape= Perfume [is] a commodity of that [person], hence \pali{gandhika} (perfumer).}}
\example[0]{teliko (tela + ṇika)\\{\upshape= oil-trader}}
\example{goḷiko (guḷa + ṇika)\\{\upshape= suger-trader}}

\bigbullet{(7) Livelihood of someone}
\example[0]{urabbhaṃ hantvā jīvatīti orabbhiko (urabbha + ṇika)\\{\upshape= [One], having killed a ram, [then] subsists on [it], hence \pali{orabbhika} (ram-hunter).}}
\example[0]{māgaviko (maga + ṇika)\\{\upshape= deer-hunter}}
\example[0]{sokariko (sūkara + ṇika)\\{\upshape= pig-farmer}}
\example{sākuṇiko (sakuṇa + ṇika)\\{\upshape= bird-hunter}}

\bigbullet{(8) In other senses}
\example[0]{jālena hato jāliko (jāla + ṇika)\\{\upshape= [One] has killed with a net, hence \pali{jālika} (fisherman).}}
\example[0]{suttena bandho suttiko (sutta + ṇika)\\{\upshape= [One] has bound with a thread, hence \pali{suttika} (thread-trapper).}}
\example[0]{cāpo assa āvudho cāpiko (cāpa + ṇika)\\{\upshape= A bow is the weapon of him, hence \pali{cāpika} (bowman).}}
\example[0]{tomariko (tomara + ṇika)\\{\upshape= a spearman}}
\example[0]{muggariko (muggara + ṇika)\\{\upshape= a malletman}}
\example[0]{mosaliko (musara + ṇika)\\{\upshape= a pestleman}}
\example[0]{vāto assa ābādho vātiko (vāta + ṇika)\\{\upshape= Wind is a disease of him, hence \pali{vātika} (one ill from wind/gas).}}
\example[0]{semhiko (semha + ṇika)\\{\upshape= one ill from phlegm}}
\example[0]{pittiko (pitta + ṇika)\\{\upshape= one ill from bile}}
\example[0]{buddhe pasanno buddhiko (buddha + ṇika)\\{\upshape= [One] is devoted in the Buddha, hence \pali{buddhika} (Buddha-devotee).}}
\example[0]{dhammikaṃ (dhamma + ṇika)\\{\upshape= Dhamma-devotee}}
\example[0]{saṅghikaṃ (saṅgha + ṇika)\\{\upshape= Saṅgha-devotee}}
\example[0]{vatthena kītaṃ bhaṇḍaṃ vatthikaṃ (vattha + ṇika)\\{\upshape= A commodity was bought by a cloth, hence \pali{vatthika} (cloth-bartered article).}}
\example[0]{kumbhikaṃ (kumbha + ṇika)\\{\upshape= pot-bartered article}}
\example[0]{phālikaṃ (phala + ṇika)\\{\upshape= fruit-bartered article}}
\example[0]{kiṅkaṇikaṃ (kiṅkaṇa + ṇika)\\{\upshape= small-bell-bartered article}}
\example[0]{sovaṇṇikaṃ (suvaṇṇa + ṇika)\\{\upshape= gold-bartered article}}
\example[0]{kumbho assa parimāṇaṃ kumbhikaṃ (kumbha + ṇika)\\{\upshape= A pot [is] the measurement of that, hence \pali{kumbhikaṃ} (one pot [of measurement]).}}
\example[0]{kumbhassa rāsi kumbhikaṃ (kumbha + ṇika)\\{\upshape= A quantity of one pot, hence \pali{kumbhikaṃ} (one pot [of measurement]).}}
\example[0]{kumbhaṃ arahatīti kumbhiko (kumbha + ṇika)\\{\upshape= [One] deserves one pot [of grains], hence \pali{kumbhika} (one worthy of a pot [of grains]).}}
\example[0]{akkhena dibbatīti akkhiko (akkha + ṇika)\\{\upshape= [One] plays with a dice, hence \pali{akkhika} (dice-player/gambler).}}
\example[0]{sālākiko (salāka + ṇika)\\{\upshape= a ticket-player/lottery-player}}
\example[0]{tindukiko (tinduka + ṇika)\\{\upshape= a tinduka-seed-player}}
\example[0]{ambaphaliko (ambaphala + ṇika)\\{\upshape= a mango-seed-player}}
\example[0]{kapiṭṭhaphaliko (kapiṭṭhaphala + ṇika)\\{\upshape= a kapiṭṭha-seed-player}}
\example{nāḷikeriko (nāḷikera + ṇika)\\{\upshape= a coconut-shell-player}}

\head{352}{352, 376. ṇa rāgā tassedamaññatthesu ca.}
\headtrans{[There is] \pali{ṇa}-paccaya after \pali{rāga} and in other sense, i.e., `this of that.'}
\transnote{In an old Thai edition, this sutta is \pali{ṇa rāgā tena rattaṃ tassedamaññatthesu ca}.}
\sutdef{ṇapaccayo hoti vā rāgamhā tena rattaṃ iccetasmiṃ atthe, tassedaṃ aññatthesu ca.}
\sutdeftrans{There is \pali{ṇa}-paccaya after \pali{rāga} in the sense of `dyed with that,' and in other sense, i.e., `this of that.'}

\bigbullet{(1) In the sense of `dyed with that'}
\example[0]{kasāvena rattaṃ vatthaṃ kāsāvaṃ (kasāva + ṇa)\\{\upshape= The colth was dyed with organic brown color, hence \pali{kāsāva} (organic brown cloth).}}
\example[0]{kosumbhaṃ (kusumbha + ṇa)\\{\upshape= safflower-dyed [cloth]}}
\example[0]{hāliddaṃ (haliddā + ṇa)\\{\upshape= turmeric-dyed [cloth]}}
\example[0]{pattaṅgaṃ (pattaṅga + ṇa)\\{\upshape= red-sandal-dyed [cloth]\footnote{In the text of Myanmar or CST edition, this word is \pali{pāṭaṅgaṃ}. It looks unlikely because \pali{paṭaṅga} means grasshopper. See also \pali{pattrāṅga} in MWD.}}}
\example[0]{rattaṅgaṃ (ratta + ṇa)\\{\upshape= red-dyed [cloth]}}
\example[0]{mañjiṭṭhaṃ (mañjiṭṭha + ṇa)\\{\upshape= crimson-dyed [cloth]}}
\example{kuṅkumaṃ (kuṅkuma + ṇa)\\{\upshape= saffron-dyed [cloth]}}

\bigbullet{(2) In the sense of `this of that'}
\example[0]{sūkarassa idaṃ maṃsaṃ sokaraṃ (sūkara + ṇa)\\{\upshape= This meat [is] of a pig, hence \pali{sokara} (pork).}}
\example{māhisaṃ (mahisa + ṇa)\\{\upshape= baffalo meat}}

\bigbullet{(3) In other various senses}
\example[0]{udumbarassa avidūre pavattaṃ vimānaṃ odumbaraṃ (udumbara + ṇa)\\{\upshape= [This] mansion existed in the neighborhood of a glamorous fig tree, hence \pali{odumbara} (near-to-fig-tree [mansion]).}}
\example[0]{vidisāya avidūre nivāso vediso (vidisā + ṇa)\\{\upshape= [This] living place [is] not far from Vidisā, hence \pali{vedisa} (near-to-Vidisā [abode]).\footnote{\pali{Vidisā} is a city's name. See \pali{Vidiśā} in MWD.}}}
\example[0]{mathurāya jāto māthuro (mathurā + ṇa)\\{\upshape= [One] was born in Mathurā, hence \pali{māthura} (Mathurā-born [one]).}}
\example[0]{mathurāya āgato māthuro (mathurā + ṇa)\\{\upshape= [One] came from Mathurā, hence \pali{māthura} (from-Mathurā [one]).}}
\example{kattikāya niyutto māso kattiko (kattikā + ṇa)\\{\upshape= [This] month corresponded with Pleiades constellation, hence \pali{kattika} (the month of Kattika, last month of rainy season).}}
\transnote{Other months can be seen in the same way, hence \pali{māgasiro, phusso, māgho, phagguno, citto, vesākho, jeṭṭho, āsaḷho, sāvaṇo, bhaddo}, and \pali{assayujo}.}
\example[0]{sikkhānaṃ samūho sikkho (sikkhā + ṇa)\\{\upshape= A multitude of training, hence \pali{sikkha}.}}
\example[0]{bhikkhānaṃ samūho bhikkho (bhikkhā + ṇa)\\{\upshape= A multitude of food, hence \pali{bhikkha}.}}
\example[0]{kāpoto (kapota + ṇa)\\{\upshape= a group of pigeons}}
\example[0]{māyūro (mayūra + ṇa)\\{\upshape= a group of peacocks}}
\example[0]{kokilo (kokila + ṇa)\\{\upshape= a group of cuckoos}}
\example[0]{buddho assa devatā buddho (buddha + ṇa)\\{\upshape= The Buddha [is] a deity of that [person], hence \pali{buddha} (Buddhotheist).}}
\example[0]{bhaddo (bhadda + ṇa)\\{\upshape= a Bhadda worshipper}}
\example[0]{māro (māra + ṇa)\\{\upshape= a Māra worshipper}}
\example[0]{māhindo (mahinda + ṇa)\\{\upshape= an Indra worshipper}}
\example[0]{vessavaṇo (vessavaṇa + ṇa)\\{\upshape= a Vessavaṇa worshipper}}
\example[0]{yāmo (yama + ṇa)\\{\upshape= a Yama worshipper}}
\example[0]{somo (soma + ṇa)\\{\upshape= a Soma worshipper}}
\example[0]{nārāyaṇo (nārāyaṇa + ṇa)\\{\upshape= a Nārāyaṇa worshipper}}
\example[0]{saṃvaccharamadhīte saṃvaccharo (saṃvacchara + ṇa)\\{\upshape= [One] studies [the counting of] year, hence \pali{saṃvacchara} (calendar maker/astronomer).}}
\example[0]{mohutto (muhutta + ṇa)\\{\upshape= an astrologer\footnote{Literally this means one who studies [auspicious] moments. See \pali{mauhūrta} in MWD.}}}
\example[0]{nemitto (nimitta + ṇa)\\{\upshape= a fortune-teller}}
\example[0]{aṅgavijjo (aṅgavijja + ṇa)\\{\upshape= a fortune-teller based on bodily marks, e.g., palm-reader}}
\example[0]{veyyākaraṇo (vyākaraṇa + ṇa)\\{\upshape= a grammarian}}
\example[0]{chando (chanda + ṇa)\\{\upshape= a student of prosody}}
\example[0]{bhāsso (bhāssa + ṇa)\\{\upshape= a student of the sun\footnote{This is the best guess. See \pali{bhāskara} in MWD.}}}
\example{cando (canda + ṇa)\\{\upshape= a student of the moon}}
\transnote{The rest of the examples, in a way, look redundant. For some of them, I cannot find viable meaning. Along with other suttas, the examples given have shown how words obtain new meaning by derivation in theoretical ways. If certain words are in use at all, we normally find them in dictionaries. So, do not worry too much about nuaunces of meaning.}

\head{353}{353, 378. jātādīnamimiyā ca.}
\headtrans{[In the sense] of being born and so on, [there are] \pali{ima}- and \pali{iya}-paccaya also.}
\sutdef{jātaiccevamādīnamatthe imaiyapaccayā honti.}
\sutdeftrans{There are \pali{ima}- and \pali{iya}-paccaya in the sense of being born and so on.}
\example[0]{pacchā jāto pacchimo (pacchā + ima)\\{\upshape= [One] was born afterward, hence \pali{pacchima} (the last-born [one]).}}
\example[0]{antimo (anta + ima)\\{\upshape= the last-born [one]}}
\example[0]{majjhimo (majjhima + ima)\\{\upshape= the middle-born [one]}}
\example[0]{purimo (pure + ima)\\{\upshape= the first-born [one]}}
\example[0]{uparimo (upari + ima)\\{\upshape= the above-born [one]}}
\example[0]{heṭṭhimo (heṭṭhā + ima)\\{\upshape= the below-born [one]}}
\example[0]{gopphimo (goppha + ima)\\{\upshape= the ankle-born [disease]}}
\example[0]{bodhisattajātiyā jāto bodhisattajātiyo (bodhisattajāti + iya)\\{\upshape= [One] was born by the birth of bodhisatta, hence \pali{bodhisattajātiya} (one born as a bodhisatta).}}
\example[0]{assajātiyo (assajāti + iya)\\{\upshape= one born as a hourse}}
\example[0]{hatthijātiyo (hatthijāti + iya)\\{\upshape= one born as an elephant}}
\example{manussajātiyo (manussajāti + iya)\\{\upshape= one born as a human being}}
\sutdef{ādiggahaṇena niyuttatthāditopi tadassatthāditopi imaiyaika\-iccete paccayā honti.}
\sutdeftrans{By taking `and so on' into account, there are \pali{ima}-, \pali{iya}-, and \pali{ika}-paccaya after the sense of `related to' and so on, and after the sense of `belonging to that' and so on.}
\example[0]{anto niyutto antimo (anta + ima)\\{\upshape= [One] was related to the end, hence \pali{antima} (the ending [one]).}}
\example[0]{antiyo (anta + iya)\\{\upshape= the ending [one]}}
\example[0]{antiko (anta + ika)\\{\upshape= the ending [one]}}
\example[0]{putto assa atthi, tasmiṃ vā vijjatīti puttimo (putta + ima)\\{\upshape= Whose son exists, or [the son] exists in that [person], hence \pali{puttima} ([one] having a son).}}
\example[0]{puttiyo (putta + iya)\\{\upshape= [one] having a son}}
\example[0]{puttiko (putta + ika)\\{\upshape= [one] having a son}}
\example[0]{kappimo (kappa + ima)\\{\upshape= [one] having a thought}}
\example[0]{kappiyo (kappa + iya)\\{\upshape= [one] having a thought}}
\example{kappiko (kappa + ika)\\{\upshape= [one] having a thought}}
\sutdef{caggahaṇena kiyapaccayo hoti niyuttatthe}
\sutdeftrans{By taking `also' into account, there is \pali{kiya}-paccaya in the sense of `engaged in.'}
\example{jātiyaṃ niyutto jātikiyo (jāti + kiya)\\{\upshape= [One] was engaged in the birth, hence \pali{jātikiya} (birth-engaged [one]).}}
\transnote{The meaning of this is far from clear. In a Thai source, it is described as \pali{jātippabhūtiyā niyutto jātikiyo}, ``[One] was appointed from the birth, hence \pali{jātikiya}.'' That is more understandable in the sense that one was endowed with a certain condition from birth. See also \pali{jātippabhava} in DPD, and \pali{pra\-bhūti} in MWD.}
\example[0]{andhe niyutto andhakiyo (andha + kiya)\\{\upshape= [One] was engaged in blind, hence \pali{andhakiya} (blind [one]).}}
\example[0]{jātiyā andho jaccandho {\upshape(tappurisa-samāsa)}\\{\upshape= [One was] blind by birth, hence \pali{jaccandha} (blind-from-birth [one]).}}
\example{jaccandhe niyutto jaccandhakiyo (jaccandha + kiya)\\{\upshape= [One] was engaged in blind-from-birth, hence \pali{jaccandhakiya} (blind-from-birth [one]).}}

\head{354}{354, 379. samūhatthe kaṇaṇā.}
\headtrans{In the sense of aggregation, [there are] \pali{kaṇa}- and \pali{ṇa}-paccaya.}
\sutdef{samūhatthe kaṇaṇaiccete paccayā honti.}
\sutdeftrans{There are \pali{kaṇa}- and \pali{ṇa}-paccaya in the sense of aggregation.}
\example[0]{rājaputtānaṃ samūho rājaputtako (rājaputta + kaṇa)\\{\upshape= An assembly of princes, hence \pali{rājaputtaka} (group of princes).}}
\example[0]{rājaputto (rājaputta + ṇa)\\{\upshape= a group of princes}}
\example[0]{mānussako (manussu + kaṇa)\\{\upshape= a group of human beings}}
\example[0]{mānusso (manussu + ṇa)\\{\upshape= a group of human beings}}
\example[0]{māyūrako (mayūra + kaṇa)\\{\upshape= a flock of peacocks}}
\example[0]{māyūro (mayūra + ṇa)\\{\upshape= a flock of peacocks}}
\example[0]{māhiṃsako (mahiṃsa + kaṇa)\\{\upshape= a herd of buffalos}}
\example{māhiṃso (mahiṃsa + ṇa)\\{\upshape= a herd of buffalos}}

\head{355}{355, 380. gāmajanabandhusahāyādīhi tā.}
\headtrans{After \pali{gāma}, \pali{jana}, \pali{bandhu}, \pali{sahāya} and so on, [there is] \pali{tā}-paccaya.}
\sutdef{gāmajanabandhusahāyaiccevamādīhi tāpaccayo hoti samūh\-atthe.}
\sutdeftrans{There is \pali{tā}-paccaya after \pali{gāma}, \pali{jana}, \pali{bandhu}, \pali{sahāya} and so on in the sense of aggregation.}
\example[0]{gāmānaṃ samūho gāmatā (gāma + tā)\\{\upshape= An aggregation of village, hence \pali{gāmatā} (group of villages).}}
\example[0]{janatā (jana + tā)\\{\upshape= an assembly of people}}
\example[0]{bandhutā (bandhu + tā)\\{\upshape= an assembly of relatives}}
\example[0]{sahāyatā (sahāya + tā)\\{\upshape= an assembly of friends}}
\example{nagaratā (nagara + tā)\\{\upshape= an aggregation of towns}}

\head{356}{356, 381. tadassa ṭhānamiyo ca.}
\headtrans{[In the sense of] `a cause of that,' [there is] \pali{iya}-paccaya also.}
\sutdef{tadassa ṭhānamiccetasmiṃ atthe iyapaccayo hoti.}
\sutdeftrans{There is \pali{iya}-paccaya in the sense of `a cause of that.'}
\example[0]{madanassa ṭhānaṃ madaniyaṃ (madana + iya)\\{\upshape= A cause of intoxication, hence \pali{madaniya}.}}
\example[0]{bandhanassa ṭhānaṃ bandhaniyaṃ (bandhana + iya)\\{\upshape= A cause of imprisonment, hence \pali{bandhaniya}.}}
\example[0]{mucchanassa ṭhānaṃ mucchaniyaṃ (mucchana + iya)\\{\upshape= A cause of infatuation, hence \pali{mucchaniya}.}}
\example[0]{rajaniyaṃ (rajana + iya)\\{\upshape= a cause of lust}}
\example[0]{kamaniyaṃ (kamana + iya)\\{\upshape= a desirable [thing]}}
\example[0]{gamaniyaṃ (gamana + iya)\\{\upshape= a worth going [place]}}
\example[0]{dussaniyaṃ (dussana + iya)\\{\upshape= a corruptible [one]}}
\example{dassaniyaṃ (dassana + iya)\\{\upshape= a worth-seeing [thing/place]}}
\transnote{In a Thai source, this paccaya is \pali{īya} instead, and this is more often seen, e.g., \pali{dassanīya}. The use of this paccaya seems to have double meaning, as you see in \pali{kamaniyaṃ} onward. For this term, it is not about moving (\pali{kamana}) but desiring (see \pali{kamanīya} in MWD).}

\head{357}{357, 382. upamatthāyitattaṃ.}
\headtrans{In metaphorical meaning, [there is] \pali{āyitatta}-paccaya.}
\sutdef{upamatthe āyitattapaccayo hoti.}
\sutdeftrans{There is \pali{āyitatta}-paccaya in metaphorical meaning.}
\example[0]{dhūmo viya dissati aduṃ vanaṃ tadidaṃ dhūmāyitattaṃ (dhūma + āyitatta)\\{\upshape= [Which] forest over there looks like smoke, that this [forest is thus] \pali{dhūmāyitatta} (smoke-like [forest]).}}
\example{timiraṃ viya dissati aduṃ vanaṃ tadidaṃ timirāyitattaṃ (timira + āyitatta)\\{\upshape= [Which] forest over there looks murky, that this [forest is thus] \pali{timirāyitatta} (murky [forest]).}}

\head{358}{358, 383. tannissitatthe lo.}
\headtrans{In the sense of `based on that,' [there is] \pali{la}-paccaya.}
\sutdef{tannissitatthe, tadassa ṭhānamiccetasmiṃ atthe ca lapaccayo hoti.}
\sutdeftrans{There is \pali{la}-paccaya also in the sense of `based on that.'}
\example[0]{duṭṭhu nissitaṃ duṭṭhullaṃ (duṭṭhu + la)\\{\upshape= [An offense was] based on a horrible [deed], hence \pali{duṭṭhulla} (horrible [offense]).}}
\example[0]{vedaṃ nissitaṃ vedallaṃ (veda + la)\\{\upshape= [A teaching was] based on insight, hence \pali{vedalla} (insight-based [teaching]).}}
\example[0]{duṭṭhu ṭhānaṃ duṭṭhullaṃ (duṭṭhu + la)\\{\upshape= horrible [offense]}}
\example{vedassa ṭhānaṃ vedallaṃ (duṭṭhu + la)\\{\upshape= insight-based [teaching]}}
\transnote{I see that \pali{nissita} and \pali{ṭhāna} are synonymous. So, the two definitions are more or less the same. The word `\pali{vedalla}' can mean many things, including one division of the Pāli canon, or simply Veda-based knowledge. See further in PTSD.}

\head{359}{359, 384. ālu tabbahule.}
\headtrans{[There is] \pali{ālu}-paccaya in [the sense of] plenty.}
\sutdef{ālupaccayo hoti tabbahulatthe.}
\sutdeftrans{There is \pali{ālu}-paccaya in the sense of plenty.}
\example[0]{abhijjhā assa pakati abhijjhālu\\{\upshape= Greed [is] one's nature, hence \pali{abhijjhālu} (greedy one)}}
\example[0]{abhijjhā assa bahulā abhijjhālu\\{\upshape= One's greed [is] plentiful, hence \pali{abhijjhālu} (greedy one)}}
\example[0]{sītālu (sīta + ālu)\\{\upshape= a cold-natured one}}
\example[0]{dhajālu (dhaja + ālu)\\{\upshape= a place full of flags}}
\example{dayālu (dayā + ālu)\\{\upshape= a compassionate one}}

\head{360}{360, 387. ṇyattatā bhāve tu.}
\headtrans{[There are] \pali{ṇya}-, \pali{tta}-, and \pali{tā}-paccaya in [the sense of] state of being.}
\sutdef{ṇyattatāiccete paccayā honti bhāvatthe.}
\sutdeftrans{There are \pali{ṇya}-, \pali{tta}-, and \pali{tā}-paccaya in the sense of state of being.}
\example[0]{alasassa bhāvo ālasyaṃ (alasa + ṇya)\\{\upshape= The state of being lazy, hence \pali{ālasya} (the laziness).}}
\example[0]{arogassa bhāvo ārogyaṃ (aroga + ṇya)\\{\upshape= The state of being disease-free, hence \pali{ārogya} (the healthiness).}}
\example[0]{paṃsukūlikassa bhāvo paṃsukūlikattaṃ (paṃsukūlika + tta)\\{\upshape= The state of being one using rag robes, hence \pali{paṃsukūlikatta}.}}
\example[0]{anodarikassa bhāvo anodarikattaṃ (anodarika + tta)\\{\upshape= The state of being one not being concerned for the stomach, hence \pali{anodarikatta}.}}
\example[0]{saṅgaṇikārāmassa bhāvo saṅgaṇikārāmatā (saṅgaṇikārāma + tā)\\{\upshape= The state of being one delighted in socializing, hence \pali{saṅgaṇikārāmatā}.}}
\example{niddārāmassa bhāvo niddārāmatā (niddārāma + tā)\\{\upshape= The state of being one delighted in sleeping, hence \pali{niddārāmatā}.}}
\transnote{Interestingly, \pali{anodarika (na + udara + ṇika)} intendedly means a monk who is not just concerned for eating.}
\transnote{As marked by \pali{tu}, \pali{ttana}-paccaya can also be used.}
\example[0]{puthujjanattanaṃ (puthujjana + ttana)\\{\upshape= the state of being an ordinary person}}
\example{vedanattanaṃ (vedanā + ttana)\\{\upshape= the state of having a feeling}}

\head{361}{361, 388. ṇa visamādīhi.}
\headtrans{[There is] \pali{ṇa}-paccaya after \pali{visama} and so on.}
\sutdef{ṇapaccayo hoti visamādīhi tassa bhāvo iccetasmiṃ atthe.}
\sutdeftrans{There is \pali{ṇa}-paccaya after \pali{visama} and so on in the sense of `the state of that.'}
\example[0]{visamassa bhāvo vesamaṃ (visama + ṇa)\\{\upshape= The state of unevenness, hence \pali{vesama} (inequality/disharmony).}}
\example{sucissa bhāvo socaṃ (suci + ṇa)\\{\upshape= The state of cleanness, hence \pali{soca} (purity).}}
\transnote{I cannot find the two words in dictionaries, but \pali{vesamma} and \pali{soceyya} in the meaning described.}

\head{362}{362, 389. ramaṇīyādito kaṇ.}
\headtrans{After \pali{ramaṇīya} and so on, [there is] \pali{kaṇ}-paccaya.}
\sutdef{ramaṇīyaiccevamādito kaṇpaccayo hoti tassa bhāvo iccetasmiṃ atthe.}
\sutdeftrans{There is \pali{kaṇ}-paccaya after \pali{ramaṇīya} and so on in the sense of `the state of that.'}
\example[0]{ramaṇīyassa bhāvo rāmaṇīyakaṃ (ramaṇīya + kaṇ)\\{\upshape= The state of delightfulness, hence \pali{rāmaṇīyaka}.}}
\example{manuññassa bhāvo mānuññakaṃ (manuñña + kaṇ)\\{\upshape= The state of delightfulness, hence \pali{mānuññaka}.}}
\transnote{This is like \pali{ka} with vuddhi transformation. It is the same as \pali{kaṇa}. But why it came in different forms is mysterious to me.}

\head{363}{363, 390. visese taratamisikiyiṭṭhā.}
\headtrans{In comparison, [there are] \pali{tara}-, \pali{tama}-, \pali{isika}-, \pali{iya}-, and \pali{iṭṭha}-paccaya.}
\sutdef{visesatthe taratamaisikaiyaiṭṭhaiccete paccayā honti.}
\sutdeftrans{There are \pali{tara}-, \pali{tama}-, \pali{isika}-, \pali{iya}-, and \pali{iṭṭha}-paccaya in the sense of distinction (comparison).}
\example[0]{sabbe ime pāpā, ayamimesaṃ visesena pāpoti pāpataro (pāpa + tara)\\{\upshape= All these [people are] evil, of these [people] this [person is] evil by superior state, hence \pali{pāpatara} (more evil [one]).}}
\example[0]{pāpatamo (pāpa + tama)\\{\upshape= the most evil [one]}}
\example[0]{pāpisiko (pāpa + isika)\\{\upshape= a more evil [one]}}
\example[0]{pāpiyo (pāpa + iya)\\{\upshape= a more evil [one]}}
\example{pāpiṭṭho (pāpa + iṭṭha)\\{\upshape= the most evil [one]}}

\head{364}{364, 398. tadassatthīti vī ca.}
\headtrans{[In the sense of] `one has that,' [there is] \pali{vī}-paccaya also.}
\sutdef{tadassatthi iccetasmiṃ atthe vīpaccayo hoti.}
\sutdeftrans{There is \pali{vī}-paccaya in the sense of `one has that.'}
\example[0]{medhā yassa atthi, tasmiṃ vā vijjatīti medhāvī (medhā + vī)\\{\upshape= Whose wisdom exists, or [wisdom is] present in that [person], hence \pali{medhāvī} (wise [one]).}}
\example{māyāvī (māyā + vī)\\{\upshape= a deceitful [one]}}
\transnote{By the word \pali{ca}, \pali{sa}-paccaya can also be the case.}
\example{sumedhā yassa atthi, tasmiṃ vā vijjatīti sumedhaso (sumedha + sa)\\{\upshape= Whose wisdom exists, or [wisdom is] present in that [person], hence \pali{sumedhasa} (wise [one]).}}

\head{365}{365, 399. tapādito sī.}
\headtrans{After \pali{tapa} and so on, [there is] \pali{sī}-paccaya.}
\sutdef{tapādito sīpaccayo hoti tadassatthi iccetasmiṃ atthe.}
\sutdeftrans{There is \pali{sī}-paccaya after \pali{tapa} and so on in the sense of `one has that.'}
\example[0]{tapo yassa atthi tasmiṃ vā vijjatīti tapassī (tapa + sī)\\{\upshape= Whose austerity exists, or [austerity is] present in that [person], hence \pali{tapassī} (austerity practitioner).}}
\example[0]{yasassī (yasa + sī)\\{\upshape= a famous one}}
\example{tejassī (teja + sī)\\{\upshape= a powerful one}}

\head{366}{366, 400. daṇḍādito ikaī.}
\headtrans{After \pali{daṇḍa} and so on, [there are] \pali{ika}- and \pali{ī}-paccaya.}
\sutdef{daṇḍādito ikaīiccete paccayā honti tadassatthi iccetasmiṃ atthe.}
\sutdeftrans{There are \pali{ika}- and \pali{ī}-paccaya after \pali{daṇḍa} and so on in the sense of `one has that.'}
\example[0]{daṇḍo yassa atthi, tasmiṃ vā vijjatīti daṇḍiko (daṇḍa + ika)\\{\upshape= Whose stick exists, or [a stick is] present in that [person], hence \pali{daṇḍika} (one carrying a stick).}}
\example[0]{daṇḍī (daṇḍa + ī)\\{\upshape= one carrying a stick}}
\example[0]{māliko (mālā + ika)\\{\upshape= one wearing a garland}}
\example{mālī (mālā + ī)\\{\upshape= one wearing a garland}}

\head{367}{367, 401. madhvādito ro.}
\headtrans{After \pali{madhu} and so on, [there is] \pali{ra}-paccaya.}
\sutdef{madhuiccevamādito rapaccayo hoti tadassatthi iccetasmiṃ atthe.}
\sutdeftrans{There is \pali{ra}-paccaya after \pali{madhu} and so on in the sense of `one has that.'}
\example[0]{madhu yassa atthi, tasmiṃ vā vijjatīti madhuro (madhu + ra)\\{\upshape= Which [thing has] honey, or [honey] exists in that, hence \pali{madhura} (sweet thing).}}
\example[0]{kuñjaro (kuñja + ra)\\{\upshape= an elephant}}
\example[0]{muggaro (mugga + ra)\\{\upshape= a kind of pea}}
\example[0]{mukharo (mukha + ra)\\{\upshape= a talkative one}}
\example[0]{susiro (susi + ra)\\{\upshape= a hollow [tree]}}
\example[0]{sīsaro (sīsa + ra)\\{\upshape= one having a head}}
\example[0]{suṅkaro (suṅka + ra)\\{\upshape= one having tax (=? king)}}
\example[0]{subharo (subha + ra)\\{\upshape= a beatiful one}}
\example[0]{suciro (suci + ra)\\{\upshape= a pure one}}
\example{ruciro (ruci + ra)\\{\upshape= a likable one}}
\transnote{For \pali{kuñja} which means `tusk,' see MWD.}

\head{368}{368, 402. guṇādito vantu.}
\headtrans{After \pali{goṇa} and so on, [there is] \pali{vantu}-paccaya.}
\sutdef{guṇaiccevamādito vantupaccayo hoti tadassatthi iccetasmiṃ atthe.}
\sutdeftrans{There is \pali{vantu}-paccaya after \pali{guṇa} and so on in the sense of `one has that.'}
\example[0]{guṇo yassa atthi, tasmiṃ vā vijjatīti guṇavā (guṇa + vantu)\\{\upshape= Whose virtue exists, or [virtue is] present in that [person], hence \pali{guṇavantu} (virtuous [one]).}}
\example[0]{yasavā (yasa + vantu)\\{\upshape= a famous one}}
\example[0]{dhanavā (dhana + vantu)\\{\upshape= a wealthy one}}
\example[0]{paññavā (paññā + vantu)\\{\upshape= a wise one}}
\example[0]{balavā (bala + vantu)\\{\upshape= a powerful one}}
\example{bhagavā (bhaga + vantu)\\{\upshape= a fortunate one}}

\head{369}{369, 403. satyādīhi mantu.}
\headtrans{After \pali{sati} and so on, [there is] \pali{mantu}-paccaya.}
\sutdef{satiiccevamādīhi mantupaccayohoti tadassatthi iccetasmiṃ atthe.}
\sutdeftrans{There is \pali{mantu}-paccaya after \pali{sati} and so on in the sense of `one has that.'}
\example[0]{sati yassa atthi, tasmiṃ vā vijjatīti satimā (sati + mantu)\\{\upshape= Whose mindfulness exists, or [mindfulness is] present in that [person], hence \pali{satimantu} (mindful [one]).}}
\example[0]{jutimā (juti + mantu)\\{\upshape= a radiant one}}
\example[0]{rucimā (ruci + mantu)\\{\upshape= a bright one}}
\example[0]{thutimā (thuti + mantu)\\{\upshape= a praiseworthy one}}
\example[0]{dhitimā (dhiti + mantu)\\{\upshape= a resolute one}}
\example[0]{matimā (mati + mantu)\\{\upshape= a wise one}}
\example{bhāṇumā (bhāṇu + mantu)\\{\upshape= a bright one}}

\head{370}{370, 405. saddhādito ṇa.}
\headtrans{After \pali{saddhā} and so on, [there is] \pali{ṇa}-paccaya.}
\sutdef{saddhāiccevamādito ṇapaccayo hoti tadassatthi iccetasmiṃ atthe.}
\sutdeftrans{There is \pali{ṇa}-paccaya after \pali{saddhā} and so on in the sense of `one has that.'}
\example[0]{saddhā yassa atthi, tasmiṃ vā vijjatībhi saddho (saddhā + ṇa)\\{\upshape= Whose faith exists, or [faith is] present in that [person], hence \pali{saddha} (faithful [one]).}}
\example[0]{pañño (paññā + ṇa)\\{\upshape= a wise one}}
\example{amaccharo (amacchara + ṇa)\\{\upshape= a non-stingy one}}

\head{371}{371, 404. āyussukārāsa mantumhi.}
\headtrans{The \pali{u}-ending of \pali{āyu} [becomes] \pali{as} because of \pali{mantu}-paccayas.}
\sutdef{āyussa anto ukāro asādeso hoti mantumhi paccaye pare.}
\sutdeftrans{There is \pali{as}-substitution for the \pali{u}-ending of \pali{āyu} because of \pali{mantu}-paccayas behind.}
\example{āyu assa atthi, tasmiṃ vā vijjatīti āyasmā (āyu + mantu)\\{\upshape= Whose longevity exists, or [longevity is] present in that [person], hence \pali{āyasmantu} (long-lived one).}}

\head{372}{372, 385. tappakativacane mayo.}
\headtrans{In the sense of `made of that,' [there is] \pali{maya}-paccaya.}
\sutdef{tappakativacanatthe mayapaccayo hoti.}
\sutdeftrans{There is \pali{maya}-paccaya in the sense of `made of that.'}
\example[0]{suvaṇṇena pakataṃ sovaṇṇamayaṃ (suvaṇṇa + maya)\\{\upshape= [A thing was] made with gold, hence \pali{sovaṇṇamaya} (made of gold).\footnote{In the text, \pali{kammaṃ} is confusingly added to the sentence, but absent in a Thai edition. This example results in vuddhi strength, but \pali{suvaṇṇamaya} is undoubtedly valid as well.}}}
\example[0]{rūpiyamayaṃ (rūpiya + maya)\\{\upshape= [a thing] made of silver}}
\example[0]{jatumayaṃ (jatu + maya)\\{\upshape= [a thing] made of resin}}
\example[0]{rajatamayaṃ (rajata + maya)\\{\upshape= [a thing] made of silver}}
\example[0]{iṭṭhakamayaṃ (iṭṭhaka + maya)\\{\upshape= [a thing] made from brick}}
\example[0]{ayomayaṃ (aya + maya)\\{\upshape= [a thing] made of iron}}
\example[0]{mattikāmayaṃ (mattikā + maya)\\{\upshape= [a thing] made of clay}}
\example[0]{dārumayaṃ (dāru + maya)\\{\upshape= [a thing] made of wood}}
\example{gomayaṃ (go + maya)\\{\upshape= cow dung}}
\transnote{For why \pali{aya} becomes \pali{ayo}, see \hyperref[sut:183]{Kacc 183}.}

\head{373}{373, 406. saṅkhyāpūraṇe mo.}
\headtrans{In [the sense of] the ordinal, [there is] \pali{ma}-paccaya.}
\sutdef{saṅkhyāpūraṇatthe mapaccayo hoti.}
\sutdeftrans{There is \pali{ma}-paccaya in the sense of the ordinal.}
\example[0]{pañcannaṃ pūraṇo pañcamo (pañca + ma)\\{\upshape= The fullness of five, hence \pali{pañcama} (the fifth).}}
\example[0]{sattamo (satta + ma)\\{\upshape= the seventh}}
\example[0]{aṭṭhamo (aṭṭha + ma)\\{\upshape= the eighth}}
\example[0]{navamo (nava + ma)\\{\upshape= the ninth}}
\example{dasamo (dasa + ma)\\{\upshape= the tenth}}

\head{374}{374, 408. sa chassa vā.}
\headtrans{[There is] \pali{sa}-substitution for \pali{cha} sometimes.}
\sutdef{chassa sakārādeso hoti vā saṅkhyāpūraṇatthe.}
\sutdeftrans{There is \pali{sa}-substitution for \pali{cha} sometimes in the sense of the ordinal.}
\example{channaṃ pūraṇo saṭṭho\\{\upshape= The fullness of six, hence \pali{saṭṭha} (the sixth) [= \pali{chaṭṭha}].}}

\head{375}{375, 412. ekādito dasassī.}
\headtrans{[There is] \pali{ī}-paccaya for \pali{dasa} after \pali{eka} and so on.}
\sutdef{ekādito dasassa ante īpaccayo hoti vā saṅkhyāpūraṇatthe.}
\sutdeftrans{There is \pali{ī}-paccaya at the end of \pali{dasa} after \pali{eka} and so on sometimes in the sense of the ordinal.}
\example[0]{eko ca dasa ca ekādasa, ekādasannaṃ pūraṇī ekādasī (ekādasa + ī)\\{\upshape= One and ten thus \pali{ekadasa} (eleven). The fullness of eleven, hence \pali{ekādasī} (the eleventh).}}
\example[0]{pañcadasī (pañcadasa + ī)\\{\upshape= the fifteenth}}
\example{cātuddasī (cātuddasa + ī)\\{\upshape= the fourteenth}}

\head{376}{376, 257. dase so niccañca.}
\headtrans{Because of \pali{dasa}, [there is] \pali{sa} [for \pali{cha}] always also.}
\sutdef{dasasadde pare niccaṃ chassa so hoti.}
\sutdeftrans{There is always \pali{sa}[-substitution] for \pali{cha} because of \pali{dasa} behind.}
\example{soḷasa {\upshape(sixteen)}}

\head{377}{377. ante niggahitañca.}
\headtrans{At the end [of those numbers, there is] \pali{niggahita}.}
\sutdef{tāsaṃ saṅkhyānaṃ ante niggahitāgamo hoti}.
\sutdeftrans{There is an insertion of \pali{niggahita} at the end of those numbers.}
\example[0]{pañcadasiṃ {\upshape(the fifteenth)}}
\example{cātuddasiṃ {\upshape(the fourteenth)}}

\head{378}{378, 414. ti ca.}
\headtrans{[At the end of those numbers, there is] \pali{ti} also.}
\sutdef{tāsaṃ saṅkhyānaṃ ante tikārāgamo hoti.}
\sutdeftrans{There is an insertion of \pali{ti} at the end of those numbers.}
\example[0]{vīsati {\upshape(twenty)}}
\example{tiṃsati {\upshape(thirty)}}

\head{379}{379, 258. la darānaṃ.}
\headtrans{[There is] \pali{la}[-substitution] for \pali{da} and \pali{ra}.}
\sutdef{dakārarakārānaṃ saṅkhyānaṃ lakārādeso hoti.}
\sutdeftrans{There is \pali{la}-substitution for \pali{da} and \pali{ra} of numbers.}
\example[0]{soḷasa {\upshape(sixteen)}}
\example{cattālīsaṃ {\upshape(forty)}}
\transnote{In a Thai source, the substitution is \pali{ḷa} instead.}

\head{380}{380, 255. vīsatidasesu bā dvissa tu.}
\headtrans{Because of \pali{vīsati} and \pali{dasa}, [there is] \pali{bā} for \pali{dvi}.}
\sutdef{vīsatidasaiccetesu dvissa bā hoti.}
\sutdeftrans{There is \pali{bā}[-substitution] for \pali{dvi} because of \pali{vīsati} and \pali{dasa}.}
\example[0]{bāvīsatindriyāni {\upshape(twenty-two faculties)}}
\example{bārasa manussā {\upshape(twelve human beings)}}
\transnote{By the word \pali{tu}, there are also \pali{du}-, \pali{di}-, and \pali{do}-substitution, as shown by the following examples.}
\example[0]{durattaṃ {\upshape(two nights)}}
\example[0]{dirattaṃ {\upshape(two nights)}}
\example[0]{diguṇaṃ {\upshape(twofold)}}
\example{dohaḷinī {\upshape(a pregnant woman)}}

\head{381}{381, 254. ekādito dassa ra saṅkhyāne.}
\headtrans{[There is] \pali{ra} for \pali{da} after \pali{eka} and so on in numbers.}
\sutdef{ekādito dasassa dakārassa rakāro hoti vā saṅkhyāne.}
\sutdeftrans{There is \pali{ra}[-substitution] for \pali{da} of \pali{dasa} after \pali{eka} and so on sometimes in numbers.}
\example[0]{ekārasa = ekādasa {\upshape(eleven)}}
\example{bārasa = dvādasa {\upshape(twelve)}}

\head{382}{382, 259. aṭṭhādito ca.}
\headtrans{After \pali{aṭṭha} and so on also, [there is \pali{ra}-substitution].}
\sutdef{aṭṭhaiccevamādito ca dasasaddassa dakārassa rakārādeso hoti vā saṅkhyāne.}
\sutdeftrans{There is \pali{ra}-substitution for \pali{da} of \pali{dasa} after \pali{aṭṭha} and so on also sometimes in numbers}
\example{aṭṭhārasa = aṭṭhadasa {\upshape(eighteen)}}

\head{383}{383, 253. dvekaṭṭhānamākāro vā.}
\headtrans{[There is] \pali{ā}[-insertion] for \pali{dvi}, \pali{eka}, and \pali{aṭṭha} sometimes.}
\sutdef{dviekaaṭṭhaiccetesamanto ākāro hoti vā saṅkhyāne.}
\sutdeftrans{There is \pali{ā}[-insertion] after the end of \pali{dvi}, \pali{eka}, and \pali{aṭṭha} sometimes in numbers.}
\example[0]{dvādasa {\upshape(twelve)}}
\example[0]{ekādasa {\upshape(eleven)}}
\example{aṭṭhārasa {\upshape(eighteen)}}

\head{384}{384, 407. catucchehī thaṭhā.}
\headtrans{After \pali{catu} and \pali{cha}, [there are] \pali{tha} and \pali{ṭha}.}
\sutdef{catuchaiccetehi thaṭhaiccete paccayā honti saṅkhyāpūraṇatthe.}
\sutdeftrans{There are \pali{tha}- and \pali{ṭha}-paccaya after \pali{catu} and \pali{cha} in the ordinal.}
\example[0]{catuttho {\upshape(the fourth)}}
\example{chaṭṭho {\upshape(the sixth)}}

\head{385}{385, 409. dvitīhi tiyo.}
\headtrans{After \pali{dvi} and \pali{ti}, [there is] \pali{tiya}.}
\sutdef{dvitiiccetehi tiyapaccayo hoti saṅkhyāpūraṇatthe}.
\sutdeftrans{There is \pali{tiya}-paccaya after \pali{dvi} and \pali{ti} in the ordinal.}
\example[0]{dutiyo {\upshape(the second)}}
\example{tatiyo {\upshape(the third)}}

\head{386}{386, 410. tiye dutāpi ca.}
\headtrans{Because of \pali{tiya}, [there are] \pali{du} and \pali{ta} also.}
\sutdef{dvitiiccetesaṃ dutaiccete ādesā honti tiyapaccaye pare.}
\sutdeftrans{There are \pali{du}- and \pali{ta}-substitution for \pali{dvi} and \pali{ti} because of \pali{tiya}-paccaya behind.}
\example[0]{dutiyo {\upshape(the second)}}
\example{tatiyo {\upshape(the third)}}

\head{387}{387,~411.~tesamaḍḍhūpapadena aḍḍhuḍḍhadivaḍḍha-\\diyaḍḍhaḍḍhatiyā.}
\headtrans{For these [\pali{catuttha}, \pali{dutiya}, and \pali{tatiya}] with \pali{aḍḍha} nearby, [there are] \pali{aḍḍhuḍḍha}-, \pali{divaḍḍha}-, \pali{diyaḍḍha}-, and \pali{aḍḍhatiya}-substitution.}
\sutdef{tesaṃ catutthadutiyatatiyānaṃ aḍḍhūpapadānaṃ aḍḍhuḍḍha\-divaḍḍhadiyaḍḍhaaḍḍhatiyādesā honti, aḍḍhūpapadena saha nippajjante.}
\sutdeftrans{There are \pali{aḍḍhuḍḍha}-, \pali{divaḍḍha}-, \pali{diyaḍḍha}-, and \pali{aḍḍhatiya}-substitution for these \pali{catuttha}, \pali{dutiya}, and \pali{tatiya} having \pali{aḍḍha} (a half) nearby. [The results are] produced with \pali{aḍḍha}.}
\example[0]{aḍḍhena catuttho aḍḍhuḍḍho \\{\upshape= The fourth by a half, hence \pali{aḍḍhuḍḍha} (three and a half).}}
\example[0]{aḍḍhena dutiyo divaḍḍho \\{\upshape= The second by a half, hence \pali{divaḍḍha} (one and a half).}}
\example[0]{aḍḍhena dutiyo diyaḍḍho \\{\upshape= The second by a half, hence \pali{divaḍḍha} (one and a half).}}
\example{aḍḍhena tatiyo aḍḍhatiyo \\{\upshape= The third by a half, hence \pali{aḍḍhatiya} (two and a half).}}
\transnote{The formation of these numbers is a little confusing. The end results are not ordinal but cardinal. In the definition, \pali{nippajjante} should be seen as \pali{nipphajjante}.}

\head{388}{388, 68. sarūpānamekasesvasakiṃ.}
\headtrans{[There is] a single remainder of terms identical in form [happening] more than once.}
\sutdef{sarūpānaṃ padabyañjanānaṃ ekaseso hoti asakiṃ.}
\sutdeftrans{There is a single remainder of terms identical in form [happening] more than once.}
\example{puriso ca puriso ca purisā \\{\upshape= A man and [another] man, hence \pali{purisā} (men).}}
\transnote{This grammatical idea came from Pāṇ 1.2.64, not found in Kāt. It sounds like several of identical terms can be collapsed into a single one.}

\head{389}{389,~413.~gaṇane dasassa dviticatupañcachasattaaṭṭha-\\navakānaṃ vīticattārapaññāchasattāsanavā yosu, yonañcīsamāsaṃṭhiritītuti.}
\headtrans{In counting, [there are substitutions of] \pali{vī}, \pali{ti}, \pali{cattāra}, \pali{paññā}, \pali{cha}, \pali{satta}, \pali{asa}, and \pali{nava} for \pali{dvi}, \pali{ti}, \pali{catu}, \pali{pañca}, \pali{cha}, \pali{satta}, \pali{aṭṭha}, and \pali{nava} of \pali{dasa} because of \pali{yo}-vibhattis. Also [there are substitutions for] \pali{yo}-vibhatti, i.e., \pali{īsaṃ}, \pali{āsaṃ}, \pali{ṭhi}, \pali{ri}, \pali{ti}, \pali{īti}, and \pali{uti}.}
\sutdef{gaṇane dasassa dvikatikacatukkapañcakachakkasattakaaṭ\-ṭhakanavakānaṃ sarūpānaṃ katekasesānaṃ yathāsaṅkhyaṃ vīticattārapaññāchasattaasanavaiccādesā honti asakiṃ yosu, yonañca īsaṃāsaṃṭhiritiītiutiiccādesā honti. pacchā puna nippajjante.}
\sutdeftrans{In counting, there are substitutions of \pali{vī}, \pali{ti}, \pali{cattāra}, \pali{paññā}, \pali{cha}, \pali{satta}, \pali{asa}, and \pali{nava} for identical terms made single respectively, i.e., \pali{dvika}, \pali{tika}, \pali{catukka}, \pali{pañcaka}, \pali{chakka}, \pali{sattaka}, \pali{aṭṭhaka}, and \pali{navaka} of \pali{dasa} [occurring] more than once because of \pali{yo}-vibhattis. Also there are substitutions for \pali{yo}-vibhatti, i.e., \pali{īsaṃ}, \pali{āsaṃ}, \pali{ṭhi}, \pali{ri}, \pali{ti}, \pali{īti}, and \pali{uti}. [The final results] are made complete subsequently.}
\example[0]{vīsaṃ {\upshape(twenty)}}
\example[0]{tiṃsaṃ {\upshape(thirty)}}
\example[0]{cattālīsaṃ {\upshape(forty)}}
\example[0]{paññāsaṃ {\upshape(fifty)}}
\example[0]{saṭṭhi {\upshape(sixty)}}
\example[0]{sattari {\upshape(seventy)}}
\example[0]{sattati {\upshape(seventy)}}
\example[0]{asīti {\upshape(eighty)}}
\example{navuti {\upshape(ninety)}}
\transnote{Even though this sutta looks complicated, the results are simple and familiar. When we learn Pāli numerals, we normally remember those final terms. This sutta describes how they come, but it gives us no logic behind.}

\head{390}{390, 256. catūpapadassa lopo tuttarapadādicassa cucopi navā.}
\headtrans{[There is] an elision of \pali{tu} of a term [having] \pali{catu}, also [there are] \pali{cu}- and \pali{co}-substitution for \pali{ca} of the outcome sometimes.}
\sutdef{catūpapadassa gaṇane pariyāpannassa tukārassa lopo hoti, uttarapadādicakārassa cucopi ādesā honti navā.}
\sutdeftrans{In counting, there is an elision of \pali{tu} contained in a term [having] \pali{catu}. There are also substitutions of \pali{cu} and \pali{co} for \pali{ca} of the outcome sometimes.}
\example[0]{cuddasa = catuddasa {\upshape(fourteen)}}
\example{coddasa = catuddasa {\upshape(fourteen)}}
\transnote{Interestingly, \pali{coddasa} is not found in the main Pāli texts. The term is instead common in Prakrit (\citealp{pind:survey}, p.~86). How the idea came here is still mysterious.}

\head{391}{391, 423. yadanupapannā nipātanā sijjhanti.}
\headtrans{Which [words] are unmentioned, that [words are] accomplished from the given.}
\sutdef{ye saddā aniddiṭṭhalakkhaṇā, akkharapadabyañjanato, itthipumanapuṃsakaliṅgato, nāmupasagganipātato, abyayībhā\-vasamāsataddhitākhyātato, gaṇanasaṅkhyākālakārakappayo\-gasaññāto, sandhipakativuddhilopāgamavikāraviparitato, vi\-bhattivibhajanato ca, te nipātanā sijjhanti.}
\sutdeftrans{Which word having unexplained characteristics; by letter, term, and syllable; by feminine, masculine, and neuter; by noun, prefix, and particle; by adverbial compound and secondary derivation of numbers; by numeral, time, kāraka, sentence, and designation; by sandhi: retention, vuddhi-strength, elision, insertion, alteration, and deformation; also by the division of vibhattis; that [the unexplained are] accomplished from the given.}
\transnote{This sutta is really puzzling. The definition is bloated, comparing to a much shorter one in Rūpa 423. The only key word here is \pali{nipātana}, which was used by Sanskrit grammatical commentators to explain instances inexplicable by rules. It was used in Mahābhāṣya, Kāśikā, and Durgasiṅha's commentary on Kāt.}
\transnote{The sutta was probably taken from Kāśikā, if not Durgasiṅha, as we see in Kāśikā on Pāṇ 3.1.123, ``\pali{yad iha lakṣaṇena anupapannaṃ tat sarvaṃ nipātanāt siddham},'' and Durgasiṅha on Kāt 4.129, ``\pali{yallakṣaṇenānutpannaṃ tat sarvaṃ nipātanāt siddham}.''}
\transnote{As far as I can make it understood, this mean `a ready-made form,' a kind of irregularly unique formation, or `a word given as such.' That is to say, if you cannot find any explanation for a certain peculiar term, it may be of a ready-made one. See further in \citealp{roodbergen:gramdict}, p.~244.}

\newpage
\boxnote{On \pali{nipātana} in Mahābhāṣya\\
\hspace{5mm}``The phrase \pali{nipātanātsiddham} is very frequently used by Patañjalī to show that some technical difficulties in the formation of a word are not sometimes to be taken into consideration, the word given by Pāṇini being the correct one.'' (\citealp{abhyankar:gramdict}, p.~221)
}

\head{392}{392, 418. dvādito ko’nekatthe ca.}
\headtrans{After \pali{dvi} and so on, [there is] \pali{ka}-paccaya in various senses also.}
\sutdef{dviiccevamādito kapaccayo hoti anekatthe ca, nipātanā sijjhanti.}
\sutdeftrans{There is \pali{ka}-paccaya after \pali{dvi} and so on in various senses also. [These are] accomplished from the given.}
\example[0]{satassa dvikaṃ dvisataṃ \\{\upshape= Two of hundred, hence \pali{dvisata} (two-hundred).}}
\example[0]{satassa tikaṃ tisataṃ {\upshape(three-hundred)}}
\example[0]{satassa catukkaṃ catusataṃ {\upshape(four-hundred)}}
\example[0]{satassa pañcakaṃ pañcasataṃ {\upshape(five-hundred)}}
\example[0]{satassa chakkaṃ chasataṃ {\upshape(six-hundred)}}
\example[0]{satassa sattakaṃ sattasataṃ {\upshape(seven-hundred)}}
\example[0]{satassa aṭṭhakaṃ aṭṭhasataṃ {\upshape(eight-hundred)}}
\example[0]{satassa navakaṃ navasataṃ {\upshape(nine-hundred)}}
\example{satassa dasakaṃ dasasataṃ {\upshape(ten-hundred/one-thousand)}}

\head{393}{393, 415. dasadasakaṃ sataṃ dasakānaṃ sataṃ sahassañca yomhi.}
\headtrans{Because of \pali{yo}-vibhatti, ten times ten [becomes] \pali{sataṃ} and hundred times ten [becomes] \pali{sahassaṃ}.}
\sutdef{gaṇane pariyāpannassa dasadasakassa sataṃ hoti, satadasa\-kassa sahassaṃ hoti yomhi pare.}
\sutdeftrans{In counting, there is \pali{sataṃ}[-substitution] for the whole \pali{dasa\-dasakaṃ} ($10\times 10$), [also] \pali{sahassaṃ}[-substitution] for \pali{satadasa\-kaṃ} ($100\times 10$), because of \pali{yo}-vibhatti behind.}
\example[0]{sataṃ {\upshape(one-hundred)}}
\example{sahassaṃ {\upshape(one-thousand)}}
\sutdef{dvikādīnaṃ taduttarapadānañca nippajjante yathāsaṅkhyaṃ.}
\sutdeftrans{The latter part of terms having `two' and so on are produced [likewise] respectively.}
\example{satassa dvikaṃ dvisataṃ\\{\upshape= Two of hundred, hence \pali{dvisata} (two-hundred).}}
\transnote{The last part looks redundant, because the examples are reproduced from the previous sutta.}

\head{394}{394, 416. yāva taduttari dasaguṇitañca.}
\headtrans{As far that it [is] beyond, [be] multiplied by ten also.}
\sutdef{yāva tāsaṃ saṅkhyānaṃ uttari dasaguṇitañca kātabbaṃ.}
\sutdeftrans{To the extent that [the counting are] beyond those numbers [mentioned above], [they] should be multiplied by ten also.}
\example[0]{dasassa gaṇanassa dasaguṇitaṃ katvā sataṃ hoti \\{\upshape= Having been multiplied by ten for the counting of ten, there is a hundred.}}
\example[0]{satassa dasaguṇitaṃ katvā sahassaṃ hoti \\{\upshape= Having been multiplied by ten for a hundred, there is a thousand.}}
\example[0]{sahassassa dasaguṇitaṃ katvā dasasahassaṃ hoti \\{\upshape= Having been multiplied by ten for a thousand, there is a ten-thousand.}}
\example[0]{dasasahassassa dasaguṇitaṃ katvā satasahassaṃ hoti \\{\upshape= Having been multiplied by ten for a ten-thousand, there is a hundred-thousand.}}
\example[0]{satasahassassa dasaguṇitaṃ katvā dasasatasahassaṃ hoti \\{\upshape= Having been multiplied by ten for a hundred-thousand, there is a ten-hundred-thousand (million).}}
\example[0]{dasasatasahassassa dasaguṇitaṃ katvā koṭi hoti \\{\upshape= Having been multiplied by ten for a ten-hundred-thousand, there is a koṭi (ten-million).}}
\example{koṭisatasahassassa sataguṇitaṃ katvā pakoṭi hoti \\{\upshape= Having been multiplied by hundred for a hundred-thousand-koṭi, there is a pakoṭi.}}

\head{395}{395, 417. sakanāmehi.}
\headtrans{[Be produced] by their own name [for the unexplained].}
\sutdef{yāsaṃ pana saṅkhyānaṃ aniddiṭṭhanāmadheyyānaṃ sakehi sakehi nāmehi nippajjante.}
\sutdeftrans{Which numbers having name unexplained [exist], [those are] produced by their own name each.}
\example[0]{satasahassānaṃ sataṃ koṭi \\{\upshape= Hundred of a hundred-thousand, hence \pali{koṭi} ($10^7$).}}
\example[0]{koṭisatasahassānaṃ sataṃ pakoṭi \\{\upshape= Hundred of a hundred-thousand-koṭi, hence \pali{pakoṭi} ($10^{14}$).}}
\example[0]{pakoṭisatasahassānaṃ sataṃ koṭipakoṭi \\{\upshape= Hundred of a hundred-thousand-pakoṭi, hence \pali{koṭipakoṭi} ($10^{21}$).}}
\example[0]{koṭipakoṭisatasahassānaṃ sataṃ nahutaṃ \\{\upshape= Hundred of a hundred-thousand-koṭipakoṭi, hence \pali{nahuta} ($10^{28}$).}}
\example[0]{nahutasatasahassānaṃ sataṃ ninnahutaṃ \\{\upshape= Hundred of a hundred-thousand-nahuta, hence \pali{ninnahuta} ($10^{35}$).}}
\example[0]{ninnahutasatasahassānaṃ sataṃ akkhobhiṇī \\{\upshape= Hundred of a hundred-thousand-ninnahuta, hence \pali{akkho\-bhiṇī} ($10^{42}$).}}
\example[0]{bindu {\upshape($10^{49}$)}}
\example[0]{abbudaṃ {\upshape($10^{56}$)}}
\example[0]{nirabbudaṃ {\upshape($10^{63}$)}}
\example[0]{ahahaṃ {\upshape($10^{70}$)}}
\example[0]{ababaṃ {\upshape($10^{77}$)}}
\example[0]{aṭaṭaṃ {\upshape($10^{84}$)}}
\example[0]{sogandhikaṃ {\upshape($10^{91}$)}}
\example[0]{uppalaṃ {\upshape($10^{98}$)}}
\example[0]{kumudaṃ {\upshape($10^{105}$)}}
\example[0]{padumaṃ {\upshape($10^{112}$)}}
\example[0]{puṇḍarikaṃ {\upshape($10^{119}$)}}
\example[0]{kathānaṃ {\upshape($10^{126}$)}}
\example[0]{mahākathānaṃ {\upshape($10^{133}$)}}
\example{asaṅkhyeyyaṃ {\upshape($10^{140}$)}}
\transnote{We can break \pali{aniddiṭṭhanāmadheyyānaṃ} in the definition to \pali{na + diṭṭha + nāma + dheyya + naṃ}.}

\head{396}{396, 363. tesaṃ ṇo lopaṃ.}
\headtrans{The \pali{ṇa} of those [paccayas get] elided.}
\sutdef{tesaṃ paccayānaṃ ṇo lopamāpajjate.}
\sutdeftrans{The \pali{ṇa} of those paccayas get elided.}
\example[0]{gotamassa apaccaṃ gotamo (gotama + ṇa)\\{\upshape= The offspring of Gotama, hence \pali{Gotama}.}}
\example[0]{vāsiṭṭho {\upshape(vasiṭṭha + ṇa)}}
\example[0]{venateyyo {\upshape(vinatā + ṇeyya)}}
\example[0]{ālasyaṃ {\upshape(alasa + ṇya)}}
\example{ārogyaṃ {\upshape(aroga + ṇya)}}

\head{397}{397, 420. vibhāge dhā ca.}
\headtrans{In [the sense of] division, [there is] \pali{dhā}-paccaya also.}
\sutdef{vibhāgatthe ca dhāpaccayo hoti.}
\sutdeftrans{There is \pali{dhā}-paccaya in the sense of division also.}
\example[0]{ekena vibhāgena ekadhā (eka + dhā)\\{\upshape= Dividing by one [portion], hence \pali{ekadhā} (in one way).}}
\example[0]{dvidhā {\upshape(in two ways)}}
\example[0]{tidhā {\upshape(in three ways)}}
\example[0]{catudhā {\upshape(in four ways)}}
\example[0]{pañcadhā {\upshape(in five ways)}}
\example{chadhā {\upshape(in six ways)}}
\transnote{Note that after adding \pali{dhā} (also the subsequent paccayas) the whole term is indeclinable. The \pali{so}-paccaya works in a similar way as shown in these examples.}
\example[0]{suttaso {\upshape(by discourse)}}
\example[0]{byañjanaso {\upshape(by letter)}}
\example{padaso {\upshape(by word)}}

\head{398}{398, 421. sabbanāmehi pakāravacane tu thā.}
\headtrans{After pronouns in the expression of manners, [there is] \pali{thā}-paccaya.}
\sutdef{sabbanāmehi pakāravacanatthe thāpaccayo hoti.}
\sutdeftrans{There is \pali{thā}-paccaya after pronouns in the expression of manners.}
\example[0]{so pakāro tathā \\{\upshape= That manner, hence \pali{tathā}.}}
\example[0]{taṃ pakāraṃ tathā \\{\upshape= That manner, hence \pali{tathā}.}}
\example[0]{tena pakārena tathā \\{\upshape= By that manner, hence \pali{tathā}.}}
\example[0]{tassa pakārassa tathā \\{\upshape= For that manner, hence \pali{tathā}.}}
\example[0]{tasmā pakārā tathā \\{\upshape= From that manner, hence \pali{tathā}.}}
\example[0]{tassa pakārassa tathā \\{\upshape= Of that manner, hence \pali{tathā}.}}
\example[0]{tasmiṃ pakāre tathā \\{\upshape= In that manner, hence \pali{tathā}.}}
\example[0]{yathā {\upshape([by/for/from/of/in] which manner)}}
\example[0]{sabbathā {\upshape([by/for/from/of/in] all ways)}}
\example[0]{aññathā {\upshape([by/for/from/of/in] other way)}}
\example{itarathā {\upshape([by/for/from/of/in] other way)}}
\transnote{By the word \pali{tu}, \pali{thatthā}-paccaya can also be used, as shown in the followings.}
\example[0]{so pakāro tathatthā \\{\upshape= That manner, hence \pali{tathatthā}.}}
\example[0]{yathatthā {\upshape([by/for/from/of/in] which manner)}}
\example[0]{sabbathatthā {\upshape([by/for/from/of/in] all ways)}}
\example[0]{aññathatthā {\upshape([by/for/from/of/in] other way)}}
\example{itarathatthā {\upshape([by/for/from/of/in] other way)}}

\head{399}{399, 422. kimimehi thaṃ.}
\headtrans{After \pali{kiṃ} and \pali{ima}, [there is] \pali{thaṃ}-paccaya.}
\sutdef{kiṃimaiccetehi thaṃpaccayo hoti pakāravacanatthe.}
\sutdeftrans{There is \pali{thaṃ}-paccaya after \pali{kiṃ} and \pali{ima} in the expression of manners.}
\example[0]{ko pakāro kathaṃ \\{\upshape= What manner, hence \pali{kathaṃ}.}}
\example[0]{kaṃ pakāraṃ kathaṃ \\{\upshape= What manner, hence \pali{kathaṃ}.}}
\example[0]{kena pakārena kathaṃ \\{\upshape= By what manner, hence \pali{kathaṃ}.}}
\example[0]{kassa pakārassa kathaṃ \\{\upshape= For what manner, hence \pali{kathaṃ}.}}
\example[0]{kasmā pakārā kathaṃ \\{\upshape= From what manner, hence \pali{kathaṃ}.}}
\example[0]{kassa pakārassa kathaṃ \\{\upshape= Of what manner, hence \pali{kathaṃ}.}}
\example[0]{kasmiṃ pakāre kathaṃ \\{\upshape= In what manner, hence \pali{kathaṃ}.}}
\example[0]{ayaṃ pakāro itthaṃ \\{\upshape= This manner, hence \pali{itthaṃ}.}}
\example[0]{imaṃ pakāraṃ itthaṃ \\{\upshape= This manner, hence \pali{itthaṃ}.}}
\example[0]{iminā pakārena itthaṃ \\{\upshape= By this manner, hence \pali{itthaṃ}.}}
\example[0]{imassa pakārassa itthaṃ\\{\upshape= For this manner, hence \pali{itthaṃ}.}}
\example[0]{imasmā pakārā itthaṃ \\{\upshape= From this manner, hence \pali{itthaṃ}.}}
\example[0]{imassa pakārassa itthaṃ\\{\upshape= Of this manner, hence \pali{itthaṃ}.}}
\example{imasmiṃ pakāre itthaṃ \\{\upshape= In this manner, hence \pali{itthaṃ}.}}

\head{400}{400, 364. vuddhādisarassa vā’saṃyogantassa saṇe ca.}
\headtrans{[There is] vuddhi-strength for vowel or non-conjunct [consonant] because of a combination of \pali{ṇa}-paccaya also.}
\sutdef{ādisarassa vā asaṃyogantassa ādibyañjanassa vā sarassa vuddhi hoti saṇakārake paccaye pare.}
\sutdeftrans{There is vuddhi-strength of vowel for the starting vowel or the starting consonant with non-conjunct ending because of a combination of \pali{ṇa}-paccaya behind.}
\example[0]{ābhidhammiko {\upshape(abhidhamma + ṇika)}}
\example[0]{venateyyo {\upshape(vinatā + ṇeyya)}}
\example[0]{vāsiṭṭho {\upshape(vasiṭṭha + ṇa)}}
\example[0]{ālasyaṃ {\upshape(alasa + ṇya)}}
\example{ārogyaṃ {\upshape(aroga + ṇya)}}
\transnote{The vuddhi-strength does not happen to conjunct starters like \pali{bhaggavo}, \pali{manteyyo}, and \pali{kunteyyo}.}

\head{401}{401, 375. mā yūnamāgamo ṭhāne.}
\headtrans{Not [vuddhi-strength], [instead] for \pali{i} and \pali{u}, [there is] insertion [of \pali{e} or \pali{o}] if suitable. }
\sutdef{iuiccetesaṃ ādibhūtānaṃ mā vuddhi hoti. tesu ca eovuddhāgamo hoti ṭhāne.}
\sutdeftrans{There is not vuddhi-strength for the initial \pali{i} and \pali{u}. There is instead an insertion of \pali{e}- or \pali{o}-vuddhi in them if suitable.}
\example[0]{byākaraṇamadhīte veyyākaraṇiko (byākaraṇa + ṇika)\\{\upshape= [One] learns grammar, hence \pali{veyyākaraṇika} (grammarian).}}
\example[0]{nyāyamadhīte neyyāyiko (nyāya + ṇika)\\{\upshape= [One] learns Nyāya, hence \pali{neyyāyika} (student of Nyāya).}}
\example[0]{byāvacchassa apaccaṃ veyyāvaccho (byāvaccha + ṇa)\\{\upshape= [One is] offspring of Byāvaccha, hence \pali{Veyyāvaccha}.}}
\example{dvāre niyutto dovāriko (dvāra + ṇika)\\{\upshape= [One] was appointed at the door, hence \pali{dovārika} (gatekeeper).}}
\transnote{We can break \pali{yūnamāgamo} to \pali{i + u + naṃ + āgama + si}. The word \pali{nyāya} can mean several subjects such as law, logic, and a system of philosophy (see MWD).}
\transnote{By reading the definition and seeing the examples, one may be baffled. There is another operation going under the surface. This is called \pali{saṃprasāraṇa} (Pāṇ 1.1.45), a substitution of vowels for their corresponding semivowel (\pali{i} for \pali{ya} and \pali{u} for \pali{va}).}

\head{402}{402, 377. āttañca.}
\headtrans{[There is] \pali{ā}[-substitution] also.}
\sutdef{iuiccetesaṃ āttañca hoti, rikārāgamo ca ṭhāne.}
\sutdeftrans{There is \pali{ā}[-substitution] for \pali{i} and \pali{u}, also \pali{ri}-insertion if suitable.}
\example[0]{isissa bhāvo ārisyaṃ (isi + ṇya)\\{\upshape= The state of a sage, hence \pali{ārisya} (sagehood).}}
\example[0]{iṇassa bhāvo āṇyaṃ (iṇa + ṇya)\\{\upshape= The state of debt, hence \pali{āṇya} (being in debt).}}
\example[0]{usabhassa bhāvo āsabhaṃ (usabha + ṇa)\\{\upshape= The state of a leader, hence \pali{āsabha} (leadership).}}
\example{ujuno bhāvo ajjavaṃ (uju + ṇa)\\{\upshape= The state of being upright, hence \pali{ajjava} (uprightness).}}

\head{403}{403, 354. kvacādimajjhuttarānaṃ dīgharassā paccayesu ca.}
\headtrans{In some places, for [vowels] at the beginning, in the middle, and at the end, [there are] an elongation and shortening because of paccayas also.}
\sutdef{kvaci ādimajjhauttaraiccetesaṃ dīgharassā honti paccayesu ca apaccayesu ca.}
\sutdeftrans{There are an elongation and shortening for [vowels] at the beginning, in the middle, and at the end, in some places because of paccayas and non-paccayas.}

\bigbullet{(1) Elongation at the beginning}
\example[0]{pākāro {\upshape(a rampart)}}
\example[0]{nīvāro {\upshape(a prevention)}}
\example[0]{pāsādo {\upshape(a mansion)}}
\example[0]{pākaṭo {\upshape([a thing] openly revealed)}}
\example[0]{pātimokkho {\upshape(the Pātimokkha)}}
\example{pāṭikaṅkho {\upshape([a thing] expected)}}

\bigbullet{(2) Elongation in the middle}
\example[0]{aṅgamāgadhiko {\upshape(residents of Aṅga and Māgadha)}}
\example{orabbhamāgaviko {\upshape(a hunter of ram and deer)}}

\bigbullet{(3) Elongation at the end}
\example[0]{khantī paramaṃ tapo titikkhā \\{\upshape= Patience, [or] endurance, [is] an excellent austerity.}}
\example[0]{añjanā giri {\upshape(the mount Añjanā)}}
\example[0]{koṭarā vanaṃ {\upshape(the Koṭarā park)}}
\example{aṅgulī {\upshape(a finger)}}

\bigbullet{(4) Shortening at the beginning}
\example{pageva (pā + eva) {\upshape(not to mention)}}

\bigbullet{(5) Shortening in the middle}
\example[0]{sumedhaso {\upshape(a wise one)}}
\example{suvaṇṇadharehi {\upshape(be one carrying gold)}}

\bigbullet{(6) Shortening at the end}
\example[0]{bhovādi nāma so hoti \\{\upshape= He is named Bhovādī.}}
\example{yathābhāvi guṇena so \\{\upshape= His [is] fated to be as such by virtue.}}

\transnote{Several examples above look shoehorned just to make the point.}

\head{404}{404, 370. tesu vuddhilopāgamavikāraviparītādesā ca.}
\headtrans{In those, [there are] vuddhi-strength-transformation, elision, insertion, alteration, deformation, and substitution.}
\sutdef{tesu ādimajjhuttaresu yathājinavacanānuparodhena kvaci vuddhi hoti, kvaci lopo hoti, kvaci āgamo hoti, kvaci vikāro hoti, kvaci viparīto hoti, kvaci ādeso hoti.}
\sutdeftrans{There is vuddhi-strength in those [vowels], at the beginning, in the middle, and at the end, in some places without going against such teachings of the Buddha. [Likewise] there is an elision in some places. There is an insertion in some places. There is an alteration in some places. There is a deformation in some places. There is a substitution in some places.}
\transnote{I found several of examples in this sutta redundant. Other some of them came from an attempt to make relevant instances, but I cannot see the points. So, I skip these examples altogether.}

\head{405}{405, 365. ayuvaṇṇānañcāyo vuddhi.}
\headtrans{For \pali{a}, \pali{i}-group, and \pali{u}-group, [there are] \pali{ā}-, \pali{e}-, and \pali{o}-vuddhi.}
\sutdef{a iti akāro, iī iti ivaṇṇo, uū iti uvaṇṇo, tesaṃ akāraivaṇṇuvaṇṇānaṃ āeovuddhiyo honti yathāsaṅkhyaṃ, āīūvuddhi ca.}
\sutdeftrans{[There are] \pali{a}, \pali{i}-group (\pali{i} and \pali{ī}), and \pali{u}-group (\pali{u} and \pali{ū}). There are \pali{ā}-, \pali{e}-, and \pali{o}-vuddhi for these \pali{a}, \pali{i}-group, and \pali{u}-group respectively, as well as \pali{ā}-, \pali{ī}-, and \pali{ū}-vuddhi.}
\example[0]{ābhidhammiko (abhidhamma + ṇika)}
\example[0]{venateyyo (vinatā + ṇeyya)}
\example{oḷumpiko (uḷumpa + ṇika)}
\transnote{The chunk in the formula can be broken to \pali{a + i + u + vaṇṇa + naṃ + ca + ā + e + o}.}

