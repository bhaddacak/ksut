\chapter{Uṇādi}

This last part of Kaccāyana, 50 suttas totally, continues describing the operations of primary derivation, mostly on nominal products. The name \pali{uṇādi} (\pali{uṇ} and so on), which was taken from Pāṇini (e.g., in Pāṇ 3.3.1, \pali{uṇādayo bahulam}), means nothing here because we have no such a paccaya.

Paccayas described in this part are listed together with other \pali{kita}-paccayas in Table \ref{tab:kitapacc}, but roots specifically named are listed in Table \ref{tab:roots-unadi} below. If a long list is mentioned, only the first one is shown.

In the suttas to come, the readers may feel frustrated when they go through the examples with peculiar analyses, not to mention the composition in a more cryptic style and poorer editorial quality. This should be warned in advance. Many of analyses might be just a speculation, and some paccayas might be postulated \pali{ad hoc}. You have definite right to accept or reject the ideas. However, the knowledge obtained here helps strengthen your Pāli substantially. Consider the following note on the same topic in Pāṇinian grammar by J.\,A.\,F. Roodbergen:

\begin{quotation}
Whereas the \pali{kṛdanta} words are usually etymologically clear, the \pali{uṇādyanta} words are not. All words derived in this way convey a meaning not necessarily resulting from the combination of stem and suffix. Whimsical attempt along the lines of \pali{Nirukta}.\footnote{See \citealp[p.~100]{roodbergen:gramdict}. Nirukta is the title of a treatise dealing with the etymology of obsolete Vedic words (p.~249).}
\end{quotation}

\setcounter{table}{0}
\begin{longtable}{%
		>{\raggedright\arraybackslash}p{0.30\linewidth}%
		>{\raggedright\arraybackslash}p{0.38\linewidth}}
\caption{Roots having particular treatments}\label{tab:roots-unadi}\\
\toprule
\upshape\bfseries Root & \bfseries Suttas \\ \midrule
\endfirsthead
\multicolumn{2}{c}{\footnotesize\tablename\ \thetable: Roots having particular treatments (contd\ldots)}\\
\toprule
\upshape\bfseries Root & \bfseries Suttas \\ \midrule
\endhead
\bottomrule
\ltblcontinuedbreak{2}
\endfoot
\bottomrule
\endlastfoot
%
\paliroot{ada} & \hyperref[sut:627]{627} \\
\paliroot{ala} & \hyperref[sut:632]{632} \\
\paliroot{asa} & \hyperref[sut:660]{660} \\
\paliroot{āma} & \hyperref[sut:664]{664} \\
\paliroot{usu} & \hyperref[sut:659]{659} \\
\paliroot{kaḍi}, \ldots & \hyperref[sut:663]{663} \\
\paliroot{kala} & \hyperref[sut:632]{632}, \hyperref[sut:633]{633} \\
\paliroot{ku} & \hyperref[sut:644]{644} \\
\paliroot{kuṭa}, \ldots & \hyperref[sut:672]{672} \\
\paliroot{khāda} & \hyperref[sut:664]{664} \\
\paliroot{khī} & \hyperref[sut:627]{627} \\
\paliroot{gamu}, \ldots & \hyperref[sut:651]{651}, \hyperref[sut:664]{664} \\
\paliroot{gaha} & \hyperref[sut:629]{629} \\
\paliroot{cara} (ā-) & \hyperref[sut:631]{631} \\
\paliroot{ci}, \ldots & \hyperref[sut:668]{668} \\
\paliroot{chada}, \ldots & \hyperref[sut:656]{656} \\
\paliroot{ṭhā} & \hyperref[sut:628]{628} \\
\paliroot{dama} & \hyperref[sut:628]{628} \\
\paliroot{dara} & \hyperref[sut:628]{628} \\
\paliroot{dava} & \hyperref[sut:644]{644} \\
\paliroot{daṃsa} & \hyperref[sut:659]{659} \\
\paliroot{dā} & \hyperref[sut:628]{628}, \hyperref[sut:644]{644} \\
\paliroot{disa} & \hyperref[sut:642]{642} \\
\paliroot{du} & \hyperref[sut:628]{628} \\
\paliroot{dhū} & \hyperref[sut:627]{627} \\
\paliroot{nuda}, \ldots & \hyperref[sut:641]{641} \\
\paliroot{pī} & \hyperref[sut:627]{627} \\
\paliroot{putha} & \hyperref[sut:666]{666} \\
\paliroot{pūja} & \hyperref[sut:643]{643} \\
\paliroot{budhi} & \hyperref[sut:643]{643} \\
\paliroot{bhasa} & \hyperref[sut:628]{628} \\
\paliroot{bhī} & \hyperref[sut:627]{627}, \hyperref[sut:628]{628}, \hyperref[sut:643]{643} \\
\paliroot{bhū} & \hyperref[sut:644]{644} \\
\paliroot{mati} & \hyperref[sut:643]{643} \\
\paliroot{matha} & \hyperref[sut:634]{634} \\
\paliroot{manu}, \ldots & \hyperref[sut:673]{673} \\
\paliroot{masu} & \hyperref[sut:630]{630} \\
\paliroot{mida}, \ldots & \hyperref[sut:658]{658} \\
\paliroot{muna}, \ldots & \hyperref[sut:669]{669} \\
\paliroot{yā} & \hyperref[sut:628]{628} \\
\paliroot{ranja}, \ldots & \hyperref[sut:659]{659}, \hyperref[sut:661]{661} \\
\paliroot{raha} & \hyperref[sut:628]{628} \\
\paliroot{ru} & \hyperref[sut:627]{627} \\
\paliroot{lū} & \hyperref[sut:627]{627} \\
\paliroot{vaja}, \ldots & \hyperref[sut:638]{638} \\
\paliroot{vada}, \ldots & \hyperref[sut:657]{657} \\
\paliroot{vamu} & \hyperref[sut:644]{644} \\
\paliroot{vā} & \hyperref[sut:627]{627} \\
\paliroot{vida}, \ldots & \hyperref[sut:670]{670} \\
\paliroot{vepu} & \hyperref[sut:644]{644} \\
\paliroot{vu} & \hyperref[sut:660]{660} \\
\paliroot{sama} & \hyperref[sut:628]{628} \\
\paliroot{sala} & \hyperref[sut:632]{632}, \hyperref[sut:633]{633} \\
\paliroot{sasu}, \ldots & \hyperref[sut:667]{667} \\
\paliroot{sā} & \hyperref[sut:628]{628} \\
\paliroot{si} & \hyperref[sut:628]{628} \\
\paliroot{sī} & \hyperref[sut:644]{644} \\
\paliroot{su} & \hyperref[sut:627]{627}, \hyperref[sut:647]{647} \\
\paliroot{sū} & \hyperref[sut:660]{660} \\
\paliroot{hana}, \ldots & \hyperref[sut:671]{671} \\
\paliroot{hi} & \hyperref[sut:627]{627}, \hyperref[sut:628]{628}, \hyperref[sut:662]{662} \\
\paliroot{hu} & \hyperref[sut:627]{627}, \hyperref[sut:644]{644} \\
\end{longtable}

\section{Chaṭṭhakaṇḍa}
\raggedbottom

\head{624}{624, 563. kattari kita.}
\headtrans{In [the sense of] agent, [there are] \pali{kita}-paccayas.}
\sutdef{kattuiccetasmiṃ atthe kitapaccayā honti.}
\sutdeftrans{There are \pali{kita}-paccayas in the sense of agent (active meaning).}
\example[0]{kāru (\paliroot{kara} + ṇu) = a doer}
\example[0]{kāruko (\paliroot{kara} + ṇuka) = a doer}
\example[0]{kārako (\paliroot{kara} + ṇvu) = a doer}
\example[0]{pācako (\paliroot{paca} + ṇvu) = a cook}
\example[0]{kattā (\paliroot{kara} + tu) = a doer}
\example[0]{janitā (\paliroot{jana} + tu) = a producer/father}
\example[0]{pacitā (\paliroot{paca} + tu) = a cook}
\example{netā (\paliroot{nī} + tu) = a leader}
\transnote{We can see the definition of \pali{kita}-paccayas in \hyperref[sut:546]{Kacc 546}.}

\head{625}{625, 605. bhāvakammesu kiccaktakhatthā.}
\headtrans{In impersonal and passive [meaning, there are] \pali{kicca}-pacca\-yas, \pali{kita}-paccayas, and \pali{kha}.}
\sutdef{bhāvakammaiccetesvatthesu kiccaktakhatthaiccete paccayā honti.}
\sutdeftrans{There are paccayas of \pali{kicca}, \pali{kita}, and in the sense of \pali{kha}-paccaya in impersonal and passive meaning.}
\transnote{In the formula and definition, \pali{kta} is dubious. It should be seen as \pali{kita}.}

\bigbullet{(1) With \pali{kicca}-paccayas}
\example[0]{upasampādetabbaṃ (upa + saṃ + \paliroot{pada} + tabba) bhavatā \\{\upshape= Ordaining should be done by you, sir.}}
\example[0]{upasampādanīyaṃ (upa + saṃ + \paliroot{pada} + anīya) bhavatā \\{\upshape= Ordaining should be done by you, sir.}}
\example[0]{sayitabbaṃ (\paliroot{sī} + tabba) bhavatā \\{\upshape= Sleeping should be done by you, sir.}}
\example[0]{kattabbaṃ (\paliroot{kara} + tabba) bhavatā \\{\upshape= [It] should be done by you, sir.}}
\example[0]{bhottabbo (\paliroot{bhuja} + tabba) odano bhavatā \\{\upshape= The boiled rice should be eaten by you, sir.}}
\example{asitabbaṃ (\paliroot{asa} + tabba) bhojanaṃ bhavatā \\{\upshape= The food should be eaten by you, sir.}}

\bigbullet{(2) With \pali{kita}-paccayas}
\example[0]{asitaṃ (\paliroot{asa} + ta) bhavatā \\{\upshape= Eating was done by you, sir.}}
\example[0]{sayitaṃ (\paliroot{sī} + ta) bhavatā \\{\upshape= Sleeping was done by you, sir.}}
\example[0]{pacitaṃ (\paliroot{paca} + ta) bhavatā \\{\upshape= Cooking was done by you, sir.}}
\example[0]{asitaṃ (\paliroot{asa} + ta) asanaṃ bhavatā \\{\upshape= The food was eaten by you, sir.}}
\example[0]{sayitaṃ (\paliroot{sī} + ta) sayanaṃ bhavatā \\{\upshape= The bed was slept on by you, sir.}}
\example{pacito (\paliroot{paca} + ta) odano bhavatā \\{\upshape= The boiled rice was cooked by you, sir.}}

\bigbullet{(3) With \pali{kha}}
\example[0]{kiñcissayo (kiñci + \paliroot{sī} + kha) \\{\upshape= a little sleep}}
\example[0]{īsassayo (īsa + \paliroot{sī} + kha) \\{\upshape= a little sleep}}
\example[0]{dussayo (du + \paliroot{sī} + kha) \\{\upshape= a difficult sleep}}
\example{sussayo (su + \paliroot{sī} + kha) \\{\upshape= an easy sleep}}
\transnote{The definition of \pali{kicca}-paccayas is stated in \hyperref[sut:545]{Kacc 545}.}

\head{626}{626, 634. kammani dutiyāyaṃ kto.}
\headtrans{When an object with an accusative vibhatti [exists, there is] \pali{kta}-paccaya.}
\sutdef[0]{kammani iccetasmiṃ atthe dutiyāyaṃ vibhattiyaṃ kattari ktapaccayo hoti.}
\sutdef{kammatthe dutiyāyaṃ vibhattiyaṃ vijjamānāyaṃ dhātūhi kattari ktappaccayo hoti.}
\sutdeftrans{When a sense of object with an accusative vibhatti exists, there is \pali{kta}-paccaya in active meaning after roots.}
\transnote{The translation is done upon the second line, a rephrased version taken from Rūpa 634.}
\example[0]{dānaṃ dinno (\paliroot{dā} + kta) devadatto \\{\upshape= [There is] Devadatta who gave alms.}}
\example[0]{sīlaṃ rakkhito (\paliroot{rakkha} + kta) devadatto \\{\upshape= [There is] Devadatta who observed the precept.}}
\example[0]{bhattaṃ bhutto (\paliroot{bhuja} + kta) devadatto \\{\upshape= [There is] Devadatta who ate food.}}
\example{garuṃ upāsito (upa + \paliroot{āsa} + kta) devadatto \\{\upshape= [There is] Devadatta who approached the teacher.}}
\transnote{This paccaya looks suspicious. It behaves like \pali{ta} with \pali{ka} elided. The elision is done by \hyperref[sut:517]{Kacc 517}, noted by Rūpa 634. The sutta mentioned is like a fallback because it can explain anything. I think \pali{kta} is made different from \pali{ta} because it only creates nominal products.}

\head{627}{627, 652. khyādīhi man ma ca to vā.}
\headtrans{After \pali{khi} and so on, [there is] \pali{man}-paccaya. Also \pali{ma} [becomes] \pali{ta} sometimes.}
\sutdef{khi\,bhī\,su\,ru\,hu\,vā\,dhū\,hi\,lū\,pī\,ada\,iccevamādīhi\ \ dhātūhi\ \  man paccayo hoti, massa ca to hoti vā.}
\sutdeftrans{There is \pali{man}-paccaya after roots such as \pali{khi}, \pali{bhī}, \pali{su}, \pali{ru}, \pali{hu}, \pali{vā}, \pali{dhū}, \pali{hi}, \pali{lū}, \pali{pī}, \pali{ada} and so on. There is also [a substitution of] \pali{ta} for \pali{ma} sometimes.}
\example[0]{khīyanti upaddavā etthāti khemo (\paliroot{khī} + man) \\{\upshape= The calamities are destroyed here, hence \pali{khema} (the state of peace).}}
\example[0]{bhāyitabboti, bhāyanti etasmāti vā bhemo (\paliroot{bhī} + man) \\{\upshape= [It is] worthy to be feared, or [they] fear from that, hence \pali{bhemo} (fearful [thing/person]).}}
\example[0]{raṃsiyo abhissavetīti somo (\paliroot{su} + man) \\{\upshape= [It] rediates a ray, hence \pali{somo} (the moon).}}
\example{ravati gacchatīti romo (\paliroot{ru} + man) \\{\upshape= [It] grows, hence \pali{romo} (hair).}}
\transnote{I cannot find a relevant meaning of \pali{ravati}. It is solely constructed by theory regardless of its actual use. It seems to mean \pali{rūhati} as suggested in a Thai source. The sentence therefore has two synonymous verbs, so do many of the subsequent examples.}
\example[0]{huvati juhvati etenāti homo (\paliroot{hu} + man) \\{\upshape= [One] makes an oblation [or] offers by that, hence \pali{homo} (offering).}}
\example[0]{paṭilomavasena vāti gacchatīti vāmo (\paliroot{vā} + man) \\{\upshape= [It] goes [or] blows in the reverse direction, hence \pali{vāma} (left/on the left).}}
\example[0]{lāmakavasena vāti gacchati pavattatīti vā vāmo (\paliroot{vā} + man) \\{\upshape= [It] happens, goes, or blows badly, hence \pali{vāma} (badness/bad).}}
\example[0]{dhunāti kampatīti dhūmo (\paliroot{dhū} + man) \\{\upshape= [It] shakes [or] wavers, hence \pali{dhūma} (smoke).}}
\example[0]{seṭṭhabhāvena hinoti pavattati cittaṃ etasminti hemo \\(\paliroot{hi} + man) \\{\upshape= The mind moves [or] urges into that by [its] excellent state, hence \pali{hema} (gold).}}
\example[0]{lunitabboti, maṃsacammāni lunāti chindatīti vā lomo \\(\paliroot{lū} + man) \\{\upshape= [It is] worthy to be cut, or [it] cuts [or] breaks the flesh and skin, hence \pali{loma} (hair).}}
\example[0]{piyanaṃ, piyāyitabboti vā pemo (\paliroot{pī} + man) \\{\upshape= Loving, or [it is] worthy to be loved, hence \pali{pema} (affection).}}
\example{sukhadukkhaṃ adati bhakkhatīti, jātijarāmaraṇādīhi adīyate bhakkhīyateti vā attā, ātumā (\paliroot{ada} + man) \\{\upshape= [It] eats [or] feeds upon happiness and suffering, or [it is] eaten [or] fed upon by birth, old age, death and so on, hence \pali{atta, ātuma} (the self).}}
\transnote{This is the first time I encounter an instance ending with a consonant (other than ṃ) in this work. So, it looks unusual to Pāli students. In fact, we should see this as \pali{maṇ} (or better, \pali{ṇma}, corresponding to \pali{ṇya} and \pali{ṇvu}), in which the \pali{ṇa}-anubandha causes vuddhi strength and gets elided.}

\head{628}{628, 653. samādīhi thamā.}
\headtrans{After \pali{sama} and so on, [there are] \pali{tha}- and \pali{ma}-paccaya.}
\sutdef{samu\,damu\,dara\,raha\,du\,hi\,si\,bhī\,dā\,yā\,sā\,ṭhā\,bhasa\,iccevam\-ādīhi\ \ dhātūhi\ \ thamapaccayā\ \ honti.}
\sutdeftrans{There are \pali{tha}- and \pali{ma}-paccaya after roots such as \pali{samu}, \pali{damu}, \pali{dara}, \pali{raha}, \pali{du}, \pali{hi}, \pali{si}, \pali{bhī}, \pali{dā}, \pali{yā}, \pali{sā}, \pali{ṭhā}, \pali{bhasa} and so on.}
\example[0]{sametīti samatho (\paliroot{sama} + tha) \\{\upshape= [It] makes still, hence \pali{samatha} (tranquillity).}}
\example[0]{damatīti, damanaṃ vā, damitabboti vā damatho \\(\paliroot{dama} + tha) \\{\upshape= [It] tames, or [it is a] taming, or [it is] worthy to be tamed, hence \pali{damatha} (self-control).}}
\example[0]{daratīti daratho (\paliroot{dara} + tha) \\{\upshape= [It] tears apart, hence \pali{daratha} (anxiety).}}
\example{jiṇṇabhāvaṃ rahissati gaṇhissatīti, dabbasambhāre rahati gaṇhātīti vā ratho (\paliroot{raha} + tha) \\{\upshape= [It] will take degeneration, or [it] holds building materials, hence \pali{ratha} (chariot).}}
\transnote{The descriptions of \pali{ratha} here look out of place. It has nothing to do with \paliroot{raha}, e.g., \pali{rahati} (to give up/abandon) or \pali{rahita} (forsaken/without). See both terms in PTSD. The descriptions given bend toward the teaching, rather than giving an accurate analysis. The term in fact came from \paliroot{ṛ} (goer). See further \pali{ratha} in MWD.}
\example[0]{davati gacchatīti, davati vuddhi viruḷhi gacchati pavattatīti uddhaṃ vā dumo (\paliroot{du} + ma) \\{\upshape= [It] goes, or [its] growth [or] development goes [or] happens upward, hence \pali{duma} (tree).}}
\example[0]{pathavīpabbatādīsu gacchati patatīti himo (\paliroot{hi} + ma) \\{\upshape= [It] goes [or] falls upon the ground, mountain and so on, hence \pali{hima} (snow).}}
\example[0]{kammavācāya bandhati etthāti, bandhitabbāti vā sīmā \\(\paliroot{si} + ma) \\{\upshape= [It] confines [the area] by/for an official proclamation in this [religion], or [it is] worthy to be confined, hence \pali{sīmā} (boundary).}}
\example[0]{bhāyanti etasmāti bhīmo (\paliroot{bhī} + ma) \\{\upshape= [They] fear from that, hence \pali{bhīma} (fearful [thing/person]).}}
\example[0]{satte avakhaṇḍenti nivārenti etenāti, mūsikādīhi khādīyati avakhaṇḍīyatīti vā dāmo (\paliroot{dā} + ma) \\{\upshape= [They] prevent [or] obstruct beings/animals by that, or [it] is bitten [or] broken by rats and so on, hence \pali{dāma} (rope).}}
\example[0]{yāti gacchatīti yāmo (\paliroot{yā} + ma) \\{\upshape= [It] goes, hence \pali{yāma} (watch of the night).}}
\example[0]{paresaṃ cittaṃ gaṇhituṃ samatthetīti sāmo (\paliroot{sā} + ma) \\{\upshape= The mind is able to pick [=? tell apart] from others, hence \pali{sāma} (dark color).}}
\example[0]{tiṭṭhanti etenāti thāmo (\paliroot{ṭhā} + ma) \\{\upshape= [They] stand by that, hence \pali{thāma} (strength).}}
\example{bhasati bhasmīkarīyatīti bhasmā (\paliroot{bhasa} + ma) \\{\upshape= [It] is made into ash, hence \pali{bhasmā} (ash).}}
\transnote{The last example is circular and poorly formed, so do not take this and some other similar cases seriously.}

\head{629}{629, 569. gahassupadhasse vā.}
\headtrans{[The vowel] of the penultimate syllable of \pali{gaha} [becomes] \pali{e} sometimes.}
\sutdef{gahaiccetassa dhātussa upadhassa akārassa etta hoti vā.}
\sutdeftrans{There is [a substitution of] \pali{e} for the letter \pali{a} of the penultimate syllable of the root \pali{gaha} sometimes.}
\example{dabbasambhāraṃ gaṇhātīti gehaṃ (\paliroot{gaha} + a) \\{\upshape= [It] holds building materials, hence \pali{geha} (house).}}
\transnote{Here, \pali{upadhā} is a technical term denoting the pre-final pho\-neme of a given unit (Pāṇ 1.1.65). See also \citealp{roodbergen:gramdict}, p.~108.}

\head{630}{630, 654. masussa sussa ccharaccherā.}
\headtrans{[There are] \pali{cchara}- and \pali{cchera}-substitution for \pali{su} of \pali{masu}.}
\sutdef{masuiccetassa pāṭipadikassa sussa ccharaccherādesā honti.}
\sutdeftrans{There are substitutions of \pali{cchara} and \pali{cchera} for \pali{su} of the nominal base of \pali{masu}.}
\example{maccharatīti maccharo, macchero (\paliroot{masu} + kvi) \\{\upshape= [One] bahaves niggardly, hence \pali{macchara/macchera} (niggardliness).}}
\transnote{Interestingly, this is the first time we see \pali{pāṭipadika} (Skt.\ \pali{prātipadika}) in use. This term is equivalent to \pali{liṅga} (nominal base) used in the former parts. See further in \hyperref[sut:284]{284} and the box note after that. However, Dhātumañjūsā treats \paliroot{masu} as a root meaning \pali{macchare}. The end product in Sanskrit seems to be \pali{matsara}, a product of \paliroot{mad} or \paliroot{mada} (\pali{ummāde}) in Pāli.}

\head{631}{631, 655. āpubbacarassa ca.}
\headtrans{For \pali{cara} with \pali{ā} in the front also, [there are \pali{cchariya}-, \mbox{\pali{cchara}-,} and \pali{cchera}-substitution].}
\sutdef{āpubbassa caraiccetassa dhātussa cchariyaccharaccherādesā honti, āpubbassa ca rasso hoti.}
\sutdeftrans{There are substitutions of \pali{cchariya}, \pali{cchara}, and \pali{cchera} for the root \pali{cara} with \pali{ā} in the front. The \pali{ā} in the front also becomes shortened.}
\example{ābhuso caritabbanti acchariyaṃ, accharaṃ, accheraṃ \\(ā +\paliroot{cara} + kvi) \\{\upshape= [It is] worthy to be practised intensely, hence \pali{acchariya/acchara/acchera} (wonder/wonderful).}}
\transnote{If you cannot find \pali{ābhusa}, try \pali{bhusa}. As marked by \pali{ca}, apart from the two forms described in the previous sutta, \pali{macchariya} can also be the case.}

\head{632}{632, 656. alakalasalehi layā.}
\headtrans{After \pali{ala}, \pali{kala}, and \pali{sala}, [there are] \pali{la}- and \pali{ya}-paccaya.}
\sutdef{alakalasalaiccetehi dhātūhi layapaccayā honti.}
\sutdeftrans{There are \pali{la}- and \pali{ya}-paccaya after the roots \pali{ala}, \pali{kala}, and \pali{sala}.}
\example[0]{alati samatthetīti allaṃ/alyaṃ (\paliroot{ala} + la/ya) \\{\upshape= [It] is capable, hence \pali{alla} (wet).}}
\example[0]{sajjati etthāti allaṃ (\paliroot{ala} + la) \\{\upshape= [Water] is attached in this, hence \pali{alla} (wet).}}
\example[0]{kalitabbaṃ saṅkhyātabbanti kallaṃ/kalyaṃ (\paliroot{kala} + la/ya) \\{\upshape= [It is] worthy to be counted, hence \pali{kalla} (proper).}}
\example{salati gacchati pavisatīti sallaṃ/salyaṃ (\paliroot{sala} + la/ya) \\{\upshape= [It] goes [or] enters [into things], hence \pali{salla} (arrow/dart).}}
\transnote{The second example, which sounds more understandable, is taken from a Thai source.}

\head{633}{633, 657. yāṇalāṇā.}
\headtrans{[There are] \pali{yāṇa}- and \pali{lāṇa}-paccaya [after the roots \pali{kala} and \pali{sala}].}
\sutdef{tehi kalasalaiccetehi dhātūhi yāṇalāṇapaccayā honti.}
\sutdeftrans{There are \pali{yāṇa}- and \pali{lāṇa}-paccaya after the roots \pali{kala} and \pali{sala}.}
\example[0]{kalitabbaṃ saṅkhyātabbanti kalyāṇaṃ (\paliroot{kala} + yāṇa) \\{\upshape= [It is] worthy to be counted, hence \pali{kalyāṇa} (good).}}
\example[0]{gaṇato paṭikkamitvā salanti etthāti paṭisalyāṇaṃ \\(paṭi + \paliroot{sala} + yāṇa) \\{\upshape= [One], having retreated from the group, goes here, hence \pali{paṭisalyāṇa} (seclusion).}}
\example[0]{sallāṇo (\paliroot{sala} + lāṇa)}
\example{paṭisallāṇo (paṭi + \paliroot{sala} + lāṇa)}

\head{634}{634, 658. mathissa thassa lo ca.}
\headtrans{[There is] \pali{la}-substitution for \pali{tha} of \pali{matha} also.}
\sutdef{mathaiccetassa dhātussa thassa lādeso hoti.}
\sutdeftrans{There is a substitution of \pali{la} for \pali{tha} of the root \pali{matha}.}
\example{aññamaññaṃ mathati viloḷatīti mallo/mallaṃ \\(\paliroot{matha} + a) \\{\upshape= [One] wrestles [or] shakes another mutually, hence \pali{malla} (wrestler/wrestling).}}
\transnote{As marked by \pali{ca}, an insertion of \pali{ka} can also be applied, hence \pali{mallako/mallakaṃ}.}

\head{635}{635, 559. pesātisaggapattakālesu kiccā.}
\headtrans{In the senses of urging, granting permission, and [telling] a proper time, [there are] \pali{kicca}-paccayas.}
\sutdef{pesa\,atisagga\,pattakāla\,iccetesvatthesu\ \ kiccapaccayā\ \ honti.}
\sutdeftrans{There are \pali{kicca}-paccayas in the senses of urging (\pali{pesa}), granting permission (\pali{atisagga}), and [telling] a proper time (\pali{pattakāla}).}
\example[0]{kattabbaṃ kammaṃ bhavatā \\{\upshape= The action should be done by you, sir.}}
\example[0]{karaṇīyaṃ kiccaṃ bhavatā \\{\upshape= The duty should be done by you, sir.}}
\example[0]{bhottabbaṃ/bhojanīyaṃ bhojanaṃ bhavatā \\{\upshape= The food should be eaten by you, sir.}}
\example{ajjhayitabbaṃ/ajjhayanīyaṃ ajjheyyaṃ bhavatā \\{\upshape= The lesson/sutta should be learned/recited by you, sir.}}
\transnote{The passive (\pali{kicca}) paccayas are defined in \hyperref[sut:545]{Kacc 545}. In the examples, we can see \pali{tabba}, \pali{anīya}, \pali{ramma}, \pali{ricca}, and \pali{ṇya} in use. In the last example, \pali{ajjheyya} is \pali{adhi + \paliroot{i} + ṇya} (see also \pali{ajjhena} in PTSD).}
\transnote{For terms used in the sutta, see \pali{preṣa} in MWD for \pali{pesa}, \pali{atisarga} for \pali{atisagga}, and \pali{prāpta} for \pali{patta}. This sutta is similar to Kāt 4.427.}

\head{636}{636, 659. avassakā’dhamiṇesu ṇī ca.}
\headtrans{In the senses of certainty and debt, [there is] \pali{ṇī}-paccaya also, [and \pali{kicca}-paccayas].}
\sutdef{avassaka\,adhamiṇa\,iccetesvatthesu\ \ ṇīpaccayo\ \ hoti, kiccā ca.}
\sutdeftrans{There is \pali{ṇī}-paccaya in the senses of certainty and debt, \pali{kicca}-paccayas also.}
\example[0]{kārī (\paliroot{kara} + ṇī) asi me kammaṃ avassaṃ \\{\upshape= [You] are certainly the doer of my work.}}
\example[0]{hārī (\paliroot{hara} + ṇī) asi me bhāraṃ avassaṃ \\{\upshape= [You] are certainly the carrier of my burden.}}
\example[0]{dāyī (\paliroot{dā} + ṇī) asi me sataṃ iṇaṃ \\{\upshape= [You] are the payer of a hundred of debt to me.}}
\example[0]{dhārī (\paliroot{dhara} + ṇī) asi me sahassaṃ iṇaṃ \\{\upshape= [You] are the holder of a thousand of debt to me.}}
\example[0]{dātabbaṃ (\paliroot{dā} + tabba) me bhavatā sataṃ iṇaṃ \\{\upshape= The debt of a hundred should be given to me by you, sir.}}
\example[0]{dhārayitabbaṃ (\paliroot{dhara} + tabba) me bhavatā sahassaṃ iṇaṃ \\{\upshape= The debt of a thousand should be held for me by you, sir.}}
\example[0]{kattabbaṃ (\paliroot{kara} + tabba) me bhavatā gehaṃ \\{\upshape= The house should be made for me by you, sir.}}
\example[0]{karaṇīyaṃ (\paliroot{kara} + anīya) me bhavatā kiccaṃ \\{\upshape= The work should be done for me by you, sir.}}
\example{kāriyaṃ (\paliroot{kara} + ṇya) me bhavatā sayanaṃ \\{\upshape= The bed should be made for me by you, sir.}}
\transnote{In the sutta, \pali{adhamiṇa} (debtor) is \pali{adhama + iṇa}. This comes from Skt.\ \pali{adhamarṇa} (see MWD). I found a similar sūtra of this in Kāt 4.428.}

\head{637}{637. arahasakkādīhi tuṃ.}
\headtrans{By [an application of] worthiness, capableness, and possibleness, [there is] \pali{tuṃ}-paccaya.}
\sutdef{araha\,sakka\,bhabba\,iccevamādīhi\ \ payoge\ \ sati\ \ sabbadhātūhi tuṃpaccayo hoti.}
\sutdeftrans{When an appication by worthiness, capableness, and possibleness exists, there is \pali{tuṃ}-paccaya after all roots.}
\example[0]{arahā bhavaṃ vattuṃ (\paliroot{vada} + tuṃ) \\{\upshape= You, sir, deserve to speak.}}
\example[0]{arahā bhavaṃ kattuṃ (\paliroot{kara} + tuṃ) \\{\upshape= You, sir, deserve to do.}}
\example[0]{sakkā bhavaṃ hantuṃ (\paliroot{hana} + tuṃ) \\{\upshape= You, sir, are capable to kill.}}
\example[0]{sakkā bhavaṃ janetuṃ/janituṃ (\paliroot{jana} + tuṃ) \\{\upshape= You, sir, are capable to generate/give birth.}}
\example[0]{sakkā bhavaṃ bhavituṃ (\paliroot{bhū} + tuṃ) \\{\upshape= You, sir, are capable to become/be born.}}
\example[0]{sakkā bhavaṃ dātuṃ (\paliroot{dā} + tuṃ) \\{\upshape= You, sir, are capable to give.}}
\example[0]{sakkā bhavaṃ gantuṃ (\paliroot{gamu} + tuṃ) \\{\upshape= You, sir, are capable to go.}}
\example{bhabbo bhavaṃ janetuṃ (\paliroot{jana} + tuṃ) \\{\upshape= You, sir, [are] likely to give birth.}}
\transnote{In the examples, \pali{arahā} and \pali{sakkā} are used as indeclinables, but \pali{bhabbo} is a masculine adjective modifying nominative \pali{bhavaṃ} (you), declined irregularly from \pali{bhavanta}.}

\head{638}{638, 660. vajādīhi pabbajjādayo nippajjante.}
\headtrans{The terms \pali{pabbajjā} and so on are produced from \pali{vaja} and so on.}
\sutdef{vajaiccevamādīhi dhātūhi, upasaggapaccayādīhi ca pabbajjādayo saddā nippajjante.}
\sutdeftrans{The terms such as \pali{pabbajjā} and so on are produced from roots such as \pali{vaja} and so on, together with prefixes and suffixes.}
\example[0]{paṭhamameva vajitabbāti pabbajjā (pa + \paliroot{vaja} + ṇya) \\{\upshape= [It is] worthy to be gone to first, hence \pali{pabbajjā} (going forth).}}
\example[0]{iñjanaṃ ejjā (\paliroot{iñja} + ṇya) \\{\upshape= [It is] a movement, hence \pali{ejjā}.}}
\example[0]{samajjanaṃ samajjā (saṃ + \paliroot{aja} + ṇya) \\{\upshape= [It is] a gathering, hence \pali{samajjā}.}}
\example[0]{nisīdanaṃ nisajjā (ni + \paliroot{sada} + ṇya) \\{\upshape= [It is] a sitting, hence \pali{nisajjā}.}}
\example[0]{vijānanaṃ vijjā (\paliroot{vida} + ṇya) \\{\upshape= [It is] a knowing, hence \pali{vijjā}.}}
\example[0]{visajjanaṃ visajjā (vi + \paliroot{saja} + ṇya) \\{\upshape= [It is] a releasing, hence \pali{visajjā}.}}
\example[0]{padanaṃ pajjā (\paliroot{pada} + ṇya) \\{\upshape= [It is] a going, hence \pali{pajjā}.}}
\example[0]{hananaṃ vajjhā (\paliroot{hana} + ṇya) \\{\upshape= [It is] a killing, hence \pali{vajjhā}.}}
\example[0]{esanaṃ icchā (\paliroot{isu} + ṇya) \\{\upshape= [It is] a seeking, hence \pali{icchā}.}}
\example[0]{atiesanaṃ aticchā (ati + \paliroot{isu} + ṇya) \\{\upshape= [It is] a great seeking, hence \pali{aticchā}.}}
\example[0]{sadanaṃ sajjā (\paliroot{sada} + ṇya) \\{\upshape= [It is] an enjoying, hence \pali{sajjā}.}}
\example[0]{sayanti etthāti seyyā (\paliroot{si} + ṇya) \\{\upshape= [They] lie down on this, hence \pali{seyyā} (bed).}}
\example[0]{sammā cittaṃ nidheti etāyāti saddhā (saṃ + \paliroot{dhā} + a) \\{\upshape= [One] places the mind rightly by that, hence \pali{saddhā} (faith).}}
\example[0]{caritabbā cariyā (\paliroot{cara} + ṇya) \\{\upshape= [It is] worthy to be practised, hence \pali{cariyā} (conduct).}}
\example[0]{karaṇaṃ kiriyā (\paliroot{kara} + ṇya) \\{\upshape= [It is] a doing, hence \pali{kiriyā}.}}
\example[0]{rujanaṃ rucchā (\paliroot{ruja} + cha) \\{\upshape= [It is] an afflicting, hence \pali{rucchā}.}}
\example[0]{padanaṃ pacchā (\paliroot{pada} + cha) \\{\upshape= [It is] a going [behind], hence \pali{pacchā}.}}
\example[0]{riñcanaṃ ricchā (\paliroot{rica} + cha) \\{\upshape= [It is] a neglecting, hence \pali{ricchā}.}}
\example[0]{tikicchanaṃ titicchā (\paliroot{kita} + cha) \\{\upshape= [It is] a healing, hence \pali{titicchā}.}}
\example[0]{saṃkocanaṃ saṃkucchā (saṃ + \paliroot{kuca} + cha) \\{\upshape= [It is] a contracting, hence \pali{saṃkucchā}.}}
\example[0]{madanaṃ macchā (\paliroot{mada} + cha) \\{\upshape= [It is] an intoxicating, hence \pali{macchā}.}}
\example[0]{labhanaṃ lacchā (\paliroot{labha} + cha) \\{\upshape= [It is] an obtaining, hence \pali{lacchā}.}}
\example[0]{raditabbāti, radanaṃ vilekhanaṃ vā racchā (\paliroot{rada} + cha) \\{\upshape= [It is] worthy to be scratched, or [it is] a scratching, hence \pali{racchā} (road).}}
\example[0]{adhobhāgena gacchatīti tiracchā/tiracchāno (\paliroot{tira} + cha) \\{\upshape= [It] goes by the lower part, hence \pali{tiracchā} (animal).}}
\example[0]{ajanaṃ acchā (\paliroot{aja} + cha) \\{\upshape= [It is] a going [of dirt], hence \pali{acchā} (clean).}}
\example[0]{titikkhatīti titikkhā (\paliroot{tija} + kha) \\{\upshape= [One] endures, hence \pali{titikkhā} (patience).}}
\example[0]{saha āgamanaṃ sāgacchā (saha + ā + \paliroot{gamu} + cha) \\{\upshape= [It is] a coming together, hence \pali{sāgacchā} (assembly).}}
\example[0]{duṭṭhu bhakkhanaṃ dobhacchā (du + \paliroot{bhakkha} + cha) \\{\upshape= [It is] a bad eating, hence \pali{dobhacchā}.}}
\example[0]{duṭṭhu rosanaṃ durucchā (du + \paliroot{rusa} + cha) \\{\upshape= [It is] a bad furiousness, hence \pali{durucchā}.}}
\example[0]{pucchanaṃ pucchā (\paliroot{puccha} + a) \\{\upshape= [It is] as asking, hence \pali{pucchā} (question).}}
\example[0]{muhanaṃ mucchā (\paliroot{muha} + cha) \\{\upshape= [It is] a deluding, hence \pali{mucchā} (confusion).}}
\example[0]{vasanaṃ vacchā (\paliroot{vasa} + cha) \\{\upshape= [It is] a living, hence \pali{vacchā}.}}
\example[0]{kacanaṃ kacchā (\paliroot{kaca} + cha) \\{\upshape= [It is] a binding, hence \pali{kacchā} (belt).}}
\example[0]{saha kathanaṃ sākacchā (saha + \paliroot{katha} + cha) \\{\upshape= [It is] a talking together, hence \pali{sākacchā} (conversation).}}
\example[0]{tudanaṃ tucchā (\paliroot{tuda} + cha) \\{\upshape= [It is] a piercing, hence \pali{tucchā}.}}
\example[0]{visanaṃ vicchā (\paliroot{visa} + cha) \\{\upshape= [It is] an entering, hence \pali{vicchā}.}}
\example[0]{pisanaṃ picchillā (\paliroot{pisa} + ṇya?) \\{\upshape= [It is] a grinding, hence \pali{picchillā}.}}
\example[0]{sukhadukkhaṃ mudati bhakkhatīti maccho (\paliroot{muda} + cha) \\{\upshape= [It] eats happiness and suffering, hence \pali{maccha} (fish).}}
\example[0]{sattānaṃ pāṇaṃ museti cajetīti maccu (\paliroot{musa} + ṇya?) \\{\upshape= [It] steals [or] abandons the life of beings, hence \pali{maccu} (death).}}
\example[0]{satanaṃ saccaṃ (\paliroot{sata} + ṇya) \\{\upshape= [It is] an existing, hence \pali{sacca} (truth).}}
\example[0]{uddhaṃ dhunāti kampatīti uddhaccaṃ (u + \paliroot{dhu} + ṇya) \\{\upshape= [It] shakes [or] trembles upward, hence \pali{uddhacca} (restlessness).}}
\example[0]{naṭanaṃ naccaṃ (\paliroot{naṭa} + ṇya) \\{\upshape= [It is] a dancing, hence \pali{nacca}.}}
\example[0]{nitanaṃ niccaṃ (\paliroot{niti} + ṇya) \\{\upshape= [It is] permanent, hence \pali{nicca}.}}
\example{tathanaṃ tacchaṃ (\paliroot{tatha} + cha) \\{\upshape= [It is] true, hence \pali{taccha}.}}
\transnote{Many of the analyses look like just a theoretical construction. I suggest a look up in Saddanīti Dhātumālā for suspicious roots first. For the change from \pali{hana} to \pali{vadha}, see \hyperref[sut:592]{Kacc 592}.}

\head{639}{639, 585. kvilopo ca.}
\headtrans{[There is] \pali{kvi}-elision also.}
\sutdef{kvilopo hoti, puna ca nippajjante.}
\sutdeftrans{There is an elision of \pali{kvi}-paccaya, then [the result] is produced further.}
\example[0]{vividhehi sīlādiguṇehi bhavatīti, visesena vā bhavatīti vibhū (vi + \paliroot{bhū} + kvi) \\{\upshape= [One] exists by various virtues such as morality and so on, or exists by a superior state, hence \pali{vibhū} (supreme being).}}
\example[0]{sayaṃ attanā bhavatīti sayambhū (sayaṃ + \paliroot{bhū} + kvi) \\{\upshape= [One] exists by oneself, hence \pali{sayambhū} (the creator god).}}
\example[0]{abhibhavitvā bhavatīti abhibhū (abhi + \paliroot{bhū} + kvi) \\{\upshape= Having conquered, [one] exists, hence \pali{abhibhū} (conqueror).}}
\example[0]{saṃ suṭṭhu dhunāti kampatīti sandhū (saṃ + \paliroot{dhu} + kvi) \\{\upshape= [It] shakes [or] trembles well together, hence \pali{sandhū}.}}
\example[0]{visesena bhāti dibbatīti vibhā (vi + \paliroot{bhā} + kvi) \\{\upshape= [It] shines [or] radiates distinctively, hence \pali{vibhā}.}}
\example[0]{nissesena bhāti dibbatīti nibhā (ni + \paliroot{bhā} + kvi) \\{\upshape= [It] shines [or] radiates entirely, hence \pali{nibhā}.}}
\example[0]{pakārena bhāti dibbatīti pabhā (pa + \paliroot{bhā} + kvi) \\{\upshape= [It] shines [or] radiates by its nature, hence \pali{pabhā}.}}
\example[0]{saha bhāsanti etthāti sabhā (saha + \paliroot{bhāsa} + kvi) \\{\upshape= [They] talk together here, hence \pali{sabhā} (assembly).}}
\example[0]{ābhuso bhāti dibbatīti ābhā (ā + \paliroot{bhā} + kvi) \\{\upshape= [It] shines [or] radiates intensely, hence \pali{ābhā}.}}
\example[0]{bhujena kuṭilena gacchatīti bhujago (bhuja + \paliroot{gamu} + kvi) \\{\upshape= [It] goes by coil [or] bending, hence \pali{bhujaga} (snake).}}
\example[0]{turitaturito gacchatīti turago (tura + \paliroot{gamu} + kvi) \\{\upshape= [It] goes quickly, hence \pali{turaga} (horse).}}
\example[0]{saṃ suṭṭhu pathaviṃ khanatīti saṅkho \\(saṃ + \paliroot{khana} + kvi) \\{\upshape= [It] digs the ground well [or] easily, hence \pali{saṅkha} (conch).}}
\example[0]{visesena yamati uparamatīti viyo (vi + \paliroot{yamu} + kvi) \\{\upshape= [One] restrains [or] abstrains distinctively, hence \pali{viya}.}}
\example[0]{suṭṭhu manati jānātīti sumo (su + \paliroot{mana} + kvi) \\{\upshape= [One] knows [or] understands well, hence \pali{suma}.}}
\example{pari samantato tanoti vitthāretīti parito (pari + \paliroot{tanu} + kvi) \\{\upshape= [It] stretches [or] spreads all around, hence \pali{parita}.}}
\transnote{I cannot find a few words in dictionaries, even in Sanskrit. So, they may be just a construction to make the point, not for general uses.}

\head{640}{640. sacajānaṃ kagā ṇānubandhe.}
\headtrans{For [roots ending] with \pali{ca} and \pali{ja}, [there are] \pali{ka}- and \pali{ga}-substitution because of \pali{ṇa}-anubandha.}
\sutdef{sacajānaṃ dhātūnamantānaṃ cajānaṃ kagādesā honti yathā\-saṅkhyaṃ ṇānubandhe paccaye pare.}
\sutdeftrans{There are substitutions of \pali{ka} and \pali{ga} for \pali{ca} and \pali{ja} of roots ending with \pali{ca} or \pali{ja} respectively because of a paccaya having \pali{ṇa}-anubandha behind.}
\example[0]{oko (\paliroot{uca} + ṇa)}
\example[0]{pāko (\paliroot{paca} + ṇa)}
\example[0]{seko (\paliroot{sica} + ṇa)}
\example[0]{soko (\paliroot{suca} + ṇa)}
\example[0]{viveko (vi + \paliroot{vica} + ṇa)}
\example[0]{cāgo (\paliroot{caja} + ṇa)}
\example[0]{yogo (\paliroot{yuja} + ṇa)}
\example[0]{bhogo (\paliroot{bhuja} + ṇa)}
\example[0]{rogo (\paliroot{ruja} + ṇa)}
\example[0]{rāgo (\paliroot{rañja} + ṇa)}
\example[0]{bhāgo (\paliroot{bhaja} + ṇa)}
\example[0]{bhaṅgo (\paliroot{bhañja} + ṇa)}
\example[0]{raṅgo (\paliroot{rañja} + ṇa)}
\example{saṅgo (\paliroot{sañja} + ṇa)}

\head{641}{641, 572. nudādīhi yuṇvūnamanānanākānanakā sakāritehi ca.}
\headtrans{After \pali{nuda} and so on, [there are] \pali{ana}-, \pali{ānana}-, \pali{aka}-, and \pali{ānanaka}-substitution for \pali{yu}- and \pali{ṇvu}-paccaya, after [roots] together with the \pali{kārita}-paccayas also.}
\sutdef{nuda\,sūda\,jana\,su\,lū\,hu\,pu\,bhū\,ñā\,asa\,samu\,iccevamādīhi\ \ dhā\-tūhi,\ \ phanda\,citi\,āṇa\,iccevamādīhi\ \ sakāritehi\ \ ca\ \ yuṇvūnaṃ\ \ paccayānaṃ\ \ ana\,ānana\,aka\,ānanak\,ādesā\ \ honti\ \ yathāsaṅkh\-yaṃ\ \ kattari, bhāvakaraṇesu ca.}
\sutdeftrans{There are substitutions of \pali{ana}, \pali{ānana}, \pali{aka}, and \pali{ānanaka} for \pali{yu}- and \pali{ṇvu}-paccaya respectively after roots such as \pali{nuda}, \pali{sūda}, \pali{jana}, \pali{su}, \pali{lū}, \pali{hu}, \pali{pu}, \pali{bhū}, \pali{ñā}, \pali{asa}, \pali{samu} and so on, also after roots such as \pali{phanda}, \pali{citi}, \pali{āṇa} and so on together with the \pali{kārita}-paccayas (causative maker) in [the meaning] of agent, state, and cause.}

\bigbullet{(1) In the meaning of agent (\pali{yu})}
\example[0]{panudatīti panūdano (pa + \paliroot{nuda} + yu) \\{\upshape= [One] dispels, hence \pali{panūdana} (dispeller)}}
\example[0]{sūdano (\paliroot{sūda} + yu) {\upshape= a cook/cleaner}}
\example[0]{janano (\paliroot{jana} + yu) {\upshape= a producer/father}}
\example[0]{savaṇo (\paliroot{su} + yu) {\upshape= a hearer}}
\example[0]{lavano (\paliroot{lū} + yu) {\upshape= a cutter}}
\example[0]{havano (\paliroot{hu} + yu) {\upshape= a sacrificer}}
\example[0]{pavano (\paliroot{pu} + yu) {\upshape= a cleaner}}
\example[0]{bhavano (\paliroot{bhū} + yu) {\upshape= a being}}
\example[0]{ñāṇo (\paliroot{ñā} + yu) {\upshape= a knower}}
\example[0]{asano (\paliroot{asa} + yu) {\upshape= an eater}}
\example{samaṇo (\paliroot{samu} + yu) {\upshape= an ascetic}}

\bigbullet{(2) In the meaning of state}
\example[0]{panudate panūdanaṃ (pa + \paliroot{nuda} + yu) \\{\upshape= Dispelling is done, hence \pali{panūdana} (dispelling)}}
\example[0]{sūdanaṃ (\paliroot{sūda} + yu) {\upshape= cooking/cleaning}}
\example[0]{jananaṃ (\paliroot{jana} + yu) {\upshape= arising}}
\example[0]{savaṇaṃ (\paliroot{su} + yu) {\upshape= hearing}}
\example[0]{lavanaṃ (\paliroot{lū} + yu) {\upshape= cutting}}
\example[0]{havanaṃ (\paliroot{hu} + yu) {\upshape= sacrificing}}
\example[0]{pavanaṃ (\paliroot{pu} + yu) {\upshape= cleansing}}
\example[0]{bhavanaṃ (\paliroot{bhū} + yu) {\upshape= being}}
\example[0]{ñāṇaṃ (\paliroot{ñā} + yu) {\upshape= knowing}}
\example[0]{asanaṃ (\paliroot{asa} + yu) {\upshape= eating}}
\example[0]{samaṇaṃ (\paliroot{samu} + yu) {\upshape= tranquillizing}}
\example[0]{sañjānanaṃ (saṃ + \paliroot{ñā} + yu) {\upshape= recognition}}
\example{kuyate kānanaṃ (\paliroot{ku} + yu) \\{\upshape= Sounding is done, hence \pali{kānana}.}}
\transnote{The last instance is dubious. It is constructed from \paliroot{ku} (\pali{sadde}). But as far as I can find, \pali{kānana} means `forest' (see MWD).}

\bigbullet{(3) With causative paccayas (\pali{yu})}
\example[0]{phandāpīyate phandāpanaṃ (\paliroot{phanda} + ṇāpe + yu) \\{\upshape= [It] makes [someone] tremble, hence \pali{phandāpana} (tremble-maker).}}
\example[0]{cetāpīyate cetāpanaṃ (\paliroot{citta} + ṇāpe + yu) \\{\upshape= [It] makes [something] variegated, hence \pali{cetāpana} (variety-maker).}}
\example{āṇāpīyate āṇāpanaṃ (\paliroot{āṇa} + ṇāpe + yu) \\{\upshape= [It] makes [someone] give an order, hence \pali{āṇāpana} (commanding).}}

\bigbullet{(4) In the meaning of cause}
\example[0]{nudanti anenāti nūdanaṃ (\paliroot{nuda} + yu) \\{\upshape= [They] dispel by this, hence \pali{nūdana} (dispelling tool).}}
\example[0]{sūdanaṃ (\paliroot{sūda} + yu) {\upshape= a cooking device}}
\example[0]{jananaṃ (\paliroot{jana} + yu) {\upshape= a way of arising}}
\example[0]{savaṇaṃ (\paliroot{su} + yu) {\upshape= a hearing device}}
\example[0]{lavanaṃ (\paliroot{lū} + yu) {\upshape= a cutting tool}}
\example[0]{havanaṃ (\paliroot{hu} + yu) {\upshape= a sacrificing tool}}
\example[0]{pavanaṃ (\paliroot{pu} + yu) {\upshape= a cleaning tool}}
\example[0]{bhavanaṃ (\paliroot{bhū} + yu) {\upshape= a way of being}}
\example[0]{ñāṇaṃ (\paliroot{ñā} + yu) {\upshape= a way of knowing}}
\example[0]{asanaṃ (\paliroot{asa} + yu) {\upshape= an eating tool}}
\example{samaṇaṃ (\paliroot{samu} + yu) {\upshape= a tranquillizer}}
\transnote{Since the examples are constructed theoretically, my translations go likewise.}

\bigbullet{(5) In the meaning of agent (\pali{ṇvu})}
\example[0]{nudatīti nūdako (\paliroot{nuda} + ṇvu) \\{\upshape= [One] dispels, hence \pali{nūdaka} (dispeller)}}
\example[0]{sūdatīti sūdakā (\paliroot{sūda} + ṇvu) \\{\upshape= [One] sprinkles, hence \pali{sūdaka} (cook)}}
\example[0]{janetīti janako (\paliroot{jana} + ṇvu) \\{\upshape= [One] gives birth, hence \pali{janaka} (father)}}
\example[0]{suṇotīti sāvako (\paliroot{su} + ṇvu) \\{\upshape= [One] hears, hence \pali{sāvaka} (disciple)}}
\example[0]{lunātīti lāvako (\paliroot{lū} + ṇvu) \\{\upshape= [One] cuts, hence \pali{lāvaka} (cutter)}}
\example[0]{juhotīti hāvako (\paliroot{hu} + ṇvu) \\{\upshape= [One] sacrifices, hence \pali{hāvaka} (sacrificer)}}
\example[0]{punātīti pāvako (\paliroot{pu} + ṇvu) \\{\upshape= [It] purifies, hence \pali{pāvaka} (fire)}}
\example[0]{bhavatīti bhāvako (\paliroot{bhū} + ṇvu) \\{\upshape= [One] becomes, hence \pali{bhāvaka} (becomer)}}
\example[0]{jānātīti jānako (\paliroot{ñā} + ṇvu) \\{\upshape= [One] knows, hence \pali{jānaka} (knower)}}
\example[0]{asatīti asako (\paliroot{asa} + ṇvu) \\{\upshape= [One] eats, hence \pali{asaka} (eater)}}
\example[0]{upāsatīti upāsako (upa + \paliroot{asa} + ṇvu) \\{\upshape= [One] attends, hence \pali{upāsaka} (devotee)}}
\example{sametīti samako (\paliroot{samu} + ṇvu) \\{\upshape= [It] makes even, hence \pali{samaka} (equal)}}

\bigbullet{(6) With causative paccayas (\pali{ṇvu})}
\example[0]{phandāpayatīti phandāpayako (\paliroot{phanda} + ṇāpaya + ṇvu) \\{\upshape= [One] makes [someone] tremble, hence \pali{phandāpayaka} (tremble-maker).}}
\example[0]{āṇāpayako (\paliroot{āṇa} + ṇāpaya + ṇvu) \\{\upshape= [One] makes [someone] give an order, hence \pali{āṇāpayaka} (comma\-nder-maker).}}
\example[0]{cetāpayako (\paliroot{citta} + ṇāpaya + ṇvu) \\{\upshape= [One] makes [something] variegated, hence \pali{cetāpayaka} (variety-maker).}}
\example{sañjānanako (saṃ + \paliroot{ñā} + ṇāpe + ṇvu) \\{\upshape= [One] makes [someone] recognize, hence \pali{sañjānanaka} (recognit\-ion-maker).}}

\head{642}{642, 588. iyatamakiesānamantassaro dīghaṃ kvaci disassa guṇaṃ do raṃ sakkhī ca.}
\headtrans{The ending vowel of \pali{i}, \pali{ya}, \pali{ta}, \pali{ma}, \pali{ki}, \pali{e}, and \pali{sa} [becomes] elongated. In some places, \pali{disa} [becomes] guṇa-strength, \pali{da} [becomes] \pali{ra}. [There are] \pali{sa}-, \pali{kkha}-, and \pali{ī}-substitution also.}
\sutdef{i\,ya\,ta\,ma\,ki\,e\,sa\,iccetesaṃ\ \ sabbanāmānamanto\ \ saro\ \ dīgha\-māpajjate, kvaci disaiccetassa dhātussa ikāro guṇamāpajjate, dakāro rakāramāpajjate, dhātvantassa sassa ca sakkhaīiccete ādesā honti yathāsambhavaṃ. ete saddā sakena sakena nāmena yathānuparodhena buddhasāsane pacchā puna nippajjante.}
\sutdeftrans{The ending vowel of pronouns such as \pali{i}, \pali{ya}, \pali{ta}, \pali{ma}, \pali{ki}, \pali{e}, and \pali{sa} becomes elongated. The letter \pali{i} of the root \pali{disa} becomes guṇa-strength (\pali{ī}) in some places. The letter \pali{da} [of the root] becomes \pali{ra} [in some places]. And there are substitutions of \pali{sa}, \pali{kkha}, and \pali{ī} relevantly for the ending \pali{sa} of the root. These terms are produced further by their own name each, by such a way not going against the teaching of the Buddha.}
\example[0]{imamiva naṃ passatīti īdiso/īriso/īdikkho/īdī \\(ima + \paliroot{disa} + kvi) \\{\upshape= [One] sees that as this, hence \pali{īdisa/īrisa/īdikkha/īdī} (like this).}}
\example[0]{yamiva naṃ passatīti yādiso/yāriso/yādikkho/yādī \\(ya + \paliroot{disa} + kvi) \\{\upshape= [One] sees that as whichever, hence \pali{yādisa/yārisa/yādikkha/ yādī} (like whichever).}}
\example[0]{tamiva naṃ passatīti tādiso/tāriso/tādikkho/tādī \\(ta + \paliroot{disa} + kvi) \\{\upshape= [One] sees that as that one, hence \pali{tādisa/tārisa/tādikkha/tādī} (like that).}}
\example[0]{mamiva naṃ passatīti mādiso/māriso/mādikkho/mādī \\(maṃ + \paliroot{disa} + kvi) \\{\upshape= [One] sees that as me, hence \pali{mādisa/mārisa/mādikkha/mādī} (like me).}}
\example[0]{kimiva naṃ passatīti kīdiso/kīriso/kīdikkho/kīdī \\(kiṃ + \paliroot{disa} + kvi) \\{\upshape= Like what does [one] see that? [= What is that like when one see it?], hence \pali{kīdisa/kīrisa/kīdikkha/kīdī} (like what?).}}
\example[0]{etamiva naṃ passatīti ediso/eriso/edikkho/edī \\(eta + \paliroot{disa} + kvi) \\{\upshape= [One] sees that as that one, hence \pali{edisa/erisa/edikkha/edī} (like that).}}
\example{samānamiva naṃ passatīti sādiso/sāriso/sādikkho/sādī \\(samāna + \paliroot{disa} + kvi) \\{\upshape= [One] sees that as the same, hence \pali{sādisa/sārisa/sādikkha/ sādī} (like the same).}}

\head{643}{643, 635. bhyādīhi matibudhipūjādīhi ca kto.}
\headtrans{After \pali{bhī} and so on, also \pali{mati}, \pali{budhi}, \pali{pūja} and so on, [there is] \pali{kta}-paccaya.}
\sutdef{bhīiccevamādīhi dhātūhi matibudhipūjādito ca ktapaccayo hoti.}
\sutdeftrans{There is \pali{kta}-paccaya after roots such as \pali{bhī} and so on, also \pali{mati}, \pali{budhi}, \pali{pūja} and so on.}
\example[0]{bhāyitabboti bhīto (\paliroot{bhī} + kta) \\{\upshape= [It is] worthy to be feared, hence \pali{bhīta} (fearful [thing]).}}
\example[0]{supitabboti sutto (\paliroot{supa} + kta) \\{\upshape= Sleeping should be done, hence \pali{sutta} (sleeping).}}
\example[0]{mijjitabbo sinehetabboti mitto (\paliroot{mida} + kta) \\{\upshape= [One is] worthy to be loved, hence \pali{mitta} (friend).}}
\example[0]{sammannitabboti, saṃ suṭṭhu mānitabbo pūjetabboti, sammānīyitthāti sammato (saṃ + \paliroot{mati} + kta) \\{\upshape= [One is] worthy to be recognized, [or one is] worthy to be respected well, [or one] was respected well, hence \pali{sammato} (respectable one).}}
\example[0]{saṃkappīyateti, saṃkappīyitthāti saṅkappito \\(saṃ + \paliroot{kappa} + kta) \\{\upshape= [It] is imagined, [or] was imagined, hence \pali{saṅkappita} (the imagined or thought upon).}}
\example[0]{sampādīyateti, sampādīyitthāti sampādito \\(saṃ + \paliroot{pada} + kta) \\{\upshape= [It] is provided, [or] was provided hence \pali{sampādita} (the provided).}}
\example[0]{avadhārīyateti, avadhārīyitthāti avadhārito \\(ava + \paliroot{dhara} + kta) \\{\upshape= [It] is selected, [or] was selected, hence \pali{avadhārita} (the selected).}}
\example[0]{bujjhitabbo ñātabboti buddho (\paliroot{budhi} + kta) \\{\upshape= [It is] worthy to be known, hence \pali{buddha} (knowledge).}}
\example[0]{ajjhayitabboti, etabbo gantabboti ito (\paliroot{i} + kta) \\{\upshape= [It is] worthy to be studied, [or] worthy to be gone to, hence \pali{ita}.}}
\example[0]{viditabbo ñātabboti vidito (\paliroot{vida} + kta) \\{\upshape= [It is] worthy to be known, hence \pali{vidita} (knowledge).}}
\example[0]{takkīyateti takkito (\paliroot{takka} + kta) \\{\upshape= [It] is pondered, hence \pali{takkita} (speculation).}}
\example[0]{pūjīyateti, pūjīyitthāti pūjito (\paliroot{pūja} + kta) \\{\upshape= [One] is honored, [or] was honored, hence \pali{pūjita} (honorable one).}}
\example[0]{apacāyitabboti apacāyito (apa + \paliroot{cāya} + kta) \\{\upshape= [One is] worthy to be respected, hence \pali{apacāyita} (respectable one).}}
\example[0]{mānitabbo pūjetabboti mānito (\paliroot{māna} + kta) \\{\upshape= [One is] worthy to be respected, hence \pali{mānita} (respectable one).}}
\example[0]{apacīyateti apacito (apa + \paliroot{ci} + kta) \\{\upshape= [One] is honored, hence \pali{apacita} (honorable one).}}
\example[0]{vandīyateti, vandīyitthāti vandito (\paliroot{vanda} + kta) \\{\upshape= [One] is bowed down, [or] was bowed down, hence \pali{vandita} (respectable one).}}
\example{sakkarīyateti, sakkarīyitthāti sakkārito (saṃ + \paliroot{kara} + kta) \\{\upshape= [One] is honored, hence \pali{sakkārita} (honorable one).}}
\transnote{The operation of \pali{kta} is mostly like \pali{ta}, but the products are nominal. As shown by the examples, most of them have passive meaning.}

\head{644}{644,~661.~vepu\,sī\,dava\,vamu\,ku\,dā\,bhū\,hvādīhi\ \ thu\,ttima\,ṇimā\ \ nibbatte.}
\headtrans{After \pali{vepu}, \pali{sī}, \pali{dava}, \pali{vamu}, \pali{ku}, \pali{dā}, \pali{bhū}, \pali{hu} and so on, [there are] \pali{thu}-, \pali{ttima}-, and \pali{ṇima}-paccaya in [the sense of] arising.}
\sutdef{vepu\,sī\,dava\,vamu\,ku\,dā\,bhū\,hu\,iccevamādīhi\ \ dhātūhi\ \ yathā\-sambhavaṃ\ \ thu\,ttima\,ṇima\,paccayā\ \ honti\ \ nibbattatthe.}
\sutdeftrans{There are \pali{thu}-, \pali{ttima}-, and \pali{ṇima}-paccaya relevantly after roots such as \pali{vepu}, \pali{sī}, \pali{dava}, \pali{vamu}, \pali{ku}, \pali{dā}, \pali{bhū}, \pali{hu} and so on in the sense of arising [of diseases, etc.].}
\example[0]{vepanaṃ vepo, tena nibbatto vepathu (\paliroot{vepu} + thu) \\{\upshape= [Which] \pali{vepa} [is] trembling, [it] was arisen from that, hence \pali{vepathu} (disease caused by trembling).}}
\example[0]{sayanaṃ sayo, tena nibbatto sayathu (\paliroot{sī} + thu) \\{\upshape= [Which] \pali{saya} [is] sleeping, [it] was arisen from that, hence \pali{sayathu} (disease caused by sleeping).}}
\example[0]{davanaṃ davo, tena nibbatto davathu (\paliroot{dava} + thu) \\{\upshape= [Which] \pali{dava} [is] playing, [it] was arisen from that, hence \pali{davathu} (disease caused by playing [sports]).}}
\example[0]{vamanaṃ vamo, tena nibbatto vamathu (\paliroot{vamu} + thu) \\{\upshape= [Which] \pali{vama} [is] vomiting, [it] was arisen from that, hence \pali{vamathu} (disease caused by vomiting).}}
\example[0]{kutti karaṇaṃ, tena nibbattaṃ kuttimaṃ (\paliroot{ku} + ttima) \\{\upshape= [Which] \pali{ku} [is] making, [it] was arisen from that, hence \pali{kuttima} (something arisen by making).}}
\example[0]{dāti dānaṃ, tena nibbattaṃ dattimaṃ. (\paliroot{dā} + ttima) \\{\upshape= [Which] \pali{dā} [is] giving, [it] was arisen from that, hence \pali{dattima} (something arisen by giving).}}
\example[0]{bhūti bhavanaṃ, tena nibbattaṃ bhottimaṃ (\paliroot{bhū} + ttima) \\{\upshape= [Which] \pali{bhū} [is] becoming, [it] was arisen from that, hence \pali{bhottima} (something arisen by becoming).}}
\example{avahuti avahanaṃ, tena nibbattaṃ ohāvimaṃ \\(ava + \paliroot{hu} + ṇima) \\{\upshape= [Which] \pali{avahu} [is] sacrificing, [it] was arisen from that, hence \pali{ohāvima} (something arisen from sacrificing).}}
\transnote{The uses of these words are hardly found. They are constructed just by speculation, I suppose. The root \paliroot{vepu} (\pali{kampane}) is found in Dhātumañjūsā. I have a difficulty to find the meaning of \paliroot{dava}. In Dhātumañjūsā, it has a circular definition as \pali{davane}, also \pali{saraṇe} and \pali{chedane} which look irrelevant. The meaning I use came from a Thai source. As suggested by Dhātvatthasaṅgaha 184 (\pali{kīḷātāpe}), it possibly means the heat (\pali{ātāpa}) generated by the playing.}

\head{645}{645, 662. akkose namhāni.}
\headtrans{In [the sense of] insulting with \pali{na} [applied], [there is] \pali{āni}-paccaya.}
\sutdef{akkosaiccetasmiṃ atthe namhi paṭisedhayutte ānipaccayo hoti dhātūhi.}
\sutdeftrans{There is \pali{āni}-paccaya after roots in the sense of insulting with the nagative \pali{na}-particle applied.}
\example[0]{na gamitabbaṃ agamāni (na + \paliroot{gamu} + āni) te jamma desaṃ \\{\upshape= The region not worthy to be gone by you [is] \pali{agamāni}, despicable you!}}
\example{na kattabbaṃ akarāṇi (na + \paliroot{kara} + āni) te jamma kammaṃ \\{\upshape= The action not worthy to be done by you [is] \pali{akarāṇi}, despicable you!}}
\transnote{As vocative case, I translate \pali{jamma} softly as `despicable you!' I expect \pali{desa} to be masculine, but the example uses it as neuter. It is worth noting that \pali{āni} becomes \pali{āṇi} because of \pali{ra}. We can see a similar case occasionally. This can be explained by Sanskrit internal sandhi rules, but it is better not to involve that here.}

\head{646}{646, 419. ekādito sakissa kkhattuṃ.}
\headtrans{For \pali{sakiṃ} after \pali{eka} and so on, [there is] \pali{kkhattuṃ}[-substitu\-tion].}
\sutdef{ekādito sakissa kkhattuṃ hoti.}
\sutdeftrans{There is [a substitution of] \pali{kkhattuṃ} for \pali{sakiṃ} (once) after [a number such as] \pali{eka} and so on.}
\example[0]{ekassa padatthassa sakiṃ vāraṃ ekakkhattuṃ \\{\upshape= One instance of the meaning of a term [is] \pali{ekakkhattuṃ} (once).}}
\example[0]{dvinnaṃ padatthānaṃ sakiṃ vāraṃ dvikkhattuṃ \\{\upshape= One instance of two [times of the] meaning of a term [is] \pali{dvikkhattuṃ} (twice).}}
\example[0]{tiṇṇaṃ padatthānaṃ sakiṃ vāraṃ tikkhattuṃ \\{\upshape= One instance of three [times of the] meaning of a term [is] \pali{tikkhattuṃ} (three times).}}
\example{catukkhattuṃ, pañcakkhattuṃ, chakkhattuṃ, sattakkhattuṃ, aṭṭhakkhattuṃ, navakkhattuṃ, dasakkhattuṃ}
\transnote{The analysis in the examples is a little confusing, but the meaning of the products is straightforward.}

\head{647}{647,~663.~sunassunass\,oṇa\,vān\,uvān\,ūn\,unakh\,un\,ā\,nā.}
\headtrans{For \pali{una} of \pali{suna}, [there are substitutions of] \pali{oṇa}, \pali{vāna}, \pali{uvāna}, \pali{ūna}, \pali{unakha}, \pali{una}, \pali{ā}, and \pali{āna}.}
\sutdef{sunaiccetassa\ \ pāṭipadikassa\ \ unassa\ \ oṇa\,vāna\,uvāna\,ūna\,un\-akha\,una\,ā\,ānādesā\ \ honti.}
\sutdeftrans{There are substitutions of \pali{oṇa}, \pali{vāna}, \pali{uvāna}, \pali{ūna}, \pali{unakha}, \pali{una}, \pali{ā}, and \pali{āna} for \pali{una} of the nominal base of \pali{suna}.}
\example{sāmikassa saddaṃ suṇātīti soṇo/svāno/suvāno/sūno/suna\-kho/suno/sā/sāno \\{\upshape= [The animal] listens the sound of the owner, hence \pali{soṇa/svāna/ suvāna/sūna/sunakha/suna/sā/sāna} (dog).}}

\head{648}{648, 664. taruṇassa susu ca.}
\headtrans{For \pali{taruṇa}, [there is] \pali{susu}-substitution also.}
\sutdef{taruṇaiccetassa pāṭipadikassa susuādeso hoti.}
\sutdeftrans{There is a substitution of \pali{susu} for the nominal base of \pali{taruṇa}.}
\example{susu (taruṇa + si) kāḷakeso {\upshape (a youth having black hair)}}
\transnote{In a Thai source, this is \pali{susū} instead.}

\head{649}{649, 665. yuvassuvassuvuvānunūnā.}
\headtrans{For \pali{uva} of \pali{yuva}, [there are substitutions of] \pali{uva}, \pali{uvāna}, \pali{una}, and \pali{unā}.}
\sutdef{yuvaiccetassa\ \ pāṭipadikassa\ \ uvassa\ \ uva\,uvāna\,una\,ūnā\,desā\ \ honti.}
\sutdeftrans{There are substitutions of \pali{uva}, \pali{uvāna}, \pali{una}, and \pali{unā} for \pali{uva} of the nominal base of \pali{yuva}.}
\example{yuvā, yuvāno, yuno, yūno {\upshape(a youth)}}

\head{650}{650, 651. kāle vattamānātīte ṇvādayo.}
\headtrans{In the time of present and past, [there are] \pali{ṇu}-paccaya and so on [\pali{yu} and \pali{ta}].}
\sutdef{kāle vattamānatthe ca atītatthe ca ṇuyutapaccayā honti.}
\sutdeftrans{There are \pali{ṇu}-, \pali{yu-, and \pali{ta}-paccaya} in the present and past meaning.}
\example[0]{akāsi, karotīti kāru (\paliroot{kara} + ṇu) \\{\upshape= [One] did [or] does, hence \pali{kāru} (artisan).}}
\example[0]{agacchi, gacchatīti vāyu (\paliroot{vā} + yu) \\{\upshape= [It] went [or] goes, hence \pali{vāyu} (wind).}}
\example{abhavi, bhavatīti bhūtaṃ (\paliroot{bhū} + ta) \\{\upshape= [It] became [or] becomes, hence \pali{bhūta} (being).}}
\transnote{In the first example, the term is synonymous to \pali{kāruka}. In a Thai source, the second example goes more sensible like this: \pali{vāyati avāyīti vāyu}. It might be better to see this \pali{ta}-paccaya as \pali{kta} instead because its products are nominal.}

\head{651}{651, 647. bhavissati gamādīhi ṇīghiṇ.}
\headtrans{In the future [meaning] after \pali{gamu} and so on, [there are] \pali{ṇī}- and \pali{ghiṇ}-paccaya.}
\sutdef{bhavissatikālatthe gamubhajasuṭhāiccevamādīhi dhātūhi ṇī\-ghiṇpaccayā honti.}
\sutdeftrans{There are \pali{ṇī}- and \pali{ghiṇ}-paccaya after roots such as \pali{gamu}, \pali{bhaja}, \pali{su}, \pali{ṭhā} and so on in the future meaning.}
\example[0]{āyatiṃ gamituṃ sīlaṃ yassa, so hotīti gāmī (\paliroot{gamu} + ṇī) \\{\upshape= Which [person has] a habit to go in the future, hence that [person] is \pali{gāmī}.}}
\example[0]{āyatiṃ bhajituṃ sīlaṃ yassa, sohotīti bhājī (\paliroot{bhaja} + ṇī) \\{\upshape= Which [person has] a habit to keep commpany [with others] in the future, hence that [person] is \pali{bhājī}.}}
\example[0]{āyatiṃ sotuṃ sīlaṃ yassa, so hotīti passāvī (pa + \paliroot{su} + ṇī) \\{\upshape= Which [person has] a habit to hear in the future, hence that [person] is \pali{passāvī}.}}
\example{āyatiṃ paṭṭhāyituṃ sīlaṃ yassa, so hotīti paṭṭhāyī \\(pa + \paliroot{ṭhā} + ṇī) \\{\upshape= Which [person has] a habit to establish [things] in the future, hence that [person] is \pali{paṭṭhāyī}.}}
\transnote{The third analysis came from a Thai source, which is more understandable than the Myanmar source. As suggested in Rūpa 647, both \pali{passāvi} and \pali{paṭṭhāyi} are also valid.}
\transnote{The use of \pali{ghiṇ} or \pali{ghiṇa} is inexplicable. This weird paccaya possibly came from Skt.\ \pali{ghinuṇ} (Pāṇ 3.2.141, see also Kāt 4.266). The products of this end with \pali{ī}, for example, \pali{śamī (\paliroot{śam} + ghinuṇ)} possibly meaning ``One having a habit of making oneself calm.'' This example came from Kāśikā. I suppose that when the author of this Kaccāyana sutta made an application of it, he changed it to \pali{ṇī} instead because the operation is similar, and left the trace of \pali{ghiṇ} to remind the origin.}

\head{652}{652, 648. kriyāyaṃ ṇvutavo.}
\headtrans{In [the sense of] action, [there are] \pali{ṇvu}- and \pali{tu}-paccaya.}
\sutdef{kriyāyamatthe ṇvutuiccete paccayā honti bhavissatikāle.}
\sutdeftrans{There are \pali{ṇvu}- and \pali{tu}-paccaya in the sense of action in the future.}
\example[0]{“karissan”ti kārako (\paliroot{kara} + ṇvu) vajati \\{\upshape= [One] goes [thinking that] ``I will do,'' hence \pali{kāraka} (one about to do).}}
\example{“bhuñjissan”ti bhottā (\paliroot{bhuja} + tu) vajati \\{\upshape= [One] goes [thinking that] ``I will eat,'' hence \pali{bhottu} (one about to eat).}}
\transnote{In the example, \pali{karissaṃ} and \pali{bhuñjissaṃ} are attanopada forms of \pali{karissāmi} and \pali{bhuñjissāmi} respectively. I suppose that \pali{vajati} here could be \pali{vadati}. In a Thai source, it is \pali{vajjati} instead.}

\head{653}{653, 306. bhāvavācimhi catutthī.}
\headtrans{In denoting a state, [there is] a dative [vibhatti].}
\sutdef{bhāvavācimhi catutthīvibhatti hoti bhavissatikāle.}
\sutdeftrans{There is a dative vibhatti in denoting a state in the future meaning.}
\example[0]{pacissate, pacanaṃ vā pāko, pākāya vajati \\{\upshape= [One] will cook, hence \pali{pacana} or \pali{pāka} (cooking). [One] goes for the cooking [= One is going to cook].}}
\example[0]{bhuñjissate, bhojanaṃ vā bhogo, bhogāya vajati \\{\upshape= [One] will eat, hence \pali{bhojana} or \pali{bhoga} (eating). [One] goes for the eating [= One is going to eat].}}
\example{naccissate, naccanaṃ vā naccaṃ, naccāya vajati \\{\upshape= [One] will dance, hence \pali{naccana} or \pali{nacca} (dancing). [One] goes for the dancing [= One is going to dance].}}
\transnote{My translation of the examples may be different from other sources, but I find this more understandable, despite verbosity.}

\head{654}{654, 649. kammani ṇo.}
\headtrans{Because of [a term having] object [nearby, there is] \pali{ṇa}-paccaya.}
\sutdef{kammani upapade ṇapaccayo hoti bhavissatikāle.}
\sutdeftrans{There is \pali{ṇa}-paccaya in the future meaning because of a term having object nearby.}
\example[0]{nagaraṃ karissati nagarakāro (nagara + \paliroot{kara} + ṇa) \\{\upshape= [One] will build a city, hence \pali{nagarakāra} (city-planner).}}
\example[0]{sāliṃ lāvissati sālilāvo (sāli + \paliroot{lu} + ṇa) \\{\upshape= [One] will reap rice, hence \pali{sālilāva} (rice-reaper).}}
\example[0]{dhaññaṃ vapissati dhaññavāpo (dhañña + \paliroot{vapa} + ṇa) \\{\upshape= [One] will sow grain, hence \pali{dhaññavāpa} (grain-sower).}}
\example[0]{bhogaṃ dadissati bhogadāyo (bhoga + \paliroot{dā} + ṇa) \\{\upshape= [One] will give wealth, hence \pali{bhogadāya} (wealth-giver).}}
\example{sindhuṃ pivissati sindhupāyo (sindhu + \paliroot{pā} + ṇa) \\{\upshape= [One] will drink river-water, hence \pali{sindhupāya} (river-water-drinker).}}
\transnote{In the text, there is \pali{vajati} in these examples, but I left it out.}

\head{655}{655, 650. sese ssaṃntumānānā.}
\headtrans{In the sense of unfinished [action], [there are] \pali{ssaṃ}-, \pali{ntu}-, \pali{māna}-, and \pali{āna}-paccaya.}
\sutdef{sesaiccetasmiṃ atthe ssaṃntumānaānaiccete paccayā honti bhavissatikāle kammūpapade.}
\sutdeftrans{There are \pali{ssaṃ}-, \pali{ntu}-, \pali{māna}-, and \pali{āna}-paccaya in the sense of unfinished [action] in the future meaning with object nearby.}
\example[0]{kammaṃ karissati kammaṃ karissaṃ (\paliroot{kara} + ssaṃ)}
\example[0]{kammaṃ karonto (\paliroot{kara} + ntu)}
\example[0]{kammaṃ kurumāno (\paliroot{kara} + māna)}
\example{kammaṃ karāno (\paliroot{kara} + āna) \\{\upshape= [One] will do work, hence one going to do work, etc.}}
\example[0]{bhojanaṃ bhuñjissati bhojanaṃ bhuñjissaṃ (\paliroot{bhuja} + ssaṃ)}
\example[0]{bhojanaṃ bhuñjanto (\paliroot{bhuja} + ntu)}
\example[0]{bhojanaṃ bhuñjamāno (\paliroot{bhuja} + māna)}
\example{bhojanaṃ bhuñjāno (\paliroot{bhuja} + āna) \\{\upshape= [One] will eat food, hence one going to eat food, etc.}}
\example[0]{khādanaṃ khādissati khādanaṃ khādissaṃ (\paliroot{khāda} + ssaṃ)}
\example[0]{khādanaṃ khādanto (\paliroot{khāda} + ntu)}
\example[0]{khādanaṃ khādamāno (\paliroot{khāda} + māna)}
\example{khādanaṃ khādāno (\paliroot{khāda} + āna) \\{\upshape= [One] will chew food, hence one going to chew food, etc.}}
\example[0]{maggaṃ carissati maggaṃ carissaṃ (\paliroot{cara} + ssaṃ)}
\example[0]{maggaṃ caranto (\paliroot{cara} + ntu)}
\example[0]{maggaṃ caramāno (\paliroot{cara} + māna)}
\example{maggaṃ carāno (\paliroot{cara} + āna) \\{\upshape= [One] will walk the path, hence one going to walk the path, etc.}}
\example[0]{bhikkhaṃ bhikkhissati bhikkhaṃ bhikkhissaṃ \\(\paliroot{bhikkha} + ssaṃ)}
\example[0]{bhikkhaṃ bhikkhanto (\paliroot{bhikkha} + ntu)}
\example[0]{bhikkhaṃ bhikkhamāno (\paliroot{bhikkha} + māna)}
\example{bhikkhaṃ bhikkhāno (\paliroot{bhikkha} + āna) \\{\upshape= [One] will beg food, one hence going to beg food, etc.}}
\transnote{The end products of these are nominal, but very close to present participles (products of \pali{māna} and \pali{anta}). That is to say, the actions mentioned get started but not finished yet.}

\head{656}{656, 666. chadādīhi tatraṇa.}
\headtrans{After \pali{chada} and so on, [there are] \pali{ta}- and \pali{traṇa}-paccaya.}
\sutdef{chada\,citi\,su\,nī\,vida\,pada\,tanu\,yata\,ada\,mada\,yuja\,vatu\,mida\-mā\,pu\,kala\,vara\,vepu\,gupa\,dā\,iccevamādīhi\ \ dhātūhi\ \ tatraṇiccete\ \ paccayā\ \ honti\ \ yathāsambhavaṃ.}
\sutdeftrans{There are \pali{ta}- and \pali{traṇa}-paccaya relevantly after roots such as \pali{chada, citi, su, nī, vi, da, pada, tanu, yata, ada, mada, yuja, vatu, mida, mā, pu, kala, vara, vepu, gupa,} and \pali{dā}.}
\example[0]{ātapaṃ chādetīti chattaṃ/chatraṃ (\paliroot{chada} + ta/traṇa) \\{\upshape= [It] prevents heat, hence \pali{chatta/chatra} (parasol).}}
\example[0]{ārammaṇaṃ cintetīti, cintenti sampayuttadhammā ethenāti vā cittaṃ/citraṃ (\paliroot{citi} + ta/traṇa) \\{\upshape= [It] thinks of sense-object, or [they] think of associated mental states by that, hence \pali{citta/citra} (mind).}}
\example[0]{ārammaṇaṃ cintetīti cittaṃ/citraṃ (\paliroot{citi} + ta/traṇa) \\{\upshape= [It] thinks of sense-object, hence \pali{citta/citra} (mind).}}
\example[0]{atthe abhissavetīti, atthe sūcetīti vā suttaṃ/sutraṃ \\(\paliroot{su} + ta/traṇa) \\{\upshape= [It] makes meanings flow into, or makes meanings clarified, hence \pali{sutta/sutra} (discourse).}}
\example[0]{satte netīti, satte icchitaṭṭhānaṃ nenti etenāti vā nettaṃ/netraṃ (\paliroot{nī} + ta/traṇa) \\{\upshape= [It] leads beings, or [they] lead beings to a desired place by that, hence \pali{netta/netra} (eye).}}
\example[0]{pakārena vidatīti pavittaṃ/pavitraṃ (pa + \paliroot{vida} + ta/traṇa) \\{\upshape= [One] knows [things] by putting together, hence \pali{pavitta/pavitra} (knower).}}
\example[0]{vividhena ākārena pāpaṃ punāti, sodhetīti pavittaṃ/pavitraṃ (\paliroot{pu} + ta/traṇa) \\{\upshape= [It] cleanses [or] purifies an evil action in various ways, hence \pali{pavitta/pavitra} (purification/filter).}}
\example[0]{sucibhāvaṃ vā pāpuṇātīti pavittaṃ/pavitraṃ \\(pa + \paliroot{vida} + ta/traṇa) \\{\upshape= [It is] the state of cleanliness, or [one] attains [that], hence \pali{pavitta/pavitra} (purified state).}}
\example[0]{padati pāpuṇātīti patto/patro (\paliroot{pada} + ta/traṇa) \\{\upshape= [One] goes [or] reaches, hence \pali{patta/patra} (goer/reacher).}}
\example[0]{āhārā patanti ettha bhājaneti, padati pavattatīti vā pattaṃ/ patraṃ (\paliroot{pada} + ta/traṇa) \\{\upshape= Foods fall into this vessel, or [it] goes [or] arises [there], hence \pali{patta/patra} (bowl).}}
\example[0]{tanoti vitthāretīti, tanitabbaṃ vitthāretabbanti vā tantaṃ/ tantraṃ (\paliroot{tanu} + ta/traṇa) \\{\upshape= [It] extends [or] spreads out, or [it is] worthy to be extended [or] spread out, hence \pali{tanta/tantra} (thread).}}
\example[0]{yatatīti, vīriyaṃ karoti etenāti vā, yatanaṃ vā yattaṃ/yatraṃ (\paliroot{yata} + ta/traṇa) \\{\upshape= [One] strives, or makes effort with that, or [it is] an endeavor, hence \pali{yatta/yatra} (endeavor).}}
\example[0]{sukhadukkhaṃ adati bhakkhatīti attā/atrā \\(\paliroot{ada} + ta/traṇa) \\{\upshape= [It] eats [or] feeds upon happiness and suffering, hence \pali{atta/ atra} (the self).}}
\example[0]{madatīti mattaṃ/matraṃ (\paliroot{mada} + ta/traṇa) \\{\upshape= [It] makes intoxicated, hence \pali{matta/matra} (intoxicant).}}
\example[0]{vatthuṃ yujjanti etenāti yottaṃ/yotraṃ (\paliroot{yuja} + ta/traṇa) \\{\upshape= [They] bind things by that, hence \pali{yotta/yotra} (rope).}}
\example[0]{vattatīti vattaṃ/vatraṃ (\paliroot{vatu} + ta/traṇa) \\{\upshape= [It] moves forward, hence \pali{vatta/vatra} (procedure/duty).}}
\example[0]{midati sinehaṃ karotīti mittaṃ/mitraṃ (\paliroot{mida} + ta/traṇa) \\{\upshape= [One] sticks [people together or] creates love, hence \pali{mitta/mitra} (friend).}}
\example[0]{midati sinehati etāyāti mettā/metrā (\paliroot{mida} + ta/traṇa) \\{\upshape= [One] sticks [people together or] makes smooth contact by that, hence \pali{mettā/metrā} (friendliness).}}
\example[0]{pari samantato sabbākārena minanti etāyāti, mānanaṃ vā mattā/matrā (\paliroot{mā} + ta/traṇa) \\{\upshape= [They] measure by every aspect all around by that, or [it is] a measurement, hence \pali{mattā/matrā} (measure/gauge).}}
\example[0]{attano kulaṃ punāti sodhetīti putto/putro (\paliroot{pu} + ta/traṇa) \\{\upshape= [One] cleanses [or] purifies one's own family, hence \pali{putta/putra} (son).}}
\example[0]{kalitabbaṃ saṅkhyātabbanti kalattaṃ/kalatraṃ \\(\paliroot{kala} + ta/traṇa) \\{\upshape= [One is] worthy to be counted [in the family], hence \pali{kalatta/ kalatra} (wife).}}
\example[0]{saṃsuṭṭhu vāreti etenāti varattaṃ/varatraṃ \\(\paliroot{vara} + ta/traṇa) \\{\upshape= [One] prevents [cloth/things] well by that, hence \pali{varatta/varatra} (strap).}}
\example[0]{vepati kampatīti vettaṃ/vetraṃ (\paliroot{vepu} + ta/traṇa) \\{\upshape= [It] shakes [by wind], hence \pali{vetta/vetra} ([bamboo] cane).}}
\example[0]{gopitabbaṃ rakkhitabbanti guttaṃ/gutraṃ/gottaṃ/gotraṃ (\paliroot{gupa} + ta/traṇa) \\{\upshape= [It is] worthy to be guarded [or] protected, hence \pali{gutta/gutra/ gotta/gotra} (clan).}}
\example{dāti avakhaṇḍati etenāti dāttaṃ/dātraṃ (\paliroot{dā} + ta/traṇa) \\{\upshape= [One] cuts [or] reaps [crops] by that, hence \pali{dāttaṃ/dātra} (sickle).}}
\transnote{The paccaya \pali{traṇ} or \pali{traṇa} is dubious because we hardly find its products in Pāli. In Pāṇ 6.4.97 and Kāt 4.19, \pali{tran} is mentioned and \pali{chattram} (parasol) is found as an example in both Kāśikā and Durgasiṅha's.}
\transnote{The example of \pali{putta} is interesting in the way that it reflects the old Vedic belief---if you do not have a son, you will be doomed in hell.}

\head{657}{657, 667. vadādīhi ṇitto gaṇe.}
\headtrans{After \pali{vada} and so on, [there is] \pali{ṇitta}-paccaya in [the sense of] group.}
\sutdef{vada\,cara\,vara\,iccevamādīhi\ \ dhātūhi\ \ ṇittapaccayo\ \ hoti\ \ gaṇ\-atthe.}
\sutdeftrans{There is \pali{ṇitta}-paccaya after roots such as \pali{vada}, \pali{cara}, \pali{vara} and so on in the sense of group.}
\example[0]{vāditānaṃ gaṇo vādittaṃ (\paliroot{vada} + ṇitta) \\{\upshape= [It is] a group of musical players, hence \pali{vāditta} (musical band).}}
\example[0]{cārittaṃ (\paliroot{cara} + ṇitta) \\{\upshape= a range of practices, custom, norm}}
\example{vārittaṃ (\paliroot{vara} + ṇitta) \\{\upshape= a range of prohibitions, moral code, law}}

\head{658}{658, 668. midādīhi ttitiyo.}
\headtrans{After \pali{mida} and so on, [there are] \pali{tti}- and \pali{ti}-paccaya.}
\sutdef{mida\,pada\,ranja\,tanu\,dhā\,iccevamādīhi\ \ dhātūhi\ \ tti\,ti\,iccete\ \ paccayā\ \ honti.}
\sutdeftrans{There are \pali{tti}- and \pali{ti}-paccaya after roots such as \pali{mida}, \pali{pada}, \pali{ranja}, \pali{tanu}, \pali{dhā} and so on.}
\example[0]{midati sinehatīti metti (\paliroot{mida} + tti) \\{\upshape= [It] sticks [people together or] makes smooth contact, hence \pali{metti} (friendliness).}}
\example[0]{padati gacchatīti patti (\paliroot{pada} + tti) \\{\upshape= [One] goes [on foot], hence \pali{patti} (foot-soldier).}}
\example[0]{ranjati etthāti ratti (\paliroot{ranja} + tti) \\{\upshape= [One] finds delight in this [time], hence \pali{ratti} (night).}}
\example[0]{tanoti vitthāretīti tanti (\paliroot{tanu} + ti) \\{\upshape= [It] extends [or] spread out, hence \pali{tanti} (string).}}
\example[0]{attano kulaṃ tanoti vitthāretīti tanti (\paliroot{tanu} + ti) \\{\upshape= [One] extends [or] spread out one's own family, hence \pali{tanti} (lineage).}}
\example[0]{paresaṃ itthīnaṃ puttaṃ dhāretīti, khīraṃ dhāretīti vā, attano sabhāvaṃ dhāretīti vā dhāti (\paliroot{dhā} + ti) \\{\upshape= [One] holds a child of other woman, or holds milk [for that child], hence \pali{dhāti} (nanny).}}
\example{attano sabhāvaṃ dhāretīti dhāti (\paliroot{dhā} + ti) \\{\upshape= [One] holds one's own nature, hence \pali{dhāti}.}}

\head{659}{659, 669. usuranjadaṃsānaṃ daṃsassa daḍḍho ḍhaṭhā ca.}
\headtrans{For \pali{daṃ} of \pali{usa}, \pali{ranja}, and \pali{daṃsa}, [there is] \pali{daḍḍha}-substi\-tution. [There are] \pali{ḍha}- and \pali{ṭha}-paccaya also.}
\sutdef{usu\,ranja\,daṃsa\,iccetesaṃ\ \ dhātūnaṃ\ \ daṃsassa\ \ daḍḍhādeso\ \ hoti, ḍhaṭhapaccayā ca honti.}
\sutdeftrans{There is a substitution of \pali{daḍḍha} for \pali{daṃ} of the roots \pali{usa}, \pali{ranja}, and \pali{daṃsa}. There are \pali{ḍha}- and \pali{ṭha}-paccaya also.}
\example[0]{usīyate uḍḍho (\paliroot{usu} + ḍha) \\{\upshape= [It] is burned, hence \pali{uḍḍha} (burned).}}
\example[0]{ranjanti etthāti raṭṭhaṃ (\paliroot{ranja} + ṭha) \\{\upshape= [They] find delight here, hence \pali{raṭṭha} (country).}}
\example{daṃsīyateti daḍḍho (\paliroot{daṃsa} + kvi) \\{\upshape= [One] is bitten, hence \pali{daḍḍha} (bitten).}}
\transnote{This sutta sounds a little confusing. This substitution of \pali{daḍḍha} is for \paliroot{daṃsa} only, and the paccayas mentioned are for other roots.}

\head{660}{660, 670. sūvusānamūvusānamato tho ca.}
\headtrans{[There is] \pali{ata}-substitution for \pali{ū}, \pali{u}, and \pali{asa} of \pali{sū}, \pali{vu}, and \pali{asa}. [There is] \pali{tha}-paccaya also.}
\sutdef{sūvuasaiccetesaṃ dhātūnaṃ ūuasānaṃ atādeso hoti, thapaccayo ca.}
\sutdeftrans{There is a substitution of \pali{ata} for \pali{ū}, \pali{u}, and \pali{asa} of the roots \pali{sū}, \pali{vu}, and \pali{asa}. [There is] \pali{tha}-paccaya also.}
\example[0]{savati hiṃsati etenāti satthaṃ (\paliroot{sū} + tha) \\{\upshape= [One] hurts [others] by that, hence \pali{sattha} (weapon)}}
\example[0]{hirottappaṃ saṃvarati etenāti vatthaṃ (\paliroot{vu} + tha) \\{\upshape= [One] prevents shame and fear by that, hence \pali{vattha} (cloth)}}
\example{saddānurūpaṃ asati bhavatīti attho (\paliroot{asa} + tha) \\{\upshape= [It] exists suitable to the sound, hence \pali{attha} (meaning)}}
\transnote{This sutta might come fro-m a speculation for some common terms. The operation looks really contrived. In Pāli, it is \paliroot{su} that means `\pali{hiṃsāyaṃ}' (it can be a possibility to differentiate \paliroot{sū} `to hurt' from \paliroot{su} `to hear'). In Sanskrit, \pali{śastra} is a product of \paliroot{śas}, a more straight formation. See Kāśikā on Pāṇ 3.2.182, also Durgasiṅha's on Kāt 4.306.}

\head{661}{661, 671. ranjudādīhi dhadiddakirā kvaci jadalopo ca.}
\headtrans{After \pali{ranja} and so on, [there are] \pali{dha}-, \pali{da}-, \pali{idda}-, \pali{ka}-, and \pali{ira}-paccaya. [There is] an elision of \pali{ja} and \pali{da} in some places also.}
\sutdef{ranja\,uda\,idi\,cadi\,madi\,khuda\,chidi\,rudi\,dala\,susa\,suca\,vaca\-vaja\,iccevamādīhi\ \ dhātūhi\ \ dha\,da\,idda\,ka\,ira\,iccete\ \ paccayā\ \ honti, kvaci jadalopo ca, puna nippajjante.}
\sutdeftrans{There are \pali{dha}-, \pali{da}-, \pali{idda}-, \pali{ka}-, and \pali{ira}-paccaya after roots such as \pali{ranja}, \pali{uda}, \pali{idi}, \pali{cadi}, \pali{madi}, \pali{khuda}, \pali{chidi}, \pali{rudi}, \pali{dala}, \pali{susa}, \pali{suca}, \pali{vaca}, \pali{vaja} and so on. [There is] an elision of \pali{ja} and \pali{da} in some places also, which is applied later.}
\example[0]{rañjitabbanti, ranjayitthāti vā randhaṃ (\paliroot{ranja} + dha) \\{\upshape= [It is] worthy to be colored, or [it] was colored [=? covered with color], hence \pali{randha} (hole/fault).}}
\example[0]{attani sannissitānaṃ macchamakarānaṃ pītisomanassaṃ undati pasavati janetīti samuddo (saṃ + \paliroot{uda} + da) \\{\upshape= [It] generates [or] produces delight and happiness in its own for fish and sea animals depending on it, hence \pali{samudda} (sea).}}
\example[0]{indati paramissariyaṃ karotīti, indattaṃ adhipatibhāvaṃ karotīti vā indo (\paliroot{idi} + da) \\{\upshape= [One] makes the highest rulership, or makes the state of rulership, hence \pali{inda} (the Indra).}}
\example[0]{canditabbo icchitabboti cando (\paliroot{cadi} + da) \\{\upshape= [It is] worthy to be wished, hence \pali{canda} (moon).}}
\example[0]{mandati hāsetīti, maditabbo hāsetabboti vā mando \\(\paliroot{madi} + da) \\{\upshape= [One] makes laugh, or should be laughed at, hence \pali{manda} (idiot).}}
\example[0]{khudati pipāsetīti khuddo (\paliroot{khuda} + da) \\{\upshape= [One] thirsts, hence \pali{khudda} (hunger).}}
\example[0]{chinditabboti chiddo (\paliroot{chidi} + da) \\{\upshape= [It is] worthy to be cut, hence \pali{chidda} (hole).}}
\example[0]{rudati hiṃsatīti ruddo (\paliroot{rudi} + da) \\{\upshape= [One] destroys, hence \pali{rudda} (destroyer).}}
\example[0]{dalati duggatabhāvaṃ gacchatīti daliddo (\paliroot{dala} + idda) \\{\upshape= [One] goes to the state of poverty, hence \pali{dalidda} (poor).}}
\example[0]{sussatīti sukkaṃ (\paliroot{susa} + ka) \\{\upshape= [It] dries, hence \pali{sukka} (dry).}}
\example[0]{sucatīti soko (\paliroot{suca} + ka) \\{\upshape= [One] grieves, hence \pali{soka} (grief).}}
\example[0]{vacitabbanti vakkaṃ (\paliroot{vaca} + ka) \\{\upshape= [It is] worthy to be said, hence \pali{vakka} (saying).}}
\example{appaṭihato hutvā vajati gacchatīti vajiraṃ (\paliroot{vaja} + ira) \\{\upshape= Having been unobstructed, [it] goes, hence \pali{vajira} (thunderbolt).}}

\head{662}{662, 672. paṭito hissa heraṇhīraṇ.}
\headtrans{After \pali{paṭi}, [there are] \pali{heraṇ}- and \pali{hīraṇ}-substitution for \pali{hi}.}
\sutdef{paṭiiccetasmā hissa dhātussa heraṇhīraṇādesā honti.}
\sutdeftrans{There are substitutions of \pali{heraṇ} and \pali{hīraṇ} for the root \pali{hi} after \pali{paṭi}.}
\example{paṭipakkhe madditvā gacchati pavattatīti pāṭiheraṃ/pāṭihīraṃ (paṭi + \paliroot{hi} + kvi) \\{\upshape= Having crushed the enemies, [one] goes [or] moves on, hence \pali{pāṭiheraṃ/pāṭihīraṃ} (miracle).}}
\transnote{In some sources, the replacements are \pali{heraṇa} and \pali{hīraṇa}. The \pali{ṇa} is just an anubandha causing vuddhi-strength to the whole word, not to the root as we have normally seen.}

\head{663}{663, 673. kaḍyādīhi ko.}
\headtrans{After \pali{kaḍi} and so on, [there is] \pali{ka}-paccaya.}
\sutdef{kaḍi\,ghaḍi\,vaḍi\,karaḍi\,maḍi\,saḍi\,kuṭhi\,bhaḍi\,paḍi\,daḍi\,raḍi\-taḍi\,isiḍi\,caḍi\,gaḍi\,aḍi\,laḍi\,meḍi\,eraḍi\,khaḍi\,iccevamādīhi\ \ dhā\-tūhi\ \ kapaccayo hoti saha paccayena ca nippajjante yathāsam\-bhavaṃ.}
\sutdeftrans{There is \pali{ka}-paccaya after roots such as \pali{kaḍi}, \pali{ghaḍi}, \pali{vaḍi}, \pali{karaḍi}, \pali{maḍi}, \pali{saḍi}, \pali{kuṭhi}, \pali{bhaḍi}, \pali{paḍi}, \pali{daḍi}, \pali{raḍi}, \pali{taḍi}, \pali{isiḍi}, \pali{caḍi}, \pali{gaḍi}, \pali{aḍi}, \pali{laḍi}, \pali{meḍi}, \pali{eraḍi}, \pali{khaḍi} and so on. [They] are also finished together with paccaya relevantly.}
\example[0]{kaṇḍitabbo chinditabboti kaṇḍo (\paliroot{kaḍi} + ka) \\{\upshape= [It is] worthy te be cut, hence \pali{kaṇḍa} (chapter).}}
\example[0]{ghaṇḍitabbo ghaṭetabboti ghaṇḍo (\paliroot{ghaḍi} + ka) \\{\upshape= [It is] worthy to be stricken, hence \pali{ghaṇḍa} (bell).}}
\example[0]{vaṇḍanti etthāti vaṇḍo (\paliroot{vaḍi} + ka) \\{\upshape= [Things] are wrapped/supported on this, hence \pali{vaṇḍa} (stem of a plant).}}
\example[0]{karaṇḍitabbo bhājetabboti karaṇḍo (\paliroot{karaḍi} + ka) \\{\upshape= [It is] worthy to be divided, hence \pali{karaṇḍa} (wicker basket).}}
\example[0]{maṇḍīyate vibhūsīyate etenāti maṇḍo (\paliroot{maḍi} + ka) \\{\upshape= [One] is adorned by that, hence \pali{maṇḍa} (ornament).}}
\example[0]{saṇḍanti gumbanti etthāti saṇḍo (\paliroot{saḍi} + ka) \\{\upshape= [Trees] gather here, hence \pali{saṇḍa} (thicket).}}
\example[0]{aṅgamaṅgāni kuṇṭhati chindatīti kuṭṭhaṃ (\paliroot{kuṭhi} + ka) \\{\upshape= [It] cuts the limbs, hence \pali{kuṭṭha} (leprosy).}}
\example[0]{bhaṇḍitabbanti bhaṇḍaṃ (\paliroot{bhaḍi} + ka) \\{\upshape= [It is] worthy to be disputed over, hence \pali{bhaṇḍa} (goods).}}
\example[0]{paṇḍati liṅgavekallabhāvaṃ gacchatīti paṇḍako \\(\paliroot{paḍi} + ka) \\{\upshape= [One] goes into the state of sexual deformation, hence \pali{paṇḍaka} (hermaphrodite).}}
\example[0]{daṇḍati āṇaṃ karoti etenāti daṇḍo (\paliroot{daḍi} + ka) \\{\upshape= [One] makes order by that, hence \pali{daṇḍa} (stick/punishment).}}
\example[0]{raṇḍati hiṃsatīti raṇḍo (\paliroot{raḍi} + ka) \\{\upshape= [One] destroys, hence \pali{raṇḍa} (destroyer).}}
\example[0]{visesena taṇḍati cāleti paresaṃ viññūnaṃ hadayaṃ kampetīti vitaṇḍo (vi + \paliroot{taḍi} + ka) \\{\upshape= [One] strikes [or] disturbs [or] shakes the heart of other wise person by superior state, hence \pali{vitaṇḍa} (sophistry).}}
\example[0]{isiṇḍati paresaṃ maddatīti isiṇḍo (\paliroot{isiḍi} + ka) \\{\upshape= [One] crushes others, hence \pali{isiṇḍa} (crusher).}}
\example[0]{caṇḍati caṇḍikkabhāvaṃ karotīti caṇḍo (\paliroot{caḍi} + ka) \\{\upshape= [One] creates the state of ferocity, hence \pali{caṇḍa} (fierce one).}}
\example[0]{gaṇḍati sannicayati samūhaṃ karoti etthāti gaṇḍo \\(\paliroot{gaḍi} + ka) \\{\upshape= [It] accumulates [or] makes a lump [of food], hence \pali{gaṇḍa} (cheek).}}
\example[0]{aṇḍīyati nibbattīyatīti aṇḍo (\paliroot{aḍi} + ka) \\{\upshape= [It] is generated, hence \pali{aṇḍa} (egg).}}
\example[0]{laṇḍitabbo jigucchitabboti laṇḍo (\paliroot{laḍi} + ka) \\{\upshape= [It is] worthy to be detested, hence \pali{laṇḍa} (dung).}}
\example[0]{meṇḍati kuṭilabhāvaṃ gacchatīti meṇḍo (\paliroot{meḍi} + ka) \\{\upshape= [It] goes into the state of being crooked, hence \pali{meṇḍa} (ram).}}
\example[0]{eraṇḍati rogaṃ hiṃsatīti eraṇḍo (\paliroot{eraḍi} + ka) \\{\upshape= [It] fights disease, hence \pali{eraṇḍa} (castor oil plant).}}
\example{khaṇḍitabbo chinditabboti khaṇḍo (\paliroot{khaḍi} + ka) \\{\upshape= [It is] worthy to be broken, hence \pali{khaṇḍa} (bit/portion).}}
\transnote{I try not to polish the translations too much, to the extent that they may stray too far from the Pāli. So, you may have to use some imagination to get the picture right. But sometimes we are barely able to access the original logic of the words (or they might just be a \emph{post hoc} construction).}
\transnote{In \hyperref[sut:672]{Kacc 672}, \pali{kuṭṭha} is analyzed as \paliroot{kuṭa} + \pali{ṭha}.}

\head{664}{664, 674. khādāmagamānaṃ khandha’ndhagandhā.}
\headtrans{For \pali{khāda}, \pali{ama}, and \pali{gamu}, [there are] \pali{khanda}-, \pali{andha}-, and \pali{gandha}-substitution.}
\sutdef{khāda\,ama\,gamu\,iccetesaṃ\ \ dhātūnaṃ\ \ khandha\,andha\,gan\-dhādesā\ \ honti, kapaccayo ca hoti.}
\sutdeftrans{There are substitutions of \pali{khanda}, \pali{andha}, and \pali{gandha} for the roots \pali{khāda}, \pali{ama}, and \pali{gamu}. There is \pali{ka}-paccaya also.}
\example[0]{jātijarāmaraṇādīhi saṃsāradukkhehi khāditabboti khandho (\paliroot{khāda} + ka) \\{\upshape= [It] should be chewed by the cyclic sufferings such as birth, old-age, death and so on, hence \pali{khandha} (the aggregate).}}
\example[0]{amati aṅgamaṅgassa rujjanabhāvaṃ gacchatīti, cakkhunā amati rujjatīti vā andho (\paliroot{āma} + ka) \\{\upshape= [One] goes into the state of injury of limbs, or is injured by an eye, hence \pali{andha} (blind).}}
\example{taṃ taṃ ṭhānaṃ vātena gacchatīti gandho (\paliroot{gamu} + ka) \\{\upshape= [It] goes by air to such and such place, hence \pali{gandha} (smell).}}
\transnote{The terms given in the examples have the \pali{ka} elided, but the paccaya can also be present, hence \pali{khandhako, andhako,} and \pali{gandhako}.}

\head{665}{665, 675. paṭādīhyalaṃ.}
\headtrans{After \pali{paṭa} and so on, [there is] \pali{ala}-paccaya.}
\sutdef{paṭa\,kala kusa\,kada\,bhaganda\,mekha\,vakka\,takka\,palla\,sad\-da\,mūla\,bila\,vida\,caḍi\,pañca\,vā\,vasa\,paci\,maca\,musa\,gotthu\,pu\-thu\,bahu\,maṅga\,baha\,kamba\,samba\,agga\,iccevamādīhi\ \ dhātūhi\ \ pāṭipadikehi ca uttarapadesu alapaccayo hoti, pacchā puna nippajjante.}
\sutdeftrans{There is \pali{ala}-paccaya after roots and nominal bases such as \pali{paṭa}, \pali{kala}, \pali{kusa}, \pali{kada}, \pali{bhaganda}, \pali{mekha}, \pali{vakka}, \pali{takka}, \pali{palla}, \pali{sadda}, \pali{mūla}, \pali{bila}, \pali{vida}, \pali{caḍi}, \pali{pañca}, \pali{vā}, \pali{vasa}, \pali{paci}, \pali{maca}, \pali{musa}, \pali{gotthu}, \pali{puthu}, \pali{bahu}, \pali{maṅga}, \pali{baha}, \pali{kamba}, \pali{samba}, \pali{agga} and so on. [They] are made finished later on.}
\example[0]{paṭe alanti paṭalaṃ (\pali{paṭa} + ala) \\{\upshape= [It is] suitable in a [piece of] cloth, hence \pali{paṭala} (bandage).}}
\example[0]{kale alanti kalalaṃ (\pali{kala} + ala) \\{\upshape= [It is] suitable in crudeness, hence \pali{kalala} (mud).}}
\example[0]{pāpake akusale dhamme kusati chindatīti, kusabhūte yathāsabhāvadhamme alanti vā kusalaṃ (\pali{kusa} + ala) \\{\upshape= [It] cuts the evil and unwholesome nature, or [it is] suitable for the true teaching, hence \pali{kusala} (wholesome).}}
\example[0]{kuse uddissa dāne alanti vā, kuse sañcaye dhammasamudāye alanti vā kusalaṃ (\pali{kusa} + ala) \\{\upshape= [It is] suitable especially in giving, or suitable in an accumulation of [good] nature, hence \pali{kusala} (wholesome).}}
\example[0]{kadde madde alanti kadalaṃ (\pali{kada} + ala) \\{\upshape= [It is] suitable in squeezing, hence \pali{kadala} (banana).}}
\example[0]{bhagande secane alanti, bhagande muttakarīsaharaṇe alanti vā bhagandalaṃ (\pali{bhaganda} + ala) \\{\upshape= [It is] suitable in dripping, or suitable in carrying urine and feces, hence \pali{bhagandala} (fistula disease).}}
\example[0]{mekhe kaṭivicitte alanti mekhalaṃ (\pali{mekha} + ala) \\{\upshape= [It is] suitable in adorning the hip, hence \pali{mekhala} (girdle).}}
\example[0]{vakke rukkhatace alanti vakkalaṃ (\pali{vakka} + ala) \\{\upshape= [It is] suitable in a tree's bark, hence \pali{vakkala} (garment made of bark).}}
\example[0]{takke rukkhasilese alanti takkalaṃ (\pali{takka} + ala) \\{\upshape= [It is] suitable in a tree's gum, hence \pali{takkala} (resin).}}
\example[0]{palle ninnaṭṭhāne alanti pallalaṃ (\paliroot{palla} + ala) \\{\upshape= [It is] suitable in a lowland place, hence \pali{pallala} (lake).}}
\example[0]{sadde harite alanti saddalaṃ (\pali{sadda} + ala) \\{\upshape= [It is] suitable in plants, hence \pali{saddala} (lawn).}}
\example[0]{mūle patiṭṭhāne alanti mulālaṃ (\pali{mūla} + ala) \\{\upshape= [It is] suitable in fixing, hence \pali{mulāla} (a kind of lotus).}}
\example[0]{biḷe nissaye alanti bilālaṃ (\paliroot{bila} + ala) \\{\upshape= [It is] suitable in supporting, hence \pali{bilāla} (cat).}}
\example[0]{vide vijjamāne alanti vidalaṃ (\paliroot{vida} + ala) \\{\upshape= [It is] suitable in existing, hence \pali{vidala} (split?).}}
\example[0]{caṇḍe alanti caṇḍālo (\paliroot{caḍi} + ala) \\{\upshape= [It is] suitable in ferocity, hence \pali{caṇḍāla} (outcaste).}}
\example[0]{pañcannaṃ rājūnaṃ alanti pañcālo (\pali{pañca} + ala) \\{\upshape= [It is] suitable for five kings, hence \pali{pañcāla} (Pañcāla).}}
\example[0]{vā gatigandhanesu alanti vālaṃ (\paliroot{vā} + ala) \\{\upshape= [It is] suitable in going and spreading odor, hence \pali{vāla} (tail hair).}}
\example[0]{vā padagamane alanti vāḷo (\paliroot{vā} + ala) \\{\upshape= [It is] suitable in going on foot, hence \pali{vāḷa} (beast).}}
\example[0]{vase acchādane alanti vasalo (\paliroot{vasa} + ala) \\{\upshape= [It is] suitable in concealment, hence \pali{vasala} (outcaste).}}
\example[0]{pace vitthāre alanti pacalo (\paliroot{paci} + ala) \\{\upshape= [It is] suitable in expanding, hence \pali{pacala} (wide?).}}
\example[0]{mace corakamme alanti macalo (\paliroot{maca} + ala) \\{\upshape= [It is] suitable for stealing, hence \pali{macala} (thief?).}}
\example[0]{muse theyye, muse pāṇacāge vā alanti musalo (\paliroot{musa} + ala) \\{\upshape= [It is] suitable in stealing, or giving up life, hence \pali{musala} (pestle).}}
\example[0]{gotte vaṃse siṅgālajātiyaṃ alanti gotthulo (\paliroot{gotthu} + ala) \\{\upshape= [It is] suitable in a species of jackal, hence \pali{gotthula} (a kind of jackal).}}
\example[0]{puthumhi vitthāre alanti puthulo (\paliroot{puthu} + ala) \\{\upshape= [It is] suitable in expanding, hence \pali{puthula} (wide).}}
\example[0]{bahumhi saṅkhyāne alanti, bahumhi vuddhimhi alanti vā bahulo (\paliroot{bahu} + ala) \\{\upshape= [It is] suitable in counting, or increasing, hence \pali{bahula} (many).}}
\example[0]{maṅgamhi gamane alanti maṅgalaṃ (\pali{maṅga} + ala) \\{\upshape= [It is] suitable in going, hence \pali{maṅgala} (auspicious).}}
\example[0]{bahumhi vuddhimhi alanti bahalaṃ (\paliroot{baha} + ala) \\{\upshape= [It is] suitable in growing, hence \pali{bahala} (thick).}}
\example[0]{kambamhi sañcalane alanti kambalaṃ (\paliroot{kamba} + ala) \\{\upshape= [It is] suitable in agitation, hence \pali{kambala} (woollen cloth).}}
\example[0]{sambamhi maṇḍale alanti sambalaṃ (\paliroot{samba} + ala) \\{\upshape= [It is] suitable in the area, hence \pali{sambala} (provision).}}
\example{agge gatikoṭille alanti aggaḷaṃ (\paliroot{agga} + ala) \\{\upshape= [It is] suitable in going crookedly, hence \pali{aggaḷa} (latch).}}
\transnote{The examples were indeed poorly formed. I have difficulties to find a suitable meaning for several words. You are supposed to feel a lot of strangeness here. Using your imagination can be helpful (sometimes). In Rūpa 675, the examples are rephrased. You may get better, undistracted ideas from that.}

\head{666}{666, 676. puthassa puthupathāmo vā.}
\headtrans{For \pali{putha}, [there are] \pali{puthu}-, and \pali{patha}-substitution. [There is] \pali{ama}-paccaya sometimes.}
\sutdef{puthaiccetassa pāṭipadikassa puthupathādesā honti, kvaci amapaccayo hoti.}
\sutdeftrans{There are substitutions of \pali{puthu} and \pali{patha} for the nominal base of \pali{putha}. In some places, there is \pali{ama}-paccaya.}
\example[0]{putha hutvā jātanti puthavī/paṭhavī (\paliroot{putha} + kvi) \\{\upshape= Having become, [it] was arisen, hence \pali{puthavī/paṭhavī} (earth).}}
\example[0]{pathame jāto pathamo/paṭhamo (\paliroot{putha} + ama) \\{\upshape= [One] was born first, hence \pali{pathama/paṭhama} (first).}}
\example{puthukilese janetīti puthujjano (\paliroot{putha} + jana + kvi) \\{\upshape= [One] produces many defilements, hence \pali{puthujjana} (ordinary person).}}

\head{667}{667, 677. sasvādīhi tudavo.}
\headtrans{After \pali{sasu} and so on, [there are] \pali{tu}- and \pali{du}-paccaya.}
\sutdef{sasu\,dada\,ada\,mada\,iccevamādīhi\ \ dhātūhi\ \ tuduiccete\ \ paccayā\ \ honti.}
\sutdeftrans{There are \pali{tu}- and \pali{du}-paccaya after roots such as \pali{sasu}, \pali{dada}, \pali{ada}, \pali{mada} and so on.}
\example[0]{aññe satte sasati hiṃsatīti sattu (\paliroot{sasu} + tu) \\{\upshape= [One] slays [or] hurts other beings, hence \pali{sattu} (enemy).}}
\example[0]{dukkhaṃ dadātīti daddu (\paliroot{dada} + du) \\{\upshape= [It] gives suffering, hence \pali{daddu} (ringworm).}}
\example[0]{dukkhena adati bhakkhati etthāti, dukkhaṃ adati anubhavati jano etenāti vā, dukkhaṃ bhājanaṃ ādhāraṃ bhāvatīti vā addu (\paliroot{ada} + du) \\{\upshape= [One] eats [or] chews with suffering here, or a person undergoes suffering by that, or [it] is a vessel [or] container of suffering, hence \pali{addu} (prison).}}
\example{madati ummattaṃ karotīti, madati maddabhāvaṃ karotīti vā maddu (\paliroot{mada} + du) \\{\upshape= [It] makes insane, or create the state of madness, hence \pali{maddu} (intoxicant).}}

\head{668}{668, 678. cyādīhi īvaro.}
\headtrans{After \pali{ci} and so on, [there is] \pali{īvara}-paccaya.}
\sutdef{cipādhāiccevamādīhi dhātūhi īvarapaccayo hoti.}
\sutdeftrans{There is \pali{īvara}-paccaya after roots such as \pali{ci}, \pali{pā}, \pali{dhā} and so on.}
\example[0]{cīyatīti cīvaraṃ (\paliroot{ci} + īvara) \\{\upshape= [It] is accumulated [piece by piece], hence \pali{cīvara} (robe).}}
\example[0]{pivatīti pīvaro, pātabbaṃ rakkhitabbanti vā pīvaraṃ \\(\paliroot{pā} + īvara) \\{\upshape= [One] drinks, or [it is] worthy to be protected, hence \pali{pīvara} (fat one).}}
\example{dhāreti dhāretvā jīvitaṃ kappetīti dhīvaro, dhīvaraṃ \\(\paliroot{dhā} + īvara) \\{\upshape= Having holds [a net], [one] makes a living, hence \pali{dhīvara} (fisherman).}}
\transnote{In the last example, \pali{dhāreti} is a product of \paliroot{dhara}, not \paliroot{dhā}.}

\head{669}{669, 679. munādīhi ci.}
\headtrans{After \pali{muna} and so on, also [there is] \pali{i}-paccaya.}
\sutdef{muna\,yata\,agga\,pata\,kava\,suca\,ruca\,mahāla\,bhaddāla\,mana\-iccevamādīhi\ \ dhātūhi, pāṭipadikehi ca ipaccayo hoti.}
\sutdeftrans{There is \pali{i}-paccaya after roots and nominal bases such as \pali{muna}, \pali{yata}, \pali{agga}, \pali{pata}, \pali{kava}, \pali{suca}, \pali{ruca}, \pali{mahāla}, \pali{bhaddāla}, \pali{mana} and so on.}
\example[0]{atthānatthaṃ munāti, ñeyyadhammaṃ lakkhaṇādivasena vā jānātīti muni (\paliroot{muna} + i) \\{\upshape= [One] knows [what is] good and not good [sensible and not sensible], or [one] knows what should be known by its characteristic and so on, hence \pali{muni} (sage).}}
\example[0]{yatati vīriyaṃ karotīti yati (\paliroot{yata} + i) \\{\upshape= [One] makes an effort, hence \pali{yati} (monk).}}
\example[0]{aggati kuṭilabhāvaṃ gacchatīti aggi (\paliroot{agga} + i) \\{\upshape= [It] goes into the state of being crooked, hence \pali{aggi} (fire).}}
\example[0]{patati seṭṭho hutvā purato gacchatīti pati (\paliroot{pata} + i) \\{\upshape= Having been the most important, [one] goes in the front, hence \pali{pati} (master).}}
\example[0]{kabyaṃ bandhatīti, kantaṃ manāpavacanaṃ vadatīti vā kavi (\paliroot{kava} + i) \\{\upshape= [One] composes a poem, or speaks charming [or] pleasant words hence \pali{kavi} (poet).}}
\example[0]{sucati parisuddhaṃ bhavatīti suci (\paliroot{suca} + i) \\{\upshape= [It] is clean, hence \pali{suci} (purity/goodness).}}
\example[0]{rucati dibbatīti ruci (\paliroot{ruca} + i) \\{\upshape= [It] radiates, hence \pali{ruci} (brightness).}}
\example[0]{mahantaṃ vibhāvaṃ bhogakkhandhaṃ lātīti mahāli \\(mahā + \paliroot{lā} + i) \\{\upshape= [One] takes a great portion of wealth, hence \pali{mahāli} (Mahāli).}}
\example[0]{bhaddaṃ yasaṃ lātīti bhaddāli (bhadda + \paliroot{lā} + i) \\{\upshape= [One] takes an excellent fame, hence \pali{bhaddāli} (Bhaddāli).}}
\example{manaṃ tattha ratane nayatīti maṇi (mana + i) \\{\upshape= The mind leads into that jewel, hence \pali{maṇi} (gem).}}

\head{670}{670, 680. vidādīhyūro.}
\headtrans{After \pali{vida} and so on, [there is] \pali{ūra}-paccaya.}
\sutdef{vida\,valla\,masa\,sida\,du\,ku\,kapu\,maya\,udi\,khajja\,kura\,icceva\-mādīhi\ \ dhātūhi, pāṭipadikehi ca ūrapaccayo hoti.}
\sutdeftrans{There is \pali{ūra}-paccaya after roots and nominal bases such as \pali{vida}, \pali{valla}, \pali{masa}, \pali{sida}, \pali{duku}, \pali{kapu}, \pali{maya}, \pali{udi}, \pali{khajja}, \pali{kura} and so on.}
\example[0]{vidituṃ alanti vidūro (\paliroot{vida} + ūra) \\{\upshape= [It is] suitable to be discovered, hence \pali{vidūra} (distant place).}}
\example[0]{vidūraṭṭhāne jāto vedūro (\paliroot{vida} + ūra) \\{\upshape= [One] was born in a distant country, hence \pali{vedūra} (foreigner).}}
\example[0]{vallati vallabhāvena bhavatīti, vallati aññamaññaṃ bandhatīti vā vallūro (\paliroot{valla} + ūra) \\{\upshape= [It] becomes by the state of restraint, or [it] binds one another, hence \pali{vallūra} (thicket).}}
\example[0]{āmasitabboti masūro (\paliroot{masa} + ūra) \\{\upshape= [It is] worthy to be rubbed, hence \pali{masūra} (lentil).}}
\example[0]{sindati siṅgārabhāvaṃ gacchatīti, sindati virocatīti vā sindūro (\paliroot{sida} + ūra) \\{\upshape= [It] goes into the state of elegance, or [it] shines, hence \pali{sindūra} (red lead).}}
\example[0]{gamituṃ alaṃ anāsannattāti dūro (\paliroot{du} + ūra) \\{\upshape= [It is] suitable to go, not in proximity, hence \pali{dūra} (far).}}
\example[0]{kuti saddaṃ karotīti kūro (\paliroot{ku} + ūra) \\{\upshape= [It] makes sound, hence \pali{kūra} (boiled rice).}}
\example[0]{attano gandhena aññaṃ gandhaṃ kapati hanati hiṃsatīti, kappati rogāpanayane samatthetīti vā kappūro (\paliroot{kapu} + ūra) \\{\upshape= [It] conceals [or] kills [or] destroys other smell by its own smell, or [it] has ability in removing [certain] disease, hence \pali{kappūra} (camphor).}}
\example[0]{mahiyaṃ ravatīti, mahiyaṃ yāti gacchatīti mayūro \\(mahī + \paliroot{yā} + ūra) \\{\upshape= [It] makes noise on earth, or [it] goes on earth, hence \pali{mayūra} (peacock).}}
\example[0]{paṃsuṃ undati pasavatīti undūro (\paliroot{udi} + ūra) \\{\upshape= [It] produces dirt, hence \pali{undūra} (rat).}}
\example[0]{khajjitabbo khāditabboti khajjūro (\paliroot{khajja} + ūra) \\{\upshape= [It is] worthy to be chewed, hence \pali{khajjūra} (date-palm).}}
\example{kurati akkosatīti kurūro (\paliroot{kura} + ūra) \\{\upshape= [One] abuses, hence \pali{kurūra} (abuser).}}
\transnote{This set of example is easier to tackle. Using MWD can elucidate several words. For date-palm, we normally see \pali{khajjūrī}.}

\head{671}{671, 681. hanādīhi ṇunutavo.}
\headtrans{After \pali{hana} and so on, [there are] \pali{ṇu}-, \pali{nu}-, and \pali{tu}-paccaya.}
\sutdef{hana\,jana\,bhā\,ri\,khanu\,ama\,ve\,dhe\,dhā\,si\,ki\,hi\,iccevamādīhi \\dhātūhi\ \ ṇunutuiccete paccayā honti.}
\sutdeftrans{There are \pali{ṇu}-, \pali{nu}-, and \pali{tu}-paccaya after roots such as \pali{hana}, \pali{jana}, \pali{bhāri}, \pali{khanu}, \pali{ama}, \pali{ve}, \pali{vedhā}, \pali{siki}, \pali{hi} and so on.}
\example[0]{bhojanaṃ hanati hiṃsati etenāti haṇu/hanu (\paliroot{hana} + ṇu/nu) \\{\upshape= [One] crushes food by that, hence \pali{haṇu/hanu} (jaw).}}
\example[0]{gamanaṃ janetītī jāṇu (\paliroot{jana} + ṇu) \\{\upshape= [It] makes the going, hence \pali{jāṇu} (knee).}}
\example[0]{bhāṇu dibbatīti bhāṇu (\paliroot{bhā} + ṇu) \\{\upshape= [It] shines, hence \pali{bhāṇu} (ray/sun).}}
\example[0]{nivāte riti gacchatītī reṇu (\paliroot{ri} + ṇu) \\{\upshape= [It] goes gently, hence \pali{reṇu} (dust).}}
\example[0]{khaṇitabbo avadāritabboti khāṇu (\paliroot{khanu} + ṇu) \\{\upshape= [It is] worthy to be dug out [or] cut down, hence \pali{khāṇu} (stump).}}
\example[0]{aṅgamaṅgassa rujjanabhāvaṃ vijjhanabhāvaṃ amati gacchatīti aṇu (\paliroot{ama} + ṇu) \\{\upshape= [It] goes into the state of injuring [or] piercing the limbs, hence \pali{aṇu} (tiny prickle).}}
\example[0]{veti tantasantāne bhavatītī, bahisāre alanti vā veṇu \\(\paliroot{ve} + ṇu) \\{\upshape= [It] has continuous texture, or [it is] suitable in the outer matter, hence \pali{veṇu} (bamboo).}}
\example[0]{vacchaṃ pāyetītī dhenu (\paliroot{dhe} + nu) \\{\upshape= [It] makes a calf drink, hence \pali{dhenu} (cow).}}
\example[0]{atthaṃ dhāretītī, gamanapacanādikaṃ kriyaṃ dhāretītī vā dhātu (\paliroot{dhā} + tu) \\{\upshape= [It] holds meaning, or holds an action such as going, cooking and so on, hence \pali{dhātu} (verbal root).}}
\example[0]{sīyatī bandhīyatītī setu (\paliroot{si} + tu) \\{\upshape= [It] is tied up, hence \pali{setu} (bridge).}}
\example[0]{uddhaṃ gacchati pavattatītī ketu (\paliroot{ki} + tu) \\{\upshape= [It] goes [or] occurs upward/upright, hence \pali{ketu} (flag).}}
\example{attano phalaṃ hinoti pavattatītī hetu (\paliroot{hi} + tu) \\{\upshape= [It] impels [or] moves forward its own result, hence \pali{hetu} (cause).}}

\head{672}{672, 682. kuṭādīhi ṭho.}
\headtrans{After \pali{kuṭa} and so on, [there is] \pali{ṭha}-paccaya.}
\sutdef{kuṭa\,kusa\,kaṭa\,iccevamādīhi\ \ dhātūhi, pāṭipadikehi ca ṭhapaccayo hoti.}
\sutdeftrans{There is \pali{ṭha}-paccaya after roots and nominal bases such as \pali{kuṭa}, \pali{kusa}, \pali{kaṭa} and so on.}
\example[0]{aṅgamaṅgaṃ kuṭati chindatītī kuṭṭhaṃ (\paliroot{kuṭa} + ṭha) \\{\upshape= [It] cuts the limbs, hence \pali{kuṭṭha} (leprosy)}}
\example[0]{dhaññena chādetabbo pūretabbotī koṭṭho (\paliroot{kusa} + ṭha) \\{\upshape= [It] should be covered [or] filled by grain, hence \pali{koṭṭha} (granary)}}
\example{kaṭitabbaṃ madditabbanti kaṭṭhaṃ (\paliroot{kaṭa} + ṭha) \\{\upshape= [It] should be crushed, hence \pali{kaṭṭha} (log/plank)}}
\transnote{In \hyperref[sut:663]{Kacc 663}, \pali{kuṭṭha} is analyzed as \paliroot{kuṭhi} + \pali{ka}.}

\head{673}{673, 683. manupūrasuṇādīhi ussanusisā.}
\headtrans{After \pali{manu}, \pali{pūra}, and \pali{suṇa}, [there are] \pali{ussa}-, \pali{nusa}-, and \pali{isa}-paccaya.}
\sutdef{manu\,pūra\,suṇa\,ku\,su\,ila\,ala\,maha\,si\,ki\,iccevamādīhi\ \ dhātūhi, pāṭipadikehi ca ussanusaisaiccete paccayā honti, puna nippajjante.}
\sutdeftrans{There are \pali{ussa}-, \pali{nusa}-, and \pali{isa}-paccaya after roots and nominal bases such as \pali{manu}, \pali{pūra}, \pali{suṇa}, \pali{kusu}, \pali{ila}, \pali{ala}, \pali{maha}, \pali{si}, \pali{ki} and so on. [They] are finished later on.}
\example[0]{kusalākusale dhamme manati jānātīti, kāraṇākāraṇaṃ manati jānātīti vā, atthānatthaṃ manati jānātīti vā manusso/mā\-nuso (\paliroot{manu} + ussa/nusa) \\{\upshape= [One] knows the wholesome and unwholesome natures, or knows what is the cause and not the cause, or knows what is sensible or not sensible, hence \pali{manussa/mānusa} (human being).}}
\example[0]{mātāpitūnaṃ hadayaṃ pūretīti, attano manorathaṃ pūretīti vā puriso/poso (\paliroot{pūra} + isa) \\{\upshape= [One] fulfills the heart of parents, or one's own wish, hence \pali{purisa/posa} (man).}}
\example[0]{sasurehi suṇitabbā hiṃsitabbāti, dvinnaṃ jānānaṃ kulasantānaṃ karotīti vā suṇisā (\paliroot{suṇa} + isa) \\{\upshape= [One is] likely to be hurt by parent-in-laws, or [one] makes the continuity of the family of two persons, hence \pali{suṇisā} (daugh\-ter-in-law).}}
\example[0]{kucchitabbanti karīsaṃ (\paliroot{ku} + isa) \\{\upshape= [It is] worthy to be detested, hence \pali{karīsa} (feces).}}
\example[0]{abbhaṃ vimocetīti suriso (\paliroot{su} + isa) \\{\upshape= [It] removes cloud, hence \pali{surisa} (sun).}}
\example[0]{tamandhakāravidhamanena sattānaṃ bhayaṃ surati hiṃs\-atīti sūriyo (\paliroot{su} + isa) \\{\upshape= [It] destroys fear of beings by taking away darkness, hence \pali{sūriya} (sun).}}
\example[0]{rogaṃ hiṃsatīti sirīso (\paliroot{su} + isa) \\{\upshape= [It] cures a disease, hence \pali{sirīsa} (a kind of tree).}}
\example[0]{ilati kampatīti, taṇhāya dubbalo hutvā ilati kampatīti vā illiso (\paliroot{ila} + isa) \\{\upshape= [One] trembles, or [one], having been weak by craving, trembles, hence \pali{illisa} (Illisa).}}
\example[0]{pāpakaraṇe alati samatthetīti alaso (\paliroot{ala} + isa) \\{\upshape= [One] has capability in doing evil, hence \pali{alasa} (lazy one).}}
\example[0]{mahitabbo pūjetabboti mahiso (\paliroot{maha} + isa) \\{\upshape= [It is] worthy to be worshipped, hence \pali{mahisa} (buffalo).}}
\example[0]{sīyati bandhīyatīti sīsaṃ (\paliroot{si} + isa) \\{\upshape= [It] is tied up, hence \pali{sīsa} (head).}}
\example{kitabbaṃ hiṃsitabbanti kisaṃ (\paliroot{ki} + isa) \\{\upshape= [One is] likely to be hurt, hence \pali{kisa} (skinny one).}}

