\chapter{Sandhi}

In the book on sandhi (joining words), there are 51 suttas totally. Mostly, it is about methodology of combining two words into one unit. There is also a preliminary part on Pāli alphabet.

The first section is general introduction. Sutta 1--9 are about Pāli letters and their terminology. Sutta 10--11 talk about conceptual ideas of sandhi methodology.

The second section, Sutta 10--22, is on sandhi methods, mainly concerning vowels, elision and substitution.

The third section, Sutta 23--29, continues on sandhi methods, mainly concerning consonants, retaining the original form, some transformations, and reduplication.

The fourth section, Sutta 30--41, continues on sandhi methods mainly concerning niggahita.

The fifth section, Sutta 42--51, continues on sandhi methods, mostly about substitutions.

\section{Paṭhamakaṇḍa}

\head{1}{1, 1. attho akkharasaññāto.}
\headtrans{Meaning [is] known by the letters.}
\sutdef{sabbavacanānamattho akkhareheva saññāyate. akkharavipattiyaṃ hi atthassa dunnayathā hoti, tasmā akkharakosallaṃ bahūpakāraṃ suttantesu.}
\sutdeftrans{The meaning of all the teachings is known only by the letters. Because of corruption of the letters, there is such difficulty of interpretation of the meaning. Therefore, [being] one having competence in the letters [is] very helpful in [the reading of] Suttas.}
\transnote{As implied by \pali{suttantesu}, \pali{vacana} here means the teaching.}

\head{2}{2, 2. akkharāpādayo ekacattālīsaṃ.}
\headtrans{The letters [are] those forty-one, such as \pali{a} and so on.}
\sutdef{te ca kho akkharā api akārādayo ekacattālīsa suttantesu sopa\-kārā.}
\sutdeftrans{These 41 letters, such as \pali{a} and so on, [are] helpful in [the reading of] Suttas.}
\transnote{In the formula, we can break \pali{akkharāpādayo} to \pali{akkharā + pi + a + ādayo}. In the definition, \pali{sopakārā} is \pali{saha + upakārā} (see \hyperref[sut:14]{Kacc 14} below). Most particles in this sentence have little meaning, so left out. All 41 letters can be arranged in groups as shown in Table \ref{tab:letters}.}
\transnote{Note that when we call out a letter, we normally add \pali{a} to it, hence \pali{ka}, for example. This will happen in all suttas below. When a letter is applied as an insertion, for instance, the \pali{a} is dropped. Just the sole letter is used. See \hyperref[sut:61]{Kacc 61} and \hyperref[sut:67]{67}, for example.}

\begin{table}[!hbt]
\centering
\caption{Pāli letters}
\label{tab:letters}
\bigskip
\begin{tabular}{|l|*{5}{c|}}
\hline sara & \multicolumn{5}{c|}{\pali{a\ \ ā\ \ i\ \ ī\ \ u\ \ ū\ \ e\ \ o}} \\
\hline
\hline \multirow{2}{*}{byañjana} & \multicolumn{2}{c|}{aghosa} & \multicolumn{2}{c|}{ghosa} & \multirow{2}{*}{nāsikā} \\
\cline{2-5} & sithila & dhanita & sithila & dhanita &  \\
\hline kaṇṭhaja & \pali{ka} & \pali{kha} & \pali{ga} & \pali{gha} & \pali{ṅa} \\
\hline tāluja & \pali{ca} & \pali{cha} & \pali{ja} & \pali{jha} & \pali{ña} \\
\hline muḍḍhaja & \pali{ṭa} & \pali{ṭha} & \pali{ḍa} & \pali{ḍha} & \pali{ṇa} \\
\hline dantaja & \pali{ta} & \pali{tha} & \pali{da} & \pali{dha} & \pali{na} \\
\hline oṭṭhaja & \pali{pa} & \pali{pha} & \pali{ba} & \pali{bha} & \pali{ma} \\
\hline avagga & \multicolumn{5}{c|}{\pali{ya\ \ ra\ \ la\ \ va\ \ sa\ \ ha\ \ ḷa\ \ aṃ}} \\
\hline
\end{tabular}
\end{table}

\head{3}{3, 3. tatthodantā sarā aṭṭha.}
\headtrans{In those [forty-one], eight, ending by \pali{o}, [are] `\pali{sara}.'}
\sutdef{tattha akkharesu akārādīsu odantā aṭṭha akkharā sarā nāma honti.}
\sutdeftrans{In those letters, eight letters such as \pali{a} and so on to \pali{o} at the end are called `\pali{sara}' (vowels).}

\head{4}{4, 4. lahumattā tayo rassā.}
\headtrans{Three [vowels in those eight] with short-meter [are] `\pali{rassa}.'}
\sutdef{tattha aṭṭhasu saresu lahumattā tayo sarā rassā nāma honti.}
\sutdeftrans{In those eight vowels, three vowels with short-meter are called `\pali{rassa}' (short).}
\transnote{The short three vowels are \pali{a, i, u}.}

\head{5}{5, 5. aññe dīghā.}
\headtrans{The other [vowels are] `\pali{dīgha}.'}
\sutdef{tattha aṭṭhasu saresu rassehi aññe pañca sarā dīghā nāma honti.}
\sutdeftrans{In those eight vowels, five vowels other than the short are called `\pali{dīgha}' (long).}
\transnote{The long five vowels are \pali{ā, ī, ū, e, o}.}

\head{6}{6, 8. sesā byañjanā.}
\headtrans{The rest [are] `\pali{byañjana}.'}
\sutdef{ṭhapetvā aṭṭha sare sesā akkharā kakārādayo niggahitantā byañjanā nāma honti.}
\sutdeftrans{Except the eight vowels, the remaining letters, such as \pali{ka} and so on to \pali{niggahita} at the end are called `\pali{byañjana}' (consonants)}

\head{7}{7, 9. vaggā pañcapañcaso mantā.}
\headtrans{[Those] by \pali{ma} at the end, [grouped] by five each, [are] `\pali{vagga}.'}
\sutdef{tesaṃ kho byañjanānaṃ kakārādayo makārantā pañcapañcaso akkharavanto vaggā nāma honti.}
\sutdeftrans{Of these consonants, such as \pali{ka} and so on to \pali{ma} at the end, those having letters grouped by five each are called `\pali{vagga}' (group).}
\transnote{There are five groups by this arrangement, i.e., \pali{ka kha ga gha ṅa, ca cha ja jha ña, ṭa ṭha ḍa ḍha ṇa, ta tha da dha na, pa pha ba bha ma}.}

\head{8}{8, 10. aṃ iti niggahitaṃ.}
\headtrans{The letter \pali{aṃ} [is] `\pali{niggahita}.'}
\sutdef{aṃ iti niggahitaṃ nāma hoti.}
\sutdeftrans{The letter \pali{aṃ} is called `\pali{niggahita}.'}

\head{9}{9, 11. parasamaññā payoge.}
\headtrans{[When] others' grammatical terms [are] in use.}
\sutdef{yā ca pana paresu sakkataganthesu samaññā ghosāti vā agho\-sāti vā, tā payoge sati etthāpi yujjante.}
\sutdeftrans{When which grammatical terms, such as \pali{ghosa} or \pali{aghosa}, [are used] in other Sanskrit texts; that [terms], when applicable, are also used here.}
\transnote{This sutta is difficult to translate cleanly. In its definition, there is an absolute construction or a relative clause with locative case (\pali{sati} is a locative form of \pali{santa}). The \pali{ya-ta} structure also makes the sentence complicated, but understandable nonetheless.}

\head{10}{10, 12. pubbamadhoṭhitamassaraṃ sarena viyojaye.}
\headtrans{[One should] separate [the term] by vowel, making the preceding consonant stand below, vowel-free.}
\sutdef{tattha sandhiṃ kattukāmo pubbabyañjanaṃ adhoṭhitaṃ assaraṃ katvā sarañca upari katvā sarena viyojaye}
\sutdeftrans{There, one who want to do a sandhi should separate the preceding consonant by vowel to make it stand below; making it vowel-free and putting the vowel above.}
\transnote{We can break \pali{pubbamadhoṭhitamassaraṃ} to \pali{pubbaṃ + adhoṭhitaṃ + assaraṃ}. The last part means vowel-free. This sutta is an operation before a sandhi is made. It is really abstract, so do not take `above' and `below' literally. This idea is in fact simple: just know the vowel so you can separate it properly (or drop it when needed).}

\head{11}{11, 14. naye paraṃ yutte.}
\headtrans{[One] should put [the above-mentioned vowel-free consonant] to the other [letter] in a proper place.}
\sutdef{assaraṃ kho byañjanaṃ adhoṭhataṃ parakkharaṃ naye yutte.}
\sutdeftrans{[One] should put the vowel-free consonant stood below to the other letter in a proper place.}
\example{tatrābhiratimiccheyya = tatra + abhiratiṃ + iccheyya}
\transnote{By the method explained above, when \pali{tatra} is put into a sandhi, make it vowel-free first, hence \pali{tatr}. Then connect it to another term properly using other rules explained below.}

