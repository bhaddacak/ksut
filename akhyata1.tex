\chapter{Ākhyāta}

This book of Kaccāyana is about Pāli verb system. Its technical term is \pali{ākhyāta (ā + \paliroot{khyā} + ta)}, literally ``[what was] said/made know.'' To be precise, we should regard this as finite verb system. For the non-finite verb system, see the next book on \pali{kibbidhāna}.

This book is relatively well-organized and quite easy to follow, comparing to other parts. It is divided into four sections, described as follows:

The first section, Sutta 406--431, introduces the eight tenses and moods of Pāli verbs, listed in Table \ref{tab:tenses}. These are indicated by groups of verbal vibhattis. To help the reader, all vibhattis are summarized in tables provided.

The second section, Sutta 432--457, describes a variety of verbal formations by using paccayas, summarized in Table \ref{tab:akhypacc}. In this section, root-groups are defined, as well as paccayas related to root-groups (\pali{vikaraṇa}). These are listed in Table \ref{tab:vikapacc}.

The third section, Sutta 458--481, explains reduplication and other special treatments upon certain roots and vibhattis/paccayas.

The fourth section, Sutta 482--523, continues the explanations on particular treatments. All roots mentioned in this and the previous section are listed in Table \ref{tab:roots-irr} toward the end.

\section{Paṭhamakaṇḍa}
\raggedbottom

\begin{table}[!hbt]
\centering
\caption{Tenses and moods in Pāli}
\label{tab:tenses}
\bigskip
\begin{tabular}{lll} \toprule
\bfseries\upshape Pāli & \bfseries\upshape English & \bfseries\upshape Suttas  \\ \midrule 
vattamānā & present & \hyperref[sut:414]{414}, \hyperref[sut:423]{423}, \hyperref[sut:431]{431}, \hyperref[sut:478]{478} \\
pañcamī & imperative & \hyperref[sut:415]{415}, \hyperref[sut:424]{424}, \hyperref[sut:431]{431}, \hyperref[sut:478]{478}, \hyperref[sut:479]{479} \\
sattamī & optative & \hyperref[sut:416]{416}, \hyperref[sut:425]{425}, \hyperref[sut:431]{431} \\
parokkhā & perfect & \hyperref[sut:417]{417}, \hyperref[sut:426]{426}, \hyperref[sut:475]{475}, \hyperref[sut:507]{507}, \hyperref[sut:516]{516} \\
hiyyattanī & imperfect & \hyperref[sut:418]{418}, \hyperref[sut:420]{420}, \hyperref[sut:427]{427}, \hyperref[sut:431]{431} \\
ajjatanī & aorist & \hyperref[sut:419]{419}, \hyperref[sut:420]{420}, \hyperref[sut:428]{428}, \hyperref[sut:507]{507}, \hyperref[sut:516]{516} \\
bhavissanti & future & \hyperref[sut:421]{421}, \hyperref[sut:429]{429}, \hyperref[sut:480]{480}, \hyperref[sut:481]{481}, \hyperref[sut:507]{507}, \hyperref[sut:516]{516} \\
kālātipatti & conditional & \hyperref[sut:422]{422}, \hyperref[sut:430]{430}, \hyperref[sut:507]{507}, \hyperref[sut:516]{516} \\
\bottomrule
\end{tabular}
\end{table}

\head{406}{406, 429. atha pubbāni vibhattīnaṃ cha parassapadāni.}
\headtrans{Now then, the first six of vibhattis [are] \pali{parassapada}.}
\sutdef{atha sabbāsaṃ vibhattīnaṃ yāni yāni pubbakāni cha padāni, tāni tāni parassapadasaññāni honti.}
\sutdeftrans{Now then, of all [twelve verbal] vibhattis, which [are] the first six terms each, those are called \pali{parassapada} (term for other).}
\transnote{As a technical term, we do not translate \pali{parassapada} (Skt.\ \pali{parasmaipada}) and use it as such. For present tense, these six are \pali{ti, anti; si, tha; mi, ma}.}

\head{407}{407, 439. parānyattanopadāni.}
\headtrans{The latter six terms [are called] \pali{attanopada}.}
\sutdef{sabbāsaṃ vibhattīnaṃ yāni yāni parāni cha padāni, tāni tāni attanopadasaññāni honti.}
\sutdeftrans{Of all [twelve verbal] vibhattis, which [are] the latter six terms each, those are called \pali{attanopada} (term for oneself).}
\transnote{Like the former, we use \pali{attanopada} (Skt.\ \pali{ātmanepada}) as a technical term. For present tense, they are \pali{te, ante; se, vhe; e, mhe}.}

\head{408}{408, 431. dve dve paṭhamamajjhimuttamapurisā.}
\headtrans{[Of all those vibhattis], the pair of terms each [are called] \pali{paṭhama}-, \pali{majjhima}-, and \pali{uttama}-purisa.}
\sutdef{tāsaṃ sabbāsaṃ vibhattīnaṃ parassapadānaṃ attanopadānañca dve dve padāni paṭhamamajjhimuttamapurisasaññāni honti.}
\sutdeftrans{Of all those vibhattis of parassapada and attanopada, the pair of terms each are called \pali{paṭhama}-, \pali{majjhima}-, and \pali{uttama}-purisa.}
\transnote{Comparing to English the names used are reverse in order, thus \pali{paṭhama} is 3rd person, \pali{majjhima} is 2nd person, and \pali{uttama} is 1st person. The pair of terms mentioned are singular (\pali{ekavacana}) and plural (\pali{bahuvacana}). All twelve verbal vibhattis of present tense are shown in Table \ref{tab:vibhat-vat}.}

\begin{table}[!hbt]
\centering
\caption{Vibhattis for present tense}
\label{tab:vibhat-vat}
\bigskip
\begin{tabular}{l*{4}{>{\itshape}l}} \toprule
\multirow{2}{*}{\textbf{purisa}} & \multicolumn{2}{c}{\textbf{parassapada}} & \multicolumn{2}{c}{\textbf{attanopada}} \\
\cline{2-5} & \bfseries\upshape\small eka & \bfseries\upshape\small bahu & \bfseries\upshape\small eka & \bfseries\upshape\small bahu \\ \midrule 
paṭhama (3rd) & ti & anti & te & ante \\
majjhima (2nd) & si & tha & se & vhe \\
uttama (1st) & mi & ma & e & mhe \\
\bottomrule
\end{tabular}
\end{table}

\head{409}{409, 441. sabbesamekābhidhāne paro puriso.}
\headtrans{In an expression [with one verb] of all persons, the person of the last [term] should be taken.}
\sutdef{sabbesaṃ tiṇṇaṃ paṭhamamajjhimuttamapurisānaṃ ekā\-bh\-idhāne paro puriso gahetabbo.}
\sutdeftrans{In an expression [with one verb] of all three persons, i.e., paṭhama, majjhima, and uttama, the person of the last [term] should be taken.}
\example[0]{so ca paṭhati, tvañca paṭhasi, tumhe paṭhatha\\{\upshape= He also recites, you also recite, [thus] you all recite.}}
\example[0]{so ca pacati, tvañca pacasi, tumhe pacatha\\{\upshape= He also cooks, you also cook, [thus] you all cook.}}
\example{so ca pacati, ahañca pacāmi, amhe pacāma\\{\upshape= He also cooks, I also cook, [thus] we cook.}}
\transnote{The point is when a sentence has multiple subjects doing the same action, the verb's person will correspond to the last subject. The last example is my addition. If you are still confused, consider the followings: \pali{so puriso ca ahaṃ cāpi pacāma} (not \pali{pacanti}), and \pali{sā kaññā ca tvaṃ cāpi pacatha} (not \pali{pacanti}), but \pali{tvaṃ ca sā kaññā cāpi pacanti} (not \pali{pacatha}).}

\head{410}{410, 432. nāmamhi payujjamānepi tulyādhikaraṇe paṭhamo.}
\headtrans{When a noun is explicitly [or implicitly] stated, being the subject of the verb, [then the verb] is paṭhama-purisa.}
\sutdef{nāmamhi payujjamānepi appayujjamānepi tulyādhikaraṇe paṭhamapuriso hoti.}
\sutdeftrans{When a noun is explicitly or implicitly stated, being the subject of the verb, [then the verb] is paṭhama-purisa (3rd person).}
\example[0]{so gacchati {\upshape (He goes.)}}
\example[0]{te gacchanti {\upshape (They go.)}}
\example[0]{gacchati {\upshape ([One] goes.)}}
\example{gacchanti {\upshape ([They] go.)}}
\transnote{The first two examples have explicit subject, the last two implicit. We have met \pali{tulyādhikaraṇa} before in kammadhāraya-samāsa (\hyperref[sut:324]{Kacc 324}). In that context, it means two terms representing the same object (apposition). In this context, it means the noun and verb equally/directly relate to each other.}

\head{411}{411, 436. tumhe majjhimo.}
\headtrans{When \pali{tumha} is explicitly [or implicitly] stated, being the subject of the verb, [then the verb] is majjhima-purisa.}
\sutdef{tumhe payujjamānepi appayujjamānepi tulyādhikaraṇe majjhimapuriso hoti.}
\sutdeftrans{When \pali{tumha} is explicitly or implicitly stated, being the subject of the verb, [then the verb] is majjhima-purisa (2nd person).}
\example[0]{[tvaṃ] yāsi {\upshape (You go.)}}
\example{[tumhe] yātha {\upshape (You all go.)}}

\head{412}{412, 437. amhe uttamo.}
\headtrans{When \pali{amha} is explicitly [or implicitly] stated, being the subject of the verb, [then the verb] is uttama-purisa.}
\sutdef{amhe payujjamānepi appayujjamānepi tulyādhikaraṇe uttamapuriso hoti.}
\sutdeftrans{When \pali{amha} is explicitly or implicitly stated, being the subject of the verb, [then the verb] is uttama-purisa (1st person).}
\example[0]{[ahaṃ] yajāmi {\upshape (I sacrifice.)}}
\example{[mayaṃ] yajāma {\upshape (We sacrifice.)}}

\head{413}{413, 427. kāle.}
\headtrans{In time.}
\sutdef{“kāle” iccetaṃ adhikāratthaṃ veditabbaṃ.}
\sutdeftrans{This `in time' should be seen as a heading.}
\transnote{The heading covers tenses and moods.}

\newpage
\head{414}{414, 428. vattamānā paccuppanne.}
\headtrans{[There is] \pali{vattamānā}-vibhatti in the present tense.}
\sutdef{paccuppanne kāle vattamānāvibhatti hoti.}
\sutdeftrans{In the present tense, there is \pali{vattamānā}-vibhatti.}
\example[0]{pāṭaliputtaṃ gacchati {\upshape ([One] goes to Pāṭaliputta.)}}
\example{sāvatthiṃ pavisati {\upshape ([One] enters Sāvatthī.)}}

\head{415}{415, 451. āṇatyāsiṭṭhe’nuttakāle pañcamī.}
\headtrans{In the sense of command and expectation in unspecific time, [there is] \pali{pañcamī}-vibhatti.}
\sutdef{āṇatyatthe ca āsīsatthe ca anuttakāle pañcamīvibhatti hoti.}
\sutdeftrans{In the senses of command and expectation in unspecific time, there is \pali{pañcamī}-vibhatti.}
\example[0]{karotu kusalaṃ {\upshape ([Let one] do good.)}}
\example{sukhaṃ te hotu {\upshape (Let your happiness exists.)}}
\transnote{In English, this is called \emph{imperative} mood. In the last example, it is common in English to say ``May you be happy.''}

\head{416}{416, 454. anumatiparikappatthesu sattamī.}
\headtrans{In the senses of consent and assumption, [there is] \pali{sattamī}-vibhatti.}
\sutdef{anumatyatthe ca parikappatthe ca anuttakāle sattamīvi\-bh\-atti hoti.}
\sutdeftrans{In the senses of consent and assumption in unspecific time, there is \pali{sattamī}-vibhatti.}
\example[0]{tvaṃ gaccheyyāsi {\upshape (You may go.)}}
\example{kimahaṃ kareyyāmi {\upshape (What should I do?)}}
\transnote{In English, this is called \emph{optative} mood.}

\newpage
\boxnote{The fifth and seventh of what?\\
\hspace{5mm}\bullet\ As I have attempted to find out for my curiosity, the order in fact came from Pāṇinian grammar when the \pali{lakāra}s (technical terms for tenses and moods) are ordered.\\
\hspace{5mm}\bullet\ If the Vedic subjunctive (\pali{leṭ}) is left out, they go like this: \pali{laṭ} (present), \pali{liṭ} (perfect), \pali{luṭ} (periphrastic future), \pali{lṛṭ} (future), \pali{loṭ} (imperative), \pali{laṅ} (imperfect), \pali{luṅ} (optative), \pali{lṛṅ} (conditional).\footnote{See \citealp{collins:grammar}, p.~14.}\\
\hspace{5mm}\bullet\ By the order above, imperative mood is in the fifth, thus \pali{pañcamī}, and optative the seventh, thus \pali{sattamī}.\\
\hspace{5mm}\bullet\ Except periphrastic future, Pāli have all these tenses and moods.
}

\head{417}{417, 460. apaccakkhe parokkhātīte.}
\headtrans{In the unseen past, [there is] \pali{parokkhā}-vibhatti.}
\sutdef{apaccakkhe atīte kāle parokkhāvibhatti hoti.}
\sutdeftrans{In the unseen past, there is \pali{parokkhā}-vibhatti.}
\example[0]{supine kilamāha {\upshape ([One] said such and such in the dream.)}}
\example{evaṃ kila porāṇāhu {\upshape (The ancients said thus such and such.)}}
\transnote{Scholars call this tense \emph{perfect}. As an alternative form, \pali{kila} is the same as \pali{kira}. The term has no English equivalent, but can be understood as `so-said' or `so-reported.'}

\head{418}{418, 456. hiyyopabhuti paccakkhe hiyyattanī.}
\headtrans{In the past starting from yesterday, seen [or unseen, there is] \pali{hiyyattanī}-vibhatti}
\sutdef{hiyyopabhuti atīte kāle paccakkhe vā apaccakkhe vā hiyyattanīvibhatti hoti.}
\sutdeftrans{In the past starting from yesterday, seen or unseen, there is \pali{hiyyattanī}-vibhatti}
\example[0]{so agamā maggaṃ {\upshape (He went to the path.)}}
\example{te agamū maggaṃ {\upshape (They went to the path.)}}
\transnote{Scholars call this tense \emph{imperfect}.}

\head{419}{419, 469. samīpe’jjatanī.}
\headtrans{In the near [past, there is] \pali{ajjatanī}-vibhatti.}
\sutdef{ajjappabhuti atīte kāle paccakkhe vā apaccakkhe vā samīpe ajjatanīvibhatti hoti.}
\sutdeftrans{In the near past starting from today, seen or unseen, there is \pali{ajjatanī}-vibhatti.}
\example[0]{so maggaṃ agamī {\upshape (He went to the path.)}}
\example{te maggaṃ agamuṃ {\upshape (They went to the path.)}}
\transnote{Scholars call this tense \emph{aorist}. In practice, the nuances of perfect, imperfect, and aorist are inconceivable. We use distinct names just for classification.}

\head{420}{420, 471. māyoge sabbakāle ca.}
\headtrans{Because of an application of \pali{mā}, [\pali{hiyyattanī}- and \pali{ajjatanī}-vibhatti] are in all times also.}
\sutdef{hiyyattanīajjatanīiccetā vibhattiyo yadā māyogā, tadā sabbakāle ca honti.}
\sutdeftrans{Whenever there are \pali{hiyyattanī}- and \pali{ajjatanī}-vibhatti applied with \pali{mā}, then they are also in all times.}
\example[0]{mā gamā {\upshape (Let one not go.)}}
\example[0]{mā vacā {\upshape (Let one not say.)}}
\example[0]{mā gamī {\upshape (Let one not go.)}}
\example{mā vacī {\upshape (Let one not say.)}}
\transnote{This simply means when \pali{mā} is used with imperfect and aorist tense, the meaning of the sentence may not only be in the past. It can be also in the present and future. This is a kind of idiomatic use of the tenses. By the word \pali{ca}, however, imperative and optative mood can also be used likewise.}
\example{mā gacchāhi {\upshape (Don't go.)}}

\head{421}{421, 473. anāgate bhavissantī.}
\headtrans{In the future [meaning, there is] \pali{bhavissanti}-vibhatti.}
\sutdef{anāgate kāle bhavissantīvibhatti hoti.}
\sutdeftrans{There is \pali{bhavissanti}-vibhatti [used for meaning] in the future.}
\example[0]{so gacchissati {\upshape (He will go.)}}
\example[0]{so karissati {\upshape (He will do.)}}
\example[0]{te gacchissanti {\upshape (They will go.)}}
\example{te karissanti {\upshape (They will do.)}}
\transnote{About the name of the future tense, we normally use short-ending \pali{bhavissanti} instead. The long-ending term likely came from Sanskrit \pali{bhaviṣyantī}.}

\head{422}{422, 475. kriyā’tipanne’tīte kālātipatti.}
\headtrans{In the past which the action did not happen, [there is] \pali{kālātipatti}-vibhatti.}
\sutdef{kriyātipannamatte atīte kāle kālātipattivibhatti hoti.}
\sutdeftrans{There is \pali{kālātipatti}-vibhatti in the past which the action just did not happen.}
\transnote{To understand the definition, we have to take a look at Rūpa 475, in which an explanation goes like this: \pali{kriyāya atipatanaṃ kriyātipannaṃ, taṃ pana sādhakasattivirahena kriyāya accantānuppatti} (The passing [= non-happening] of an action is \pali{kriyātipanna}, the definite non-arising of the action with the absence of ability to bring about the effect).}
\transnote{There are some difficult words here, such as \pali{sādhakasattivirahena (sādhaka + satti + viraha + nā)}. This appears in the last part of my translation. Another one is \pali{accantānuppatti (accanta + anuppatti)}. The latter part of this can mean two opposite things: (1) attainment, \pali{anu + pa + \paliroot{ap} + ti} (Skt.\ \pali{anuprāpti}), and (2) non-arising, \pali{na + ud + \paliroot{pad} + ti} (Skt.\ \pali{anutpatti}). In this context, the second meaning is intended. For \pali{atipatana}, see MWD.}
\example[0]{so ce taṃ yānaṃ alabhissā, agacchissā \\{\upshape= If he had got that carriage, he would have gone.}}
\example{te ce taṃ yānaṃ alabhissaṃsu, agacchissaṃsu \\{\upshape= If they had got that carriage, they would have gone.}}

\head{423}{423, 426. vattamānā ti anti, si tha, mi ma, te ante, se vhe, e mhe.}
\headtrans{\pali{Vattamānā} [is] \pali{ti anti, si tha, mi ma, te ante, se vhe, e mhe}.}
\sutdef{vattamānā iccesā saññā hoti ti anti, si tha, mi ma, te ante, se vhe, e mhe iccetesaṃ dvādasannaṃ padānaṃ.}
\sutdeftrans{\pali{Vattamānā} is the designation of these twelve terms (vibhattis), i.e., \pali{ti anti, si tha, mi ma, te ante, se vhe, e mhe}.}
\transnote{It is better to see these in Table \ref{tab:vibhat-vat}.}

\head{424}{424, 450. pañcamī tu antu, hi tha, mi ma, taṃ antaṃ, ssu vho, e āmase.}
\headtrans{\pali{Pañcamī} [is] \pali{tu antu, hi tha, mi ma, taṃ antaṃ, ssu vho, e āmase}.}
\sutdef{pañcamī iccesā saññā hoti tu antu, hi tha, mi ma, taṃ antaṃ, ssu vho, e āmase iccetesaṃ dvādasannaṃ padānaṃ.}
\sutdeftrans{\pali{Pañcamī} is the designation of these twelve terms (vibhattis), i.e., \pali{tu antu, hi tha, mi ma, taṃ antaṃ, ssu vho, e āmase}.}
\transnote{See Table \ref{tab:vibhat-pan}. In Thai tradition, \pali{āmase} is \pali{āmhase} instead.}

\setcounter{table}{2}
\begin{table}[!hbt]
\centering
\caption{Vibhattis for imperative mood}
\label{tab:vibhat-pan}
\bigskip
\begin{tabular}{l*{4}{>{\itshape}l}} \toprule
\multirow{2}{*}{\textbf{purisa}} & \multicolumn{2}{c}{\textbf{parassapada}} & \multicolumn{2}{c}{\textbf{attanopada}} \\
\cline{2-5} & \bfseries\upshape\small eka & \bfseries\upshape\small bahu & \bfseries\upshape\small eka & \bfseries\upshape\small bahu \\ \midrule 
paṭhama (3rd) & tu & antu & taṃ & antaṃ \\
majjhima (2nd) & hi & tha & ssu & vho \\
uttama (1st) & mi & ma & e & āmase \\
\bottomrule
\end{tabular}
\end{table}

\head{425}{425, 453. sattamī eyya eyyuṃ, eyyāsi eyyātha, eyyāmi eyyāma, etha eraṃ, etho eyyāvho, eyyaṃ eyyāmhe.}
\headtrans{\pali{Sattamī} [is] \pali{eyya eyyuṃ, eyyāsi eyyātha, eyyāmi eyyāma, etha eraṃ, etho, eyyāvho, eyyaṃ eyyāmhe}.}
\sutdef{sattamī iccesā saññā hoti eyya eyyuṃ, eyyāsi eyyātha, eyyāmi eyyāma, etha eraṃ, etho, eyyāvho, eyyaṃ eyyāmhe iccetesaṃ dvādasannaṃ padānaṃ.}
\sutdeftrans{\pali{Sattamī} is the designation of these twelve terms (vibhattis), i.e., \pali{eyya eyyuṃ, eyyāsi eyyātha, eyyāmi eyyāma, etha eraṃ, etho, eyyāvho, eyyaṃ eyyāmhe}.}
\transnote{See Table \ref{tab:vibhat-sat}. In Thai tradition, \pali{eyyāvho} is \pali{eyyavho} instead.}

\begin{table}[!hbt]
\centering
\caption{Vibhattis for optative mood}
\label{tab:vibhat-sat}
\bigskip
\begin{tabular}{l*{4}{>{\itshape}l}} \toprule
\multirow{2}{*}{\textbf{purisa}} & \multicolumn{2}{c}{\textbf{parassapada}} & \multicolumn{2}{c}{\textbf{attanopada}} \\
\cline{2-5} & \bfseries\upshape\small eka & \bfseries\upshape\small bahu & \bfseries\upshape\small eka & \bfseries\upshape\small bahu \\ \midrule 
paṭhama (3rd) & eyya & eyyuṃ & etha & eraṃ \\
majjhima (2nd) & eyyāsi & eyyātha & etho & eyyāvho \\
uttama (1st) & eyyāmi & eyyāma & eyyaṃ & eyyāmhe \\
\bottomrule
\end{tabular}
\end{table}

\head{426}{426, 459. parokkhā a u, e ttha, aṃ mha, ttha re, ttho vho, iṃ mhe.}
\headtrans{\pali{Parokkhā} [is] \pali{a u, e ttha, aṃ mha, ttha re, ttho vho, iṃ mhe}.}
\sutdef{parokkhā iccesā saññā hoti a u, e ttha, aṃ mha, ttha re, ttho vho, iṃ mhe iccetesaṃ dvādasannaṃ padānaṃ.}
\sutdeftrans{\pali{Parokkhā} is the designation of these twelve terms (vibhattis), i.e., \pali{a u, e ttha, aṃ mha, ttha re, ttho vho, iṃ mhe}.}
\transnote{See Table \ref{tab:vibhat-par}.}

\begin{table}[!hbt]
\centering
\caption{Vibhattis for perfect tense}
\label{tab:vibhat-par}
\bigskip
\begin{tabular}{l*{4}{>{\itshape}l}} \toprule
\multirow{2}{*}{\textbf{purisa}} & \multicolumn{2}{c}{\textbf{parassapada}} & \multicolumn{2}{c}{\textbf{attanopada}} \\
\cline{2-5} & \bfseries\upshape\small eka & \bfseries\upshape\small bahu & \bfseries\upshape\small eka & \bfseries\upshape\small bahu \\ \midrule 
paṭhama (3rd) & a & u & ttha & re \\
majjhima (2nd) & e & ttha & ttho & vho \\
uttama (1st) & aṃ & mha & iṃ & mhe \\
\bottomrule
\end{tabular}
\end{table}

\head{427}{427, 455. hiyyattanī ā ū, o ttha, aṃ mhā, ttha tthuṃ, se vhaṃ, iṃ mhase.}
\headtrans{\pali{Hiyyattanī} [is] \pali{ā ū, o ttha, aṃ mhā, ttha tthuṃ, se vhaṃ, iṃ mhase}.}
\sutdef{hiyyattanī iccesā saññā hoti ā ū, o ttha, aṃ mhā, ttha tthuṃ, se vhaṃ, iṃ mhase iccetesaṃ dvādasannaṃ padānaṃ.}
\sutdeftrans{\pali{Hiyyattanī} is the designation of these twelve terms (vibhattis), i.e., \pali{ā ū, o ttha, aṃ mhā, ttha tthuṃ, se vhaṃ, iṃ mhase}.}
\transnote{See Table \ref{tab:vibhat-hiy}. In Thai tradition, \pali{mhā} is \pali{mha} instead.}

\begin{table}[!hbt]
\centering
\caption{Vibhattis for imperfect tense}
\label{tab:vibhat-hiy}
\bigskip
\begin{tabular}{l*{4}{>{\itshape}l}} \toprule
\multirow{2}{*}{\textbf{purisa}} & \multicolumn{2}{c}{\textbf{parassapada}} & \multicolumn{2}{c}{\textbf{attanopada}} \\
\cline{2-5} & \bfseries\upshape\small eka & \bfseries\upshape\small bahu & \bfseries\upshape\small eka & \bfseries\upshape\small bahu \\ \midrule 
paṭhama (3rd) & ā & ū & ttha & tthuṃ \\
majjhima (2nd) & o & ttha & se & vhaṃ \\
uttama (1st) & aṃ & mhā & iṃ & mhase \\
\bottomrule
\end{tabular}
\end{table}

\head{428}{428, 468. ajjatanī ī uṃ, o ttha, iṃ mhā, ā ū, se vhaṃ, aṃ mhe.}
\headtrans{\pali{Ajjatanī} [is] \pali{ī uṃ, o ttha, iṃ mhā, ā ū, se vhaṃ, aṃ mhe}.}
\sutdef{ajjatanī iccesā saññā hoti ī uṃ, o ttha, iṃ mhā, ā ū, se vhaṃ, aṃ mhe iccetesaṃ dvādasannaṃ padānaṃ.}
\sutdeftrans{\pali{Ajjatanī} is the designation of these twelve terms (vibhattis), i.e., \pali{ī uṃ, o ttha, iṃ mhā, ā ū, se vhaṃ, aṃ mhe}.}
\transnote{See Table \ref{tab:vibhat-ajj}.}

\begin{table}[!hbt]
\centering
\caption{Vibhattis for aorist tense}
\label{tab:vibhat-ajj}
\bigskip
\begin{tabular}{l*{4}{>{\itshape}l}} \toprule
\multirow{2}{*}{\textbf{purisa}} & \multicolumn{2}{c}{\textbf{parassapada}} & \multicolumn{2}{c}{\textbf{attanopada}} \\
\cline{2-5} & \bfseries\upshape\small eka & \bfseries\upshape\small bahu & \bfseries\upshape\small eka & \bfseries\upshape\small bahu \\ \midrule 
paṭhama (3rd) & ī & uṃ & ā & ū \\
majjhima (2nd) & o & ttha & se & vhaṃ \\
uttama (1st) & iṃ & mhā & aṃ & mhe \\
\bottomrule
\end{tabular}
\end{table}

\head{429}{429, 472. bhavissantī ssati ssanti, ssasi ssatha, ssāmi ssāma, ssate ssante, ssase ssavhe, ssaṃ ssāmhe.}
\headtrans{\pali{Bhavissanti} [is] \pali{ssati ssanti, ssasi ssatha, ssāmi ssāma, ssate ssante, ssase ssavhe, ssaṃ ssāmhe}.}
\sutdef{bhavissantī iccesā saññā hoti ssati ssanti, ssasi ssatha, ssāmi ssāma, ssate ssante, ssase ssavhe, ssaṃ ssāmhe iccetesaṃ dvādasannaṃ padānaṃ.}
\sutdeftrans{\pali{Bhavissanti} is the designation of these twelve terms (vibhattis), i.e., \pali{ssati ssanti, ssasi ssatha, ssāmi ssāma, ssate ssante, ssase ssavhe, ssaṃ ssāmhe}.}
\transnote{See Table \ref{tab:vibhat-bha}.}

\begin{table}[!hbt]
\centering
\caption{Vibhattis for future tense}
\label{tab:vibhat-bha}
\bigskip
\begin{tabular}{l*{4}{>{\itshape}l}} \toprule
\multirow{2}{*}{\textbf{purisa}} & \multicolumn{2}{c}{\textbf{parassapada}} & \multicolumn{2}{c}{\textbf{attanopada}} \\
\cline{2-5} & \bfseries\upshape\small eka & \bfseries\upshape\small bahu & \bfseries\upshape\small eka & \bfseries\upshape\small bahu \\ \midrule 
paṭhama (3rd) & ssati & ssanti & ssate & ssante \\
majjhima (2nd) & ssasi & ssatha & ssase & ssavhe \\
uttama (1st) & ssāmi & ssāma & ssaṃ & ssāmhe \\
\bottomrule
\end{tabular}
\end{table}

\head{430}{430, 373. kālātipatti ssā ssaṃsu, sse ssatha, ssaṃ ssāmhā, ssatha ssisu, ssase ssavhe, ssiṃ ssāmhase.}
\headtrans{\pali{Kālātipatti} [is] \pali{ssā ssaṃsu, sse ssatha, ssaṃ ssāmhā, ssatha ssisu, ssase ssavhe, ssiṃ ssāmhase}.}
\sutdef{kālātipatti iccesā saññā hoti ssā ssaṃsu, sse ssatha, ssaṃ ssāmhā, ssatha ssisu, ssase ssavhe, ssiṃ ssāmhase iccetesaṃ dvādasannaṃ padānaṃ.}
\sutdeftrans{\pali{Kālātipatti} is the designation of these twelve terms (vibhattis), i.e., \pali{ssā ssaṃsu, sse ssatha, ssaṃ ssāmhā, ssatha ssisu, ssase ssavhe, ssiṃ ssāmhase}.}
\transnote{See Table \ref{tab:vibhat-kal}. In Thai tradition, \pali{ssisu} is \pali{ssiṃsu}, and \pali{ssiṃ} is \pali{ssaṃ} instead.}

\begin{table}[!hbt]
\centering
\caption{Vibhattis for conditional mood}
\label{tab:vibhat-kal}
\bigskip
\begin{tabular}{l*{4}{>{\itshape}l}} \toprule
\multirow{2}{*}{\textbf{purisa}} & \multicolumn{2}{c}{\textbf{parassapada}} & \multicolumn{2}{c}{\textbf{attanopada}} \\
\cline{2-5} & \bfseries\upshape\small eka & \bfseries\upshape\small bahu & \bfseries\upshape\small eka & \bfseries\upshape\small bahu \\ \midrule 
paṭhama (3rd) & ssā & ssaṃsu & ssatha & ssisu \\
majjhima (2nd) & sse & ssatha & ssase & ssavhe \\
uttama (1st) & ssaṃ & ssāmhā & ssiṃ & ssāmhase \\
\bottomrule
\end{tabular}
\end{table}

\head{431}{431, 458. hiyyattanī sattamī pañcamī vattamānā sabbadhātukaṃ.}
\headtrans{[The vibhattis of] hiyyattanī, sattamī, pañcamī, and vattamānā [are] \pali{sabbadhātuka}.}
\sutdef{hiyyattanādayo catasso vibhattiyo sabbadhātukasaññā honti.}
\sutdeftrans{The four [groups of] vibhattis of hiyyattanī and so on (sattamī, pañcamī, and vattamānā) are called \pali{sabbadhātuka}[-vibhatti] (applicable to all roots).}
\example[0]{agamā {\upshape ([One] went.)}}
\example[0]{gaccheyya {\upshape ([One] may go.)}}
\example[0]{gacchatu {\upshape (Let [one] go.)}}
\example{gacchati {\upshape ([One] goes.)}}

