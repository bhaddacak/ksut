\section{Pañcamakaṇḍa}

\head{42}{42, 32. go sare puthassāgamo kvaci.}
\headtrans{[There is an insertion of] \pali{ga} for \pali{putha} because of [the other] vowel in some places.}
\sutdef{puthaiccetassa ante sare pare kvaci gakārāgamo hoti.}
\sutdeftrans{There is an insertion of \pali{ga} at the end of [this] \pali{putha} in some places because of the other vowel.}
\example{puthageva = putha + eva}
\transnote{\pali{Puthaiccetassa} can be cut as \pali{putha + iti + etassa}. The \pali{iti} and \pali{eta} add no specific meaning. This sutta is an example of a treatment of a single peculiar instance.}

\head{43}{43, 33. pāssa canto rasso.}
\headtrans{Also for \pali{pā}, the ending [vowel] shortened.}
\sutdef{pāiccetassa ante sare pare kvaci gakārāgamo hoti, anto ca saro rasso hoti.}
\sutdeftrans{There is an insertion of \pali{ga} at the end of \pali{pā} in some places because of the other vowel. The ending vowel also becomes shortened.}
\example{pageva = pā + eva}

\head{44}{44, 24. abbho abhi.}
\headtrans{[There is a substitution of] \pali{abbha} for \pali{abhi}.}
\sutdef{abhiiccetassa sare pare abbhādeso hoti.}
\sutdeftrans{There is a substitution of \pali{abbha} for \pali{abhi}.}
\example[0]{abbhudīritaṃ = abhi + udīritaṃ}
\example{abbhuggacchati = abhi + uggacchati}

\head{45}{45, 25. ajjho adhi.}
\headtrans{[There is a substitution of] \pali{ajjha} for \pali{adhi}.}
\sutdef{adhiiccetassa sare pare ajjhādeso hoti.}
\sutdeftrans{There is a substitution of \pali{ajjha} for \pali{adhi}.}
\example[0]{ajjhokāse = adhi + okāse}
\example{ajjhāgamā = adhi + āgamā}

\head{46}{46, 26. te na vā ivaṇṇe.}
\headtrans{Sometimes [there are] no [substitutions of] those because of \pali{i}-group [vowels].}
\sutdef{te ca kho abhiadhiiccete ivaṇṇe pare abbho ajjhoitivuttarūpā na honti vā.}
\sutdeftrans{There are no substitutions of \pali{abbha} and \pali{ajjha} form mentioned in those \pali{abhi} and \pali{adhi} because of \pali{i}-group in the other vowel sometimes.}
\example[0]{abhicchitaṃ = abhi + icchitaṃ}
\example{adhīritaṃ = adhi + īritaṃ}

\head{47}{47, 23. atissa cantassa.}
\headtrans{[No substitution] of \pali{ca} at the end of \pali{ati}.}
\sutdef{atiiccetassa antabhūtassa tisaddassa ivaṇṇe pare “sabbo caṃ tī”ti vuttarūpaṃ na hoti.}
\sutdeftrans{There is no substitution [of \pali{ca}] for \pali{ti} at the end of \pali{ati} as mentioned in ``\pali{sabbo caṃ ti}'' (\hyperref[sut:19]{Kacc 19}) because of \pali{i}-group in the other vowel.}
\example[0]{atīsigaṇo = ati + isigaṇo}
\example{atīritaṃ = ati + iritaṃ}

\head{48}{48, 43. kvaci paṭi patissa.}
\headtrans{In some places, [there is] \pali{paṭi}-substitution for \pali{pati}.}
\sutdef{patiiccetassa sare vā byañjane vā pare kvaci paṭiādeso hoti.}
\sutdeftrans{There is a substitution of \pali{paṭi} for \pali{pati} in some places because of the other vowel or consonant.}
\example[0]{paṭaggi = pati + aggi}
\example{paṭihaññati = pati + haññati}

\head{49}{49, 44. puthassu byañjane.}
\headtrans{[There is] \pali{u} for [the end of] \pali{putha} because of consonant.}
\sutdef{puthaiccetassa anto saro byañjane pare ukāro hoti.}
\sutdeftrans{There is \pali{u} as the last vowel of \pali{putha} because of the other consonant.}
\example[0]{puthujjano = putha + jano}
\example{puthubhūtaṃ = putha + bhūtaṃ}
\transnote{\pali{Puthassu} can be cut as \pali{puthassa + u}.}

\head{50}{50, 45. o avassa.}
\headtrans{[A substitution of] \pali{o} for \pali{ava}.}
\sutdef{avaiccetassa byañjane pare kvaci okāro hoti.}
\sutdeftrans{There is a substitution of \pali{o} for \pali{ava} in some places because of the other consonant.}
\example{onaddhā = ava + naddhā}

\head{51}{51, 59. anupadiṭṭhānaṃ vuttayogato.}
\headtrans{For those unspecified, [use] the application from what are said [previously].}
\sutdef{anupadiṭṭhānaṃ upasagganipātānaṃ sarasandhīhi byañjanasandhīhi vuttasandhīhi ca yathāyogaṃ yojetabbaṃ.}
\sutdeftrans{[The uses of] suffixes and particles not specified [previously] should be applied as said by the sandhis of vowel, consonant, and niggahita.}
\example[0]{pāpanaṃ = pa + āpanaṃ}
\example[0]{parāyaṇaṃ = para + āyaṇaṃ}
\example[0]{upāyanaṃ = upa + āyanaṃ}
\example[0]{upāhanaṃ = upa + āhanaṃ}
\example[0]{nyāyogo = ni + ā + yogo}
\example[0]{nirupadhi = ni + upadhi}
\example[0]{anubodho = anu + bodho}
\example[0]{duvūpasantaṃ = du + upa + santaṃ}
\example[0]{suvūpasantaṃ = su + upa + santaṃ}
\example[0]{dvālayo = du + ālayo}
\example[0]{svālayo = su + ālayo}
\example[0]{durākhyātaṃ = du + ākhyātaṃ}
\example[0]{svākhyāto = su + ākhyāto}
\example[0]{udīritaṃ = u + īritaṃ}
\example[0]{samuddiṭṭhaṃ = saṃ + uddiṭṭhaṃ}
\example[0]{viyaggaṃ = vi + aggaṃ}
\example[0]{vijjhaggaṃ = vi + adhi + aggaṃ}
\example[0]{byaggaṃ = vi + aggaṃ}
\example[0]{avayāgamanaṃ = ava + āgamanaṃ}
\example[0]{anveti = anu + eti}
\example[0]{anupaghāto = ana + upa + ghāto}
\example[0]{anacchariyaṃ = ana + acchariyaṃ}
\example[0]{pariyesanā = pari + esanā}
\example[0]{parāmāso = para + āmāso}
\example[0]{pariggaho = pari + gaho}
\example[0]{paggaho = pa + gaho}
\example[0]{pakkamo = pa + kamo}
\example[0]{parakkamo = para + kamo}
\example[0]{nikkamo = ni + kamo}
\example[0]{nikkasāvo = ni + kasāvo}
\example[0]{nillayanaṃ = ni + layanaṃ}
\example[0]{dullayanaṃ = du + layanaṃ}
\example[0]{dubbhikkhaṃ = du + bhikkhaṃ}
\example[0]{dubbuttaṃ = du + buttaṃ}
\example[0]{sandiṭṭhaṃ = saṃ + diṭṭhaṃ}
\example[0]{duggaho = du + gaho}
\example[0]{viggaho = vi + gaho}
\example[0]{niggato = ni + gato}
\example[0]{abhikkamo = abhi + kamo}
\example{paṭikkamo = paṭi + kamo}
\transnote{\pali{Anupadiṭṭhānaṃ} is \pali{na + upadiṭṭhānaṃ}. Before a vowel, \pali{na} (not) takes \pali{ana} form.}





